% this is a part of the habilitation thesis of Max Neunhoeffer

\chapter{Finding reductions for subfield and semilinear groups}
\label{chap:subsemi}

This chapter is about a polynomial-time reduction algorithm 
for groups of semilinear or subfield class.
The contents of this
chapter are joint work with Jon F.~Carlson and Colva M.~Roney-Dougal and 
are already published as \cite{subfieldpaper}.

% Rest is copied from the subfield paper.
% Changes:
%  \  after \GAP:
%  {theorem} --> {Theo}
%  {proposition} --> {Prop}
%  {lemma} --> {Lemm}
%  {corollary} --> {Cor}
%  {definition} --> {Def}
%  {hyp} --> {Hyp}
%  \BBF --> \F
%  \bF --> \F
%  \boldmath --> \mathbf
%  label main --> label subsemi:main
%  label sec:complexity --> label subsemi:sec:complexity
%  label timings --> label subsemi:timings
%  cite Asch --> cite aschbacher
%  cite CurtisReiner --> cite CR0
%  cite GAP --> cite GAP4
%  cite SMASH --> cite smashnormal
%  cite Primitive --> cite smashprim
%  cite MeatAxeHR --> MeatAxeHoltRees
%  cite MeatAxe --> MeatAxeRP
%  cite issac --> AkosMaxISSAC
%  cite AkosPermGrp --> Ser


\section{Introduction}\label{sec:intro}

The matrix group recognition project was begun some years ago 
by Neumann and Praeger in a groundbreaking paper
\cite{neumann-praeger}. Their results answered the question of how 
one can determine computationally whether a given set 
of invertible matrices with entries in a finite 
field $\F_q$ generates the group $\SL(d,q)$. 
Since then many algorithms for computing with matrix groups 
over finite fields have been developed. Given a collection 
$g_1, \ldots, g_m$ of  matrices in $\GL(d, q)$, the 
basic problem is to find a composition series for the 
group $G$ that they generate and to be able to express arbitrary 
group elements as straight line programs in the generators.  
\index{straight line program}\index{SLP}%
An overview of the  aims of the recognition 
project  is given in \cite{MatGrpProj}.

The overall approach of  the project relies on a fundamental 
theorem of Aschbacher \cite{aschbacher} on the maximal
subgroups of classical groups. The theorem says that every subgroup of
$\GL(d,q)$ lies in at least one of nine classes $\CC1, \ldots, \CC9$.
Classes $\CC1, \ldots, \CC8$ are treated by reducing to 
some sort of easier setting, and there are algorithms  
for these cases. However, the
complexity of some of them has not been analysed and many
do not run in polynomial time. The overall project is currently in a 
second phase, producing provably polynomial-time algorithms for
each class. 

The basic approach (see \cite{MatGrpProj, AkosMaxISSAC}) 
is first to reduce 
the problem by either finding a proper nontrivial homomorphism 
from $G$ or finding an isomorphism to a representation 
with a smaller ambient group (for example to $\GL(d, q_0)$ 
for  $q_0 < q$).  
If a homomorphism is found then the kernel and image are treated
separately, eventually producing a
\emph{composition tree}, whose leaves are either simple groups or
groups that can be constructively  recognised by other means.
\index{leaf}%

Let 
$\overline{G} \leq \PGL(d, q)$ be the corresponding projective group. 
In this chapter we present a fully analysed, polynomial-time Las
Vegas algorithm to find a reduction for the case that  $G$ or $\overline{G}$
 is not in $\CC1$, but is in $\CC3$ or $\CC5$, 
or has non-absolutely-irreducible derived group. 
Class $\CC1$ (reducible groups)
 is completely under control using MeatAxe methods
\cite{MeatAxeHoltRees, IL, MeatAxeRP}.
In the case that $G$ or $\overline{G}$ is in \CC3 or \CC5 
we either find a proper nontrivial homomorphism from $G$ to a permutation, matrix or projective group, or an isomorphism writing $G$ or $\overline{G}$ over a smaller field, or in a smaller dimension. 
  In addition, for some groups in classes \CC2, \CC4 or \CC6,
 we find a nontrivial reduction homomorphism: 
this is important as there is as yet no fully 
polynomial-time analysis for these classes. 

Our algorithms are efficient in the sense that they use a 
number of field operations that is bounded by a low-degree polynomial 
in $m$ (the number of generators), $d$ and $\log q$: 
the input size is  $O(m d^2 \log q)$. We analyse 
the complexity of all algorithms during the course 
of the chapter, and have implemented our work in \GAP\ \cite{GAP4}. 
We avoid the use of a discrete logarithm oracle. We use $3$ for
 the exponent of matrix multiplication, as although the theoretical 
exponent is lower than this, for practical implementations this is 
more realistic. 

In addition to developing reduction algorithms, we characterise 
the groups which have a faithful absolutely irreducible module
on which the derived group acts by scalar matrices. 
We also develop efficient Monte Carlo 
methods (see Section~\ref{montevegas})
for generating subgroups of matrix groups that behave 
like normal subgroups. The use of Clifford's Theorem 
upgrades these algorithms to Las Vegas. 

One of the motivations for this work is a 
recent article by Glasby, Leedham-Green and 
O'Brien \cite{GLGOB}, who develop an algorithm 
to recognise  groups $G$ in class \CC5,  
generalising   
\cite{GlasbyHowlett}. The algorithm in \cite{GLGOB} 
is polynomial time provided that the commutator 
subgroup acts absolutely 
irreducibly. Here we address the case where $G'$ 
is not absolutely irreducible by providing fully 
analysed algorithms for all 
actions of $G'$, including the case where $G'$ 
consists only of scalars.  In the paper \cite{GLGOB} 
the authors appear to misstate the complexity of 
generating $G'$. To the best of our knowledge,
the best published complexity for this is 
$O(d^7 \log^2 q)$. In this chapter, we develop Las Vegas methods to generate 
a normal subgroup of $G$ that is contained in $G'$ in $O(d^4 \log q)$ field 
operations. 

Our approach in Sections~\ref{subsec:semilin} 
to \ref{subsec:tensor} is heavily influenced 
by {\sc Smash} \cite{smashnormal} and we have reused 
\index{Smash@\textsc{Smash}}%
many subroutines. There are two main differences between 
these sections and the original treatment in 
{\sc Smash}. Firstly, we have analysed the probability of having
generators for a subgroup 
that has the same submodule lattice as a normal 
subgroup of $G$. Secondly, we have improved algorithms and 
complexity estimates for finding an irreducible 
submodule of a normal subgroup. Hence we are able to derive 
tighter upper bounds on the complexity
of our algorithm.

The layout of this chapter is as follows. In Section~\ref{sec:defns}
 we present basic definitions and our main result. In 
Section~\ref{sec:describe_scalars} we characterise the groups with 
faithful absolutely irreducible representations over arbitrary fields
 whose derived group 
is mapped into the scalar matrices. In 
Section~\ref{sec:generation} we prove probabilistic results 
about the number of random elements required to generate  a matrix group. 
In Section~\ref{realisesubfield} we present an 
algorithm for writing irreducible matrix groups over a smaller field. 
In Section~\ref{subsemi:main} we present the main body of our algorithm,
followed by Section~\ref{subsemi:sec:complexity} which summarises the complexity 
results and Section~\ref{implcomplexity} which reports on our implementation 
of these algorithms. 

The first author is grateful to the RWTH in Aachen for their kind 
hospitality and to the Humboldt Foundation for support while parts of
this chapter were being written. 

\section{Definitions and main result}\label{sec:defns}

Throughout this chapter (except for Sections~\ref{sec:describe_scalars}    
and~\ref{realisesubfield}), we assume that                              
$g_1, \ldots, g_m \in \GL(d,q)$ 
and let $G = \left< g_1, \ldots,
g_m\right> \le \GL(d,q)$ be the corresponding matrix 
group. By considering each of $g_1, \ldots, g_m$ to be 
defined only up to scalar multiplication, we also 
define a group $\overline G = 
\left< \overline{g_1}, \ldots, \overline{g_m} \right> \leq
\PGL(d,q)$, which is the projective group generated by 
the given matrices. Two matrices represent the same elements of $\overline{G}$ 
if one is a scalar multiple of the other, 
so replacing any of the $g_i$ by scalar multiples will 
alter the matrix group but not the projective group.
We assume throughout that $G$ acts irreducibly on the natural
module $V = \F_q^d$ and that $\overline{G}$ acts irreducibly
on the corresponding projective space $\overline{V}$.

%The class \CC3 is defined as follows. 

\begin{Def} \label{def:semilin}
The group $G$ lies in \CC3 (the class of semilinear groups) 
if there is an $\F_q$-vector space identification between
$\F_q^d$ and $\F_{q^e}^{d/e}$ such that
for $1 \le i \le m$ 
there exist automorphisms $\alpha_i \in \Gal(\F_{q^e}/\F_q)$ 
with \[(v+\lambda
w)g_i = v  g_i + \lambda^{\alpha_i}  w  g_i\] for
all $v,w \in \F_{q^e}^{d/e}$ and all $\lambda \in \F_{q^e}$.
 The 
group $\overline{G}$ lies in \CC3 if and only if $G$ lies in \CC3: 
note that multiplying $g_1, \ldots, g_m$ by scalars from $\F_q$ 
does not affect the semilinearity of $G$. 
\end{Def}

If the $\alpha_i$ generate a proper subgroup
of $\Gal(\F_{q^e}/\F_q)$ then
multiplication by elements of the corresponding invariant 
subfield produces $\F_qG$-endomorphisms that are not $\F_q$-scalar, so $V$ 
is not absolutely irreducible.
Conversely, if $V$ is not absolutely irreducible, then
there is a divisor $e'$ of $d$ 
such that we can view $V$ as a $d/e'$-dimensional vector space
over $\F_{q^{e'}}$ on which the action of $G$
is $\F_{q^{e'}}$-linear. That is, $G$ lies in class \CC3 with
trivial automorphisms.

\begin{Def}
The group $G$ lies in \CC5 if there exists a subfield $\F_{q_0} \subsetneq
\F_q$, a $t \in \GL(d,q)$, and $\beta_1, \ldots, \beta_m
\in \F_q^\times$ such that $t^{-1}g_it = \beta_i  h_i$ with
$h_i \in \GL(d,q_0)$. The group $\overline{G}$ lies in \CC5 if and only if 
$G$ lies in \CC5: note 
that multiplying $g_1, \ldots, g_m$ 
 by scalars from $\F_q$ does not change the membership of $G$ in \CC5. 
\end{Def}
If $\beta_i = 1$ for all $i$ then $G$ can be
written over  $\F_{q_0}$. In general, $G$ lies in
\CC5 if $G$ can be written over $\F_{q_0}$ modulo
scalars.
Note
that $\overline{G}$ being in \CC5 implies that 
$\overline{G} \cong \langle \overline{h_1}, 
\ldots, \overline{h_m} \rangle$ embeds naturally in $\PGL(d,q_0)$.

We assume that the input to our algorithm is an irreducible group
$G$: see Lemma~\ref{MeatAxe} 
for the complexity of proving this. The MeatAxe run which shows $G$ 
to be irreducible also computes the endomorphism ring $E = \End_{\F_q G}(V)$.
  If $G$ is irreducible  but not
absolutely irreducible, the ring $E$ is an
extension field of $\F_q$. This provides an explicit $E$-vector 
space structure on $V$ and an $E$-linear action of the
group generators.

\vspace{3mm}
\noindent We now summarise our algorithm, see the relevant sections for
more details. 


\begin{enumerate}
\item Let $G$ be irreducible with endomorphism ring $E$ of degree $e
\geq 1$ over $\F_q$. If $e > 1$ find an explicit base change to express
the generators over $\F_{q^e}$.
\item Check whether $G$ can be written over 
a subfield $\F_{q_0}$ with $\beta_i = 1$ for $1 \leq i \leq m$, 
using the standard basis 
technique described in Section~\ref{realisesubfield}, and find the
degree $f$ of $\F_q$ over  $\F_{q_0}$.
\item If $e > f^2$ then  return a homomorphism into $\GL(d/e, q^e)$. 
Otherwise, if $f > 1$ then return a monomorphism into $\GL(d, q_0)$.
\item Compute 10 commutators of random elements of $G$. If they are all 
scalar, choose any non-scalar generator $g_i$  and check whether 
$[g_i, g_j]$ is scalar for all $j$. If so, 
jump to step 10. 
\item Compute a normal subgroup $N$ of the derived 
subgroup $G'$ of $G$ as in Section~\ref{subsec:derived}. 
\item If $N$ is absolutely irreducible, check 
whether $N$ can be written over a smaller field, as in
 Section~\ref{subsec:subfield_scalars}. If $G$ is not contained in
\CC5, return {\tt false} as $G$ is not in \CC3 or \CC5.
\item If $N$ is irreducible but not absolutely 
irreducible find a semilinear decomposition 
of $G$, as in Section~\ref{subsec:semilin}.
\item If $N$ is reducible with more than one 
homogeneous component, find an imprimitive 
decomposition of $G$ as in  
Section~\ref{subsec:imprim}.
\item If $N$ is reducible with a single 
homogeneous component with irreducible $N$-submodules of
dimension greater than $1$, find a tensor
decomposition of $G$ as in
Section~\ref{subsec:tensor}. 
\item If $[g_i, g_j]$ is scalar for some non-scalar $g_i$ and all $j$, find 
a nontrivial homomorphism from $G$ to $\F_q^\times$ as in
 Sections~\ref{sec:describe_scalars} and~\ref{subsec:scalars}. 
\end{enumerate}


We will show that all of our methods
can be applied to both matrix and projective groups, 
because the success or failure of each step is unaffected 
by multiplying generating matrices by scalars. 
The following theorem summarises the main 
algorithmic results of this article.

\begin{Theo} [Main Theorem]
Let $G \leq \GL(d, q)$ or $\overline{G} \leq \PGL(d, q)$ be an absolutely 
irreducible group that lies in \CC3, \CC5 or 
whose derived group is not absolutely irreducible. 
There exists an $O(d^4 \log (\log d) \log q + md^3)$ Las Vegas 
algorithm, where $m$ is the number of generators,  to find a nontrivial reduction of $G$ or $\overline{G}$, respectively.
\end{Theo}

The complete procedure is Las Vegas in that we can prescribe an upper
bound $\delta$ for the failure probability. The algorithm can succeed by
returning a homomorphism or reporting {\tt false}; or it can report {\tt
fail} with a prescribed probability bound
 $\delta$. If success is reported, the result is
guaranteed to be correct. If the algorithm reports {\tt false} then some
additional information may be deduced: for example, that if $G$ lies in
\CC2 then $G'$ is transitive on all possible sets of blocks.



\section{Characterisation of groups with scalar derived group }
\label{sec:describe_scalars}

In this section we do not require the global hypothesis that 
$G$ is given by matrices or projective matrices.
Instead, we assume only that  $G$ is a finite group with a faithful 
representation $\rho$. We investigate groups $G$ which
satisfy the following hypothesis.

\begin{Hyp} \label{group-info}
The group $G$ is a finite group, given by  
$G = \langle g_1, \ldots, g_m\rangle$. 
Assume that $G$ has a faithful absolutely irreducible representation
$\rho: G \longrightarrow \GL(d,k)$ for some $d > 1$ and field $k$ such that 
for all $g \in G^\prime$, the matrix $\rho(g)$ is scalar. 
\end{Hyp} 

Equivalently, the group $G$ has a faithful 
nonlinear absolutely irreducible representation
$\rho$ such that $\rho(G)$ is projectively abelian. 

Let $V$ be the module afforded by  $\rho$.
The hypothesis implies the following facts.

\begin{Lemm} \label{lem:facts}
Suppose that $G$, $\rho$ and $V$ are as above. The following all hold. 
\begin{enumerate}
\item The derived group $G'$ is contained in the centre $Z(G)$. 
\item The group $G$ is nilpotent of class 2 and hence is a direct 
product of its Sylow subgroups. 
\item The centre and the derived group of $G$ are cyclic.
\item If the characteristic of $k$ is $p > 0$ then 
the order of $G$ is not divisible by $p$.
\item Let $[g,h] = g^{-1}h^{-1}gh$ be the commutator of elements 
$g$ and $h$ in $G$. Then 
\[ 
[g,h_1h_2] = [g,h_1][g,h_2], \qquad [h_1h_2,g] = [h_1,g][h_2,g]
\] 
for all $g,h_1,h_2$ in $G$. 
\item Let $k^\times$ denote the multiplicative group of $k$. For any $g \in 
G$ there is a homomorphism 
$\psi_g:G \longrightarrow k^\times$ given by $\psi_g(h) = \rho([h,g])$. These
homomorphisms satisfy 
$\psi_{g_1g_2}(x) =
\psi_{g_1}(x)\psi_{g_2}(x)$ for all $x, g_1, g_2 \in G$.
Moreover, $\psi_g$ is a constant function if and only if $g \in Z(G)$.
\end{enumerate}
\end{Lemm}

\begin{proof}
Part $(1)$ is true because $\rho$ is faithful. 
Part $(2)$ follows from $(1)$.  It is
well known that finite nilpotent groups are the direct product of 
their Sylow subgroups. 

Part $(3)$ is true because $\rho(G)$ is 
absolutely irreducible and $\rho$ is faithful, 
so $Z(\rho(G))$ is a group of scalar matrices, 
which must be cyclic. 
Part $(4)$ then follows from the fact that a Sylow $p$-subgroup of $G$
would contribute a factor of $p$ to the centre of $\rho(G)$, and hence
to the centre of $G$.

$(5)$ is a straightforward calculation. That is,
\[
h_1 h_2 g [g, h_1 h_2] = g h_1 h_2 = h_1 g [g, h_1] h_2 = h_1 g h_2 [g, h_1] 
= h_1 h_2 g [g, h_1] [g, h_2]
\]
since $[g, h_1], [g,h_2] \in Z(G)$ by $(1)$. 
The second equation follows by inverting the first. 
Part $(6)$  is a direct
consequence of $(5)$.
\end{proof}

The lemma allows us to prove the following. 

\begin{Prop} \label{exponents}
Let $z$ be the order of $Z(G)$, and let $c$ be the 
order of $G'$. Then 
\begin{enumerate}
\item The exponent of $G/Z(G)$ is at most $c$, 
\item The exponent of $G$ is at most $cz$, and 
\item The order of $G$ is a divisor of $c^{l}z$ where $l$ is the number of 
non-central generators of $G$.
\end{enumerate}
\end{Prop}

\begin{proof} 
The group $G$ is generated by 
elements $g_1, \ldots, g_m$. Let $\psi_i:G \rightarrow
\rho(G')$ be given by $\psi_i(h) = \rho([h,g_i])$. 
Then $\psi_i$ is a homomorphism by Lemma~\ref{lem:facts}.6, 
and the  kernel $K_i$ of $\psi_i$ has index in $G$ 
at most $c$, since $\mathrm{Im}(\psi_i)$ has order dividing $c$. 
Let $\psi_i^c$ be defined by  $\psi_i^c(x):= \psi_i(x)^c$, 
then $\psi_i^c$ 
is the constant homomorphism from $G$ to $\{1\}$. From this and         
Lemma~\ref{lem:facts}.5 it follows that $g_i^c$ is in the centre of $G$ 
for all $i$. This proves $(1)$.                                         

Part $(2)$
is a direct consequence of $(1)$ since the exponent of $Z(G)$ is $z$.
Finally, $(3)$ is a simple count based on the fact that
$\cap_i K_i \leq  Z(G)$. 
\end{proof}


We know by hypothesis that $V$ is absolutely 
irreducible and by Lemma~\ref{lem:facts}.2 that 
$G$ is a direct product $G =  S_1 \times S_2 \times \cdots 
\times S_t$ of its Sylow subgroups. From this we get the next lemma.

\begin{Lemm} \label{lem:tensor}
The module $V$ is a tensor product 
\[
V \ \  \cong \ \ V_1 \otimes \cdots \otimes V_t
\]
where each $V_i$ is an absolutely irreducible module for $S_i$ on 
which $S_j$ acts trivially for $i \neq j$. 
\end{Lemm}

\begin{proof}
It is well known that irreducible 
modules of direct products are tensor 
products (see for instance \cite[51.13]{CR0}).  
Since $V$ is absolutely irreducible, so is each factor.
\end{proof}

\begin{Prop} [Eigenspace decomposition] \label{prop:eigenspace} 
Let
$g \in G$ have all eigenvalues in  $k$.
Then $V$ is a vector space direct sum 
\[
V = V_{\lambda_1} \oplus V_{\lambda_2} \oplus \cdots \oplus V_{\lambda_s}
\]
where $\lambda_1, \ldots, \lambda_s$ are the eigenvalues of $\rho(g)$ and
$V_{\lambda_i}$ is the $\lambda_i$-eigenspace. 
Moreover, if $h \in G$, then $V_{\lambda_i}\rho(h) = V_{\lambda_j}$ 
where $\lambda_j = \psi_g(h)\lambda_i$.
\end{Prop}

\begin{proof}
The space  $V$ is a direct sum of eigenspaces of $\rho(g)$
since $k \langle g \rangle$ is 
semi-simple and split by $k$. For the 
next statement, let $v \in V_{\lambda_i}$. Then as asserted
\[ 
(v \rho(h))\rho(g) = v \rho( gh [h, g])= 
\lambda_i v \rho([h, g])\rho(h) = \lambda_i \psi_g(h) v \rho(h).
\]
\end{proof}

If $g \in Z(G)$, then in Proposition~\ref{prop:eigenspace}, $V$ is only a single eigenspace for the action of $g$. 
%We know that
%the order of $Z(G)$ divides $|k^\ast|$ since the representation
%$\rho$ is faithful. 

%We show that there is an element $g$ with $g \notin 
%Z(G)$ such that $g$ satisfies the hypothesis of  
%Proposition~\ref{prop:eigenspace}.

\begin{Theo} [Non-central element of prime order] \label{thm:exists_g}
Let $G$ be an  $r$-group %such that $G' \leq Z(G)$
satisfying Hypothesis~\ref{group-info}. Then either 
$G$ is the quaternion group of order $8$ or $G$ 
has an element $g$ of order $r$ such that 
$g \not \in Z(G)$. %Hence if $G$ is not a 
%quaternion group of order $8$ then $G$ is imprimitive.
\end{Theo}

\begin{proof}
Since $Z(G)$ is cyclic, it suffices to prove 
that either $G$ is quaternion of order $8$ or 
$G$ contains more than one cyclic subgroup of order $r$. 

It is well-known (see for instance \cite[3.15]{Zassenhaus}) 
that the only $r$-groups which contain a single 
subgroup of order $r$ are the cyclic groups and 
the generalised quaternion groups. Since $\rho(G)$ 
is absolutely irreducible and $d > 1$, the group 
$G$ is not cyclic and so either $G$ has a non-central 
element of order $r$ or $G$ is generalised quaternion.

The generalised quaternion group of order $2^i$ 
for $i \geq 3$ has presentation 
$\langle a, b \ | a^{2^{i-1}} = b^4 = 
a^{2^{n-2}}b^{-2} =  b^{-1} a ba = 1 \rangle$. 
A short calculation shows that the derived group 
contains non-central elements for $i > 3$, 
and hence if $G$ is a generalised quaternion 
group then $|G| = 8$.
\end{proof}

We finish this section with our characterisation 
of the groups satisfying Hypothesis~\ref{group-info}.

\begin{Theo} [Characterisation Theorem] \label{thm:classify_scalars}
Let $G$ have a faithful absolutely irreducible 
representation $\rho$ of dimension $d > 1$ over an arbitrary field $k$ 
such that the derived group of $G$ is mapped
into the set of scalar matrices. 
Then either $d = 2$ and $\rho(G)$ 
is an extension by scalars of the quaternion 
group of order $8$ acting semilinearly, 
or $\rho(G)$ is imprimitive.
\end{Theo}

\begin{proof}
By Lemma~\ref{lem:tensor} we may consider $V$ 
as a tensor product, with a distinct Sylow 
subgroup of $G$ acting on each tensor factor. 
Let $V_i$ be one such factor. 

The first possibility is that $V_i$ is $1$-dimensional, and $S_i$ is cyclic. 
Secondly, if the action on $V_i$  is $Q_8$ then the 
dimension of $V_i$ is $2$ (see for instance \cite[\S 47]{CR0}).

Otherwise, by Theorem~\ref{thm:exists_g} 
the  action has a
 non-central element  $\rho(g)$ of order $r$.
The group $G$ also contains a central element of order $r$, which is mapped to a scalar.  
 Hence all of the eigenvalues 
of $\rho(g)$ lie in $k$. Since $\rho(g)$ is 
not central, it has more than one eigenvalue.  
It follows from Proposition~\ref{prop:eigenspace} 
that all elements of $G$ permute the eigenspaces. 
Since $G$ acts irreducibly, this action on 
the eigenspaces is transitive and hence $\rho(G)$ is imprimitive.

If at least one of the induced actions is imprimitive 
then the action of $\rho(G)$ is imprimitive. Not 
all of the Sylow subgroups can be cyclic since $d > 1$.
\end{proof}

Extraspecial $r$-groups over finite fields $\F_q$
with $d = r^n$ and $r$ dividing $(q-1)$ lie in this class, but so do other $r$-groups. For example, the subgroup $G$ of $\GL(3, 19)$ of order $3^4$ generated by
$$\left[ \begin{array}{ccc}
0 & 0 & 1 \\
17 & 0 & 0 \\
0 & 11 & 0 \end{array}\right], 
\left[ \begin{array}{ccc}
1 & 0 & 0 \\
0 & 7 & 0 \\
0 & 0 & 11 \end{array} \right]$$
satisfies $G' \leq Z(\GL(3, 19))$ but does not contain an 
extraspecial group of order $3^{1+2}$.

%{\bf Note from CMRD:} I've now redone this 
%section to only talk about the field $k$ rather 
%than mixing $k$ and $\F_q$. Jon, could 
%you check I've done this right? Thanks!

\section{Generation of matrix groups by random elements} \label{sec:generation}

In this section we analyse the generation 
of a subgroup $H = \langle g_1, \ldots, g_n \rangle$ 
of a normal subgroup $N$ of a matrix group 
$G$, and in particular provide bounds on $n$ 
for $H$ to have the same submodule structure and endomorphism ring as $N$, with probability at least $1 - \delta$.
Perhaps surprisingly, we do this via results for permutation groups. 

%We shall need the following lemma.

\begin{Lemm} %[{See \cite[Corollary 2.2]{babai95}}]
    \label{babailemma}
    Let $X_1, X_2, \ldots$ be a sequence of\/ $0$-$1$ valued random
    variables such that $\Pro(X_i = 1) \ge p$ for any values of the
    previous $X_j$ (but $X_i$ may depend on these $X_j$). 
    
    Then, for
    all integers $t$ and all\/ $0 < \epsilon < 1$,
    \[ \Pro\left( \sum_{i=1}^t X_i \le (1-\epsilon)pt\right) \le
       e^{-\epsilon^2pt/2}. \]
   \end{Lemm}

\begin{proof} See \cite[Corollary 2.2]{babai95} or \cite[Lemma
    2.3.3]{Ser}.
\end{proof}

The following proposition is based on \cite[2.3.7]{Ser}, where it is    
proved for the case $G$ transitive. Note that the assumptions are       
slightly more general than for $G$ given as a group of permutations, as 
(for example) this can be applied to any finite group equipped with a   
permutation action.                                                     

\begin{Prop} [Correct orbits of subgroup] \label{prop:orbits}
Suppose that a finite group $G = \langle S \rangle$ 
acts on a finite set $\Omega$, with $\alpha$ 
orbits. Let $1 > \delta > 0$ be arbitrary. 
Then with probability at least $1 - \delta$, a sequence
of $O(\log \delta^{-1} + \log |\Omega|)$ 
uniformly distributed random elements of $G$ 
generates a subgroup of $G$ that has the same 
orbits on $\Omega$ as $G$.
\end{Prop}

\begin{proof}
Let $t = c \log |\Omega|$ where 
$c \geq \max\{24 \log \delta^{-1}/\log |\Omega|, 45\}$.
Let $g_1, \ldots, g_t$ be uniformly distributed random elements 
of $G$. For $1 \leq i \leq t$ let 
$G_i = \langle g_1, \ldots, g_i \rangle$, 
let $N_i$ be the number of $G_i$-orbits on $\Omega$, 
and let $M_i$ be the 
number of $G_i$-orbits that coincide with $G$-orbits. 
Let $k_i = N_i - M_i$.
Note that $N_i \geq \alpha$ for $1 \leq i \leq t$, 
that $M_i \leq \alpha$ for $1 \leq i \leq t$, and
that $N_i = \alpha$ if and only if $M_i = \alpha$. 
Hence $k_i$ is either $0$ or at least $2$.

We claim that if $N_{i-1} > \alpha$, then 
\[
\Pro\left(k_i \leq \frac{7}{8}k_{i-1}\right) \geq \frac{1}{3}.
\]
To see this,  let $k = k_{i-1}$ and 
let $\Delta_1, \ldots, \Delta_k$ be 
the $G_{i-1}$ orbits on $\Omega$ that are 
\emph{not} $G$-orbits. For  $1 \leq j \leq k$, 
let $X_j = 1$ if $\Delta_j^{g_i} \neq \Delta_j$ 
and let $X_j = 0$ otherwise. 
Now,  $\Delta_j$ lies 
in an orbit of length at least two in the action 
of $G$ on subsets of $\Omega$. Therefore at most half of the 
elements of $G$ fix $\Delta_j$, and 
so $E(X_j) \geq 1/2$ for $1 \leq j \leq k$. Let $X = \sum_{j = 1}^{k} X_j$, then
$E(X)  \geq k/2$. 

Let $p$ be the probability that $X \leq k/4$. Then with probability $p$
the 
variable $X$ takes value at most $k/4$ whilst $X$
takes value greater than $k/4$ and less than or equal to $k$ with 
probability $1-p$. Therefore
\[
\frac{pk}{4} + (1-p) k \geq E(X) \geq k/2,
\]
so $p \leq 2/3$.
Hence, with probability at least $1/3$, at 
least $k/4$ of the $G_{i-1}$-orbits that are not $G$-orbits 
are proper subsets of orbits of $G_{i}$. Thus
 the number of orbits of $G_{i}$ which are 
not orbits of $G$ is at most $7k/8$, and the claim follows. 

Define $Y_1, \ldots, Y_t$ by 
\[\begin{array}{ll}
Y_i = 0 & \mbox{ if $k_i > 0$ and }k_i > \frac{7}{8}k_{i-1} \\
Y_i = 1 & \mbox{ if $k_i = 0$ or }k_i  \leq \frac{7}{8}k_{i-1}. \end{array}\]
By the previous claim, $\Pro(Y_i = 1) \geq 1/3$. 

Now, $N_0 = |\Omega|$, so $k_0 \leq |\Omega|$. 
Clearly,  $k_t \leq k_0 \left(\frac{7}{8}\right)^{\sum_{i = 1}^t Y_i}$. 
The group $G_t$ has the same orbits as $G$ if 
and only if $k_t \leq 1$, which will follow if 
$|\Omega| \left( \frac{7}{8} \right)^{\sum_{i = 1}^t Y_i} \leq 1$. 
In turn this simplifies to 
$|\Omega| \leq \left( \frac{8}{7} \right)^{\sum Y_i}$, which gives
\[
\sum_{i = 1}^t Y_i \geq \frac{ \log |\Omega|}{ \log \frac{8}{7}}.
\]
Then by Lemma~\ref{babailemma}, with $p = 1/3$, $t = c \log |\Omega|$ and $\epsilon = 1 - 3/(c \log (8/7))$ we get % \cite[Corollary 2.2]{babai95},   
$$\Pro(\sum_{i = 1}^t Y_i \leq \frac{\log 
|\Omega|}{\log \frac{8}{7}}) \leq e^{-\frac{1}{6}(1- \frac{3}{c \log (8/7)})^2 c \log |\Omega|} \leq e^{- 
c \log |\Omega|/24}$$ for $c \geq 45$. 
In turn this is less than or equal to $\delta$.  
%Thus we choose $t$ large enough that $e^{-t/24} 
%\leq \delta$, and the result follows. 
\end{proof}
 
We now apply the previous theorem to matrix groups, by considering their action on vectors. 

\begin{Theo} [Correct action of subgroup]
Let $G \leq \GL(d, q)$, and let $1 > \delta > 0$ 
be arbitrary. With probability at least 
$1 - \delta$,
a sequence of $O(\log \delta^{-1} + d \log q)$ 
uniformly distributed random elements of $G$ 
generate a subgroup $H$ of $G$ with the same 
submodule lattice as $G$ on $V = \F_q^d$.
Furthermore, if $G$ is irreducible then  $\End_{\F_qG}(V) = \End_{\F_qH}(V)$
with probability $1 - \delta$.
\end{Theo}

\begin{proof}
First consider $G$ as a permutation group 
on $|\Omega| = q^d$ points. By 
Proposition~\ref{prop:orbits}, any group 
$H$ generated by $O(\log \delta^{-1} + d \log q)$ 
uniformly distributed random elements of $G$ 
has the same orbits as $G$ with probability $1-\delta$. 

A submodule for $G$ is a union of orbits of 
$G$ in its action on vectors that is closed 
under addition and scalar multiplication, 
and a submodule is irreducible if and only 
if it contains no smaller unions of orbits 
that are closed under addition and scalar 
multiplication, except for $\{0\}$, so the first claim follows. 

For the second claim, let $G$ be irreducible and $\End_{\F_qG} = \F_{q^f}$.
Let $H_0$ be generated 
by $O(\log \delta^{-1} + d \log q)$ random 
elements of $G$, so that $H_0$ is irreducible 
with probability $1-\delta/2$. Let $t = c \log d$ 
for $c \geq \max\{22,  16\log (2\delta^{-1}) /(3 \log d)\}$
and let $h_1, \ldots, h_t$ be further random 
elements of $G$. For $1 \leq i \leq t$ let 
$H_i = \langle H_0, h_1, \ldots, h_i \rangle$. 
Notice that $H = H_t$ is generated by 
$O(\log \delta^{-1} + d \log q)$ elements of $G$. 


Since $H_0$ is irreducible,  $\End_{\F_qH_0}(V) = \F_{q^s}$ for some 
$s$ that is a multiple of $f$ and divides $d$. 
For $1 \leq i \leq t$ let $N_i$ be the
 degree of $\End_{\F_q H_i}(V)$ over $\F_{q^f}$. 

We claim first that if $N_{i-1} > 1$ then 
$\Pro(N_i \leq N_{i-1}/2) \geq 1/2$.  To 
see this let $x$ generate
 $\End_{\F_q H_{i-1}}(V)$. Then since 
$x$ is not centralised by  $G$, at 
most half of the elements of $G$ commute 
with $x$. If $[h_i, x] \neq 1$ 
then $N_i$ is a divisor of $N_{i-1}$ so the claim follows. 

Now for $1 \leq i \leq t$ define $Y_i = 0$ 
if $N_{i-1} > 1$ and $N_i = N_{i-1}$ and 
$Y_i = 1$ otherwise. If 
$\sum_{i = 1}^t Y_i \geq \log_2 d \geq \log_2 (d/f)$ 
then $\End_{\F_q H}(V) = \End_{\F_q G}(V)$. Now, 
$\Pro(Y_i = 1) \geq 1/2$ by the claim, so 
by Lemma~\ref{babailemma} with $p = 1/2$, $t = c \log d$ and $\epsilon = 1 - 2/(c \log 2)$ % \cite[Corollary 2.2]{babai95} 
 $$\Pro(\sum_{i = 1}^t Y_i \leq \log d) \leq e^{-\frac{1}{4}(1- \frac{2}{c \log 2})^2 c \log d} \leq e^{-\frac{3}{16}c \log d}$$ since $c \geq 22$. 
This is at most $\delta/2$ so the result follows. 
\end{proof}


\begin{Cor} [Correct action of normal subgroup] \label{cor:right_action}
For $G \leq \GL(d, q)$, let $N \unlhd G$ and let $1 > \delta > 0$ 
be arbitrary. With probability $1-\delta$ 
any group $H$ generated by $O(\log \delta^{-1} + d \log q)$ 
uniformly distributed elements of $N$ 
has the same submodule lattice as $N$, the same homogeneous components as $N$
and, if $N$ is irreducible, then $\End_{\F_q H}(V) = \End_{\F_q N}(V)$.
\end{Cor}

We will apply this result to commutators from $G$. A \emph{normal} generating set for $N \unlhd G$ is a set of elements of $N$
which generate a group whose normal closure is $N$. 
 Any set of elements of $G'$ form a normal 
generating set for a normal subgroup of $G$ which is contained in $G'$. 


\section{Writing $\mathbf{G}$ over a smaller field}
\label{realisesubfield}

In this section unless indicated otherwise we let $K$ be a finite field, let
 $G = \langle g_1, \ldots, g_m \rangle \leq \GL(d, K)$, 
and let $V = K^{1 \times d}$ be the natural right
$KG$-module. We assume that $V$ is irreducible but not necessarily
absolutely irreducible. We want to determine whether there exists
 a $t \in \GL(d,K)$ such that for
$1 \le i \le m$ the matrices
$t^{-1}g_it$  have entries over some proper subfield of $K$. 
If such a $t$ exists, we want to construct it for 
the smallest possible subfield $F$ of $K$. 
We first
analyse when such a $t$ exists.

%Let $q = q_0^f$ and denote $\F_q$ by $K$ and $\F_{q_0}$ by $F$.


The $K$-algebra $K \otimes_F FG$ is isomorphic  as a $K$-algebra to
the group algebra $KG$ by the $F$-linear map given by $x \otimes g \mapsto xg$ 
for $x \in K$ and $g \in G$. This isomorphism makes the tensor
product $K \otimes_F \tilde V$ into
a $KG$-module,  for any $FG$-module $\tilde V$. 
%Using these concepts we get:

\begin{Lemm}
\label{extsca}
There exists a $t \in \GL(d,K)$ such that $t^{-1}Gt \in \GL(d,F)$
if and only if there exists an irreducible $F G$-module 
$\tilde{V}$
such that $V \cong K \otimes_{F} \tilde{V}$ as $KG$-modules.
\end{Lemm}

\begin{proof}
If there exists a $\tilde{V}$ such that $V \cong K \otimes_{F} \tilde{V}$ as
$KG$-modules then there is an irreducible representation of
$G$ over $F$ which is $K$-equivalent to the natural representation of $G$ on $V$,
 hence there is a $t$ as required.
On the other hand, such a $t$ 
gives rise to a representation of $G$ over $F$ and thus to an $FG$-module
$\tilde{V}$. The extension of scalars $K \otimes_F \tilde V$ of 
$\tilde{V}$ to $K$ is isomorphic
to $V$. If $\tilde{V}$ had a non-trivial $FG$-invariant subspace then
$V$ would have a non-trivial $KG$-invariant subspace, thus $\tilde{V}$
is irreducible.
\end{proof}

For a subfield $F$ of $K$ we denote by $F[G]$ the set of $F$-linear
combinations of the elements of $G$ as an $F$-subalgebra of $K^{d \times d}$.
This is also called the \emph{$F$-enveloping algebra of $G$}. For any finite field $K$ we denote the prime field by $K_0$. 


\begin{Prop}[Prime field enveloping algebra I] 
\label{pfenvalgI}
Let $G \leq \GL(d,F)$ for $F$ finite, 
and let $V := F^{1 \times d}$ be irreducible.
Let $E := \End_{F[G]}(V)$ with  $e = [E:F]$ and $d' = d/e$. Identify $V$ with 
with $E^{1 \times d'}$ %and $G \le GL(d',E) \le GL(d,F)$, 
so that $F[G] = E^{d' \times d'}$. Set $L := F_0[G] \cap (E \cdot
\mathbf{1})$.
 Then $F_0[G] \cong L^{d' \times d'}$ as an
$F_0$-algebra.
\end{Prop}
\begin{proof}
Clearly $e = 1$ and  $E=F$ if and only $V$ is absolutely irreducible.

Choosing an $F$-basis $(c_1,\ldots,c_e)$ of $E$ we can express each element 
of $E$ as an $(e\times e)$-matrix over $F$.
The set $V$ is an $E$-vector space and if $(b_1, \ldots, b_{d'})$ is an
$E$-basis of $V$, then $(c_ib_j)_{i,j}$ is
an $F$-basis of $V$. Since the action of $G$ on $V$ is
$E$-linear, this choice of basis fixes an embedding
$E^{d' \times d'} \subseteq F^{d \times d}$. Therefore we may assume that
$G \le \GL(d',E) \le \GL(d,F)$.
By the Density Theorem (see \cite[(3.27)]{CRI}), since $V$ is an
irreducible $F[G]$-module, $F[G] = E^{d' \times d'}$.

Now consider $B := F_0[G]$, which is an $F_0$-subalgebra of 
$F[G]$ such that $FB = E^{d'\times d'} = F[G]$. We first show
that $B$ is a simple algebra. If $J$ is a nilpotent 
two-sided ideal of $B$, then $FJ$ is a nilpotent two-sided ideal in 
$FB = E^{d' \times d'}$, contradicting its simplicity. So $B$ has no 
nilpotent two-sided 
ideals and hence is semi-simple. It follows that $B$ is a direct sum of
simple algebras. The identity elements in these simple summands form an
orthogonal set of central idempotents in $B$. A central
idempotent in $B$ is also central in $FB = E^{d' \times d'}$, and hence is 
the identity.
Consequently, $B$ is a simple algebra.

By the usual Wedderburn Theorems there exists an isomorphism
 $\psi: L^{s \times s} \rightarrow B$
for some $s$,  and some extension $L$ of $F_0$. 
The field $L$ need not contain $F$. However,  the elements of $B$ corresponding
to scalar matrices in $L^{s \times s}$ are central in $B$ and hence also
central in $FB = E^{d' \times d'}$. Therefore we can identify
$L$ with the centre of $B$ and thus with some subfield of $E$.
That is, $L = F_0[G] \cap (E \cdot \mathbf{1}) 
\subseteq E^{d' \times d'}$.

This produces a homomorphism of rings
\[ \varphi: E^{s \times s} \cong E \otimes_L L^{s \times s} \cong
   E \otimes_L F_0[G] \to E^{d' \times d'} = F[G] \]
given by $\varphi(x \otimes b) = xb$. Since $\varphi$ is surjective and
$E^{s \times s}$ is simple, $\varphi$ is an isomorphism and thus
$s = d'$.
\end{proof}

\begin{Prop}[Prime field enveloping algebra II]
\label{pfenvalgII}
Let $G$ be as in Proposition~\ref{pfenvalgI}, and suppose
additionally  that 
there is no subfield $D$ of $F$ such that there exists $t \in \GL(d,F)$
with $G^t \leq \GL(d,D)$. Then $F_0[G] = F[G]$.
\end{Prop}
\begin{proof}
Let $\psi : L^{d' \times d'} \to F_0[G]$ be the isomorphism given
by Proposition~\ref{pfenvalgI}. 
Let $e_{i,j} \in F_0[G] $ for $1 \le i,j \le d'$ 
be the image under $\psi$ of a set of matrix units in  $L^{d' \times d'}$. 
Then $e_{i,j} e_{k,l} = \delta_{j,k} e_{i,l}$ %for all $i, j, k, l$
%in $\{1,\ldots, d'\}$ 
and the $e_{i,j}$ are an $L$-basis of $F_0[G]$.

We claim that the $e_{i,j}$ are an $E$-basis of $E^{d \times d}$.
 To see linear independence, let
\[ \sum_{i,j=1}^{d'} \lambda_{i,j} e_{i,j} = 0, \]
then multiplying on the left by $e_{k,k}$ and
on the right by $e_{l,l}$ shows that $\lambda_{k,l} e_{k,l} = 0$ and thus
$\lambda_{k,l} = 0$ for all $k,l$.
On the other hand, since $F_0[G]$ is the $F_0$-span of the elements
of $G$, the $e_{i,j}$ span $E[G] = E^{d' \times d'}$ as an
$E$-vector space. Thus they are an $E$-basis of $E^{d' \times d'}$.

Since $\mathbf{1_{d' \times d'}} = \sum_{i=1}^{d'} e_{i,i}$ gives rise to a
decomposition of $E^{1 \times d'}$ as an $E$-vector space in which the
direct summands are the row spaces of the $e_{i,i}$, it follows that
these row-spaces are all one-dimensional.

Let $b'_1 \in E^{1 \times d'}$ such that $\left<b'_1\right>_E$ is the 
row space of $e_{1,1}$ and set $b'_i := b'_1 e_{1,i}$. Then 
$b'_i e_{j,k} = \delta_{i,j} b'_k$. If $t^{-1} \in E^{d' \times d'}$ has
rows $b'_1, b'_2, \ldots, b'_{d'}$, then $t^{-1} e_{i,j} t$ has
 $1_E$ in position $(i,j)$ and zeroes elsewhere.
Thus $F_0[G^t] = L^{d' \times d'}$.

If $V$ is absolutely irreducible then $E=F$ and so $L \le F$.
By assumption $F$ is the
smallest possible field over which $G$ can be written, so
$L = F$ as required.

Now consider the case $F < E$,  and assume by way of contradiction that
$l := [E:L] > 1$. Since $F[G^t] = F[G] = E^{d'
\times d'}$ it follows that $F \cdot L^{d' \times d'} = E^{d' \times d'}$.
Therefore, $E$ is the smallest field containing both $F$ and $L$
implying that  $l$ and $e = [E:F]$ are coprime.
Let $D := F \cap L$. Then $D$ is a field with $[F:D] = l$ and $[L:D] = e$.
We claim that $G$ can be written over $D$, contradicting our
assumption that $F$ is the minimal field over which $G$ can be written. 
This contradiction will give $L = E$ and hence prove the proposition. 

Let $(c'_1, \ldots, c'_e)$ be a $D$-basis of $L$. Then it is also
an $F$-basis of $E$, since every element of $E$ is an $F$-linear
combination of elements of $L$. Now  change basis in $V=F^{1 \times d}$
from the $F$-basis $(c_i b'_j)_{i,j}$ to $(c'_i b'_j)_{i,j}$ 
using a base change $s \in \GL(d,F)$, to get $G^{ts} \leq D^{d
\times d}$.
\end{proof}

\begin{Theo}[Characterisation of smallest possible field]
\label{charsmallest}
Let $G \leq \GL(d, K)$ act
irreducibly on its natural module. Then there is a
unique smallest subfield $F$ of $K$ such that there exists $t \in
\GL(d,K)$ with $G^t \le \GL(d,F)$. This $F$ is uniquely determined by 
$K_0[G] \cap (K \cdot \mathbf{1}_{d \times d}) 
= F \cdot \mathbf{1}_{d \times d}$. Furthermore,   $K_0[G] \cong E^{d/e
\times d/e}$ where $E = \End_{F[G^t]}(F^{1 \times d})$ is an 
extension field of $F$ of degree $e$.
\end{Theo}
\begin{proof}
Since $K$ is finite there is a smallest subfield $F$ of
$K$ such that there exists a $t \in \GL(d,K)$ with $G^t \le \GL(d,F)$.
By Lemma~\ref{extsca} the natural $F[G^t]$-module $V := F^{1 \times d}$ is
irreducible.
Then Proposition~\ref{pfenvalgII},  applied to $G^t$,
shows that $F_0[G^t] = F[G^t] = E^{d/e \times d/e}$.
Since the
$F$-scalar matrices are central in $F^{d \times d}$, the theorem
follows immediately.
\end{proof}


{} From now on we assume that the equivalent statements in Lemma~\ref{extsca}
hold. We now develop some theory which leads to 
an algorithm that finds a $t$ and the smallest possible subfield $F$, or proves
that none exists.

Let $B = \{b_1, \ldots, b_f\}$ be an $F$-basis for $K$, such that $b_1 = 1$.
We start by noting that if the natural $KG$-module
$V \cong K \otimes_F \tilde{V}$ as $KG$-modules, then
$V \cong \bigoplus_{i = 1}^f b_i \otimes_F \tilde{V}$ as $FG$-modules.
We therefore identify $V$ 
with $K \otimes_F \tilde{V}$ and $\tilde{V}$
with $1 \otimes_F \tilde{V}$ respectively via this second
isomorphism and thus write $b_i v$ instead of $b_i \otimes_F v$ and $v$ for
$1 \otimes_F v \in \tilde{V}$. 

\begin{Lemm} 
\label{degsplittingfield}
The $F$-dimension of\/ $\End_{FG}(\tilde{V})$ is  $e$.
\end{Lemm}

\begin{proof}
This result is a special case of the fact that
\[\End_K(K \otimes_F \tilde{V}) \cong K \otimes_F
\End_F(\tilde{V}),\] see for
example \cite[(2.38)]{CRI}.
To assist the reader and set up some notation, we
 first prove that $\End_{K}(V) \cong K\otimes_{F} \End_{F}(\tilde{V})$. 
If $(m_1,\ldots,m_d)$ is an $F$-basis of $\tilde V$, then 
$(b_i m_j)_{1 \le i \le f, 1 \le j \le d}$ is an $F$-basis of $V$ and
$(m_j)_{1 \le j \le d} = (1 \otimes_F m_j)_{1 \le j \le d}$ is a
$K$-basis of $V$. Hence 
every $\varphi \in \End_K(V)$ can be written in a unique way as
\[ \varphi = \sum_{i=1}^f b_i \varphi_i \]
with $\varphi_i \in \End_F(\tilde V)$. Therefore
$\End_K(V) \cong K \otimes_F \End_F(\tilde V)$.

Next we show that $\End_{KG}(V) \cong K \otimes_F \End_{FG}(\tilde V)$. 
If $\psi \in \End_K(V)$ then $\psi \in \End_{KG}(V)$ if and only if
$\psi(vg) - \psi(v)g = 0$ for all $g \in G$ and  $v \in V$. By $K$-linearity, 
it suffices to check this for $v \in (m_j)_{1 \le j \le d}$. Writing
$\psi = \sum_{i=1}^f b_i \psi_i$ with $\psi_i \in \End_F(\tilde V)$ shows that
\[\psi \in \End_{KG}(V) \Leftrightarrow
 \sum_{i=1}^f b_i (\psi_i(m_jg)-\psi_i(m_j)g) = 0\]
for all $1 \le j \le d$ and all $g \in G$ and thus by the uniqueness 
above to $\psi_i(m_j g)-\psi_i(m_j)g=0$ for all $i$, $j$, and $g$. This
proves that $\End_{KG}(V) \cong K \otimes_{F} \End_{FG}(\tilde{V})$.

Now the $F$-dimension of $K\otimes_{F} \End_{FG}(\tilde{V}) = f
\dim_F(\End_{FG}(\tilde{V}))$ and the $F$-dimension of $\End_{KG}(V)$ is $e f$,
so $e = \dim_F(\End_{FG}(\tilde{V}))$, as required.
\end{proof}

\begin{Lemm}
\label{lindep}
Let $w \in \tilde{V}$ and $x_1, \ldots, x_k \in FG$. A set of vectors of the 
form $\{\sum_{i = 1}^f b_i w x_j \ | \ 1 \leq j \leq k\}$ are linearly
independent over $K$ if and only if they are linearly independent over
$F$.
\end{Lemm}

\begin{proof}
For $1 \leq j \leq k$ let $c_j = \sum_{i = 1}^f b_i w x_j$. Let $a =
\sum_{i = 1}^f b_i$ and for $1 \leq j \leq k$ let $d_j = a^{-1}c_j = w x_j
\in \tilde{V}$. The $d_j$ are linearly independent over $K$ if and only
if the $c_j$ are linearly independent over $K$. Since each $c_j$ has
been multiplied by the \emph{same} element $a \in K$, the same statement
is true over $F$.

The set $\{d_j \ : \ 1 \leq j \leq k\}$ is linearly independent over $F$
if and only if $\{1 \otimes d_j \ : \ 1 \leq j \leq k\}$ is linearly
independent over $K$. The
result follows from the identification of the two sets. 
\end{proof}


\bigskip
We are now in a position to attack the original problem of this
section. The method for finding the matrix $t$ as in Lemma~\ref{extsca}
is an instance of the 
``standard basis method'' which is usually used for finding
homomorphisms from irreducible modules into arbitrary modules. In fact, we
use the $FG$-module isomorphism $V \cong 
\bigoplus_{i=1}^f b_i\tilde{V}$ 
and then find an $FG$-homomorphism
from $\tilde{V}$ to $V$. We will show that the $K$-span of the image is $V$
and the representing matrices with respect to that basis
are the same as those on $\tilde{V}$ and thus are over $F$.

To describe this, we first define the term ``standard basis'', which
is most easily done by means of an algorithm. Note that this concept was
described by Parker in \cite[Section~6]{MeatAxeRP}.

\begin{Def}[Standard basis] \label{spinup}
\index{standard basis}%
Let $G = \langle g_1, \ldots, g_m \rangle$ be a group, 
let $V$ be a right $FG$-module,
and let $0 \neq v \in V$. The \emph{standard basis} starting at $v$ with
respect to $(g_1,\ldots,g_m)$  is a
list of vectors.

Starting with $(v)$, successively apply each of $g_1, \ldots, g_m$
in this order to each vector in the list, 
finding all $m$ images of one vector before progressing to the next.  
 Whenever the result
is not contained in the $F$-linear span of  the previous vectors, 
add it to the end of the list. This produces
 a basis $SB(V,v,(g_1, \ldots, g_m))$ for a 
non-trivial $G$-invariant subspace, which 
is $V$ itself  if $V$ is irreducible.

%For the sake of completeness we present this algorithm in 
%Figure~\ref{spinup} as pseudocode.
\end{Def}

%\begin{figure}[ht]
%\rule{\textwidth}{1pt}
%\begin{tabbing}
%\textbf{Input:} $G = \left< g_1, \ldots, g_m 
%\right>$, $V$ a right $FG$-module,
%$0 \neq v \in V$. \\
%$b := [v]$, $i := 1$ \\
%whi\=le $i \le \mbox{Length}(b)$ do \\
%\> for \=$j$ from $1$ to $m$ do \\
%\>\>  $x := b[i] \cdot g_j$ \\
%\>\>  if $x \notin F\mbox{-span}(b)$ then append $x$ to $b$ \\
%\> $i := i + 1$ \\
%\textbf{Output:} $SB(V,v,(g_1, \ldots, g_m)) := b$
%\end{tabbing}
%\caption{Spinup algorithm to compute/define standard basis}
%\rule{\textwidth}{1pt}
%\label{spinup}
%\end{figure}

We next present a theorem which is useful for isomorphism testing with
an irreducible module. Although the ideas are described in
\cite[Section 6]{MeatAxeRP}, we include the exact formulation and a proof,
since these arguments are used in an intricate way later in the
determination of the matrix $t$ from Lemma~\ref{extsca}.

\begin{Theo}[Isomorphism test]
\label{isotest}
\index{isomorphism test}%
Let $G=\left< g_1, \ldots, g_m\right>$ be a group, $V$ a finite-dimensional,
irreducible $FG$-module and 
$E := \End_{FG}(V)$ its endomorphism ring. Furthermore let 
$W$ be a finite-dimensional $FG$-module, 
and
$c \in FG$ an element such that $\dim_F(\ker_V(c)) = \dim_F(E)$. 
Let $N := \ker_W(c) = \{w \in W \ : \ wc = 0\}$. There are two possibilities:

\begin{itemize}
\item If $N = \{0\}$, then $V \not\cong W$ as $FG$-modules. 

\item If $N \neq \{0\}$,  let $0 \neq w \in N$. 
Then $SB(W,w,(g_1, \ldots, g_m))$ spans $W$ if and only if 
$V \cong W$. If $V \cong W$ as $FG$-modules, then the $F$-linear
map $\varphi : V \to W$ mapping $SB(V,v,(g_1, \ldots, g_m))$ to
$SB(W,w,(g_1, \ldots, g_m))$ is an $FG$-isomorphism.

Hence, if $V \cong W$, then for any non-zero $w_1, w_2 \in N$ there is an 
$FG$-automorphism of $W$ mapping $w_1$ to $w_2$.
\end{itemize}
\end{Theo}

\noindent\textsc{Remark:} 
This provides an efficient algorithm to test
whether $V \cong W$ as $FG$-modules and if so to construct an explicit
isomorphism, provided $V$ is known to be irreducible and $\dim_F(E)$ is known.
 The algorithm finds $c$, 
computes $SB(V,v,(g_1, \ldots, g_m))$, and then computes
$N$, looking for $0 \neq w \in N$. 
If an appropriate $c$ is found and $N \neq 0$ then the algorithm computes
$SB(W,w,(g_1, \ldots, g_m))$. This computation verifies
whether $\varphi$ is an $FG$-isomorphism. Thus the algorithm
either computes an isomorphism $\varphi$ or proves that none exists.

\smallskip
\begin{proof}
$\ker_V(c)$ is $E$-invariant and thus
a vector space over $E$. By assumption its $E$-dimension is 1. 
Every $FG$-module isomorphism between $V$ and $W$ maps
$\ker_V(c)$ into $N$. If $W \cong V$, then
$\dim_F(N) = \dim_F(\ker_V(c))$ and so $N$ is a $1$-dimensional
vector space over $E' := \End_{FG}(W)$. Therefore, for all
$(w,w') \in N \times N$ with $w \neq 0 \neq w'$ there is an
automorphism $e' \in E'$ with $e'(w)=w'$. Thus, if we pick
any $0 \neq v \in \ker_V(c)$ and any $0 \neq w \in N$, then there
is an isomorphism $\varphi : V \to W$ that maps $v$ to $w$. This
isomorphism necessarily maps $SB(V,v,(g_1, \ldots, g_m))$ to
$SB(W,w,(g_1,\ldots,g_m))$ proving our claims.
\end{proof}

%\noindent We will need the following lemma:

%\begin{Lemm}
%\label{kerdec}
%Let $c \in FG$. Then
%\[ \ker_V(c) = \bigoplus_{i=1}^f b_i \ker_{\tilde{V}}(c). \]
%\end{Lemm}

%\begin{proof}
%The direct sum is a decomposition as $FG$-modules. Thus taking kernels
%of $c \in FG$ respects this decomposition.
%\end{proof}


\begin{Theo}[Writing $G$ over a smaller field]
\label{writing}
%Let $G = \langle g_1, \ldots, g_m \rangle \leq \GL(d, K)$, 
%and let $V$ be the natural\/ $KG$-module. 
%Assume that $V$ is irreducible,
%$E := \End_{KG}(V)$ and $e := \dim_K(E)$ is the degree of the
%splitting field. 
Let the global assumptions for this section on $G$ apply. 
There exists a $c \in K_0 G$ such that $\dim_K \ker_V(c) = e$.
Let $w \in \ker_V(c)$ with $w \neq 0$,  let $B := SB(V,w,(g_1, \ldots, g_m))$
and let $t^{-1} \in \GL(d,K)$ have the vectors in $B$ as rows.

If $F$ is a subfield of $K$ for which there is an $r \in \GL(d,q)$
such that $r^{-1}Gr \leq \GL(d,F)$, then $t^{-1}Gt \leq \GL(d,F)$ as well. 
That is, $t^{-1}Gt$ writes $G$ over the smallest possible field.

Let $1 > \delta > 0$ 
be arbitrary. There is an algorithm to find $c$ and construct $t$ in 
Las Vegas $O(md^3 \log \delta^{-1})$ field operations
with failure probability at most
$\delta$.
If the algorithm is allowed to run indefinitely, then it
finishes with probability $1$ and the expected number of attempts to find $c$
 is bounded above by a constant which does not depend on $d$ or $|K|$.
Each attempt needs $O(md^3)$ field operations.
\end{Theo}

\begin{proof}
We apply the standard basic technique to 
$V \cong \bigoplus_{i=1}^f b_i\tilde{V}$, 
where $\{b_1, \ldots, b_f\}$ is an $F$-basis for $K$, as before.
Note that we assume that  the isomorphism exists and do not yet
have it explicitly! 
We attempt to compute an $FG$-homomorphism
$\varphi : \tilde{V} \to \bigoplus_{i=1}^f b_i\tilde{V}$, 
and will either succeed or show that $f = 1$. 

1. First we look for $c \in FG$ such that
$\dim_F(\ker_{\tilde V}(c)) = \dim_F(\End_{FG}(\tilde V))$. 
We do not know $F$ nor have $\tilde V$, 
but by Lemma~\ref{degsplittingfield}, $\dim_K(E) = e = 
\dim_F(\End_{FG}(\tilde V))$ and 
\[f \cdot \dim_F(\ker_{\tilde V}(c)) = \dim_F(\ker_V(c)) =
f \cdot \dim_K(\ker_V(c)).\] Given a possible $c$, 
we can compute $\dim_K(\ker_V(c))$, and stop if this is equal to $e$.

To find $c$ we produce random $K_0$-linear 
(and hence $F$-linear) combinations $c$ of elements of $G$
and stop if $\dim_K(\ker_V(c)) = e$. 
 The results in \cite{MeatAxeHoltRees} and \cite{IL}
    show that there is an upper bound $b$ not depending on $|K|$, $|F|$ and 
    $d$ for
    the probability that a random element 
    $c \in K_0[G] = F_0[G] \cong F[G^t] \cong E^{d/e \times d/e}$ has
    $\dim_{K}(\ker_V(c)) \neq e$. Thus $\log_b \delta^{-1}$ 
    tries will succeed with probability at least $1 - \delta$.  


2. Assume that we have found such a $c \in FG$, given in its action on $V$.
We compute a non-zero $w \in \ker_V(c)$.
This has the form $w = \sum_{i=1}^f b_i w_i$
for some $w_i \in \tilde{V}$, and since $\ker_V(c) = \bigoplus_{i=1}^f b_i \ker_{\tilde{V}}(c)$
 all of the $w_i$
lie in $\ker_{\tilde{V}}(c)$.

By Theorem~\ref{isotest} and the fact that
all summands $b_i \tilde{V}$ are isomorphic to $\tilde{V}$
as $FG$-modules, we conclude that for 
all $w$ and all nonzero $v \in \ker_{\tilde{V}}(c)$, 
there is a unique $FG$-monomorphism from $\tilde{V}$ into $V$,  mapping
$SB(\tilde{V},v,(g_1,\ldots,g_m))$ to $SB(V,w,(g_1,\ldots,g_m))$.
Note that the latter is a basis for the image of this $FG$-homomorphism, which is an $F$-subspace. 
%which has $F$-dimension $d$. 

We do yet know $F$, so we cannot yet
 test for $F$-linear independence. 
%Furthermore,  Lemma~\ref{lindep} is not yet applicable, 
%as the vectors do not all use the same 
%basis vector. 
We now make some observations which allow us  to apply Lemma~\ref{lindep}. 
By the last statement of
Theorem~\ref{isotest}, for $1 \leq i \leq f$
there is an automorphism $\alpha_i \in \End_{FG}(\tilde{V})$
mapping $w_i$ to $w_1$. Thus, there is an automorphism $\alpha$
of the $FG$-module $V \cong \oplus_{i = 1}^f b_i \tilde{V}$ 
with $\alpha(b_i w_i) = b_i w_1$ for all $i$ and thus
$\alpha(w) = \sum_{i=1}^f b_i w_1$.

By Lemma~\ref{lindep}, with $x_1, x_2, \ldots, x_k \in FG$,
the tuple $t := (\sum_{i=1}^f b_i w_1 x_j)_{1 \le j \le k}$ is $F$-linearly
independent if and only if it is $K$-linearly independent. 
This in turn holds if and only if the tuple 
$(w x_j)_{1 \le j \le k}$ is $K$-linearly independent, since it is 
mapped to $t$ by $\alpha$. We can now use the standard basis 
algorithm from Definition~\ref{spinup}
with $w$ in place of $v$, testing for 
$K$-linear independence of the resulting vectors. In fact,
this will test for $F$-linear independence as is required 
--- note that this works without knowing $F$ explicitly, provided we
only take linear combinations over $K_0$ to find
$c \in FG$!

3. The result $SB(V,w,(g_1, \ldots, g_m))$  is an $F$-basis of an $FG$-submodule
of $V$ that is isomorphic to $\tilde{V}$. In particular, the representing
matrices for the $g_i$ expressed with respect to this basis contain
only coefficients from $F$. Since $SB(V,w,(g_1, \ldots, g_m))$ is 
$K$-linearly independent
it clearly is a $K$-basis of $V$ and we have found our base change
matrix $t$ explicitly.

Finally, we
 determine the smallest subfield $F$ of $K$ containing all
coefficients of $t^{-1}g_it$ for $1 \leq i \leq m$.

    All individual steps are $O(md^3)$ proving our claims.
If the search for $c$
 is repeated indefinitely, the probability of success tends to $1$
    and the expected number of tries is $1/(1-b)$.
\end{proof}

\noindent
In summary, our Las Vegas 
algorithm to write $G$ over a subfield proceeds as follows.
We assume that we have already tested $V$ for absolute irreducibility, 
and hence know the degree $e = \dim_K(\End_{KG}(V))$ of the splitting
field.
\begin{enumerate}
\item Choose a uniformly distributed random element $c \in K_0 G$
    in its action on $V$ and compute $\ker_V(c)$.
%\item Produce a few random elements of $G$ and compute a $F_0$-linear
%combination of them to get the action on $V$ of an element $c \in F_0G$.
Repeat this until $\dim_K(\ker_V(c)) = e$ or 
fail after $O(\log \delta^{-1})$ tries. 
\item Take $0 \neq w \in \ker_V(c)$ and compute $B := SB(V,w,(g_1, \ldots,
g_m))$ using $K$-linear independence.
\item Let $t^{-1} \in \GL(d,K)$ have the vectors in $B$ as rows, and
find the smallest subfield of $K$ containing all entries of all
$t^{-1}g_it$.
\end{enumerate}

By Theorem~\ref{writing} this algorithm either fails in step 1 with
bounded probability or finds the smallest possible subfield $F$ of $K$
together with an explicit base change matrix $t$ to write $G$ over $F$.
If $G$ cannot be written over a smaller field then $F = K$ in step $3$.


\section{Restriction to a subgroup of the derived group}
\label{subsemi:main}

We now consider the case of a matrix group $G = \langle g_1, \ldots, g_m
\rangle \leq \GL(d, q)$ acting absolutely
irreducibly on the natural module $V = \F_q^d$, that cannot be 
written over a smaller field with trivial scalars. Our algorithm
finds a reduction provided that $G$ lies in \CC3 or \CC5, 
or the derived group of $G$ is not absolutely irreducible. If none of these 
is the case, it might still find a reduction 
but might also report that $G$ does not lie in \CC3 or \CC5 and that $G'$ is absolutely irreducible.

In Sections~\ref{subsec:caseanalysis} and following we refer to a
normal subgroup $N$ of $G$ that is contained in the derived group
$G'$. In Section~\ref{subsec:derived} we describe a method of computing a
subgroup $H$ of such an $N$ which can be used instead. In each of the
following sections, we analyse the complexity of the algorithms used
in terms of number of field operations.

Note that some of these complexity results involve a prescribed bound
$\delta$ for the failure probability. If we do several such
steps consecutively, we have to adjust the individual bounds because
the complete procedure fails if any of the intermediate steps
fails. We analyse the overall picture in
Section~\ref{subsemi:sec:complexity}.

\subsection{Computing a normal subgroup of the derived group} 
\label{subsec:derived}

%We compute the derived group using the Monte Carlo algorithm described
%in \cite[2.3.4]{Ser}. Let $l_G$ be an upper bound for the
%length of subgroup chains in $G$ --- note that since $|\GL(d, q)| =
%O(q^{d^2})$ we may certainly take $l_G = d^2 \log q$. We start by
%computing a set $T$ containing $\mathrm{min}\{m^2, cl_G\}$ random
%commutators, where $c$ is a constant. Then $T$ is a normal generating
%set for $G'$, with probability $1$ if $|T| = m^2$ and with probability
%depending on $c$ otherwise.

% We then apply fast normal closure to $T$ to construct a generating set
% of size $O(l_G)$ in $O(l_G(|T| + m + l_G))$ matrix multiplications.

We proceed differently depending on whether $G$ has at most $7$ 
non-scalar generators. If this holds then we start by making all 
commutators of generators. If these are all scalar then $G'$ is 
scalar, and the methods of Section~\ref{subsec:scalars} apply with $G'$ known to be scalar. If any of them is found to be non-scalar we proceed to generate $H$ as described below.

If there are more than $7$ non-scalar generators then we compute $10$
commutators of random elements of $G$. If these are all scalar 
then for the first non-scalar generator $g_i$ of $G$ we test 
whether $[g_i, g_j]$ is scalar for $j > i$. If this holds for 
all $j$ then Proposition~\ref{prop:scal_reduct} applies. 

Otherwise, we now have a nonempty set $S$ of non-scalar commutators. We 
compute a subgroup $H$ of a normal subgroup $N$ of $G$ 
that is contained in $G'$ by 
 the methods of Section~\ref{sec:generation}. 
Namely,  we use a variant of \textsc{Rattle} \cite{LGMurray, LGO97} to 
\index{Rattle@\textsc{Rattle}}%
produce a set $T$ of $O(\log \delta^{-1} + d \log q)$ elements of 
$N = \langle S \rangle^G$. Computing $T$ has complexity 
$O(d^3 \log \delta^{-1} + d^4 \log q)$ 
field operations, since one random element can be produced
at cost $O(d^3)$ after initial setup.

By Corollary~\ref{cor:right_action} 
the group $H = \langle T \rangle \leq N$  has the 
same submodule structure and (if $N$ is irreducible) centraliser algebra 
as $N$ with probability at least $1 -
\delta$. That is, we can use $H$ instead of
$N$ in the methods described in subsequent sections.
%We will generally refer to working with $N$ instead of $H$, 
%with additional remarks as required. 


\subsection{A case analysis for $\mathbf{N \unlhd G}$ with
$\mathbf{N \leq G'}$}
\label{subsec:caseanalysis}

{} From now on we assume that $H$ is given by $s$ generators 
and is a subgroup of a normal subgroup $N$ of $G$ that is
contained in $G'$.  Note that
$s = O(\log \delta^{-1} + d \log q)$ if we use the method from
Section~\ref{subsec:derived}, but our algorithms in 
Sections~\ref{subsec:subfield_scalars} to~\ref{subsec:tensor} can be applied 
to any normal subgroup. The group $H$ might be smaller than $N$, but with 
probability $1-\delta$ the structure of the natural module is the same for both
groups. 

Since $N \unlhd  G$ there are only five possibilities, 
by Clifford's Theorem. 
\begin{enumerate}
\setlength{\itemsep}{0pt}\setlength{\parskip}{0pt}
\item $N$ is absolutely irreducible on $V$.
\item $N$ is irreducible but not absolutely irreducible.
\item $N$ is reducible, and $V$ is a direct sum of 
more than one homogeneous component.
\item $N$ is reducible, and $V$ splits into a direct 
sum of isomorphic $N$-submodules of dimension greater than $1$, so that in particular $N$ is non-scalar.
\item $N$ is reducible, and $V$ splits into a direct sum of isomorphic $1$-dimensional submodules, so that $N$ is scalar.
\end{enumerate}

We proceed differently in each of these five cases, but, assuming $G$
is in \CC3 or \CC5 or $N$ is not absolutely irreducible, in each case
we find a reduction with probability $\delta$ of failure. 
By a reduction we mean a non-trivial homomorphism onto a
smaller group or an isomorphism to a situation with smaller input
size. To distinguish these cases, we first run the MeatAxe on $H$ 
in place of $N$.
This uses $O(d^3 s \log \delta^{-1})$ 
field operations since $H$ is given by $s$ generators, where
$\delta$ is the upper bound for the failure probability for this step.
This MeatAxe run decides whether we run the algorithms for
case~1, case~2, or one of
cases~3 and 4. In case~2, it returns a field generator of the endomorphism
algebra. In cases 3 and 4, it returns a proper $H$-submodule. Note
that if we use the methods in Section~\ref{subsec:derived} to compute
$H$, then case~5 is never found here because it is detected earlier on
 (see Section~\ref{subsec:scalars}).


\subsection{Absolutely irreducible normal subgroup}
\label{subsec:subfield_scalars}

We continue to assume that $N$ is a normal subgroup of $G$ that is
contained in $G'$, and add the assumption that the MeatAxe has shown that
$H$ and hence $N$ act absolutely irreducibly.

We first note the following lemma, which rules out case~1 for \CC3.

\begin{Lemm}
\label{C3notabsirred}
If $G$ lies in class \CC3 then $G'$, and 
hence $H$ and $N$, are not absolutely irreducible.
\end{Lemm}
\begin{proof}
Assume that there is  an $\F_{q^e}$-vector space
structure on $V$, such that $G$ acts
semilinearly. Then
$G'$ acts $\F_{q^e}$-linearly and thus $\End_{\F_q
G'}(V) \neq \F_q$. Thus $G'$ is not absolutely
irreducible.
\end{proof}


We can therefore assume in this case that $G$
lies in class \CC5, provided $G$ is in \CC3 or in \CC5 or 
$G'$ is not absolutely irreducible.

\begin{Lemm} [{Compare \cite[Lemma~4.1]{GLGOB}}]
If\/ $G$ can be written over\/ $\F_{q_0}$ modulo scalars in\/ $\F_q$, 
then $N \leq G'$ can be written over\/ $\F_{q_0}$.
\end{Lemm}

\begin{proof}
Multiplying each of $g, h \in G$ by
a fixed scalar fixes  the value of $[g, h]$.
\end{proof}




\begin{Theo} [Recognition of \CC5]
Consider $G = \langle g_1, \ldots, g_m \rangle 
\leq \GL(d, q)$ or its projective version and let $1 > \delta > 0$ be given. 
Let $N = \langle n_1, \ldots, n_s \rangle \unlhd G$ be a subgroup of the
derived group known (by
a MeatAxe run) to be absolutely irreducible. 
Then in $O(sd^3 \log \delta^{-1} + md^3)$ field operations
we construct a homomorphism to\/ $\PGL(d, q_0)$ for minimal $q_0$
 or prove that $G$ and
$\overline{G}$ are not in \CC5. The algorithm returns {\tt fail} with
probability at most $\delta$.
\end{Theo}

\begin{proof}
Since $N$ is absolutely irreducible we use the methods of
Section~\ref{realisesubfield} to find a matrix $t$ such that $t^{-1}Nt
\leq \GL(d, q')$
with $\delta$ as an upper bound for the failure probability.
This automatically finds the smallest prime power $q'$ with this       
property. Notice that the $\F_{q'}$-subspace chosen in the kernel 
vector by the standard basis method is unique up to multiplication 
by elements of $\End_{\F_q N}(V) = \F_q$.   From this point on 
the algorithm is guaranteed to determine   
whether $G$ lies in~\CC5.

Now examine $h_i:= t^{-1}g_it$ and check whether it can be written as a
product of a scalar $\lambda_i \in \F_{q}$ and an element of $\GL(d,
q_0)$ for $q' < q_0 < q$. 
To do so, notice that if $h_i \in \lambda_i \GL(d,q_0)$,
then the quotient between any two
nonzero entries in $h_i$ lies in $\F_{q_0}$. 
Therefore we may take
$\lambda_i$ to be any nonzero entry of $h_i$ and then find the
minimal field $\F_{q_0}$ containing all entries of $h_i / \lambda_i$. 
This enables us to set up a homomorphism from $G$ to $\PGL(d, q_0)$
with kernel $G \cap Z(\GL(d, q))$, and so a reduction has been
completed. 
For the projective group $\overline{G}$,
we get a homomorphism into $\PGL(d,q_0)$, which could be
an isomorphism. Even if this is the case, we have reduced to a smaller field. 

If no smaller field is found, the procedure reports that
 $G$ does not lie in \CC3 or \CC5 and that $G'$ is absolutely irreducible.
\end{proof}

Note that although we work with $H$ instead of $N$, 
since $H$ is absolutely irreducible so is $N$. 


\subsection{Irreducible but not absolutely irreducible} \label{subsec:semilin}

We continue to assume that $N$ is a normal subgroup of $G$ that is
contained in $G'$. As described in
Section~\ref{subsec:caseanalysis} we assume that the MeatAxe has proved
that $H$ acts irreducibly but not absolutely
irreducibly, and so $N$ is guaranteed to act irreducibly, but with 
probability at most $\delta$ the endomorphism ring $\End_{\F_q N}(V)$ may 
be smaller than $\End_{\F_q H}(V)$. 
We will deal with this possibility at the end of this section, 
and in general talk about $N$ rather than $H$.

%The following proposition shows that $G$ is in \CC3 in the case of
%the present section. We then demonstrate how this can be used to find
%a reduction. 

\begin{Prop}
If $G$ is absolutely irreducible and $N \unlhd G$ is irreducible but not
absolutely irreducible then $G$ is semilinear.
\end{Prop}

\begin{proof}
Since $N$ is irreducible but not absolutely irreducible, $E = \End_{\F_qN}(V) = \F_{q^e}$ for some $e > 1$.
 Let $C \in \GL(d, q)$ generate the multiplicative group of
$E$.

For all $h \in N$, $g \in G$, by definition $hC = Ch$, thus
$h^g C^g = C^g h^g = h_1 C^g = C^g h_1$, for some $h_1 \in N$. As
$h$ varies over $N$ the element $h_1$ takes every value in $N$, therefore
$\langle C \rangle ^g = \langle C \rangle$, and so $C^g = C^k$ for some
 $k$. Suppose that $C^i + C^j = C^l$, 
then $(C^i)^g + (C^j)^g = (C^l)^g$
so $g$ acts as field automorphisms on $\F_{q^e}$ and thus $G$ is
semilinear.
\end{proof}


\begin{Theo} [Recognition of \CC3]
Let $G = \langle g_1, \ldots, g_m \rangle \leq \GL(d, q)$ or its
projective version.
Let $N = \langle n_1, \ldots, n_s \rangle \unlhd G$ be  known 
to be irreducible but not absolutely irreducible. 
In deterministic $O(d^4\log q + m d^3)$ field operations we construct
 two homomorphisms, one to
the cyclic group of order $e$ for some divisor $e$ of $d$ and a second
from the kernel of the
first to\/ $\GL(d/e, q^e)$ or\/ $\PGL(d/e, q^e)$.
\end{Theo} 

\begin{proof}
%We can apply the method of Section~\ref{subsec:subfield_scalars} to
%check whether $G$ is in \CC5, since the algorithms
%in Section~\ref{realisesubfield} only need $N$ to act irreducibly.
%This takes $O((m+s\log(\delta^{-1})) d^3)$ field operations.
%We assume that this is done, and the index $f$ of $\F_{q_0}$ in
%$\F_q$ is stored. At the end, if both methods succeed then the \CC3 or
%the \CC5 method is chosen depending on whether $e > f^2$.
When  $N$ is irreducible but not absolutely irreducible,
the MeatAxe returns a generator $C$ of
the field $\F_{q^e} = \End_{\F_q N}(V)$ realised as a matrix in
$\GL(d,q)$ together with  $e$. Note that $e \le d$. 
The matrix $C$ need not generate the
multiplicative group of $\F_{q^e}$, but its powers $C^0, C^1, \ldots,
C^{e-1}$ are $\F_q$-linearly independent. 

If $G$ acts via field automorphisms on $\F_{q^e}$, then we can immediately
read off this action using $O(md^3)$ field operations by computing
the matrices $C, C^q, C^{q^2}, \ldots, C^{q^{e-1}}$, conjugating
$C$ with the generators of $G$ and looking up the result. 
Computing these matrices requires at most $O(d^4 \log q)$ field
operations and space for $O(d)$ matrices, since $e \le d$.
This provides a homomorphism from $G$ to the cyclic group of order
$e$.

In addition,  $C$ gives an
explicit $\F_{q^e}$-vector space structure on  $V$.
To get the $\F_{q^e}$-span of a vector $v \in V$ we  compute
$v, vC, vC^2, \ldots, vC^{e-1}$. In this way we can 
perform a spinning algorithm for $V$ as an $\F_{q^e}$-vector space. All
computations are with vectors over $\F_q$ but whenever
we produce a new vector $v$ that does not lie in the $\F_q$-span of what we
already have, we not only add $v$ but also $vC, vC^2, \ldots,
vC^{e-1}$ by repeatedly multiplying with $C$. 
This spinning algorithm needs at most $O(md^3)$ field
operations.

%If we now apply the base change with the newly found $\F_{q^e}$-adapted
%basis, we can read off the action of elements of $G$ on $\F_{q^e}$
%by conjugating matrices. We get this action as a linear action on
%the $e$-dimensional $\F_q$-vector space $\F_{q^e}$ thus providing
%a homomorphism onto a smaller matrix group. If we can solve the
%discrete logarithm problem in $\F_{q^e}$ more easily, directly a homomorphism
%onto the automorphism group of the multiplicative group of $\F_{q^e}$.
The kernel of the action as field automorphisms acts $\F_{q^e}$-linearly
on the original space and we read off this action using
the above base change to the $\F_{q^e}$-adapted basis. This therefore also
 leads to a reduction for the kernel by reducing the input size to
$(d/e \times d/e)$-matrices over $\F_{q^e}$. 

Altogether, we have found a significant reduction 
using $O(d^3(d\log q +m))$ field operations and memory for $O(d)$ matrices.

%{\bf CMRD: Not quite sure what to say about complexity of this para,
%in particular how do we produce an element of order $q^i-1$ given the
%elements $C, \ldots, C^{e-1}$?}.

Note that since scalars from $\F_q$ do not alter the
action of elements of $G$ as field automorphisms on $\F_{q^e}$ the
same procedure works for the projective case $\overline{G}$.

The homomorphism is the same as in the matrix case, the kernel 
is a subgroup of $\PGL(d,q)$ and we construct a map from the kernel into
$\PGL(d/e,q^e)$ by writing the matrices over the bigger field. This
map in turn has a kernel, since we divide out more scalars. However, this
second kernel only contains $\F_{q^e}$-scalars modulo
$\F_q$-scalars, which can be handled easily. Thus this case can
be handled in the projective situation.
\end{proof}

We finish with a discussion of the possibility that 
$\F_{q^e} = \End_{\F_q H}(V) \neq \End_{\F_q N}(V)$, which happens 
with probability bounded by $\delta$. 
If  $\End_{\F_q H}(V) \neq \End_{\F_q N}(V)$ then the elements 
of $G$ need not
act as field automorphisms on $\F_{q^e}$. However, if $G$ is contained in 
\CC3 then they
\emph{will} act as field automorphisms on some subfield of $\F_{q^e}$
that properly contains $\F_q$. Thus, if $e$ has not too many divisors,
we test for each divisor $i$ of $e$ whether $G$ acts as field automorphisms
on $\F_{q^i}$.
In theory we could use the fact that $e$ has  $O(\log d)$ prime divisors 
(by the Prime Number Theorem) to 
find a subfield $\F_{q^r} \subseteq \End_{\F_q N}(V)$ and hence a reduction
 in 
$O(\log d)$ attempts, but in practice we always find the divisors of $e$.

To find a generating element $C'$ for the field $\F_{q^i}$ we proceed
as follows. First we find a polynomial $h \in \F_q[x]$ of degree less
than $e$ such that $h(C)$ has order $q^e-1$. This involves only
polynomial arithmetic using the minimal polynomial of $C$. Evaluating
$h(C)$ can be done in at most $O(d^4)$ field operations. We then
power up $h(C)$ to exponent $(q^e-1)/(q^i-1)$, which can be done
using at most $O(d^4 \log q)$ field operations.

If no divisor works
then the algorithm reports {\tt fail}, that $G$ is not in \CC3 and 
that $N$ and the derived group are absolutely irreducible. Note that if
$N$ is not absolutely irreducible then the algorithm is guaranteed to
find that $G$ is in \CC3 at this point, therefore if failure is reported
the algorithm adds generators to $H$ until it is absolutely irreducible,
and then returns to the test of Section~\ref{subsec:subfield_scalars}.
However this can only happen with probability $\delta$.



 If there
are too many divisors, we simply report {\tt fail}. This happens
with probability at most $\delta$ by Corollary~\ref{cor:right_action}, 
and never in practice. 

\subsection{More than one homogeneous component} \label{subsec:imprim}

We continue to assume that $N$ is a normal subgroup of $G$ that is
contained in $G'$. As described in
Section~\ref{subsec:caseanalysis} we assume that the MeatAxe 
has proved that $H$ acts reducibly by finding an explicit
proper nontrivial submodule $V'$ of the natural module.

First we prove a lemma which will eventually be used to find an 
irreducible $H$-submodule that with probability $1 - \delta$ 
is an $N$-submodule. 
\begin{Lemm} [Finding an irreducible submodule] \label{lem:find_irred}
Let $N = \langle n_1, \ldots, n_s \rangle \leq \GL(d, q)$ act reducibly
on the natural module $V$ and let $1 > \delta > 0$ be given. 
Given a submodule $V' < V$, an irreducible
$N$-subfactor can be found in Las Vegas $O(s d^3 \log (\delta^{-1} \log
d))$ field operations, with probability of failure at most
$\delta$.
\end{Lemm}

\begin{proof}
We repeatedly use the MeatAxe to find an irreducible subfactor
of $V|_{\F_q N}$. Initially, we have a submodule $V' < V$.
We run the MeatAxe either on  $V/V'$ or on $V'$, whichever has the
smaller dimension. If we find a proper submodule, we repeat the same
technique. Since we halve the dimension in each step, this terminates after
 at most
$\log_2 d$ runs of the MeatAxe using at most 
$O(d^3s2^{-3i}\log\delta'^{-1})$ field operations in step $i$, where
$\delta'$ is an upper bound for the failure probability in each step.
To bound the overall failure probability of this whole
procedure by  $\delta$, we define $\delta' :=
\delta/\log_2 d$. Since $\sum_{i=1}^\infty 2^{-3i} < 1$, the overall
cost for finding an irreducible subfactor is 
$O(sd^3 \log(\delta^{-1} \log d))$.
\end{proof}

\begin{Theo} [Construction of block action] \label{thm:imp_reduct}
Let $G = \langle g_1, \ldots, g_m \rangle \leq \GL(d, q)$ or its
projective version, and let $1 > \delta > 0$ be given. Let $N =
\langle n_1, \ldots, n_s \rangle \unlhd G$ be known (by a MeatAxe
run) to be reducible. 

In Las Vegas $O(sd^3 \log (\delta^{-1} \log
d))$ field operations  we can either construct a homomorphism from $G$ to a
permutation group with kernel the pointwise stabiliser of a set of blocks 
 or prove that $V|_{\F_q N}$ has a single homogeneous component.
The probability of failure is bounded from above by $\delta$.
\end{Theo}

\begin{proof}
By assumption  $V|_{\F_q N}$, as an $\F_q N$-module, is a
direct sum of  homogeneous components $C_1, \ldots, C_k$ with $k > 1$.
The $C_i$ form a block system exhibiting an imprimitive action
of $G$ and $N$ is a normal subgroup of the kernel of the action on
blocks. We only have to find the action on this block system to find a
reduction.

By Lemma~\ref{lem:find_irred} we can find an irreducible $N$-subfactor
in $O(sd^3 \log (\delta^{-1} \log d))$ field operations with probability
of failure $\delta$. One application of the isomorphism testing
procedure described in Theorem~\ref{isotest} gives a homomorphism of the
irreducible factor into $V|_{\F_q N}$ and thus an irreducible submodule
$S$ with $O(sd^3)$ field operations. Note that when we
proved the final subfactor to be irreducible  we constructed the algebra
word $c$ that is needed for isomorphism testing, namely a word 
describing an algebra element with nullity the dimension of
the centraliser of~$N$.

Such an irreducible module $S$ is all we need to run the {\sc
MinBlocks} procedure described in \cite{smashprim} which needs
$O(sd^3)$ field operations to compute the block system or reports
that there is none. If the latter occurs, then there is a single 
homogeneous constituent and we apply the algorithms of 
Section~\ref{subsec:tensor}. Otherwise
this provides a non-trivial homomorphism onto a permutation
group and thus a reduction. The overall complexity 
is $O(sd^3\log(\delta^{-1} \log d))$.

Since $\F_q$-scalars act trivially on the homogeneous
components, the homomorphism onto the permutation group has all
scalars in its kernel. Therefore we can use the same homomorphism
for the projective situation with $\overline{G}$. Thus,  this case
can be handled in the projective situation.
\end{proof}

Of course in practice we work with a subgroup $H$ of $N$. If $H$
has a submodule that is \emph{not} an $N$-submodule then it is
possible that we will not be able to find a homomorphism from the
irreducible $H$-subfactor to $V$. In this case the algorithm will report
{\tt fail}: note that this occurs with probability at most $\delta$. 
However, it is possible to rerun the algorithm starting at
Sections~\ref{subsec:caseanalysis} with a new version of $H$ that has
submodule structure closer to that of $N$. To see this, note that
the subfactor is described by two $H$-submodules of $V$, at least one
of which is not preserved by $N$. Therefore a simple argument shows that at
least half of the elements of $N$ must fail to fix at least one of the
two $H$-submodules. Thus we add a new generator to $H$ and return to the
MeatAxe run of Section~\ref{subsec:caseanalysis} to determine whether
the new $H$ is (absolutely) irreducible.



\subsection{Isomorphic irreducible submodules of dimension at least 2} 
\label{subsec:tensor}

We continue to assume that $N$ is a normal subgroup of $G$ that is
contained in $G'$. As described in
Section~\ref{subsec:caseanalysis} we assume that the
MeatAxe has proved that $H$ acts reducibly by finding an explicit
proper nontrivial submodule $V'$ of the natural module.

As described in Theorem~\ref{thm:imp_reduct} we first find an
irreducible $H$-submodule $S$ and run the {\sc MinBlocks} procedure. If
this fails to find a block system, then (assuming $H$ has the same
submodule structure as $N$) there is only one homogeneous component,
corresponding to $S$.

\begin{Theo} [Reduction for single homogeneous component]
Let  $G = \langle g_1, \ldots, g_m \rangle \leq \GL(d, q)$ or its
projective version be absolutely irreducible, and let $1 > \delta > 0$ be given.
Let $N = \langle n_1, \ldots, n_s \rangle  \unlhd G$ be reducible, with a single homogeneous component of dimension $n > 1$, and let an irreducible $N$-submodule $S$ be given.
In deterministic $O((s+m)d^3)$ field operations we construct a proper nontrivial 
homomorphism from $G$ into\/ $\PGL(d/n, q^e)$ for $e$ the degree of 
the centraliser of $N$ on $S$. 
\end{Theo}

\begin{proof}
We first find an explicit decomposition of $V|_{\F_q N}$
as a direct sum of copies of $S$. This can be done using a variant
of the isomorphism testing procedure described in
Theorem~\ref{isotest} to compute a basis of the space of all
homomorphisms of $S$ into $V|_{\F_q N}$. Namely, we compute the action on
$V$ of the algebra word $c \in \F_q N$ that proved that $S$ is simple
and determine its kernel $K$. Since $V|_{\F_q N}$ is isomorphic to a direct
sum of copies of $S$, we can choose an arbitrary nonzero vector from $K$,
compute the standard basis with respect to the generators of $N$ starting
at that vector and thereby find a summand $S_1$ of $V|_{\F_q N}$ together
with an explicit isomorphism of $S$ to $S_1$. By choosing further vectors
from $K$ that are not contained in the direct sum of previous
copies of $S$ and repeating this procedure, we inductively get an explicit
direct sum decomposition of $V|_{\F_q N}$ into summands that are all
isomorphic to $S$. This automatically leads to a base change
such that every element of $N$ is represented by a block
diagonal matrix in which all diagonal blocks are identical of size $n :=
\dim_{\F_q}(S)$ in $O(sd^3)$ field operations.

As $N \unlhd G$, for all $h \in N$ and $g \in G$,
 the product $g^{-1}hg \in N$ and thus $g^{-1} h g$ is also 
a block diagonal matrix in which all $n \times n$-blocks along the diagonal
are identical. Fixing $g$, we conclude that $g\cdot (g^{-1}hg) = hg$ for all
$h \in N$. If we now cut $g$ into $n \times n$-blocks, we get:

\begin{eqnarray*}
   &g \cdot (g^{-1}hg) & 
 = \left[ \begin{array}{c|c|c|c}
      g_{1,1} & g_{1,2} & \cdots & g_{1,d/n} \\ \hline
      g_{2,1} & g_{2,2} & \cdots & g_{2,d/n} \\ \hline
      \vdots  & \vdots  & \ddots & \vdots    \\ \hline
      g_{d/n,1}&g_{d/n,2}& \cdots& g_{d/n,d/n} \end{array} \right]
\cdot \left[ \begin{array}{c|c|c|c}
      D^g(h) & 0   & \cdots &      0    \\ \hline
         0   &D^g(h)&\cdots &      0    \\ \hline
      \vdots  & \vdots  & \ddots & \vdots    \\ \hline
         0    &    0    & \cdots& D^g(h) \end{array} \right] \\
 &=& \left[ \begin{array}{c|c|c|c}
      D(h)    & 0       & \cdots &     0    \\ \hline
         0    &D(h)     &\cdots &      0    \\ \hline
      \vdots  & \vdots  & \ddots & \vdots    \\ \hline
         0    &    0    & \cdots& D(h)   \end{array} \right]
\cdot \left[ \begin{array}{c|c|c|c}
      g_{1,1} & g_{1,2} & \cdots & g_{1,d/n} \\ \hline
      g_{2,1} & g_{2,2} & \cdots & g_{2,d/n} \\ \hline
      \vdots  & \vdots  & \ddots & \vdots    \\ \hline
      g_{d/n,1}&g_{d/n,2}& \cdots& g_{d/n,d/n} \end{array} \right]
 = hg
\end{eqnarray*}
where the $g_{i,j}$ are $n \times n$-matrices, $D(h)$ is a matrix
representing $h$ on the module $S$ and $D^g(h) = D(g^{-1}hg)$ is the
same representation twisted by the element $g$. By the block diagonal
structure of the matrices in $N$ we get 
$g_{i,j} \cdot D^g(h) = D(h) \cdot g_{i,j}$ for all $i$ and $j$ and 
all $h \in N$.

But by hypothesis, the matrix representations $D$ and $D^g$ of $N$
are isomorphic. Thus there is a nonzero matrix $T \in \F_q^{n \times n}$ with
$T \cdot D^g(h) = D(h) \cdot T$ for all $h \in N$. By Schur's lemma and
since the representation $D$ is irreducible,
the matrix $T$ is invertible and unique 
up to multiplication by an element of $C_{\GL(n,q)}(D(N))$, which
is isomorphic  as a group to the group of units of the extension field 
$\End_{\F_q N}(S) \cong \F_{q^e}$.

This shows that for every pair $(i,j) \in \{ 1, \ldots, d/n \} \times 
\{ 1, \ldots, d/n \}$ there
is a unique element $e_{i,j} \in \End_{\F_q N}(S)$ (possibly $0$) with 
$g_{i,j} = T \cdot e_{i,j}$. Thus we have shown that with respect to
the above choice of basis, every element $g$ is equal to a Kronecker 
product of some matrix in $U \in \F_{q^e}^{d/n \times d/n}$ with a matrix
$T \in \F_q^{n \times n}$. Since $g$ is invertible both 
$U$ and $T$ are invertible. 

This provides an explicit embedding of 
$\F_q^d$ into a tensor product $\F_{q^e}^{d/n} \otimes_{\F_q} \F_q^n$, 
where one factor can be over an extension field if the $\F_q N$-module
$S$ is not absolutely irreducible. This embedding can be computed
explicitly because the above base change is constructive.
Using another $O(md^3)$ field operations we compute the generators
of $G$ after the base change from which we can read off the tensor
decomposition.

Thus we get a non-trivial homomorphism of $G$ into $\PGL(d/n,q^e)$ with
$N$ lying in the kernel which is a significant reduction. The kernel of
this homomorphism can immediately be reduced further since its 
elements are block
diagonal matrices with identical $n \times n$-diagonal blocks.

The projective situation can be handled identically,
by viewing the kernel as a  projective group.

\end{proof}

If  $H$ is a proper subgroup of $N$, then
our algorithm can fail in two ways. Firstly,
$V|_{\F_q H}$ might not be isomorphic to a direct sum of
copies of the irreducible $H$-module $S$. In this case
there are not enough homomorphisms from $S$ into
the socle of $V|_{\F_q H}$ to span the whole of $V$. Secondly,
even if $V|_{\F_q H}$ is a direct sum of copies of $S$,  the
generators of $G$ might not be not Kronecker products after a corresponding
base change, which we detect during the setup of the homomorphism. In
both cases, the error is detected and the algorithm reports {\tt fail}. However,
by Corollary~\ref{cor:right_action} this happens with probability at most $\delta$.

\subsection{Normal subgroup is scalar} \label{subsec:scalars}

The remaining case is that the restriction of the natural module to $N$
has only one homogeneous component and all irreducible $N$-constituents
are one-dimensional, so that $N$ consists of scalars.  
We may also use the algorithms in this section if we have found a 
non-scalar generator $g_i$ of $G$ such that $[g_i, g_j]$ is scalar for all $j$. 



We start with a proposition giving a homomorphism 
into the multiplicative group of the field that need 
not necessarily correspond to an imprimitive 
decomposition of the natural module.

\begin{Prop} [Scalar homomorphism] \label{prop:scal_reduct}
Let $G = \langle g_1, \ldots, g_m \rangle \leq \GL(d, q)$ or 
its projective version
be an absolutely irreducible group such that the commutator 
of a non-scalar generator $g_i$ with all other generators 
is known to be scalar. Then 
we can construct a nontrivial homomorphism from $G$ 
into the multiplicative group of\/ $\F_q$ at no further cost.
%, and construct a presentation for the image.
\end{Prop}

\begin{proof}
We are given a non-scalar generator $g_i$, 
and we already have calculated the homomorphism $\psi_{g_i}(g_j) = [g_j, g_i]$ from Lemma~\ref{lem:facts} 
for all  $g_j$. 
This
kernel of $\psi_{g_i}$ is $C_G(g_i)$. Since $g_i$ is non-central, $\psi_{g_i}$ 
is nontrivial.
Multiplying generators by scalars does not 
change commutators, 
so these algorithms will also work in the projective case.  %
\end{proof}


%Since the image is cyclic the order $o$ of 
%the image is simply the lowest common multiple of 
%the orders $a_i$ of the finite field elements $\psi_g(g_i)$, 
%each of which is calculated by a single discrete logarithm call. 

%If $o_i$ is small then we can compute the kernel by 
%finding Schreier generators for the identity in the image. 
%If the order of the image is $O(q)$ then this would produce 
%too many generators. Let $\psi_g(h)$ be a generator for the 
%image. Then the set $\{g_i h^{o/a_i}, h^o : 1 \leq i \leq m\}$ 
%is a normal generating set for the kernel. 
Since $g_i \in C_G(g_i)$ the kernel will 
be a non-absolutely irreducible group, and may not be irreducible. 
If $G'$ is known to be scalar then
 the derived group of the kernel $C_G(g_i)$ is also
central, and hence a hint can be passed to the kernel to return 
to the techniques of this section once an absolutely irreducible 
representation has been found.



Finally we give a deterministic decomposition algorithm 
for groups with scalar derived group that are not $r$-groups. 
We can apply this algorithm if $G$ has 
at most $7$ non-scalar generators, so that all commutators of generators 
have been calculated --- in this case the $m^2$ vanished
from the complexity. 
This algorithm can easily be modified to decompose any 
black box group with order oracle that is known to be 
nilpotent and not a $p$-group. 
The assumption that the prime factors of $q^i-1$ 
are known for $i \leq d$ is reasonable in practice, 
and is relied upon for many other algorithms: see 
\cite{brillart} for details of currently maintained lists of such factors.


\begin{Lemm} 
Let $G = \langle g_1, \ldots, g_m \rangle \leq \GL(d, q)$ 
be an absolutely irreducible group whose derived group 
consists only of scalars. Suppose that the order of $G$ 
is divisible by $k$ primes for some $k > 1$, and that 
the prime divisors of $q^i-1$ are known for $1 \leq i \leq d$. 
Then $k < \log (q-1)$ and in  $O(m^2 d^3 \log q \log( d \log q))$ 
field operations we can compute a homomorphism from $G$ 
whose kernel and image have order divisible by $\lfloor k/2 \rfloor$ 
and $\lceil (k+1)/2 \rceil$ primes respectively. Both 
the kernel and image have at most $m$ generators. 
\end{Lemm}

\begin{proof}
By Proposition~\ref{exponents}.3 the order of $G$ is a 
divisor of $o:= (q-1)^{m+1}$, which is divisible by 
less than $\log (q-1)$ distinct primes. 

The group $G$ is a direct product of its Sylow subgroups 
by Lemma~\ref{lem:facts}.2. We compute the order $o_i$ of 
$g_i$ for $1 \leq i \leq m$ in $O(m d^3 \log q \log (d \log q))$ 
field operations (see \cite{CellLeedOrder}), and find a set of 
primes $\{p_1, \ldots, p_k\}$ such that each $o_i$ is 
a product of powers of these primes. 
For $1 \leq i \leq k$ we find the highest exponent $\alpha_i$ 
such that $p_i^{\alpha_i}$ divides $o$. %this step is 

Let $a:= \lfloor k/2 \rfloor$ and define $r = p_1^{\alpha_1} \cdots
p_a^{\alpha_a}$ and $r' = p_{a+1}^{\alpha_{a+1}} \cdots p_k^{\alpha_k}$.
First run the extended Euclidean algorithm to find $s$ and $s'$ such
that $1 = sr + s'r'$, then let $N = s'r'$ and $M = sr$. Clearly for all
$g \in G$ the order of $g^N$ divides $r$ whilst $|g^M|$ divides $r'$.
Therefore for all $g \in G$ the only way to write $g$ as a product of
an element of order dividing $r$ and an element of order dividing $r'$
is $g = g^N g^M$. Since $G$ is nilpotent, the map $x \mapsto x^N$ is a
homomorphism from $G$ to $\Syl(G, p_{a+1}) \times \cdots \times \Syl(G,
p_k)$ with kernel $\Syl(G, p_1) \times \cdots \times \Syl(G, p_{a})$.

Notice that $p_i^{\alpha_i}$ divides $o$ for 
all $i$  so  $N < o$, and hence we can raise each 
generator to the power $N$ in $O(m^2 d^3 \log q)$ field 
operations to get at most $m$ generators for the image. 
Multiplying each generator by the inverse of its image 
in $O(m d^3)$ field operations will produce at most $m$ 
generators for the kernel.
\end{proof}



%We finish with a theorem which draws together the results of this section.
%\begin{Theo}
%Let $G = \langle g_1, \ldots, g_m \rangle \leq \GL(d,q)$ be 
%absolutely irreducible such that $G'$ consists only of scalars. 
%Then in $O(m d^3 \log^2 q \log \log q^d)$ field operations, 
%plus $O(m \log q)$ discrete logarithm calls, we can construct 
%a homomorphism from $G$ with nontrivial image and kernel. 
%\end{Theo}

\section{Complexity summary}\label{subsemi:sec:complexity}

In this section we summarise our complexity results, mainly for the
sake of a good overview, but also to explicitly give our assumptions. 

We begin by describing the complexity of a ``MeatAxe run''. Although
this result is well-known we want to 
say exactly what assumptions we make and what results underlie
our complexity analysis.

\begin{Hyp}
\label{hypMtx}
\index{MeatAxe}%
Assume that we can generate a sufficiently evenly distributed random
element in the group generated by a set $X$ of matrices, and in its 
normal closure  in
 a  group $G \le \GL(d,q)$, in $O(d^3)$ elementary field
operations.
Assume further that we can generate a sufficiently evenly distributed
random element in a matrix algebra $\mathcal{A} \le \F_q^{d \times d}$ generated by a set $X$ of matrices
in $O(d^3)$ elementary field operations.
\end{Hyp}

Note that this assumption is valid in practice by using Product
Replacement and Rattle
\index{Rattle@\textsc{Rattle}}%
 (see \cite{ProdRep, LGMurray, LGO97}) 
 for groups and normal closure, and by taking random 
linear combinations of random products of generators for algebras,
at least after an initialisation phase.
However, these assumptions are not proven to hold in general!

\begin{Lemm}[MeatAxe]
\label{MeatAxe}
\index{MeatAxe}%
Let $F$ be a finite field, $\mathcal{A}$ a finite dimensional $F$-algebra,
$V$ an $\mathcal{A}$-module of $F$-dimension $d$, given by the action
of $m$ generators of $\mathcal{A}$ as matrices in $F^{d \times d}$,
and let $0 < \delta < 1$ be given. 
There is a Las Vegas algorithm with failure probability less than $\delta$ 
that determines whether $V$ is irreducible
in $O(md^3\log \delta^{-1})$ elementary field operations.
In the case of success the result is either a proper nontrivial
submodule or the answer ``irreducible'' together with a field generator
of the endomorphism ring\/ $\End_{\mathcal{A}}(V)$. Running the algorithm
until success gives a deterministic algorithm in which a step needs
$O(md^3)$ field operations and the expected value of the number of such
steps is bounded by a constant not depending on $|F|$, $d$ and $m$.
\end{Lemm}

\begin{proof}
All of this is proved in \cite{MeatAxeHoltRees, IL}, since it is shown that
a certain percentage (not depending on $|F|$ or $d$ or $m$) of all 
matrix algebra elements are usable to
reach a decision and all operations in one step are $O(md^3)$.
Note that we assume that we are able to create sufficiently random elements
as in Hypothesis~\ref{hypMtx}.
\end{proof}

We now summarise our complexity results, all assuming
Hypothesis~\ref{hypMtx} and all given in terms of number of field
operations. The whole procedure contains several subalgorithms of Las
Vegas type, namely in steps 2, 5, 6, 7 and 8. However, at most 4 of
them are possibly executed sequentially (namely in the execution path 
with steps 1, 2, 3, 4, 5, 8, 9). Thus if we prescribe a failure probability
of $\delta/4$ in each Las Vegas step, we get a Las Vegas algorithm with overall
failure probability bounded from above by $\delta$. Notice that the factor of 4 does not affect the ``big O'' complexity. 
We follow the numbering in our summary of the 
complete procedure in Section~\ref{sec:defns}:

\begin{enumerate}
\item Assume $G = \left< g_1, \ldots, g_m \right> \le \GL(d,q)$ acting
irreducibly, $E = \End_{\F_q G}(\F^d) = \F_{q^e}$ and
field generator $C \in \GL(d,q)$ of $E^\ast$ are known. If $e > 1$ find an explicit
base change in $O(md^3)$ field operations.
\item Try to write $G$ over a subfield with $\beta_i = 1$ for $1 \leq i \leq m$ 
in $O(md^3\log \delta^{-1})$ field operations
(see Theorem~\ref{writing}).
\item Immediately get a reduction in either the non-absolutely irreducible
or the subfield case.
\item Compute $10$ commutators of random elements of $G$. If all are
scalar test in $O(md^3)$ whether all commutators of some non-scalar
$g_i$ with other generators are scalar. If so, jump to step 10.
\item Compute $s = O(d\log q + \log \delta^{-1})$ generators for 
    $H \le N \unlhd G$ with $N$ contained in $G'$ in 
    $O(d^3\log \delta^{-1} + d^4 \log q)$ field operations
    (see Section~\ref{subsec:derived}). 

Run the MeatAxe to distinguish cases for $N$ in 
$O(sd^3 \log \delta^{-1})$ field operations
(see Section~\ref{subsec:caseanalysis}).
\item If $N$ is absolutely irreducible, check whether $G$ is in \CC5
in $O(sd^3\log \delta^{-1} + md^3)$ field operations
(see Section~\ref{subsec:subfield_scalars}).
\item If $N$ is irreducible but not absolutely irreducible, check
whether $G$ is in \CC3 in $O(d^4\log q +m d^3)$ field operations
    as in Section~\ref{subsec:semilin}.
\item If $N$ is reducible, look for more than one 
homogeneous component and if so find an imprimitive 
decomposition of $G$ in $O(sd^3\log( \delta^{-1} \log d))$ field operations as in
Section~\ref{subsec:imprim}.
\item If $N$ is reducible with a single 
homogeneous component with irreducible $N$-submodules 
dimension greater than $1$, find a tensor
decomposition of $G$ in $O((s+m)d^3)$ field operations as in 
Section~\ref{subsec:tensor}.
\item If one non-scalar 
generator has only scalar commutators with other generators we have 
already constructed
a nontrivial homomorphism from $G$ to $\F_q^\times$
 as in
Section~\ref{subsec:scalars}.
\end{enumerate}

The above algorithm can stop after either of steps 3, 6, 7, 8, 9 or 10.
We summarise the complexity statements for each of the possible paths
through the above steps in Table~\ref{casescomplexity}.

The worst cases are the last two, where the overall complexity is
bounded from above by
\[ O(md^3 \log \delta^{-1} +d^4\log q
              + sd^3 \log( \delta^{-1} \log d)), \]
where $s = O(\log \delta^{-1} + d\log q)$, that is by
\[ O(d^3\log \delta^{-1}(m + \log( \delta^{-1} \log d))
              + d^4 \log( \delta^{-1} \log d) \log q). \]
Fixing $\delta > 0$ this simplifies to $O(d^4 \log (\log d) \log q + md^3)$. 
\begin{table}
\begin{center}
\begin{tabular}{|l|l|}
\hline
Path & Cost \\
\hline
\hline
1,2,3 & $O(md^3 \log \delta^{-1} )$ \\
1,2,3,4,10 & $O(md^3 \log \delta^{-1} )$ \\
1,2,3,4,5,6 & $O(md^3 \log \delta^{-1}  +d^4\log q
              + sd^3\log \delta^{-1})$ \\
1,2,3,4,5,7 & $O(md^3 \log \delta^{-1} +d^4\log q + sd^3 \log \delta^{-1})$ \\
1,2,3,4,5,8 & $O(md^3 \log \delta^{-1} +d^4\log q
              + sd^3\log(\delta^{-1} \log d))$ \\
1,2,3,4,5,8,9 & $O(md^3 \log \delta^{-1}  +d^4\log q
              + sd^3\log(\delta^{-1} \log d))$ \\
\hline
\end{tabular}
\\[3mm]
where $s = O(\log \delta^{-1} + d\log q)$.
\end{center}
\caption{Complexity of algorithm for different cases}
\label{casescomplexity}
\end{table}

\section{Implementation and performance}
\label{implcomplexity}

All of our algorithms have been implemented in the
forthcoming \textsf{GAP} package \textsf{recog} for constructive group 
\index{recog package@\textsf{recog} package}%
recognition. In general 
we make $\lfloor (d\log q)/20 \rfloor$
random elements when producing generators for $N$ (see
Section~\ref{subsec:derived}), but always at least $5$ and 
at most $40$, which seems
to work well in practice. The division by $20$ 
indicates that our analysis of the generation of a
sufficiently large subgroup of $N$ underestimates the probability of success in
most cases.

In Table~\ref{subsemi:timings} we give timing results. 
All experiments have been done on a machine with an Intel Core2 Quad 
CPU Q6600 running at 2.40GHz with 8~GB of main memory using the
development version of \textsf{GAP} and the \textsf{recog} package. All
times are in milliseconds and were
\index{recog package@\textsf{recog} package}%
averaged over several runs. Note that due to randomisation the runtimes
can vary significantly between runs.


\begin{table}
\begin{center}
    \begin{tabular}{|r|l|r|l|r|c|r|l|r|r|}
\hline
No & Group      &    $d$&  $q$     & $m$& Case     & $d'$& $q'$ & Time & Total\\
\hline
1  & $M_{11}$   &  $10$ & $3^5$    &  2 & Subfield & $10$& $3$  &    9 &    70\\
2  & $S.M_{11}$ &  $10$ & $3^5$    &  2 & \CC5     & $10$& $3$  &   25 &   110\\
3  & $J_2$      &  $13$ & $3^{10}$ &  2 & Subfield & $13$& $3^2$&   25 &  2791\\
4  & $S.J_2$    &  $13$ & $3^{10}$ &  2 & \CC5     & $13$& $3^2$&  184 &  1687\\
5  & $Co_3$     &  $22$ & $2^{16}$ &  2 & Subfield & $22$& $2$  &  103 &  3509\\
6  & $S.Co_3$   &  $22$ & $2^{16}$ &  2 & \CC5     & $22$& $2$  &  788 &  4173\\
7  & $2.\mathrm{A}_8$    &  $24$ & $5^6$    &  2 & Subfield & $24$& $5^2$&  119 &  1147\\
8  & $S.(2.\mathrm{A}_8)$&  $24$ & $5^6$    &  2 & \CC5     & $24$& $5^2$&  954 &  1711\\
9  & $\GL_{90}(7)$& $90$ & $7^5$    &  2 & Subfield & $90$& $7$  & 8849 &   ---\\
10 & $S.\GL_{90}(7)$& $90$ & $7^5$  &  2 & \CC5     & $90$& $7$  &51133 &   ---\\
11 & $J_2$      &  $28$ & $2$      &  2 & NotAbsIrr& $14$& $2^2$&   52 &  1255\\
12 & $J_2.2$    &  $28$ & $2$      &  2 & \CC3     & $14$& $2^2$&   27 &   888\\
13 & $M_{22}$   &  $90$ & $3$      &  2 & NotAbsIrr& $45$& $3^2$&  622 & 10262\\
14 & $(S.M_{24}).A$&$69$& $5$      &  2 & \CC3     & $23$& $5^3$&  594 &  3096\\
15 & $3.3^9$    &  $81$ &$19$      &256 & Scalar   & $81$&$19$  & 5140 & 50085\\
16 & $\mathrm{S}_{14} \wr C_5$&$320$&$2$     & 3 & Components&$320$&  $2$& 2117 &106100\\
17 & $3^{1+4} \otimes HS$&$189$&25 &  8 & Tensor   & $9$  & $25$&24053 &150226\\
\hline
    \end{tabular}
\end{center}
\caption{Timing results for a few example groups}
\label{subsemi:timings}
\end{table}

Columns ``$d$'' and ``$q$'' give the matrix dimension
and field size of the input matrix group. Column
 ``$m$'' states the number of generators.
Column ``Group'' contains a structural description of the input
group. The notation ``$S.G$''  indicates
that the input group is a central extension of a group $G$ by all scalars 
of $\F_q$. The notation ``$G.A$'' 
indicates that the input group is a group $G$ extended by
 a group $A$ of field automorphisms on a centralising matrix.


Column ``Case''
describes the type of reduction found. Here,
``Subfield'' means that an immediate base change to write the group
over a smaller field was found. ``NotAbsIrr'' means that the input
group did not act absolutely irreducibly and was
written over a larger field with smaller dimension.
``\CC5'' means that a subgroup $H < N$ was computed which acted
irreducibly and then a base change was found to write the (projective)
group over a smaller field. ``\CC3'' means that a subgroup $H < N$ was
computed which acted irreducibly (but not absolutely irreducibly) and
an action as field automorphisms was found. This automatically gives the information required to write the kernel in a smaller dimension over the appropriate extension field. 
``Components'' means that an imprimitive action 
 was found. ``Tensor'' means that a tensor
decomposition was found. ``Scalar'' means that the group was
reduced using commutators with a single non-central element.

Columns ``$d'$'' and ``$q'$'' contain the dimension and
field size after the reduction. The ``Time'' column
indicates the time needed for the first reduction. The  ``Total''
column contains the runtime to build a complete composition
tree for the given group. This value is occasionally omitted, which 
indicates that not enough leaf methods are implemented yet to
recognise this group  fully.

To produce the examples, we first changed the field by embedding
or blowing up and then conjugated by a random element in the general
linear group over the new field.

To make example  $14$ we took the Mathieu group $M_{24}$ represented
in $\GL_{23}(5)$, multiplied the generators by scalars in $\F_{5^3}$,
blew everything up into $\GL_{69}(5)$ and added a new generator
that acts as a field automorphism of $\F_{5^3}$ when conjugating the centralising matrix.
In example $14$ the algorithm then
writes the kernel as a subgroup of $\GL_{23}(5^3)$ and finally 
recognises the $\CC5$ case and goes back to $\GL_{23}(5)$.

Example  $15$ is an absolutely irreducible $3$-group in $\GL_{81}(19)$ such 
that all commutators are scalar matrices. After several reductions by the 
commutator action the kernel becomes reducible.

Example  $16$ is a wreath product of the symmetric group $\mathrm{S}_{14}$
with the cyclic group of order $5$. We started with an absolutely irreducible 
$64$-dimensional representation of $\mathrm{S}_{14}$ over $\F_2$ and made
 a $320$-dimensional absolutely irreducible representation for 
$\mathrm{S}_{14} \wr C_5$ over $\F_2$. Our algorithm computes the action on 
the five homogeneous components, then tells the kernel node that the 
group is reducible and in block form so that a MeatAxe 
call is not necessary. 


Example $17$ is a central product of an extraspecial $3$-group of
order $243$ in its irreducible representation of dimension $9$ over
$\F_{25}$ with the sporadic finite simple group $HS$ in an irreducible
representation of dimension $21$ over $\F_{25}$. The latter
representation came from one over $\F_5$ where the representing
matrices of the group generators were multiplied by elements from 
$\F_{25}$. The extraspecial factor vanishes
when computing a subgroup of the derived group, since it is 
one example of our characterisation in 
Section~\ref{sec:describe_scalars},
 exhibiting the tensor
decomposition. In subsequent nodes the extraspecial
factor is taken apart using commutators as in
Section~\ref{subsec:scalars}. %It seems that whenever our
%procedure runs into this tensor product case at least one of the tensor
%factors is one of the groups in Section~\ref{sec:describe_scalars}.

