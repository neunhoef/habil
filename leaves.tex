% this is a part of the habilitation thesis of Max Neunhoeffer

\chapter{Leaves of the composition tree}
\label{chap:leaves}

This chapter is about constructive recognition (see
Problem~\ref{ProbCR3}) of the leaves of a
composition tree (see Section~\ref{recapproach}. By the arguments in
Chapter~\ref{chap:findhom} this basically means that we have to be
able to solve Problem~\ref{ProbCR3} for the subgroups $G \le \GL(n,q)$
that lie in classes \DD8 and \DD9 (see Sections~\ref{descD8} and
\ref{descD9}. Because of the obvious homomorphism from $\GL(n,q) \to
\PGL(n,q)$ and our setup of the composition tree we basically have
to work with the projective version $\bar G \le \PGL(n,q)$.

We do not want to describe the details of the methods used here but
rather give an overview and refer the reader to the literature.
One reason for this is that the current state of the art does not seem
to be final, another is, that the author has not contributed to the
development of algorithms in this area up to know.

This chapter is structured as follows. We begin by describing methods
based on permutation groups, which we call ``direct methods for
constructive recognition'' in Section~\ref{solvedirect}. The next
Section~\ref{nonconstructive} then explains the concept of
``non-constructive recognition'' which basically means to determine
the name of a given group. This in turn is needed to apply more
specialised methods for constructive recognition which are explained
in Section~\ref{solveD8} for classical groups and
in Section~\ref{solveD9} for quasi-simple groups.

\section{Direct methods for constructive recognition}
\label{solvedirect}

For ``small groups'' one can use direct methods to solve
Problem~\ref{ProbCR3} for a group $G \le \GL(n,q)$. By

\section{Non-constructive recognition}
\label{nonconstructive}

\section{The classical case: \DD8}
\label{solveD8}

\section{The almost simple case: \DD9}
\label{solveD9}

