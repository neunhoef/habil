% this is a part of the habilitation thesis of Max Neunhoeffer

\chapter{Leaves of the composition tree}
\label{chap:leaves}

This chapter is about constructive recognition (see
Problem~\ref{ProbCR3}) of the leaves of a
composition tree (see Section~\ref{recapproach}. By the arguments in
Chapter~\ref{chap:findhom} this basically means that we have to be
able to solve Problem~\ref{ProbCR3} for the subgroups $G \le \GL(n,q)$
that lie in classes \DD8 and \DD9 (see Sections~\ref{descD8} and
\ref{descD9}). Because of the obvious homomorphism from $\GL(n,q) \to
\PGL(n,q)$ and our setup of the composition tree we basically have
to work with the projective version $\bar G \le \PGL(n,q)$.

We do not want to describe the details of the methods used here but
rather give an overview and refer the reader to the literature.
One reason for this is that the current state of the art does not seem
to be final, another is that the author has not contributed to the
development of algorithms in this area up to know.

This chapter is structured as follows. We begin by describing
methods based on permutation groups, which we call ``direct methods
for constructive recognition'' in Section~\ref{solvedirect}. The
next Section~\ref{nonconstructive} then explains the concept of
``non-constructive recognition'' which basically means to determine
the isomorphism type of a given group. Once this is known the
concept of standard generators applies, which is introduced in
Section~\ref{standardgens}. These concepts in turn are needed to apply
more specialised methods for constructive recognition over which an
overview is given in Section~\ref{solveD8} for classical groups in
their natural representation and in
Section~\ref{solveD9} for quasi-simple groups.

\section{Direct methods for constructive recognition}
\label{solvedirect}

For ``small groups'' one can use permutation group methods to solve
Problem~\ref{ProbCR3} for a group $G \le \GL(n,q)$. The basic idea is
to find a permutation action. This immediately gives a homomorphism
into a permutation group which will be an isomorphism if the group is
simple. Even if there is a non-trivial kernel the composition tree setup
will take care of this.

Matrix groups and projective groups of course act on their natural
module and in there on vectors and subspaces. So finding any other
action is not difficult. To get good performance however can be
tricky. To this end, we want to find short orbits. Heuristic methods
for this for matrix groups can be found in \cite{shortorbits}.

Another possibility are low index methods.
Here one would guess the point stabiliser of a point with a short
orbit, restrict the natural module to it and use the MeatAxe to find a
proper invariant subspace. The orbit of this subspace would then be
relatively short. However, although this approach looks promising and
occasionally works, it is not clear how to find generators of such a
point stabiliser, even with random methods.

Once we have an action and thus a homomorphism into a permutation 
group we can either use the methods described in Section~\ref{permgrps}
or immediately compute a stabiliser chain for the projective or matrix
group using the Schreier-Sims method (see \cite{nearlylin} or
\cite{Ser}).

\section{Non-constructive recognition}
\label{nonconstructive}

The term ``non-constructive recognition'' for a group means finding
the isomorphism type. In situations where direct methods for constructive 
recognition (see the previous Section~\ref{solvedirect}) fail, the usual
approach is to first determine the isomorphism type of the group and
then use additional knowledge about the group in question to do the
constructive part of the recognition. In particular to use the
technique of standard generators (see the next
Section~\ref{standardgens}), the isomorphism type has to be known in
advance.

For the non-constructive recognition problem we can for example use
element order statistics. We produce uniformly distributed elements in
the group and compute their orders. If we see an element order that
does not occur in a group of a certain isomorphism type, we can
immediately rule out this type. If we fail to see an element order that
occurs very frequently in a certain isomorphism type after some tries,
we can rule out this type with a known error probability.

See Sections~\ref{solveD8} and \ref{solveD9} for an overview over the
current state of art of non-constructive recognition for classical
groups in their natural representation and for groups in class \DD9
respectively.

\section{Standard generators}
\label{standardgens}

In this section we give a definition of the term
``standard generators''. As in this whole chapter we do not want to go
into detail too much but rather give the reader an idea of the
concept.

Once the isomorphism type of a group $G = \left< g_1, \ldots,
g_k\right>$ is known (after successful non-con\-struc\-tive recognition as in
Section~\ref{nonconstructive}), one wants to find an explicit
isomorphism of $G$ to a ``standard copy $\hat G$''. Note that such an
automorphism is not automatically ``computable'' in the sense that 
we can map group elements back and forth, as we will see below.
However, in the end we strive to use previously acquired and stored
knowledge about $\hat G$ and transfer it over to $G$ via that
isomorphism to eventually solve the constructive recognition problem (see
\ref{ProbCR3}) for $G$.

The concept of standard generators serves this purpose.

\begin{Def}[Standard generators]
    Let $G$ be a finite group and $\Aut(G)$ its automorphism group. Then
    $\Aut(G)$ acts on tuples of elements of $G$ componentwise. We choose
    one orbit of this action that contains tuples, whose entries
    generate $G$ as a group, and call exactly those tuples in this
    orbit \emph{standard generators} for $G$. This choice is done once
    and forever for every isomorphism type of finite group and the
    chosen orbit is described by giving a set of properties of the
    tuples that uniquely determines the orbit.
\end{Def}

\begin{Rem}
Note that indeed this choice has to be done individually for every
isomorphism type of finite groups and the properties have to be
determined intelligently such that finding a tuple of standard
generators is possible efficiently (see Section~\ref{goodstandgens}).
\end{Rem}

This rather vague definition can best be filled with life by an
example:

\begin{Exa}[Standard generators for the sporadic simple Mathieu group
    $M_{11}$]
    \label{ExaM11}
    This description is taken from \cite[$M_{11}$ page]{WWWAtlas} and
    is derived in \cite[Example~11]{standgens}.

    Standard generators of $M_{11}$ are $(a,b)$ where $a$ has order $2$, 
$b$ has order $4$, $ab$ has order $11$ and $ababababbababbabb$ has 
order $4$. Note that it is a theorem that these properties uniquely
determine an orbit of $\Aut(M_{11})$ on pairs of elements of $M_{11}$.

\smallskip
To find standard generators for $M_{11}$:
\begin{enumerate}
        \setlength{\parskip}{0pt}
    \item Find an element of order $4$ or $8$. This powers up to $x$ of order 
        $2$ and $y$ of order $4$.

      [The probability of success at each attempt is $3$ in $8$.]
\item Find a conjugate $a$ of $x$ and a conjugate $b$ of $y$ such that $ab$ 
    has order $11$.

      [The probability of success at each attempt is $16$ in $165$.]
  \item If $ababbabbb$ has order $3$, then replace $b$ by its inverse.
  \item Now $ababbabbb$ has order $5$, and standard generators of $M_{11}$ 
      have been obtained.
\end{enumerate}
It is a theorem that this procedure produces standard generators, the
probabilities can be read off the character table of $M_{11}$ and assume
that we produce uniformly distributed random elements.
Note that choosing random conjugates of $x$ that are uniformly distributed in the
conjugacy class of $x$ can be achieved by conjugating $x$ with a random element
that is uniformly distributed in the group.
\end{Exa}

\begin{Prop}[The virtue of standard generators]
If $(s_1, \ldots, s_m) \in G^m$ and $(t_1, \ldots, t_m) \in G^m$ are 
both standard generators for a group $G$, then the equations
\[ \varphi (s_i) = t_i \qquad\mbox{for all } 1 \le i \le m \]
uniquely define an automorphism $\varphi$ of $G$.

That is, if we have a tuple of standard generators $(s_1, \ldots, s_m)$ for $G$
and $\hat G$ is a standard isomorphic copy of $G$ for which we know
a tuple of standard generators $(u_1, \ldots, u_m)$, then the
equations
\[ \psi (s_i) = u_i \qquad\mbox{for all } 1 \le i \le m \]
uniquely define an explicit isomorphism $\psi$ from $G$ to $\hat G$.
\end{Prop}
\proofbeg
Since by definition both tuples lie in the same orbit under $\Aut(G)$
and both tuples generate $G$, there is exactly one automorphism
$\varphi \in \Aut(G)$ mapping $s_i$ to $t_i$ for all $1 \le i \le m$.
The hypothesis that $\hat G$ is isomorphic to $G$ takes care of the
second statement.
\proofend

\begin{Rem}[The problem of mapping elements]
Note that even if we have found standard generators in $G$, the above
definition of the explicit isomorphism $\psi$ to $\smash{\hat G}$ does in fact 
\emph{not}
enable us to map arbitrary elements of $G$ via $\psi$, because for
this we would have to express an arbitrary element of $G$ as a
straight line program in $(s_1, \ldots, s_m)$, which is exactly the
constructive recognition problem we want to solve!

However, if we want to store certain elements or subgroups of $\hat
G$ beforehand, we can store them as straight line programs in $(u_1,
\ldots, u_m)$ and can then evaluate these straight line programs
in $(s_1, \ldots, s_m)$ to actually get their images under $\psi^{-1}$
in $G$. This fact helps to transfer previously acquired knowledge from
$\hat G$ to $G$.
\end{Rem}

\begin{Rem}[Good standard generators]
    \label{goodstandgens}

We want to comment only briefly on this topic. Basically, the choice
of the standard generators for an isomorphism type of group, that is 
the choice of the $\Aut(G)$-orbit 
in the tuples of elements of $G$, is ``good'', if it is relatively
easy to find a tuple of standard generators by random methods. The
example in Section~\ref{ExaM11} exhibits this. The probabilities to find
the right elements in the algorithm presented there are quite good,
such that very few random elements will usually lead to success.

A large collection of such good choices of standard generators together 
with algorithms to find them can be found on the WWW-Atlas of Group
Representations, see \cite{WWWAtlas}.
\end{Rem}

\begin{App}[Storing hints for stabiliser chains]
One immediate application of standard generators is the following. If
a group has a subgroup $U$ with a relatively low index, then we can store
generators for this subgroup as a straight line program in standard
generators. Once we have recognised the isomorphism type of $G$
using non-constructive recognition and have found standard generators
in $G$, we can evaluate this straight line program, get a
generating set of a subgroup $U$ of $G$ with this low index, restrict
the natural module to $U$ and find a proper invariant subspace. The
$G$-orbit of this subspace in the natural action on subspaces then
gives a permutation action of $G$ which is isomorphic to the one on
cosets of $U$. Thus, using standard generators in this way, we can collect hints
to find ``good'' actions for the different absolutely irreducible
matrix representations of a group.

Eamonn O'Brien and Robert Wilson have for example done exactly this
for the sporadic simple groups. Their hints data is available in the
{\MAGMA} system and will be used in the composition tree
implementation in the {\GAP} system as well.
\end{App}

\section{The classical case in natural representation: \DD8}
\label{solveD8}

The seminal paper by Neumann and Praeger \cite{neumann-praeger} which
presents an algorithm to decide whether a given group $G \le \GL(n,q)$
contains the special linear group was the starting point of a whole
industry of papers concerned with non-constructive and constructive
recognition of groups.

An algorithm to recognise classical
groups in their natural representation non-constructively is given 
in \cite{classicalnonconstructive}. Once this is done, the results in
\cite{peteconstructiveclassical} give algorithms to solve the
constructive recognition problem.
The basic idea of these constructive recognition algorithms is to find
a certain tuple of standard generators and then perform a base change
such that linear algebra methods can be used to express arbitrary
elements as straight line programs in the standard generators.

ADD REFERENCE to recent preprint Leedham-Green O'Brien.


\section{The almost simple case: \DD9}
\label{solveD9}

If a group $G \le \GL(n,q)$ with $Z := G \cap Z(\GL(n,q))$ 
is contained in class \DD9, there is a
non-abelian simple group $\bar N$ and a group $T$ with $\bar N
\leq T \leq \Aut(\bar N)$ and $G/Z \cong T$. The first problem for
non-constructive recognition is that $G/Z$ is itself not necessarily 
simple. However, the Schreier conjecture, which follows from the
classification of finite simple groups, says that $\Aut(\bar N) / \bar
N$ is solvable for all finite simple groups $\bar N$. In fact, this
outer automorphism group is rather small in most cases occurring in
practice.

Explain about going to simple group

References:


