% this is a part of the habilitation thesis of Max Neunhoeffer

\chapter{Leaves of the composition tree}
\label{chap:leaves}

This chapter is about constructive recognition (see
Problem~\ref{ProbCR3}) of the leaves of a
composition tree (see Section~\ref{recapproach}. By the arguments in
Chapter~\ref{chap:findhom} this basically means that we have to be
able to solve Problem~\ref{ProbCR3} for the subgroups $G \le \GL(n,q)$
that lie in classes \DD8 and \DD9 (see Sections~\ref{descD8} and
\ref{descD9}). Because of the obvious homomorphism from $\GL(n,q) \to
\PGL(n,q)$ and our setup of the composition tree we basically have
to work with the projective version $\bar G \le \PGL(n,q)$.

We do not want to describe the details of the methods used here but
rather give an overview and refer the reader to the literature.
One reason for this is that the current state of the art does not seem
to be final, another is that the author has not contributed to the
development of algorithms in this area up to know.

This chapter is structured as follows. We begin by describing
methods based on permutation groups, which we call ``direct methods
for constructive recognition'' in Section~\ref{solvedirect}. The
next Section~\ref{nonconstructive} then explains the concept of
``non-constructive recognition'' which basically means to determine
the isomorphism type of a given group. Once this is known the
concept of standard generators applies, which is introduced in
Section~\ref{standardgens}. These concetps in turn are needed to apply
more specialised methods for constructive recognition over which an
overview is given in Section~\ref{solveD8} for classical groups and in
Section~\ref{solveD9} for quasi-simple groups.

\section{Direct methods for constructive recognition}
\label{solvedirect}

For ``small groups'' one can use permutation group methods to solve
Problem~\ref{ProbCR3} for a group $G \le \GL(n,q)$. The basic idea is
to find a permutation action. This immediately gives a homomorphism
into a permutation group which will be an isomorphism if the group is
simple. Even if there is a non-trivial kernel the composition tree setup
will take care of this.

Matrix groups and projective groups of course act on their natural
module and in there on vectors and subspaces. So finding any other
action is not difficult. To get good performance however can be
tricky. To this end, we want to find short orbits. Heuristic methods
for this for matrix groups can be found in \cite{shortorbits}.

Another possibility are low index methods.
Here one would guess the point stabiliser of a point with a short
orbit, restrict the natural module to it and use the MeatAxe to find a
proper invariant subspace. The orbit of this subspace would then be
relatively short. However, although this approach looks promising and
occasionally works, it is not clear how to find generators of such a
point stabiliser, even with random methods.

Once we have an action and thus a homomorphism into a permutation 
group we can either use the methods described in Section~\ref{permgrps}
or immediately compute a stabiliser chain for the projective or matrix
group using the Schreier-Sims method (see \cite{nearlylin} or
\cite{Ser}).

\section{Non-constructive recognition}
\label{nonconstructive}

\section{Standard generators}
\label{standardgens}

In this section we give a rather vague definition of the term
``standard generators''. As in this whole chapter we do not want to go
into detail too much but rather give the reader an idea of the
concept.

Once the isomorphism type of a group $G = \left< g_1, \ldots,
g_k\right>$ is known (after successful non-con\-struc\-tive recognition as in
Section~\ref{nonconstructive}), one wants to find an explicit
isomorphism of $G$ to a ``standard copy $\hat G$''. With
``computable'' we mean, that we want to be able to map group elements
back and forth, such that for example a previously determined and
stored solution to the constructive recognition problem (see
\ref{ProbCR3}) in $\hat G$ can be transferred to a solution for $G$.
%FIXME

The concept of standard generators serves exactly this purpose.

\begin{Def}[Standard generators]
    Let $G$ be a finite group and $\Aut(G)$ its automorphism group. Then
    $\Aut(G)$ acts on tuples of elements of $G$ componentwise. We choose
    one orbit of this action that contains tuples, whose entries
    generate $G$ as a group, and call exactly those tuples in this
    orbit \emph{standard generators} for $G$. This choice is done once
    and forever for every isomorphism type of finite group and the
    chosen orbit is described by giving a set of properties of the
    tuples that uniquely determines the orbit.
\end{Def}

\begin{Rem}
Note that indeed this choice has to be done individually for every
isomorphism type of finite groups and the properties have to be
determined intelligently such that finding a tuple of standard
generators is possible efficiently (see Section~\ref{goodstandgens}).
\end{Rem}

This rather vague definition can best be filled with life by an
example:

\begin{Exa}[Standard generators for the sporadic simple Mathieu group
    $M_{11}$]
    This description is taken from \cite[$M_{11}$ page]{WWWAtlas} and
    is derived in \cite[Example~11]{standgens}.

    Standard generators of $M_{11}$ are $(a,b)$ where $a$ has order $2$, 
$b$ has order $4$, $ab$ has order $11$ and $ababababbababbabb$ has 
order $4$. Note that it is a theorem that these properties uniquely
determine an orbit of $\Aut(M_{11})$ on pairs of elements of $M_{11}$.

\smallskip
To find standard generators for $M_{11}$:
\begin{enumerate}
    \item Find an element of order $4$ or $8$. This powers up to $x$ of order 
        $2$ and $y$ of order $4$.

      [The probability of success at each attempt is $3$ in $8$.]
\item Find a conjugate $a$ of $x$ and a conjugate $b$ of $y$ such that $ab$ 
    has order $11$.

      [The probability of success at each attempt is $16$ in $165$.]
  \item If $ababbabbb$ has order $3$, then replace $b$ by its inverse.
  \item Now $ababbabbb$ has order $5$, and standard generators of $M_{11}$ 
      have been obtained.
\end{enumerate}
Note that choosing random conjugates of $x$ that are uniformly distributed in the
conjugacy class of $x$ can be achieved by conjugating $x$ with a random element
that is uniformly distributed in the group.
\end{Exa}

\begin{Prop}[The virtue of standard generators]
If $(s_1, \ldots, s_m) \in G^m$ and $(t_1, \ldots, t_m) \in G^m$ are 
both standard generators for a group $G$, then the equations
\[ \varphi (s_i) = t_i \qquad\mbox{for all } 1 \le i \le m \]
uniquely define an automorphism $\varphi$ of $G$.

That is, if $\hat G$ is a standard isomorphic copy of $G$ for which we know
a tuple of standard generators $(u_1, \ldots, u_m)$ and have 
\end{Prop}
\proofbeg
Since by definition both tuples lie in the same orbit under $\Aut(G)$
and both tuples generate $G$, there is exactly one automorphism
$\varphi \in \Aut(G)$ mapping $s_i$ to $t_i$ for all $1 \le i \le m$.
\proofend

\begin{Rem}[Good standard generators]
    \label{goodstandgens}
\end{Rem}

\section{The classical case: \DD8}
\label{solveD8}

\section{The almost simple case: \DD9}
\label{solveD9}

