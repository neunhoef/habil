% this is a part of the habilitation thesis of Max Neunhoeffer

\chapter{Leaves of the composition tree}
\label{chap:leaves}
\index{leaf}%

This chapter is about constructive recognition (see
Problem~\ref{ProbCR3}) of the leaves of a
composition tree (see Section~\ref{recapproach}). By the arguments in
Chapter~\ref{chap:findhom} this means that we have to be
able to solve Problem~\ref{ProbCR3} for the subgroups $G \le \GL(n,q)$
that lie in classes \DD8 and \DD9 (see Sections~\ref{descD8} and
\ref{descD9}). Because of the homomorphism from $\GL(n,q) \to
\PGL(n,q)$ and our setup of the composition tree we only have
to work with the projective version $\bar G \le \PGL(n,q)$.

We do not want to describe the details of the methods used here but
instead give an overview and refer the reader to the literature.
One reason for this is that the current state of the art does not seem
to be final, another is that the author has not contributed to the
development of algorithms in this area up to now.

This chapter is structured as follows. We begin by describing
methods based on permutation groups, which we call ``direct methods
for constructive recognition'' in Section~\ref{solvedirect}. The
next Section~\ref{nonconstructive} explains the concept of
``non-constructive recognition'' which means to determine
the isomorphism type of a given group. Once this is known the
concept of standard generators applies, which is introduced in
Section~\ref{standardgens}. These concepts in turn are usually needed to apply
more specialised methods for constructive recognition, of which an
overview is given in Section~\ref{solveD8} for classical groups in
their natural representation and in
Section~\ref{solveD9} for almost simple groups.

\section{Direct methods for constructive recognition}
\label{solvedirect}

For ``small groups'' one can use permutation group methods to solve
Problem~\ref{ProbCR3} for a group $\bar G \le \PGL(n,q)$. The basic idea is
to find a permutation action. This immediately gives a homomorphism
into a permutation group which will be an isomorphism if the group is
simple. Even if there is a non-trivial kernel the composition tree setup
will take care of this.

Matrix groups and projective groups of course act on their natural
module and thus on vectors and subspaces. So finding some
action is not difficult. Achieving good performance however can be
tricky. To this end we want to find short orbits. Heuristic methods
for this for matrix groups can be found in \cite{shortorbits}.

Another possibility is low index methods.
Here one would guess the point stabiliser of a point with a short
orbit, restrict the natural module to it and use the MeatAxe to find a
proper invariant subspace. The orbit of this subspace would then be
relatively short. However, although this approach looks promising and
occasionally works, it is not clear how to find generators of such a
point stabiliser, even with random methods.

Once we have an action and thus a homomorphism into a permutation 
group we can either use the methods described in Section~\ref{permgrps}
or immediately compute a stabiliser chain for the projective or matrix
group using the Schreier-Sims method (see \cite{nearlylin} or
\cite{Ser}). Both approaches solve the constructive recognition
problem \ref{ProbCR3} even if the group is not a simple group.

Note that for larger dimensions and in particular for classical groups
we do not even want to try these direct methods because any orbit we
can possible find would be prohibitively large.


\section{Non-constructive recognition}
\label{nonconstructive}
\index{non-constructive recognition}%

The term ``non-constructive recognition'' for a group means finding
the isomorphism type. In situations where direct methods for constructive 
recognition (see the previous Section~\ref{solvedirect}) fail, the usual
approach is to determine the isomorphism type of the group first and
then use additional knowledge about the group in question to do the
constructive part of the recognition. In particular to use the
technique of standard generators (see the next
Section~\ref{standardgens}), the isomorphism type has to be known in
advance.

For the non-constructive recognition problem we can for example use
element order statistics. We produce uniformly distributed elements in
the group and compute their orders. If we see an element order that
does not occur in a group of a certain isomorphism type, we can
immediately rule out this type. If we fail to see an element order that
occurs very frequently in a certain isomorphism type after some tries,
we can rule out this type with a known error probability.

See Sections~\ref{solveD8} and \ref{solveD9} for an overview of the
methods for non-constructive recognition for classical
groups in their natural representation (class \DD8) and for groups in class \DD9
respectively.

\section{Standard generators}
\label{standardgens}
\index{standard generators}%

In this section we give a definition of the term
``standard generators''. As in this whole chapter we do not want to go
into too much detail but instead give the reader an idea of the
concept.

Once the isomorphism type of a group $G = \left< g_1, \ldots,
g_k\right>$ is known (after successful non-con\-struc\-tive recognition as in
Section~\ref{nonconstructive}), one wants to find an explicit
isomorphism of $G$ to a ``standard copy $\hat G$''. Note that such an
isomorphism is not automatically ``computable'' in the sense that 
we can map group elements back and forth, as we will see below.
However, in the end we strive to use previously acquired and stored
knowledge about $\hat G$ and transfer it over to $G$ via that
isomorphism to eventually solve the constructive recognition problem (see
\ref{ProbCR3}) for $G$.

The concept of standard generators serves this purpose.

\begin{Def}[Standard generators]
\index{standard generators}%
    Let $G$ be a finite group and $\Aut(G)$ its automorphism group. Then
    $\Aut(G)$ acts on tuples of elements of $G$ componentwise. We choose
    one orbit of this action that contains tuples whose entries
    generate $G$ as a group, and call exactly those tuples in this
    orbit \emph{standard generators} for $G$. This choice is done once
    and forever for every isomorphism type of finite group and the
    chosen orbit is described by giving a set of properties of the
    tuples that uniquely determines the orbit.
\end{Def}

\begin{Rem}
Note that this choice has to be done individually for every
isomorphism type of finite groups and the properties have to be
determined intelligently, such that finding a tuple of standard
generators is possible efficiently (see Section~\ref{goodstandgens}).
\end{Rem}

This rather vague definition can best be filled with life by an
example:

\begin{Exa}[Standard generators for the sporadic simple Mathieu group
    $M_{11}$]
    \label{ExaM11}
    This description is taken from \cite[$M_{11}$ page]{WWWAtlas} and
\index{WWW-Atlas of group representations}%
    is derived in \cite[Example~11]{standgens}.

    Standard generators of $M_{11}$ are $(a,b)$ where $a$ has order $2$, 
$b$ has order $4$, $ab$ has order $11$ and $ababababbababbabb$ has 
order $4$. Note that it is a theorem that these properties uniquely
determine an orbit of $\Aut(M_{11})$ on pairs of elements of $M_{11}$.

\smallskip
To find standard generators for $M_{11}$:
\begin{enumerate}
        \setlength{\parskip}{0pt}
    \item Find an element of order $4$ or $8$. This powers up to $x$ of order 
        $2$ and $y$ of order $4$.

      [The probability of success at each attempt is $3$ in $8$.]
\item Find a conjugate $a$ of $x$ and a conjugate $b$ of $y$ such that $ab$ 
    has order $11$.

      [The probability of success at each attempt is $16$ in $165$.]
  \item If $ababbabbb$ has order $3$, then replace $b$ by its inverse.
  \item Now $ababbabbb$ has order $5$, and standard generators of $M_{11}$ 
      have been obtained.
\end{enumerate}
It is a theorem that this procedure produces standard generators and the
probabilities can be read off the character table of $M_{11}$. Of course 
we assume uniformly distributed random elements for these
probabilities to be correct.
Note that choosing random conjugates of $x$ that are uniformly distributed in the
conjugacy class of $x$ can be achieved by conjugating $x$ with a random element
that is uniformly distributed in the group.
\end{Exa}

\begin{Prop}[The virtue of standard generators]
\index{standard generators}%
If $(s_1, \ldots, s_m) \in G^m$ and $(t_1, \ldots, t_m) \in G^m$ are 
both standard generators for a group $G$, then the equations
\[ \varphi (s_i) = t_i \qquad\mbox{for all } 1 \le i \le m \]
uniquely define an automorphism $\varphi$ of $G$.

That is, if we have a tuple of standard generators $(s_1, \ldots, s_m)$ for $G$
and $\hat G$ is a standard isomorphic copy of $G$ for which we know
a tuple of standard generators $(u_1, \ldots, u_m)$, then the
equations
\[ \psi (s_i) = u_i \qquad\mbox{for all } 1 \le i \le m \]
uniquely define an explicit isomorphism $\psi$ from $G$ to $\hat G$.
\end{Prop}
\proofbeg
Since by definition both tuples lie in the same orbit under $\Aut(G)$
and both tuples generate $G$, there is exactly one automorphism
$\varphi \in \Aut(G)$ mapping $s_i$ to $t_i$ for all $1 \le i \le m$.
The hypothesis that $\hat G$ is isomorphic to $G$ takes care of the
second statement.
\proofend

\begin{Rem}[The problem of mapping elements]
Note that even if we have found standard generators in $G$, the above
definition of the explicit isomorphism $\psi$ to $\smash{\hat G}$ does in fact 
\emph{not}
enable us to map arbitrary elements of $G$ via $\psi$, because for
this we would have to express an arbitrary element of $G$ as a
straight line program in $(s_1, \ldots, s_m)$, which is exactly the
\index{straight line program}\index{SLP}%
constructive recognition problem we want to solve!

However, if we want to store certain elements or subgroups of $\hat
G$ beforehand, we can store them as straight line programs in $(u_1,
\ldots, u_m)$ and can then evaluate these straight line programs
in $(s_1, \ldots, s_m)$ to actually get their images under $\psi^{-1}$
in $G$. This fact helps to transfer previously acquired knowledge from
$\hat G$ to $G$.
\end{Rem}

\begin{Rem}[Good standard generators]
\label{goodstandgens}
\index{standard generators}%

We want to comment only briefly on this topic. Basically, the choice
of the standard generators for an isomorphism type of group, that is 
the choice of the $\Aut(G)$-orbit 
in the tuples of elements of $G$, is ``good'', if it is relatively
easy to find a tuple of standard generators by random methods. The
example in Section~\ref{ExaM11} exhibits this. The probabilities to find
the right elements in the algorithm presented there are quite good,
such that very few random elements will usually lead to success.

A large collection of such good choices of standard generators together 
with algorithms to find them can be found on the WWW-Atlas of group
\index{WWW-Atlas of group representations}%
representations, see \cite{WWWAtlas}.
\end{Rem}

\begin{App}[Storing hints for stabiliser chains]
\label{hintsstabchains}
\index{hints for stabiliser chains}%
One immediate application of standard generators is the following. If
a group has a subgroup $U$ with a relatively low index, then we can store
generators for this subgroup as a straight line program in standard
generators. Once we have recognised the isomorphism type of $G$
using non-constructive recognition and have found standard generators
in $G$, we can evaluate this straight line program, get a
generating set of a subgroup $U$ of $G$ with this low index, restrict
the natural module to $U$ and find a proper invariant subspace. Provided
that $G$ acts irreducibly on its natural module, the
$G$-orbit of this subspace in the natural action on subspaces then
gives a permutation action of $G$ which is isomorphic to the one on
cosets of $U$. Note that the existence of such a proper invariant subspace 
of course depends not only on the isomorphism type of $G$ but rather
also on the actual representation $G$ comes in.
Thus, using standard generators in this way, we can collect hints
to find ``good'' actions for the different absolutely irreducible
matrix representations of a group.

Eamonn O'Brien and Robert Wilson have for example done exactly this
for the sporadic simple groups. Their hints data is available in the
{\MAGMA} system (see \cite{Magma}) and will be used in the composition tree
implementation in the {\GAP} system as well.
\end{App}
\index{standard generators}%

\section{The classical case in natural representation: \DD8}
\label{solveD8}

The seminal paper by Neumann and Praeger \cite{neumann-praeger} which
presents an algorithm to decide whether a given group $G \le \GL(n,q)$
contains the special linear group was the starting point of a whole
industry of papers concerned with non-constructive and constructive
recognition of groups.

An algorithm to recognise classical groups in their
natural representation non-constructively is given in
\cite{classicalnonconstructive}. Once this is done, other algorithms
apply: In \cite{slrecogconstr} an algorithm to recognise $\SL(n,q)$
in its natural representation constructively is presented with
effective cost $O(n^4q)$. For arbitrary classical groups in their
natural representation the results in \cite{peteconstructiveclassical}
give algorithms to solve the constructive recognition problem with
effective cost $O(n^5 \log^2 q)$, however these algorithms need an
$\SL(2,q)$ oracle, that is, they rely on a solution of the constructive
recognition problem for $\SL(2,q)$. Recently a new preprint (see
\cite{recogclassicalodd}) has appeared which handles the case of odd
characteristic.

The basic idea of all these constructive recognition algorithms is to find
a tuple of standard generators and then perform a base change
such that linear algebra methods can be used to express arbitrary
elements as straight line programs in the standard generators.


\section{The almost simple case: \DD9}
\label{solveD9}
\index{almost simple}%

This part of the whole group recognition project is probably the one
for which the currently known methods are least satisfying. In particular
because many isomorphism types of groups have to be dealt with in a case by
case fashion a lot of work still needs to be done.
In this section we try to give an overview of the known methods by
means of references to the literature. We intentionally leave out
complexity results for the sake of brevity and because the last word on
these does not yet seem to be spoken. A more detailed account of
the state of the art can be found in \cite{OB}.

We begin with a discussion of the problem with the ``almost'' in
``almost simple''.

If a group $G \le \GL(n,q)$ with $Z := G \cap Z(\GL(n,q))$ 
is contained in class \DD9, there is a
non-abelian simple group $\bar N$ and a group $T$ with $\bar N
\leq T \leq \Aut(\bar N)$ and $G/Z \cong T$. The first problem for
both constructive and non-constructive recognition is that $G/Z$ is
itself not necessarily simple, after all, $G/Z$ is only ``almost simple''.
However, the Schreier conjecture, which follows from the classification
of finite simple groups, says that $\Aut(\bar N) / \bar N$ is solvable
for all finite simple groups~$\bar N$. 

In fact, this outer automorphism
group is rather small for groups occurring in practice. 
If $\bar N$ is alternating or sporadic, then
$|\Out(\bar N)| = 2$ except for $|\Out(A_6)| = 4$.
In \cite[Lemma 1.4]{LucchiniMorigi} it is shown
that for a simple group $\bar N$ of Lie type in cross-characteristic, 
$|\Out(\bar N)| \le \beta \log n$ for some global constant
$\beta$. If $\bar N$ is a simple group of Lie type in its defining
characteristic, then $|\Out(\bar N)| \le 2(n+1)\log q$ by
\cite[Proof of Lemma 1.3]{LucchiniMorigi}.

So the first step for recognition is to find the simple subgroup
isomorphic to $\bar N$ of $G/Z$, we denote this by $N/Z$ in the following. 
One approach is to go down the derived series
until a group is reached which is ``probably perfect''. To test for
the latter, one uses the algorithm by Leedham-Green and O'Brien in
\cite[5.3]{RecogTensInd}. This algorithm estimates the order of an element in a
factor group if the group and the normal subgroup are given.
This technique produces generators for $N$ efficiently.

Most publications in this area seem to concentrate on the
non-constructive and constructive recognition of simple groups.
However, to completely solve Problem~\ref{ProbCR3} for all groups in
\DD9 one has to be able to do constructive recognition for all
almost simple groups. There seems to be no method known to get hold of
the factor group $G/N$ in general since in most cases
the restriction of the natural module to $N$ is absolutely
irreducible.  We will ignore this problem here, in particular since $G/N$
is very small in most cases, such that solving the constructive recognition
problem for $N$ together with a few coset case distinctions can solve the
problem in $G$ satisfactorily.

Next we try to give an overview of the known techniques for
non-constructive recognition of simple groups. Alternating groups and
sporadic groups can be recognised non-constructively by looking at
element orders of random elements. For Lie type groups there is an
algorithm for non-constructive recognition in \cite{blackboxlienonconstr}.
One outstanding case in this paper is covered by methods in
\cite{altseimer}. However, both algorithms need to know the defining
characteristic of the Lie type group in advance. To determine this,
there are three concurrent methods, one is described in
\cite{primpowgraphs}, a newer one in \cite{findingcharlie} and thirdly there
is unpublished work by Seress which uses statistics about the two largest 
projective element orders occurring.

In 2001 Malle and O'Brien developed a practical implementation of all 
the algorithms for non-constructive recognition known at the time,
which is distributed with the {\MAGMA} system.

Assuming from now on that the non-constructive recognition is achieved we
conclude this section with a list of references to work that has been done
to solve the constructive recognition problem for classes of simple groups.
Note that many of these solutions include the corresponding almost simple
groups thereby solving specific cases of \DD9 groups completely.

An algorithm to recognise the alternating group $A_k$ and the symmetric
group $S_k$ constructively in an arbitrary 
representation is described in \cite{bbsymaltconstr}. An alternative
algorithm was developed in \cite{bratuspak}.

For classical groups in another than their natural representation (see
Section~\ref{solveD8} for the \DD8 case) there is an algorithm in
\cite{bbclassical}. The particularly important case of the special linear
group $\SL(2,q)$ of rank $2$ in non-natural but defining characteristic
representation is dealt with in \cite{classicallargefield}
and \cite{psl2qconstr} in the sense that an algorithm is described to find
an explicit computable epimorphism to $\PSL(2,q)$ in its natural
representation. Further work to improve the algorithms for simple classical 
groups in arbitrary representations can be found in \cite{bbomega},
\cite{bbunitary}, \cite{bbpsldq}, \cite{computingmatrix} and
\cite{bbortho}. Recently the preprint \cite{smalldegreegl} appeared which
deals with recognising small degree representations of general linear
groups specifically.

For Lie type groups work on constructive recognition has appeared in
\cite{recogSL3}, \cite{rybaid} and \cite{suzukiconstr}. An article about 
constructive recognition of exceptional groups of Lie type by Kantor and Magaard is
in preparation. However, this area is still work in progress.

For the sporadic simple groups the data collected by O'Brien and Wilson
about subgroup chains (see Section~\ref{hintsstabchains}) can be used to
solve the constructive recognition problem after non-con\-struc\-tive
recognition using the direct methods mentioned in
Section~\ref{solvedirect}. 

We finish this section by mentioning two other approaches to the
constructive part of the recognition problem.

One is published
in the preprint \cite{bbconstrmember}. It uses involution centralisers
and the fact that generators for them can be computed efficiently (see
\cite{BrayInv}). One instance of the constructive membership test problem
for a group $G$ is translated into three instances of the corresponding
problem in the centralisers of certain involutions which come up during the
run of the algorithm.

Finally we want to mention the paper \cite{gensift}, which introduces a
generic framework for constructive membership testing in a group whose
isomorphism type is known. It relies on standard generators (see
Section~\ref{standardgens}) and prepared subgroup chains stored as straight
line programs in the standard generators.
\index{straight line program}\index{SLP}%

% REFERENCE: Kantor, Seress: Black box classical groups???
