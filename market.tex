\documentclass[11pt]{article}

\usepackage[a4paper,top=1in,bottom=1in]{geometry}
\usepackage[mtpluscal]{mathtime}
\usepackage{times}

\begin{document}
\title{Target audience and market competitors}
\author{Max Neunh\"offer}
\maketitle

My book ``Constructive Recognition of Matrix Groups'' describes the
current status of the so called ``Matrix Group Recognition Project''.
This project is ongoing, started in the 1990s and in the meantime
some 30-50 mathematicians worldwide are more or less involved.
Its ultimate aim is that one can work with finite matrix groups
on a computer essentially as well as one can do so already now with
permutation groups. Thus, the results of this project in form of
implemented algorithms are interesting for a much larger audience.

The audience I had in mind when writing this book is graduate
students, researchers beginning to work in the area of
algorithms for matrix groups, as well as those already working
in the area. It both serves as an introduction to the field and
as a description of its current state. Obviously, it also contains
original contributions.

The book is suitable to base a graduate course on it. Such a course
would be a good preparation for a variety of students: Most naturally, 
those who go on working on algorithms for matrix groups get
a good introduction and are carefully lead to the frontier of research.
They are made aware of the state of the art and of the typical problems
for matrix group algorithms. They will also get a good understanding
of the complexity analysis of algorithms and its importance for the
development of efficient algorithms.
However, also graduate students who do not continue to work
on matrix groups profit greatly from such a course. They learn about 
the exciting area between theoretical methods from group and representation
theory and practical algorithms together with their analysis. The expertise
the students acquire from this is useful for whatever carrier they pursue
afterwards.

The second major audience group are established mathematicians who would
like to get an insight into the subject area, be it out of general interest
in new computational advances, or because they want to work on matrix
group algorithms themselves. This book will serve them well by providing
a relatively easily accessible path into the area with lots of references
into the original literature, should this be needed.

Last but not least I have to mention the experts in the area of
matrix group algorithms.
This book is also written as an account of the current
state of the art for the members of the community.
I expect it to be useful as a kind of milestone, to look back at what has
been achieved in the past 20 years, as well as to look forward and to
discuss what still has to be done.

Furthermore, quite a few of the parts of this book are interesting
independently. For example, Chapter~II describes the efficient implementation 
of matrices over finite fields on a computer, Chapter~III describes
the computation of characteristic and minimal polynomials which
are important in many other situations than matrix group recognition, 
and the arguments in Chapter~VI provide a hands-on walk through the
structure of subgroups of finite classical groups.

I am not aware of any other published work providing such a complete
overview of the subject area of matrix group algorithms. 
Of course, there are the three survey
articles \cite{CLG} by Charles Leedham-Green and \cite{EOB} and
\cite{EOBBath}
by Eamonn O'Brien but the
first two are 10 respectively 5 years old and all three do not aim 
at the same level of completeness as my work.

On the opposite end of the specialisation spectrum is the Handbook of
Computational Group Theory \cite{handbook} by Derek Holt et.~al.
This book covers all of computational group theory with the notable 
exception of algorithms for matrix groups. It only has 8 pages on
methods using base and strong generating sets and a brief description
of the Aschbacher approach. Therefore, the Handbook
is not a competitor for my book, rather, the two complement each other
nicely.

Another important book in the area is \cite{akosperm}, which is a
standard reference for algorithms for permutation groups. Its content 
is disjoint from my book in that it covers permutation groups. I could
very well imagine that in 15 or 20 years time a book similar to
\cite{akosperm} could be written about algorithms for matrix groups,
however, at this stage quite a lot of research still needs to be done.
Therefore, my book cannot achieve the completeness of \cite{akosperm}
for matrix groups,
rather, as mentioned above, its aim is to describe the current state
of the art.

All in all I expect that this monograph is interesting for a
variety of people and I imagine that quite a few mathematical
libraries will order a copy. I would certainly recommend the book
to every PhD-student starting to work on a thesis in the subject area.

\begin{thebibliography}{[1]}
\bibitem{CLG}
 Charles R.~Leedham-Green.  \newblock
 The computational matrix group project. \newblock
 Groups and computation, III (Columbus, OH, 1999), 
 229--247, Ohio State Univ. Math. Res. Inst. Publ., 8, de Gruyter,
Berlin,  2001. 
                
\bibitem{EOB}
 Eamonn A.~O'Brien.  \newblock
 Towards effective algorithms for linear groups. \newblock
 Finite geometries, groups, and computation, 
 163--190, Walter de Gruyter, Berlin,  2006. 

\bibitem{EOBBath}
 Eamonn A.~O'Brien.  \newblock
 Algorithms for matrix groups. \newblock
 Groups St Andrews, (Bath), to appear 2011.

\bibitem{handbook}
 Derek F.~Holt, Bettina Eick, Eamonn A.~O'Brien. \newblock
 Handbook of Computational Group Theory. \newblock
 Chapman and Hall/CRC, 2005.

\bibitem{akosperm}
 {\'A}kos Seress.  \newblock
 Permutation group algorithms. \newblock
 Cambridge Tracts in Mathematics, \textbf{152}, Cambridge University
 Press, 2003.

\end{thebibliography}

\end{document}
