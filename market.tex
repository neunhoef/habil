\documentclass[11pt]{article}

\usepackage[a4paper,top=1in,bottom=1in]{geometry}
\usepackage[mtpluscal]{mathtime}
\usepackage{times}

\begin{document}
\title{Target audience and market competitors}
\author{Max Neunh\"offer}
\maketitle

The book ``Constructive Recognition of Matrix Groups'' describes the
current status of the so called ``Matrix Group Recognition Project''.
This project is ongoing, started in the 1990s and in the meantime
some 30-50 mathematicians worldwide are more or less involved.
Its ultimate aim is that one can work with finite matrix groups
on a computer essentially as well as one can do so already now with
permutation groups. Thus, the results of this project in form of
implemented algorithms are interesting for a much larger audience.

The audience I had in mind when writing this book is graduate
students, researchers beginning to work in the area of
algorithms for matrix groups, as well as those already working
in the area. It both serves as an introduction to the field and
as a description of its current state. Obviously, it also contains
original contributions.

I am not aware of any other published work providing such a complete
overview of the subject area. Of course, there are the two survey
articles \cite{CLG} by Charles Leedham-Green and \cite{EOB} 
by Eamonn O'Brien but they
are 10 respectively 5 years old and do not aim at the same level of
completeness.

Therefore, I expect that a monograph would be interesting for a
variety of people and I imagine that quite a few mathematical
libraries would order a copy. I would certainly recommend the book
to every PhD-student starting to work on a thesis in the subject area.

\begin{thebibliography}{[1]}
\bibitem{CLG}
 Leedham-Green, Charles R.  \newblock
 The computational matrix group project. \newblock
 Groups and computation, III (Columbus, OH, 1999), 
 229--247, Ohio State Univ. Math. Res. Inst. Publ., 8, de Gruyter,
Berlin,  2001. 
                
\bibitem{EOB}
 O'Brien, E. A. \newblock
 Towards effective algorithms for linear groups. \newblock
 Finite geometries, groups, and computation, 
 163--190, Walter de Gruyter, Berlin,  2006. 

\end{thebibliography}

\end{document}
