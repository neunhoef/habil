% this is a part of the habilitation thesis of Max Neunhoeffer

\chapter*{Preface}
\addcontentsline{toc}{chapter}{Preface}

Ultimate goal: Computing with matrix groups, mention projective.

History: Permutation groups, stabiliser chains, base and strong generators,
many highly efficient algorithms, complexity theory as tool, large
base groups, do different things in different situations, use
randomisation!

First goal: Constructive recognition (compare to sifting)
Matrix group recognition project. A bit of history.

Basic approach, rough sketch.

This book gives an overview over the current state of the art and
shows some contributions of the author. Collaborations.
Implementations.

Work to do (in particular C9 groups).

Not mentioned: New methods for matrices over finite fields using
floating point numbers, details on C9, Mark Stather's work, further
algorithms building on top of constructive recognition. Black box
groups. Global analysis of the whole procedure of building a
composition tree.

Structure of this book. Go through chapters. Explain authors.

Acknowledgements. Thanks.

In particular:

RWTH, Gerhard Hiss, Alice Cheryl, UWA, Akos, Ohio State, St Andrews, Colva, Jon,
Steve. Proofreading: Colva, Papa, Anja.

