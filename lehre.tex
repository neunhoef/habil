\documentclass[12pt,ngerman]{article}

\pagestyle{empty}
\parindent0pt
\usepackage{babel}
\usepackage[a4paper,headheight=0mm,top=1in,bottom=1.4in,left=1in,right=1in]{geometry}
%\usepackage{newcent}
\usepackage{charter}
\usepackage[latin1]{inputenc}
\usepackage{graphicx}

\begin{document}
\begin{center}
{\Large �bersicht �ber die Lehrerfahrung}

\vspace*{1cm}
\begin{tabular}{lp{5in}}
1992--1997 & Tutor, Universit�t Heidelberg \\
           & Betreuung von �bungsgruppen zu den Vorlesungen 
\begin{itemize}
\item Lineare Algebra~I,
\item Analysis I--III,
\item Einf�hrung in die Stochastik, 
\item Statistik I\&II und
\item Algebra I
\end{itemize} \\
1997--2007 & Wissenschaftlicher Assistent am Lehrstuhl D f�r Mathematik,
RWTH \\
& Betreuung der Vorlesungen 
\begin{itemize}
\item Lineare Algebra I\&II, 
\item Diskrete Strukturen,
\item Algebra I\&II,
\item Computeralgebra und
\item Algebraische Topologie
\end{itemize} \\
2001--2003 & Entwicklung des e-Learning-Systems OKUSON (mit Frank
L�beck) \\[2mm]
2003--2006 & Vorlesungsmodul ``Lineare Algebra'' beim Vorkurs Mathematik der
RWTH Aachen f�r Studienanf�nger aller Fachrichtungen \\[2mm]
WS 2007/08 & Reading Course ``Representation Theory'' (4th year
undergraduate), \\
& University of St Andrews \\[2mm]
SS 2008 & Vorlesung ``Finite Fields'' (4th year undergraduate), \\
& University of St Andrews \\[2mm]
SS 2009 & Vorlesung ``Lie Algebras'' (4th year undergraduate), \\
& University of St Andrews \\[2mm]
seit 2008 & Kobetreuung von $3$ Doktoranden in St Andrews \\
& (dort hat jeder Doktorand 2 Betreuer) \\[2mm]
\end{tabular}

\end{center}

\end{document}
