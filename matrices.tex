% this is a part of the habilitation thesis of Max Neunhoeffer

\chapter{Matrices over finite fields}

This chapter covers the implementation of the basic operations for matrices 
over finite fields. We begin with a description of a new concrete 
representation of such matrices on nowadays computers in 
Section~\ref{sec:ffematrices}, both in main memory and on storage.
We then analyse the performance and complexity of matrix arithmetic 
for this new representation in Section~\ref{sec:matarith} and compare
it to previous implementations. In Section~\ref{sec:basalgmat}, we give
an overview over the most basic algorithms for finite field matrices,
before we conclude this chapter with the description of a new method
to compute minimal polynomials of matrices over finite fields in
Section~\ref{sec:minpoly}, which is used to compute projective orders
of matrices.

\section{Representing matrices over finite fields}
\label{sec:ffematrices}

\subsection{The idea}

If $p$ is a prime then elements of the finite field $\F_p$ with $p$
elements can be represented by the integers $0, 1, \ldots, p-1$. Thus,
storing one such element on a computer needs only $\lceil \log_2(p)
\rceil$ bits. The finite extension field  $\F_q$ with for $q = p^k$ is built
as quotient $\F_p[x]/c_k \F_p[x]$ using the Conway polynomial
$c_k$ (see \cite{Nickel} or on the web at \cite{ConwayFL}). Since the
Conway polynomials are monic by definition, one can represent an element
of $\F_q$ by one polynomial over $\F_p$ of degree less then $k$ and thus
by storing $k$ elements of $\F_p$ using $\lceil \log_2(p^k) \rceil$ bits.


\index{Matrix}

\section{Matrix arithmetic}
\label{sec:matarith}

\section{Basic algorithms for matrices}
\label{sec:basalgmat}

\section{Computing minimal polynomials}
\label{sec:minpoly}

