% This is the main file of the Habilitation thesis of Max Neunhoeffer

\documentclass[openany,11pt,british]{book}

\usepackage[latin1]{inputenc}
\usepackage{amssymb}
\usepackage{makeidx}
\usepackage[british]{babel}
%\usepackage{stmaryrd}
\usepackage[mtbold,subscriptcorrection,mtpluscal]{mathtime}
%\usepackage[heavybold,uprightgreek,subscriptcorrection,mtpluscal]{mathtime}
\usepackage{theorem}
\usepackage[all,ps]{xy}
%\usepackage{showkeys}
\usepackage{ifthen}
\usepackage[bf,center]{caption}
%\renewcommand{\captionfont}{\sffamily}
\usepackage{calc}
\usepackage{tocloft}
\setlength{\cftchapnumwidth}{7mm}
\setlength{\cftsecnumwidth}{10mm}
%\usepackage{titlesec}
\usepackage{fancyhdr}
\usepackage{color}
\usepackage{longtable}

\usepackage[numbib,numindex,notlot,notlof]{tocbibind}

%\definecolor{RoyalBlue}{rgb}{0.0236,0.0894,0.6179}
%\definecolor{RoyalGreen}{rgb}{0.0236,0.6179,0.0894}
%\definecolor{RoyalRed}{rgb}{0.6179,0.0236,0.0894}
\definecolor{MyBlue}{rgb}{0.01,0.05,0.5}
\definecolor{MyGreen}{rgb}{0.01,0.4,0.05}
\definecolor{MyRed}{rgb}{0.7,0.01,0.05}

% The paper format:
%\usepackage{a4wide}
%So am Ende:
\usepackage[a4paper,top=1in,bottom=2in,width=5.5in,
            headheight=1\baselineskip,verbose]{geometry}
%So zum Korrekturlesen mit breitem rechtem Rand:
%\usepackage[a4paper,width=6.125in,headheight=1\baselineskip,left=5mm,twoside=false]{geometry}

\usepackage[colorlinks=true,backref=section,%hypertex,
            linkcolor=MyBlue,urlcolor=MyRed,citecolor=MyGreen,
            pdftitle={Computing with Matrix Groups},
            pdfauthor={Max Neunh\string\�ffer},
            pdfsubject={Matrix Groups},
            pdfkeywords={constructive recognition}]{hyperref}
%\usepackage{myhyppdf}
% Bemerkung: showkeys und hyperref funktionieren nicht zusammen!

% The page headings:
\fancyhead[LO]{\nouppercase{\rightmark}}
\fancyhead[RE]{\nouppercase{\leftmark}}
\fancyhead[LE,RO]{\thepage}
\fancyhead[FCO,FCE]{}
\pagestyle{fancy}

\setcounter{tocdepth}{1}

% Change arrows in xypic:
\SelectTips{cm}{11}
\UseTips

\makeindex

\parindent0pt

% Mathematische Operatoren wie z.B. rad:
\makeatletter
\newcommand{\maop}[1]{%
\ensuremath{\mathop{\operator@font #1}\nolimits}}
\newcommand{\maopl}[1]{%
\ensuremath{\mathop{\operator@font #1}\limits}}
\makeatother

\newcommand{\rad}{\maop{rad}}
\newcommand{\soc}{\maop{soc}}
\newcommand{\myimplies}{\ensuremath{\Longrightarrow}}
\newcommand{\myiff}{\ensuremath{\iff}}
\newcommand{\Hom}{\maop{Hom}}
\newcommand{\End}{\maop{End}}
\newcommand{\Tr}{\maop{Tr}}
\newcommand{\id}{\maop{id}}
\newcommand{\Ima}{\maop{Im}}
\newcommand{\Char}{\maop{char}}
\newcommand{\GL}{\maop{GL}}
\newcommand{\PGL}{\maop{PGL}}
\newcommand{\SL}{\maop{SL}}
\newcommand{\PSL}{\maop{PSL}}
\newcommand{\tens}[1][]{\ifthenelse{\equal{#1}{}}{\maopl{\otimes}}%
{\mathop{\raisebox{0.4mm}{$\scriptstyle\maopl{\otimes}_{#1}$}}}}
\newcommand{\smtens}[1][]{\ifthenelse{\equal{#1}{}}{\maopl{\otimes}}%
{\mathop{\raisebox{0.15mm}{$\scriptscriptstyle\maopl{\otimes}_{#1}$}}}}
\newcommand{\myle}{\leqslant}
\newcommand{\myge}{\geqslant}

% Einige Definitionen fuer haeufig vorkommende Buchstaben:

\newcommand{\F}{\ensuremath{\mathbb{F}}}
\let\ll=\l   % fuer alle Faelle!
\renewcommand{\l}{\ensuremath{\ell}}
\newcommand{\C}{\ensuremath{\mathbb{C}}}
\newcommand{\N}{\ensuremath{\mathbb{N}}}
\newcommand{\Q}{\ensuremath{\mathbb{Q}}}
\newcommand{\R}{\ensuremath{\mathbb{R}}}
\newcommand{\Z}{\ensuremath{\mathbb{Z}}}

% Ich will spaeter noch umkonfigurieren:
\newcommand{\Emph}[1]{{\boldmath\textbf{#1}}}
\newcommand{\ba}[1]{\overline{\rule{0mm}{1.2ex}#1}}

% Einige Gimmicks:
\newcommand{\Proof}{Proof: }
\newcommand{\ProofEnd}{\hfill$\square$\par}

\newenvironment{compactlist}{\begin{list}{}{\setlength{\itemsep}{0pt}%
\setlength{\labelwidth}{1cm}\addtolength{\labelsep}{3mm}%
\addtolength{\leftmargin}{3mm}}}{\end{list}}

\renewcommand{\thechapter}{\Roman{chapter}}
\renewcommand{\thesubsection}{(\arabic{section}.\arabic{subsection})}
\newcommand{\mychapter}[1]{\chapter{#1}}
\makeatletter
\let\@oldrefstepcounter=\refstepcounter
\def\refstepcounter#1{\@oldrefstepcounter{#1}%
\ifthenelse{\equal{#1}{subsection}}%
{\gdef\@currentlabel{\thechapter.\thesubsection}}{}}
\makeatother

\theoremstyle{changebreak} 
\theoremheaderfont{\bfseries\boldmath}
\newtheorem{Prop}[subsection]{Proposition}
\newtheorem{Theo}[subsection]{Theorem}
\newtheorem{Lemm}[subsection]{Lemma}
\newtheorem{Corr}[subsection]{Corollary}
\newtheorem{DefProp}[subsection]{Definition/Proposition}
\newtheorem{Conj}[subsection]{Conjecture}
\theorembodyfont{\upshape}
\newtheorem{Def}[subsection]{Definition}
\newtheorem{Rem}[subsection]{Remark}
\newtheorem{Rems}[subsection]{Remarks}
\newtheorem{Not}[subsection]{Notation}
\newtheorem{Hyp}[subsection]{Hypothesis}
\newtheorem{Exa}[subsection]{Example}

\newcommand{\ruecke}{\mbox{\phantom{\rm\bf\thesubsection\ }}}

%\includeonly{}

\begin{document}

\tableofcontents
\thispagestyle{fancy}

% this is a part of the habilitation thesis of Max Neunhoeffer

\chapter{Introduction}
\label{chap:intro}

This chapter covers a few fundamental concepts and algorithms which
will be used in the rest of the book. We start talking about the
complexity of algorithms, go on with the concept of straight line
programs in groups and finish by mentioning the idea of randomisation
in algorithms and a way to produce nearly uniformly distributed
elements in a finite group.

\section{Some notes on complexity theory}
\label{sec:complexity}

\index{complexity theory}%
Already in this chapter, but all the more in the rest of the book, we
talk about the analysis of algorithms. In this section we want to
discuss briefly what we mean by this at all.

An algorithm is usually designed to solve a whole family of problems of
different sizes. Obviously, for a single, particular instance of a
computational problem one can simply store the answer and look it up
when needed in nearly no time. But this is usually not what we intend
to do when we develop an algorithm. Furthermore, it is clear that a
small instance of a computational problem usually will need less time
to solve than a big instance of the same problem. It takes for example longer 
to multiply two $10000\times 10000$-matrices with entries in the finite 
field $\F_q$ than to multiply two $100\times 100$-matrices
with entries in the same finite field $\F_q$,
even if these two situations are instances of the same
computational problem ``matrix multiplication over $\F_q$''.

Therefore, when we talk about analysing an algorithm that solves a
certain family of computational problems, we assign each instance of
this problem a ``size'' and ask how the runtime of the algorithm grows
in comparison to the size of different instances. This is in vague terms
what is meant by ``complexity of an algorithm'', and ``complexity
\index{complexity}%
theory'' is the study of the complexity of algorithms. The size of an
instance of the above mentioned matrix multiplication problem of two $n
\times n$-matrices could for example be measured by the value $n$.

Of course, different computers are running at different speeds, so
what we usually do is to count the number of steps in the algorithm
needed to complete a particular instance of the computational problem
as a function of the size of the problem instance.
We try to keep the different steps comparable. We might for example
count the number of elementary field operations (addition,
\index{elementary field operation}%
subtraction, multiplication, inversion) during the execution of a
matrix multiplication. The standard simple-minded approach to matrix
multiplication would then need $(2n-1)\cdot n^2$ elementary field
operations since each entry of the result is a sum of $n$ products of
numbers in $\F_q$, that is, we need $n$ products and $n-1$ additions
for each of the $n^2$ matrix entries. The complexity of this algorithm
in terms of the size $n$ of the input would then be $(2n-1)\cdot n^2$
elementary field operations.

If the growth rate of the complexity of an algorithm A is slower
\index{growth rate}%
than the one for another algorithm $B$ solving the same problem, then 
we would consider A to be ``better'' than B. Note that it is
possible that the complexity of B is in fact smaller than
the one of A for small problem sizes. In that case we might want to
implement both algorithms and use B for smaller instances and A for
bigger instances. In this way complexity theory helps to come up with
better implementations. If we had for example an algorithm B to multiply
two $n\times n$-matrices over $\F_q$ using $n^4/20$ elementary field
operations, it would be worthwhile to use it as opposed to the
standard algorithm A for small matrices, since
$n^4/20 \le (2n-1)\cdot n^2$ for $n \le 39$. However, for big
matrices A would outperform B dramatically.

Counting the steps in an algorithm can be very tedious and much more
difficult if there are decisions and case distinctions during the
execution. Randomisation as described in Section~\ref{montevegas}
makes this even more difficult. Therefore we are usually happy to come
up with an upper bound for the number of steps necessary. In addition,
when we are only interested in the growth rate of the complexity, we
\index{growth rate}%
are only interested in the type of function expressing this upper
bound in terms of the problem size and are actually not interested in
the constants. 

So, to determine that the growth rate of the complexity
\index{growth rate}%
of the example algorithm A above is smaller than the one of B, we do
not need the exact number of steps $(2n-1)\cdot n^2$ and $n^4/20$
respectively, but we only need to know that the first behaves ``like
$n^3$'' and the second ``like $n^4$''. The higher exponent $4$ in the
second function eventually beats the smaller constant $1/20$.
Note however that without knowing the constants we would in fact not
know that the break-even point of where we should start using A in
favour of B is at $n=39$. Having noticed this, we would like to comment
that knowing the constants in the complexity of algorithms can be very
interesting for coming up with good implementations.

Different parts of this book go differently about this. Whenever
possible we have tried to include constants in the complexity
estimates. However, sometimes this would have been too tedious and
would have complicated things dramatically. In these cases we are
content with determining the growth rate. For this purpose, we adopt
\index{growth rate}%
the standard big-$O$ notations, which we will briefly repeat here.
It is used to formulate statements about the asymptotic behaviour of
functions.

\begin{Def}[Capital-$O$-notation]
    \label{capO}\index{O-notation@$O(-)$-notation}%
    Let $\R^+$ be the set of positive real numbers and
    $g : \R^+ \to \R$.
    We say that a function $f : \R^+ \to \R$ is $O(g)$ if there are
    two positive real constants $C$ and $D$ such that 
    $|f(x)| \le C \cdot |g(x)|$ for all $x > D$.
\end{Def}

This will be used in the analysis of algorithms by saying for
example: Algorithm A above has complexity $O(n^3)$ whereas algorithm B
\index{complexity}%
above has complexity $O(n^4)$. Obviously it has to be clear from the
context, which family of computational problems the algorithm deals
with and how the size $n$ of an instance of the problem is measured.

There is a certain sloppiness in this usage of Definition~\ref{capO}.
Strictly speaking the following statement is true as well: 
``The function $n \mapsto (2n-1)\cdot n^2$ is $O(n^5)$.'' This follows
from the fact that the function $n \mapsto n^3$ is $O(n^5)$. However,
this statement is not as strong as possible. Obviously, we are
interested in the fact that the exponent $3$ is the \emph{least
possible $e$} for which the statement ``the complexity of algorithm A is
$O(n^e)$'' is true. Sometimes we can prove that our statement is best
possible and sometimes we cannot.

Up to now we have concentrated on the ``time complexity'' of
\index{time complexity}%
algorithms, that is, the asymptotic behaviour of the runtime with
growing problem sizes. Although this is usually the most interesting
aspect, it is of course also necessary to keep an eye on other
computational resources like memory usage. We speak of ``space
complexity'' in this case and use a similar setup and language. This
\index{space complexity}%
only plays a minor role in this book.

The point of view we adopt in this book is that the complexity analysis of
algorithms is a tool to come up with good implementations. Complexity
results tell us, which algorithms are suited best for which problem
classes and possibly which problem sizes, and they tell us, what is
considered to be a ``satisfactory'' algorithm for a problem class and
what is considered to be ``lacking''. In general we would like to
devise algorithms which have a polynomial as an asymptotic bound of their
time complexity in terms of the problem size. We call these algorithms
``polynomial-time algorithms''.


\section{Straight line programs}
\label{slp}

\index{straight line program}\index{SLP}%
Let $G$ be a group that is given by a tuple of generators $(g_1,
\ldots, g_k)$. That is, every element in $G$ can be expressed as a
finite product of powers of the $g_i$ and their inverses. However,
since $G$ is not necessarily abelian, the generators may occur more
than once and the products can be quite long. We call such a product a 
\emph{word in the generators $(g_1, \ldots, g_k)$}. 
\index{word in generators}%
For finite groups we do not need the
inverses of the generators since they all have finite order and can
thus be expressed by positive powers.

Quite often in applications we want to encode rather long words in
generators on a computer. One reason for this is that we want to store certain
elements in a known group in terms of a generating tuple (see in
particular Section~\ref{standardgens}). Another reason is for example that to
evaluate a group homomorphism $\varphi:G \to H$ on arbitrary elements
$g \in G$, if we only know the images $\varphi(g_i)$ of a tuple of
generators for $G$, we need to express $g$ explicitly in terms of
the generators $(g_1, \ldots, g_k)$. Finally, the problem of constructive
recognition (see Problem~\ref{ProbCR3}) involves sometimes rather
\index{constructive recognition}%
long words in a tuple of generators, too.

To store and evaluate such words efficiently is the purpose of
\emph{straight line programs (SLP)}.

In general, straight line programs are programs that have no branches
or loops, their execution follows a ``straight line'' under all
circumstances. For the purpose of storing and evaluating words in
generators in a group, we further restrict this to the following:

\begin{Def}[Straight Line Program (SLP)]
    \label{defslp}\index{straight line program}\index{SLP}%
    \index{straight line program!definition of}\index{SLP!definition of}%
    A \emph{straight line program} is a procedure that takes as input a
    $k$-tuple $(g_1, \ldots, g_k)$ of group elements in a 
    common group and consists of a finite 
    sequence of steps, which are each one of the following:
\begin{itemize}
    \item Compute the product of two stored elements, or
    \item compute the inverse of a stored group element.
\end{itemize}
    An SLP starts with the elements $g_1, \ldots, g_k$,
    stores all intermediate results and returns one or more of the 
    group elements collected during its execution. The number of
    steps in the SLP is called its \emph{length}.
\end{Def}

\begin{Rem}
A straight line program of length $l$ can encode words in its input
of length up to $2^l$. Obviously, it cannot encode all words of that
length.
\end{Rem}

Implementations and data structures for straight line programs are
available in the {\GAP} (see \cite{GAP4}) and {\MAGMA} (see
\cite{Magma})
computer algebra systems. The
WWW-Atlas of group representations uses straight line programs to
\index{WWW-Atlas of group representations}%
store generators for maximal subgroups of groups. Note that the
current implementations of straight line programs in these systems
provide an even more compact storage by allowing arbitrary finite
products of powers of previously stored group elements in each step.


\section{Randomised algorithms}
\label{montevegas}
\index{randomised algorithm}\index{algorithm!randomised}%
\enlargethispage{1\baselineskip}

Traditionally, an algorithm is a completely deterministic procedure to
achieve a certain goal. Whenever it is executed, it performs the same
steps and thus behaves in the same way when called twice with the same input.

However, there is a certain limitation in this paradigm. In particular
in situations, in which we want to find some result that can later
be verified to be correct easily, randomised methods excel. By
randomised
methods we mean algorithms that involve a certain amount of random
choices. That is, the sequence of steps performed by a randomised
algorithm depends on certain random choices done during the algorithm.
Of course, in practice we will usually use pseudo random numbers to
do these random choices.

We do not want to go into too much detail here, but there are many
examples in which randomised algorithms can greatly outperform
deterministic algorithms. However, how do we measure or analyse
the performance of a randomised algorithm, given that it does
different things on different calls with the same input, and thus
has different runtimes on different occasions?

One possibility for performance analysis is to look at the average
or the expected runtime. Although this is a good and
interesting thing to look at, this type of analysis often stays a
bit unsatisfactory, since one never knows, how long the algorithm will
take at most in a particular instance.

Therefore the most common approach for randomised algorithms is to do a 
worst-case analysis. However, clearly the absolutely worst case is
that by incredible bad luck all random choices turn out to be wrong
and the algorithm does not succeed even after a very long time. To get
rid of this problem we have to allow our algorithms to fail in some
way, most commonly simply by giving up with \textsc{Fail} as answer
after a certain time.
Using this exit route, we can devise algorithms that are
guaranteed to terminate after a certain number of steps or a given
amount of time. To be useful in practice, we of course want to have a
bound for the probability with which this failure occurs. Optimally, we
want to prescribe an upper bound for the failure probability
beforehand. The guaranteed upper bound for the runtime then might
depend on the choice of the failure probability bound.

In general we distinguish between so-called ``Monte Carlo'' and
``Las Vegas'' algorithms as defined in the following.

\begin{Def}[Monte Carlo algorithm]
\index{randomised algorithm}\index{algorithm!randomised}%
\index{Monte Carlo algorithm}%
    A \emph{Monte Carlo algorithm with failure probability $\epsilon$}
    is a randomised algorithm that is guaranteed to terminate after
    a finite amount of time with some result, if the probability for
    returning a wrong result is bounded by $\epsilon$.
\end{Def}

A bit more satisfying is the following.

\begin{Def}[Las Vegas algorithm]
\index{randomised algorithm}\index{algorithm!randomised}%
\index{Las Vegas algorithm}%
    A \emph{Las Vegas algorithm with failure probability $\epsilon$}
    is a randomised algorithm that is guaranteed to terminate after
    a finite amount of time with either the correct result or
    \textsc{Fail} indicating failure, if the probability for failure
    is bounded by $\epsilon$.
\end{Def}

The two concepts are sometimes related by the following.

\begin{Rem}[Upgrading Monte Carlo to Las Vegas by verification]
\index{verification}%

Assume that there is an efficient way to verify the correctness of the output
of a Monte Carlo algorithm. Then we can immediately upgrade the
algorithm to be of Las Vegas type by following it with a verification
step that returns \textsc{Fail}, if the result was incorrect in the first
place. ``Efficient'' here means that the verification does not take
much longer than the Monte Carlo computation in the first place.
\end{Rem}

We will use the terms ``Monte Carlo algorithm'' and ``Las Vegas
algorithm'' in this sense throughout this book.

\section{Random elements in finite groups}
\label{randomelts}

Randomised algorithms in group theory need random elements in groups.
Moreover, to allow a proper analysis of the behaviour of such
algorithms one needs to know quite a lot about the distribution of
the random elements in the group. Usually, the best with respect to
analysis is to have a source of uniformly distributed random elements.

For most applications we are content with pseudo randomness, that is, with a
deterministic procedure which produces from some initial seed a sequence of 
elements with a good uniform distribution. Choosing different seeds (or
maybe actually choosing the seed at random) then leads to a different
behaviour of the algorithm in each call.

There are well-known methods to produce good uniformly distributed
pseudo random numbers (see for example \cite[Chapter~3]{AOCP2}).
Building on these, there is a method to produce pseudo random elements
in a finite group given by generators, which works astonishingly
well in the sense that the distribution of the elements is very close
to uniform. In the sequel we describe this method briefly but refer
for proofs to the literature. After this we discuss some of the
advantages and limitations of this method.

\begin{algorithm}
\caption{$\quad$ \sc RattleStep}
\label{rattlestep}
\index{Rattle@\textsc{Rattle}}%
\begin{algorithmic}
\STATE \textbf{Input:} A pair $(a,(h_1, \ldots, h_n))$ with $a \in G$
and $G = \left< h_1, \ldots, h_n \right>$.
\STATE \textbf{Output:} A modified pair $(a,(h_1, \ldots, h_n))$ with $a \in G$
and $G = \left< \smash{h_1, \ldots, h_n} \right>$.
\vspace*{2mm}
\STATE $i := \textsc{Random}(\{ 1,2,\ldots,n\}$
\STATE $j := \textsc{Random}(\{1,2,\ldots,n-1\}$
\STATE $b := \textsc{Random}(\{1,2\}$
\IF {$j = i$}
    \STATE $j := j + 1$
\ENDIF
\IF {$b = 1$}
    \STATE $h_i := h_i \cdot h_j$
\ELSE
    \STATE $h_i := h_j \cdot h_i$
\ENDIF
\STATE $a := a \cdot h_i$
\STATE \textbf{Return} modified $(a,(h_1,\ldots,h_n))$
\end{algorithmic}
\end{algorithm}

\begin{Def}[Random elements with \textsc{Rattle}]
    \label{rattle}
\index{Rattle@\textsc{Rattle}}%
Let $G$ be a group given by a tuple $(g_1, \ldots, g_k)$ of generators
and $n, N \in \N$ with $n \ge k$. The \textsc{Rattle} method to
produce random elements in $G$ is the following procedure:

It uses a variable $(a,(h_1,\ldots,h_n)) \in G \times G^n$
which is changed during the runtime by calls to
Algorithm~\ref{rattlestep}.

It first initialises $(a,(h_1,\ldots,h_n))$
by $a := \mathbf{1}_G$ and $h_i := g_i$ for $1 \le i \le k$ and $h_i
:= \mathbf{1}_G$ for $k < i \le n$.

Then it calls Algorithm~\ref{rattlestep} $N$
times with $(a,(h_1, \ldots, h_n))$ as argument
thereby changing it and
finally returns the last value of $a$ as a random element $a_0$ in $G$.

After this initialisation phase it produces a sequence of random
elements $(a_i)_{i \in \N}$ by calling 
Algorithm~\ref{rattlestep} repeatedly with 
$(a,(h_1, \ldots, h_n))$ as argument thereby changing it and
assigning the value of $a$ to $a_i$ after call number $i$.
\end{Def}

\begin{Rem}[Variant of product replacement]
\index{Rattle@\textsc{Rattle}}%
The \textsc{Rattle} method described above is a variant of the
``product replacement algorithm'', because the main part of a step replaces 
an element by a product of it with another one.
\end{Rem}

\begin{Rem}[Comments on the implementation of \textsc{Rattle}]
\index{Rattle@\textsc{Rattle}}%
It is not completely clear how to choose the parameters $n$ and $N$. In
principal $n$ could be chosen equal to $k$. However, what a good length
$N$ of the initialisation phase is depends on the group $G$ and
on the generating tuple $(g_1, \ldots, g_k)$. In practice one chooses $n$
slightly bigger than $k$ and $N$ around $100$ unless $k$ is very big.
For large $k$ the value of $N$ has to be chosen bigger. It is clear
that a constant value for $N$ will not work well for $|G| \to \infty$.
\end{Rem}

\begin{Prop}[\textsc{Rattle} converges to the uniform distribution]
    \label{proprattle}
\index{Rattle@\textsc{Rattle}}%
    The distribution of the element $a_0$ in the \textsc{Rattle} procedure (see
    Definition~\ref{rattle}) converges for $N \to \infty$ to the
    uniform distribution.
\end{Prop}
\proofbeg
See \cite[Section~4]{LGMurray}.
\proofend

\subsubsection{A brief discussion of \textsc{Rattle}}

\index{Rattle@\textsc{Rattle}}%
Although it can be proved that for every finite group $G$ and every
generating tuple $(g_1, \ldots, g_k)$ the distribution of the element
$a_0$ for $N \to \infty$ tends to the uniform distribution (see
Proposition~\ref{proprattle}), there is
not much known about the rate of convergence. So, picking the value for
$N$ in practice is difficult, in particular if we do not know anything
about $G$.

If we use the sequence produced by the \textsc{Rattle} method
after initialisation as a sequence of random elements in $G$, 
subsequent elements seem to be uniformly distributed, but adjacent
elements in the sequence are clearly not distributed independently.
Obviously the next element in the sequence depends heavily on the
previous state $(a,(h_1,\ldots,h_n))$ and the previous element is
equal to the $a$ value of this state. However, the state contains of
course more information than simply the value $a$.

Despite these obvious deficiencies, the algorithm
works surprisingly well in practice. The computational cost for the 
initialisation is 200 multiplications and after that 2 more
multiplications for every further element in the sequence.
The memory requirements are minimal and the produced sequence of
random elements is good enough for most purposes. Analysing algorithms
that use \textsc{Rattle}
\index{Rattle@\textsc{Rattle}}%
with the assumption that it produces uniformly distributed random
elements in the group provides good predictions on how
well these algorithms work in practice.

Throughout this book the \textsc{Rattle} method is used to produce
random elements in group and the above assumption is made.
\index{Rattle@\textsc{Rattle}}%


% this is a part of the habilitation thesis of Max Neunhoeffer

\chapter{Matrices over finite fields}

This chapter covers the implementation of the basic operations for matrices 
over finite fields. We begin with a description of a new concrete 
representation of such matrices on nowadays computers in 
Section~\ref{sec:ffematrices}, both in main memory and on storage.
We then analyse the performance and complexity of matrix arithmetic 
for this new representation in Section~\ref{sec:matarith} and compare
it to previous implementations. In Section~\ref{sec:basalgmat}, we give
an overview over the most basic algorithms for finite field matrices,
before we conclude this chapter with the description of a new method
to compute minimal polynomials of matrices over finite fields in
Section~\ref{sec:charminpoly}, which is used to compute projective orders
of matrices.

%FIXME

\section{Representing vectors and matrices over finite fields}
\label{sec:ffematrices}

If you already know something about compact vectors and matrices over 
finite fields you can safely skip the next subsection and directly
proceed to Section~\ref{ssec:cvec}. We first explain the basic idea.

\subsection{The idea}

If $p$ is a prime then elements of the finite field $\F_p$ with $p$
elements can be represented by the integers $0, 1, \ldots, p-1$. Thus,
storing one such element on a computer needs only $\lceil \log_2(p)
\rceil$ bits. The finite extension field  $\F_q$ with $q = p^k$ elements 
is built
as quotient $\F_p[x]/c_k \F_p[x]$ using the Conway polynomial
$c_k$ (see \cite{Nickel} or on the web at \cite{ConwayFL}). Since the
Conway polynomials are monic by definition, an element
of\, $\F_q$ can be represented by one polynomial over\, $\F_p$ of degree 
less then $k$ and thus by storing $k$ elements of\, $\F_p$ using $\lceil
\log_2(p^k) \rceil$ bits.

Since linear algebra operations over finite fields can be performed
rather efficiently by modern microprocessors, the limiting factor 
for such computations is memory access. Therefore, it is performance
critical to store vectors and matrices using as little memory as possible,
and to choose the data structure in a way that allows for fast memory
access. We call such memory efficient data structures ``compact''.

This idea is quite old (see ADDREFERENCE) and has already been used
extensively (see for example \cite{CMeatAxe} or \cite{GAP4} or
various other systems).

There is a fundamental difference between the characteristic $p=2$ case
and all other characteristics. The reason for this is that in
characteristic $2$ the addition of vectors can be implemented by using
the XOR (exclusive or) operation available in every microprocessor
instruction set. In odd characteristic, the available instructions
do not fit so well to finite field arithmetic. Therefore, previous
implementations mostly rely on table lookups to perform arithmetic
operations. Since the memory space available for lookup tables is
limited, this method is limited to relatively small fields (usually up to
fields with $256$ elements) and has to use single byte accesses
as opposed to processor word accesses, for which modern machines
seem to be optimised. These limitations seem to become more serious
as word lengths in standard microprocessors increase.

The main novelty in the approach presented here is to overcome this
problem by choosing the data structures in a way that allows to use
processor word operations for all arithmetic operations in all
characteristics. Also this idea has been used before (see
\cite{EssenLinAlg}), however, no implementation of this seems to
be available any more and, more importantly, the approach still
insisted on storing whole elements of\, $\F_q$ in one machine word.
We will pack as many prime field elements as possible into a machine
word and distribute the various prime field coefficients of a single
element of\, $\F_q$ into distinct machine words to allow for better
memory usage in the case of extension fields.


\subsection{Compact vectors}
\label{ssec:cvec}

We first describe in detail how a vector of length $n$ over the field
$\F_q$ with $q=p^k$ elements is stored on a machine with word length $32$ 
or $64$ bits respectively. The ideas for the step from $32$ to $64$ bits
can be applied analogously for future microprocessors with even larger
word length. We ignore architectures with funny word lengths not being
a multiple of $32$ bits.

Let $e$ be $\lceil \log_2(2p-1)\rceil$ for $p > 2$ and $1$ for $p=2$. 
This is the number of bits
necessary to store an integer in the range $0,1, \ldots, 2p-2$ except
for $p=2$, where it is $1$. This number $e$ is the number of bits we 
reserve for every prime field element we want to store. For $p=2$ this
is evidently best possible, whereas for odd $p$, we seem to waste
one bit per prime field element, since we seem to need only store
numbers in the range $0,1,\ldots p-1$. This additional bit in the
data structure allows us to represent a sum of two numbers in the
range $0,1,\ldots, p-1$ using $e$ bits in odd characteristic.

We start with the prime field case $q=p$.

We pack as many prime field elements as possible into a machine
word, that is, $w := \lfloor 32/e \rfloor$ prime field elements per word
on a machine with word length $32$ bit. On machines with $64$ bits
we pack $w := 2 \lfloor 32/e \rfloor$ prime field elements into one word.
Note that we do not use $\lfloor 64/e \rfloor$ elements which can be one
more, since then the conversion between the different data formats for
different word lengths becomes too expensive and awkward.

We always imagine the least significant bit in a machine word as being 
on the right hand side. To store a vector, we start filling machine words 
from right to left, always using $e$ bits for one prime field element and
packing $w$ prime field elements into a word.

We illustrate this layout in the following example for $p=5$ and thus
$e=4$ on a machine with $32$ bit word length:

\begin{verbatim}
  bit             3322|2222|2222|1111|1111|1100|0000|0000
  number:         1098|7654|3210|9876|5432|1098|7654|3210
  prime field         |    |    |    |    |    |    |
  element number: 7777|6666|5555|4444|3333|2222|1111|0000
\end{verbatim}

Within each block of $e$ bits, we represent prime field elements
by the binary representation of a number in the range $0,1,\ldots,p-1$
with the least significant bit of this binary representation on the
right hand side. Thus, in our example $1+1+1+1 \in \F_5$ would be
represented by the bit sequence \texttt{0100} and $1+1+1$ by \texttt{0011}.
Note that the most significant bit in this representation is always $0$.

The first $w$ prime field elements in a vector are stored in the first
machine word of the vector, the next $w$ in the next word and so on.

We proceed now to the extension field case $q=p^k$. Let $e$ and $w$ be
exactly the same values as above. We now have to store $k$ prime field
elements for every element of $\F_q = \F_p[x]/c_k\F_p[x]$, namely the 
coefficients of the unique residue class representative of degree smaller 
than $k$, where $c_k$ is the Conway polynomial used to construct the 
finite field extension $\F_q$ over $\F_p$ (for details see \cite{Nickel}
or on the web \cite{ConwayFL}). Namely, if $a \in \F_q$ is represented by
the polynomial $\sum_{i=0}^{k-1} a_i x^i$, then we have to store the prime 
field elements $a_0, a_1, \ldots, a_{k-1}$. 

In a compact vector of length $n$ of elements of $\F_q$ we distribute
those numbers in the following way: The first $k$ machine words in the
memory representation of the vector are used to store the first $w$
elements of the vector. The first machine word holds all the coefficients
$a_0$ of those $\F_q$ elements, the second the coefficients $a_1$ and so
on. Since one machine word can hold up to $w$ prime field elements this
fills the first $k$ words rather satisfactorily. The second $k$ machine words
in the vector then hold the vector elements with indices 
$w+1, w+2, \ldots, 2w$ in exactly the same way.

Note again that for machines with a word length of $64$ bit we choose the
value $w$ only twice as big as for $32$ bit machines even if one more
prime field element would fit into the machine word.

The only natural limit of this implementation is that 
$\lceil \log_2(2p-1) \rceil$ must be smaller or equal to the word length
of the microprocessor.

We can now explain, how we can add vectors in the above representation
only by doing a series of word operations.

\subsection{Adding compact vectors}

The basic addition formula for finite field elements is rather simple:
For prime field elements we just have to add two numbers in the range 
from $0$ to $p-1$, and
subtract $p$, if the sum is greater or equal to $p$. The extension
field elements are done component-wise. However, we have to solve
the problem of doing this simple operation using word operations and 
thus doing this for $w$ prime field elements at the same time.
This is easy for $p=2$, since we can use the standard XOR (exclusive or)
operation.

The idea to overcome this problem for $p > 2$ is the following. 
Assume $a$ and $b$
are two integers in the range from $0$ to $p-1$. By adding
$2^{e-1}-p$ to the sum $a+b$ we get $t := a+b+2^{e-1}-p$, which
has the property, that $t \ge 2^{e-1}$ if and only if $a+b \ge p$
(remember $2p-1 \le 2^e$ and thus $p-1 < 2^{e-1}$). That is, if
the number $t$ is represented using binary expansion with $e$ bits,
then the most significant bit is set if and only if $a+b \ge p$.

This idea is now used for two words $a$ and $b$, containing $w$
prime field elements each. Every prime field element uses exactly $e$ bits
in its word and we call these sections of $e$ adjacent bits in a word 
``components'' for the moment. 

We prepare an ``offset'' word $o$ that contains in each component the
number $2^{e-1}-p$ and a ``mask'' word $m$ that contains in each component
the number $2^{e-1}$ meaning, that in each component only the most significant
bit is set and all others are zero. In addition, we keep a word $n$
containing the number $p$ in each component.

The finite field sum $c$ of $a$ and $b$ is now computed in the following way:
First $s := a+b$ and $t := a+b+o$ are computed. We then use an AND 
operation for words to extract exactly the most significant bits of
those components, in which the sum was greater or equal to $p$.
This is done by computing $r := t \ \&\  m$ (we use the \& symbol to
indicate bit-wise AND operations). Bit-shifting
the word $r$ by $e-1$ bits to the right (we use the notation
$r \gg (e-1)$ for this) and subtracting the
result from $r$ now results in a word $u := r - (r \gg (e-1))$
having the number $2^{e-1}-1$
in those components, in which the sum was greater or equal to $p$ and
$0$ in the others. Finally, doing a bitwise AND operation of $u$ with
$n$ results in exactly the right word to subtract from $s$ to get
the correct result.

Thus, the complete formula is
\[ a+b - \Big(\big(r - (r \gg (e-1))\big) \ \&\ n \Big)
   \qquad \mbox{where}\quad r = (a+b+o) \ \&\  m \]

We illustrate this by an example for $p=3$ on a machine with
$32$ bit word length. In this case, $e = 3$ and $w = 10$. We want to
add the words shown below in the rows depicted by \texttt{a} and \texttt{b}.
We also show the prepared words $o$, $m$, and $n$. The bits marked
with \texttt{X} are not used and are all equal to zero.

\begin{verbatim}
  bit             33|222|222|222|211|111|111|110|000|000|000
  number:         10|987|654|321|098|765|432|109|876|543|210
                    |   |   |   |   |   |   |   |   |   |
  a:              XX|000|010|001|000|010|001|000|010|001|000
  b:              XX|000|010|010|010|001|001|001|000|000|000
  o:              XX|001|001|001|001|001|001|001|001|001|001
  m:              XX|100|100|100|100|100|100|100|100|100|100
  n:              XX|011|011|011|011|011|011|011|011|011|011
\end{verbatim}

In the following table we show some intermediate results and repeat
the input values for easier verification:

\begin{verbatim}
  a+b:            XX|000|100|011|010|011|010|001|010|001|000
  a+b+o:          XX|000|101|100|011|100|011|010|011|010|001
  r:              XX|000|100|100|000|100|000|000|000|000|000
  u:              XX|000|011|011|000|011|000|000|000|000|000
  u&n:            XX|000|011|011|000|011|000|000|000|000|000
  result:         XX|000|001|000|010|000|010|001|010|001|000
  a:              XX|000|010|001|000|010|001|000|010|001|000
  b:              XX|000|010|010|010|001|001|001|000|000|000
\end{verbatim}

Of course, it is a coincidence here that $u$ is equal to 
$u \ \&\ n$, since $p = 2^{e-1}-1$.

This means, that the addition of two words containing $w$ prime field
elements can be done in $7$ word operations for $p > 2$. For the
$p=2$ case, we only need one XOR operation. Thus we have proved:

\begin{Prop}[Addition of compact vectors]
\label{addvec}
We assume a machine with $32$ bit word length. For $2 < p < 2^{31}$,
two compact vectors of length $n$ over the field of $q=p^k$
elements can be added using $7k\cdot \lceil n/w \rceil$ word operations
(plus memory fetches and stores), where $w = \lfloor 32/e \rfloor$
and $e = \lceil \log_2(2p-1) \rceil$. 

For $p=2$, only $k \cdot \lceil n/32 \rceil$ word operations are needed.

For machines with $64$ bit word length, the number of word operations
is halved in both cases and we can work with primes $p < 2^{63}$.
\end{Prop}
\Proof See above. \ProofEnd

\begin{Rem}
Note that since the amount of memory needed for a vector of length $n$
is $k \cdot \lceil n/w \rceil$ for $p > 2$ and $\lceil n/32 \rceil$ for
$p=2$ this means that the addition needs $7$ word operations for each
word of a vector for $p>2$ and $1$ word operation for $p=2$.

Assuming a microprocessor
with a long enough instruction pipeline, we can conclude that all this
can be done as fast as accessing the main memory to fetch $a$ and $b$ and
store the result somewhere, such that the number $7$ does not hurt at all. 
Compare Section~\ref{ssec:discussion} and see Section~\ref{sec:cache}
for some additional comments on processor caches.
\end{Rem}

Next we consider multiplication of vectors by scalars.

\subsection{Multiplication by scalars}

To explain the method, we restrict our attention to the case that one
compact vector shall be multiplied by one scalar and the result 
shall be stored in some other memory location.

Let $e$ and $w$ be defined as in Section~\ref{ssec:cvec}.

We start by discussing the prime field case $\F_p$.
Since a scalar $s \in \F_p$ is represented by an integer in the range $0$ 
to $p-1$ in its binary expansion, we can multiply a compact vector $v$
by $s$ by repeatedly adding vectors to themselves starting with $v$ and
adding up those multiples
whose corresponding bits in the binary expansion of $s$ are set. That is,
if $s = \sum_{i=0}^{e-2} s_i 2^i$ with $s_i \in \{0,1\}$ and again
$e = \lceil \log_2(2p-1) \rceil$, we compute $s\cdot v$ by computing
$v, 2^1 \cdot v, 2^2 \cdot v, \ldots, 2^{e-2} \cdot v$ and then summing
$\sum_{i=0}^{e-2} s_i \cdot (2^i \cdot v)$.
All this can be done with at most $2(e-2)$ vector additions.

We now proceed to the extension field case $\F_q = \F_p[x]/c_k \F_p[x]$ 
with the Conway polynomial $c_k$ and $q = p^k$.
Here again a scalar $s \in \F_q$ is represented by an expansion
$s = \sum_{i=0}^{k-1} s_i x^i + c_k \cdot \F_p[x]$. For the rest of this
section we omit the ``$+ c_k \cdot \F_p[x]$'' and denote cosets
by their representing polynomials of degree less than $k$.

We reduce the problem to scalar multiplications of vectors with
prime field elements. To this end, we have to be able to multiply
a vector with the primitive root $x$.

Considering a single scalar $t = \sum_{i=0}^{k-1} t_i x^i \in \F_q$, 
we see that 
\[ xt = \sum_{i=1}^{k-1} (t_{i-1}x^i) 
+ \sum_{i=0}^{k-1} t_{k-1} \cdot (x^k - c_k) \in \F_q \] 
(remember that we compute in $\F_p[x]$ modulo $c_k$).
Thus, the multiplication by $x$ can be achieved by a shift, one
multiplication of $x^k - c_k$ with the prime field element $t_{k-1}$,
and an addition.

Since we distribute the prime field elements belonging to a single
extension field element in our vector into adjacent words, we can 
do the shift basically for free by accessing a shifted memory location.
But we still have to deal with the fact, that we have $w$ possibly different
highest coefficients $t_{k-1}$ stored together in one word. However,
since we have to multiply all of them with the same prime field element
coming from the expansion of $x^k - c_k = \sum_{i=0}^{k-1} b_i x^i$ 
we can use the method described
for the prime field case above to compute every word to be added to the
shifted vector. That is, for each $k$ adjacent words 
$(a_{jk},a_{jk+1},\ldots,a_{(j+1)k-1})$ in our vector we have to
add the words shifted by one $(0,a_{jk},a_{jk+1},\ldots,a_{(j+1)k-2})$
to the words $(a_{(j+1)k-1} \cdot b_0, \ldots, a_{(j+1)k-1} \cdot b_{i-1})$.
Therefore, the total cost is exactly the same as for one multiplication of
a vector by a prime field scalar and one addition of vectors.

For the full computation of $s \cdot v$, we have to perform this multiplication
by $x$ altogether $k-1$ times, multiply each intermediate result by the
prime field scalar $s_i$ and sum everything up. Thus, the total cost
of the multiplication $s \cdot v$ is the same as $k$ multiplications
of a vector by a prime field scalar and $k-1$ additions of vectors.
Thus, we have proved the following Proposition:

\begin{Prop}[Multiplication of vectors by scalars]
\label{multvec}
We assume a machine with $32$ bit word length.
Let $p$ be a prime, $q = p^k$, and $w$ and $e$ as above:
$w = \lfloor 32/e \rfloor$ and $e = \lceil \log_2(2p-1) \rceil$.

For $2 < p < 2^{31}$, the total cost of a multiplication of a compact 
vector $v$ of length $n$ over $\F_q$ by a scalar $s \in \F_p$ is at most 
$14k(e-2)\cdot \lceil 32/w \rceil$ word operations (plus memory fetches
and stores). For $p=2$, the scalar
can only be $0$ or $1$ and therefore no computation is necessary at all.

For $2 < p < 2^{31}$, the total cost of a multiplication of a compact
vector $v$ of length $n$ over $\F_q$ by a scalar $s \in \F_q$ is at most
$7k^2(2e-3)\cdot \lceil 32/w \rceil$. For $p = 2$, the total cost is
at most $2k(k-1) \cdot \lceil n/32 \rceil$ word operations.
\end{Prop}

\Proof See above and just add up the number of word operations for up
to $k$ multiplications of a vector by a prime field scalar and $k-1$
additions of vectors. Note for $p=2$ that the vector $n$ consists of
$k \cdot \lceil n/32 \rceil$ words and that we have to count one
vector addition for the ``multiplication with a scalar'', since 
the word $a_{(j+1)k-1}$ has to be XORed to those words whose corresponding
bit is set in $x^k - c_k$.
\ProofEnd

\begin{Rem}
Since a vector of length $n$ needs $k \cdot \lceil n/w \rceil$ words
for $p > 2$, this means, that the number of word operations per word of
the vector needed for one multiplication with a scalar is at most 
$7k(2e-3)$ for $p > 2$.
\end{Rem}

\subsection{Memory throughput in real implementations}

In this section we present timing results in real implementations. We 
compared two implementations: The first is an implementation of the
ideas presented in this chapter in the {\sf GAP} package {\sf cvec}
(see \cite{cvec}) by the author, and the second is the standard implementation
of compact vectors in the {\sf GAP} kernel written by Steve Linton
(see \cite{GAP4}).
The latter uses byte-oriented table lookup for fields $\F_q$ with 
$3 \le q \le 256$. We compared vector additions over $\F_2$ and $\F_7$, 
and multiplications of vectors by scalars over $\F_7$ in both implementations.
Finally, we tested multiplications of vectors by scalars over $\F_{3^k}$
for $1 \le k \le 5$ in the new implementation.
Note that the {\sf GAP} library does not offer direct access to the
operation $z := v+w$ without memory allocation. Therefore this 
measurement is missing.

We used three relatively new machines with popular micro processors, 
namely two different machines
with an Intel Pentium 4 with 1024 kB second level cache running at 3.2 GHz, 
and a machine with a AMD Athlon 64 X2 Dual Core Processor 3800+ with 512 kB
second level cache running at 2.0 GHz. The two Pentium 4 machines were
equipped with memory modules with the same specifications (PC 3200, CL3). 

In all runs, no other processes
were consuming significant amounts of CPU time such that nothing should
have interfered with the caches. Strangely enough, the two nearly identical
machines show different performance. 
We cannot explain the differences in memory throughputs.
Therefore we have presented both
results to show that such measurements have a certain fluctuation
obviously involving parameters which cannot easily be determined.
For a discussion of these results see below.

To demonstrate the effect of second level caches, we used different 
lengths of vectors. We used vectors using $32 \cdot 10^6$ bytes each in
order to make sure that none of the data is in the second level cache,
and vectors using $125000$ and $62500$ bytes each to make sure that
most accesses should lead to cache hits in the second level cache.
Of course, all operations are repeated many times to get a high accurracy
of the measured time for one operation by averaging.

We tested three different operations. The first, which we denote by 
``$z := v+w$'', is addition of two different vectors and writing the
result to some other memory location. The time for memory allocation
was not considered. The second operation, which we denote by 
``$v := v+w$'', is also addition of two different vectors, but the result
is written into the same memory location as one of the summands.
The third operation is denoted by ``$w := sw$'' and is multiplication
of a vector by a (non-zero) scalar in place, that is, the result overwrites
the original vector. We chose the scalar $6 + 7\Z \in \F_7$, since its
binary expansion is $110$ and it is thus one of the most ``expensive'' scalars
for multiplication. Note that for the case of table lookups the chosen
scalar does not matter.

For vector times scalar computations over $\F_{3^k}$ we always chose as
scalar the coset of a polynomial with no zero coefficients, which is
the worst with respect to performance.

All results are memory throughput values in megabytes per second. 
That is, a result of $1000$ means that altogether $1000$ times
$1024^2$ bytes of memory have been accessed. ``Altogether'' means, that for
the operation $z := v+w$ both read operations (for $v$ and $w$) and the
write operation into $z$ are counted, that is, for vectors of length
$125000$ one such addition needs to access $375000$ bytes in memory,
namely reading $250000$ and writing $125000$. The same holds for
$v := v + w$, whereas the operation $w := sw$ only reads the memory once
and writes it again, which means that one such operation has to access
$250000$ bytes of memory.

All our results are shown in Table~\ref{memthrough}. The columns marked
with ``C'' are results obtained using the {\sf cvec} package and those 
marked with ``L'' are results obtained using the {\sf GAP} library.

In the next section we discuss some aspects of these results.

\begin{table}[ht]
\begin{center}
\begin{tabular}{|l|r|r|r|r|r|r|}
\hline
Test            & C, P4$_1$ & C, P4$_2$ & C, Ath & 
                  L, P4$_1$  & L, P4$_2$  & L, Ath \\
\hline
\hline
$\F_2$, vectors $32\cdot 10^6 B$, $z := v+w$ 
& 2764 & 2684 & 1940 & ---  & --- & --- \\
\hline                                                    
$\F_2$, vectors $32\cdot 10^6 B$, $v := v+w$ 
& 3629 & 2965 & 2643 & 3577 & 2812 & 2677 \\
\hline                                                    
$\F_2$, vectors $125\,000 B$, $z := v+w$     
& 6839 & 4249 & 8499 & ---  & --- & --- \\
\hline                                                    
$\F_2$, vectors $125\,000 B$, $v := v+w$     
& 6167 & 3791 &12076 & 4818 & 3298 & 12116 \\
\hline                                                    
$\F_2$, vectors $62\,500 B$, $z := v+w$      
& 5643 & 4379 & 7560 & ---  & --- & --- \\
\hline
$\F_2$, vectors $62\,500 B$, $v := v+w$      
& 6458 & 3969 &11237 & 5109 & 3406 &11635 \\
\hline                                                    
\hline
$\F_7$, vectors $32\cdot 10^6 B$, $z := v+w$ 
& 2380 & 1691 & 1902 & ---  & --- & --- \\
\hline
$\F_7$, vectors $32\cdot 10^6 B$, $v := v+w$ 
& 2624 & 1749 & 2515 & 1088 & 708 & 488  \\
\hline                                                    
$\F_7$, vectors $32\cdot 10^6 B$, $w := sw$  
& 2289 & 1504 & 2267 & 1205 & 870 & 422  \\
\hline                                                    
$\F_7$, vectors $125\,000 B$, $z := v+w$     
& 2671 & 1761 & 5745 & ---  & --- & --- \\
\hline
$\F_7$, vectors $125\,000 B$, $v := v+w$     
& 2843 & 1816 & 5356 & 1102 & 713 & 496  \\
\hline                                                    
$\F_7$, vectors $125\,000 B$, $w := sw$      
& 2388 & 1490 & 4273 & 1375 & 876 & 427  \\
\hline                                                    
$\F_7$, vectors $62\,500 B$, $z := v+w$      
& 2765 & 1742 & 5472 & ---  & --- & --- \\
\hline
$\F_7$, vectors $62\,500 B$, $v := v+w$      
& 3080 & 1878 & 5529 & 1170 & 737 & 498 \\
\hline                                                    
$\F_7$, vectors $62\,500 B$, $w := sw$       
& 2386 & 1565 & 5138 & 1342 & 898 & 434 \\
\hline
\hline
$\F_3$, vectors $32\cdot 10^6 B$, $w := sw$
& 2297 & 1471 & 2219 & --- & --- & --- \\
\hline
$\F_9$, vectors $32\cdot 10^6 B$, $w := sw$
& 153 & 98 & 336 & --- & --- & --- \\
\hline
$\F_{27}$, vectors $32\cdot 10^6 B$, $w := sw$
& 99 & 68 & 232 & --- & --- & --- \\
\hline
$\F_{81}$, vectors $32\cdot 10^6 B$, $w := sw$
& 88 & 58 & 204 & --- & --- & --- \\
\hline
$\F_{243}$, vectors $32\cdot 10^6 B$, $w := sw$
& 70 & 47 & 170 & 1166 & 867 & 422 \\
\hline
\end{tabular}
\end{center}
\caption{Memory throughput for vector operations, C: {\sf cvec} package, 
L: {\sf GAP} library}
\label{memthrough}
\end{table}

\subsection{Discussion of results}
\label{ssec:discussion}

The results for the $\F_2$ case show the throughput for raw memory access
since the XOR operations for addition within the processor registers cost
nearly nothing. 

We first look at the top $\F_2$ part of Table~\ref{memthrough}.

We observe, that both implementations perform very similar, which is
only to be expected, since they use the exact same method. For
main memory access we seem to be nearly identical throughputs. For 
second level cache accesses we see some differences, which can be
explained for the given architectures by looking at the C code and the
produced assembler code, which would lead too far here. However, 
the general picture is the same.

One can clearly see in the $\F_2$ part of
Table~\ref{memthrough} that the second level cache seems to help noticably
when the vectors involved are shorter. This is to be expected, but shows
only a factor of between $1.5$ and $3$ for the Pentium~4 and up to 
a factor of $5$ for the Athlon processor.

Theoretically, one could expect a higher speedup factor, but modern
processors and memory interfaces seem to be highly optimised for 
linear memory accesses using so-called bursts. This means that the
processor has a special method of accessing large amounts of consecutive
memory words, which we seem to use here.

We turn now to the middle part of the table concerning the field $\F_7$.
Here one can see two important things: The first is that the $7$ word
operations per word stated in Proposition~\ref{addvec} can be done as
fast as main memory access (compare main memory throughputs for $p=2$ and
$p=7$ on all architectures). If nearly all accesses lead to hits in 
the second level cache, then the $7$ operations reduce the memory
throughput, but only by a factor of about $2$ to $3$, depending on
the processor architecture.

The second thing is that the byte oriented table lookup is noticably
slower than the word accesses. When working in the second level cache,
this can cripple the memory throughput by a factor of up to $10$, at least on
the Athlon architecture. Note that the vector sizes are chosen small
enough, such that not only the three vectors but also the lookup tables
should fit together completely in the second level cache.
The same observation holds for the multiplication with a scalar. 

Of course, for other finite fields $\F_q$ with $q \le 256$ the situation
is different. Among these fields the number $7k(2e-3)$ of word
operations per vector word is at most $105$. The experiments shown in
the third part of Table~\ref{memthrough} concerning fields $\F_{3^k}$ 
indicate, that
if this number is assumed, the performance of multiplication of a vector
by a scalar is much slower than with table lookup. It follows that there 
is no method that is better in all situations.

However, in the light of the grease method explained in
Section~\ref{ssec:vecmat} computing too many vector times scalar 
operations can often be avoided, which renders the new method
superior in many real life examples.


\index{Matrix}

\section{Matrix arithmetic}
\label{sec:matarith}

Traditionally, matrices are implemented in the {\sf GAP} system as
lists of (row-) vectors. This means, that the list object only
holds references to the subobjects, which are the rows of the matrix.
Therefore, one can for example permute rows very efficiently without
actually touching the bulk of the data in the rows. Instead, one can
just redirect references. In addition, different matrices can share
rows, which leads to the fact that the modification of one matrix
can in fact modify the other matrix. Although this can give rise to
nasty bugs, it can also help to increase efficiency both with respect
to memory usage and with respect to performance. We call matrices
implemented with this approach ``row list matrices''.

An alternative way would be to actually embed the row data into the
matrix object itself. A row would then no longer be a {\sf GAP} object
in its own right, but only a part of one. The advantage here is that
we need fewer memory allocations (only one per matrix) and can better
control cache issues, since we know better, where in memory our row data
is stored. We call matrices implemented with this approach ``flat
matrices''.

When devising efficient algorithms dealing with matrices one has to know
which type of matrix one is using, because one has to know, when the
system actually copies data and when it only passes references.

For the {\cvec} package we chose the row list matrix approach since it fits
better into the existing {\GAP} library and allows for easier code reusage.
We also discuss this approach further in this section. We call a list
of compact vectors of the same length over the same field a ``compact
matrix''.

Note however, that Beth Holmes and Richard Parker are currently working
on an implementation of flat matrices in the {\GAP} system.


\subsection{Addition and multiplication with scalars}

This section is extremely short, since to add matrices or
to multiply a matrix by a scalar simply means to add corresponding
row vectors or multiply all rows by the scalar respectively.
Thus we immediately get from Propositions~\ref{addvec} and
\ref{multvec}:

\begin{Cor}[Matrix addition and multiplication of matrices by scalars]
We assume a machine with $32$ bit word length. Let $M$ and $M'$ be 
two compact matrices with $m$ rows and $n$ columns over $\F_q$, where $q
= p^k$.

For $2 < p < 2^{31}$, the matrices $M$ and $M'$
can be added using $7km\cdot \lceil n/w \rceil$
word operations
(plus memory fetches and stores), where $w = \lfloor 32/e \rfloor$
and $e = \lceil \log_2(2p-1) \rceil$. 
For $p=2$, only $mk \cdot \lceil n/32 \rceil$ word operations are needed.

For $2 < p < 2^{31}$, the total cost of a multiplication of
$M$ by $s \in \F_p$ is at most 
$14km(e-2)\cdot \lceil 32/w \rceil$ word operations (plus memory fetches
and stores). For $p=2$, the scalar
can only be $0$ or $1$ and therefore no computation is necessary at all.

For $2 < p < 2^{31}$, the total cost of a multiplication of $M$
by $s \in \F_q$ is at most
$7k^2m(2e-3)\cdot \lceil 32/w \rceil$.
 For $p = 2$, the total cost is
at most $2k(k-1)m \cdot \lceil n/32 \rceil$ word operations.

For machines with $64$ bit word length, the number of word operations
is halved in both cases and we can work with primes $p < 2^{63}$.
\end{Cor}
\Proof This is immediate from Propositions~\ref{addvec} and \ref{multvec}.
\ProofEnd

\subsection{Vector-matrix multiplication}
\label{ssec:vecmat}

This section is about the multiplication of a row vector 
$v \in \F_q^{1 \times m}$ by a matrix $M \in \F_q^{m \times n}$. The result
is a vector $vM \in \F_q^{1 \times n}$. Since by convention we are working 
mainly with row vectors, this is the only operation between vectors
and matrices we have to consider. The left multiplication of a column
vector by a matrix can be simulated using the transposed matrix.

The multiplication $vM$ can be understood as taking a linear combination
of the rows of $M$, the vector $v$ just contains the coefficients. This
kind of reasoning already leads to the most efficient way to implement
this operation: We just multiply the $i$-th row of $M$ by the scalar
$v_i$ and add the product vector to the final result for $1 \le i \le m$.
Thus this operation can be done using at most $m$ multiplications
of a vector by a scalar plus $m-1$ vector additions.

If we are considering only one vector-matrix multiplication this is basically
all one can say. However, in a situation in which we apply one given
matrix repeatedly to lots of vectors (for example during matrix 
multiplication (see below) or when enumerating orbits), there is a
trick called ``grease'', which was invented by Richard Parker and 
probably others.

The idea is that if the base field is small, there are not so many
different linear combinations of a finite number of vectors. Thus,
we can distribute the rows of our matrix into small groups, say of
$\ell$ adjacent rows each, and precompute all possible linear combinations of
all vectors in each group. If we now want to multiply a vector $v$ and the 
matrix, we extract the entries of $v$ corresponding to each group,
look up the linear combination and add it to the result. The number
$\ell$ of elements in each group is called ``grease level''.

\begin{figure}[ht]
\begin{center}
\input{grease.pstex_t}
\end{center}
\caption{Illustration for Vector-matrix multiplication and grease}
\label{grease}
\end{figure}

This technique is illustrated pictorially in Figure~\ref{grease}. There
one can see a vector-matrix multiplication. The vector is divided into
sections of length $\ell$, beginning from the left, leaving a section of
possibly less than $\ell$ elements at the right hand side. Each such section
corresponds to $\ell$ (or less at the end) adjacent rows in the matrix, we 
call such a submatrix a ``grease block''.
To do grease level $\ell$, we compute all possible linear combinations of
each block of $\ell$ rows and store them. If we have computed and stored
all this data, we call the matrix ``greased''.
Then a vector-matrix multiplication
with this matrix only has to add $\lceil m/\ell-1 \rceil$ vector additions
of looked up vectors. More precisely, we have the following result:

\begin{Theo}[Grease: expected cost and gain]
\label{theogrease}
Let $M$ be a matrix with $m$ rows and $n$ columns over the field\/ $\F_q$
for $q = p^k$, and let $\ell \in \Z_{> 0}$.

Then greasing the matrix $M$ with grease level $\ell$ multiplies the amount
of memory needed for $M$ by at most $q^\ell$ and costs at most 
$\lceil m/\ell \rceil \cdot (q^\ell-k\cdot \ell -1)$ row vector additions 
and $m \cdot (k-1)$ multiplications of a row vector with a primitive
root of $\F_q$.

If $M$ is greased, a vector $v$ can be multiplied from the right with $M$
using only $\lceil m/\ell -1 \rceil$ row vector additions plus the 
cost of extracting the finite field elements from the vector $v$, which
is usually neglectible.

The standard approach for a vector-matrix multiplication needs 
approximately $m(q-1)/q$ multiplications of a row vector by a non-zero
scalar from $\F_q$ plus $(m-1)(q-1)/q$ row vector additions, assuming that 
about every $q$-th element of $v$ is equal to zero.
\end{Theo}
\Proof We first discuss the precomputation step to compute the greased
matrix. Since there are $q^\ell$ different linear combinations of $\ell$ 
vectors the statement about the needed memory is clear.

We have to do the following for every grease block and thus 
all costs are multiplied by $\lceil m/\ell \rceil$. 
After multiplying each vector in the
block $k-1$ times by the primitive root of $\F_q$ represented by the
polynomial $x \in \F_p[x]$ and storing the intermediate results, 
we can compute all $\F_q$-linear combinations
of the vectors in the block by computing all $\F_p$-linear combinations
of the vectors we already have. Thus, by Lemma~\ref{alllinkomb} below we can
compute all those linear combinations with 
$p^{k\cdot \ell} - k\cdot \ell - 1 = 
q^\ell - k \cdot \ell - 1$ row vector additions, which is the cost in the
theorem. Note that we compute here the scalar multiplications by the
primitive root of $\F_q$ as full multiplication although it can be done
more efficiently since the element represented by the polynomial $x$
is sparse, such that the scalar multiplication with it is faster.

After having greased the matrix, a multiplication of a vector with
the matrix can be done by adding appropriate vectors. We have to add
one for every grease block, leading to $\lceil m/\ell - 1\rceil$ additions.
\ProofEnd

\begin{Rems}
\begin{itemize}
\item The value $(2m-1)(q-1)/q$ for the standard approach is a
good estimate for practical considerations about when to grease and when
not. It can be replaced by the upper bound $2m-1$ neglecting
possible zero entries in the vector.
\item It is crucial to implement the extraction of entries
from the vector $v$ as efficient as possible. If this is not done
right, the cost of extracting elements in one vector can cancel out
the benefit of fewer scalar multiplications.
\item The fact that we need very few scalar multiples in the
precomputation phase and none in the vector times matrix phase is
very useful with respect to our new implementation in the
{\cvec} package, since there additions are sometimes much cheaper
than multiplication with scalars.
\end{itemize}
\end{Rems}
 
The following lemma is used in the proof of Theorem~\ref{theogrease}:

\begin{Lemm}[Computing all\/ $\F_p$-linear combinations]
\label{alllinkomb}
Let $v_1, \ldots, v_{j}$ be $\F_p$-linearly  
independent vectors over a finite field $\F_q$ with $q = p^k$
for some $j \in \Z_{> 0}$. If
all $\F_p$-linear combinations of $v_1, \ldots, v_{j-1}$ are already
computed, then all $\F_p$-linear combinations of $v_1, \ldots, v_j$ can
be computed with $p^j - p^{j-1} - 1$ vector additions.

This implies in particular that all $\F_p$-scalar multiples of a vector
can be computed with $p-2$ vector additions, and inductively, that
all $\F_p$-linear combinations of $v_1, \ldots, v_j$ can be be computed
without assumption using $p^j-j-1$ vector additions.
\end{Lemm}
\Proof All non-zero $\F_p$-scalar multiples of $v_j$ can be computed by
successively adding $v_j$ altogether $p-2$ times. Then every $\F_p$-linear
combination of $v_1, \ldots, v_j$ is either one of the already
computed $\F_p$-linear combinations of $v_1, \ldots, v_{j-1}$ or the
non-zero multiples of $v_j$, or a sum of two such vectors. Thus, we 
can compute all missing $p^j - p^{j-1} - (p-1)$ vectors with one
vector addition each, which proves the first statement. 

This includes the statement for $j=1$, which handles the case of computing
all $\F_p$-scalar multiples of one vector. Without assumtions, we can
apply the first statement inductively for $i=1, 2, \ldots, j$ proving
the last statement, since
\[ \sum_{i=1}^j (p^i - p^{i-1} - 1) = p^j - j - 1. \]
\ProofEnd

\begin{Rem}
Thus computing all $\F_p$-linear combinations can be done
with one addition per vector that is not already given, assuming
the zero vector and the input vectors to be already there.
\end{Rem}

\subsection{Matrix multiplication}

Let $M \in \F_q^{m \times n}$ and $N \in \F_q^{n \times s}$ with $q = p^k$.
The matrix multiplication $M \cdot N$ can be computed by multiplying the
rows of $M$ from the right with the matrix $N$ and putting the $m$ resulting
rows of length $s$ into the resulting matrix in $\F_q^{m \times s}$.

There are at least three methods to improve the performance of this
computation: The first is using grease (see the previous section), which
we will analyse in detail in this section. 

The second is using
methods to arrange the same computations in a different order to 
achieve that more memory fetches and stores are handled by the second
level cache. This approach is used in some implementations and can
speed up computations. However, we do not want to go into detail here.
For some discussion of this topic see Section~\ref{sec:cache}. 

The third method is to use the methods of Strassen and Winograd to
get a lower exponent than $3$ in the complexity of matrix multiplication
of square matrices.

We now discuss grease before we analyse for which sizes of
matrices Strassen/Winograd is worthwhile.

When the matrices we want to multiply are big enough such that the number
of rows of $M$ is large in comparison to $q^\ell$
for a suitable chosen grease level $\ell$, it is not necessary to grease
the complete matrix $N$ before beginning the multiplication. Rather,
we can proceed with the whole multiplication "`by grease blocks"', that
is, we compute in a first step all possible linear combinations of the
first $\ell$ rows of $N$, then run through all rows of $M$ looking
only on the first $\ell$ entries, and add each one looked up
linear combination of the rows of $N$. After that we can forget
our precomputed linear combinations and proceed to the next grease block
in $N$ and so on.

The implementation in the {\cvec} package decides about whether it
should use this method by looking at the number of rows of $M$. It compares
the numbers and
estimating whether $m/\ell$ row vector additions

\begin{figure}[ht]
\begin{center}
\includegraphics[width=14cm]{matmulF2_all}
\end{center}
\caption{Matrix multiplication times over $\F_2$ against matrix dimension}
\end{figure}

\begin{figure}[ht]
\begin{center}
\includegraphics[width=14cm]{matmulF2_low}
\end{center}
\caption{Matrix multiplication times over $\F_2$ against matrix dimension}
\end{figure}


\subsection{Matrix inversion}

Greasing.

\subsection{Semi echelonisation}

Cleaning, spinning.

\section{Cache issues}
\label{sec:cache}

Expected performance gain, problems.

\section{Characteristic and minimal polynomial}
\label{sec:charminpoly}

\section{Basic algorithms for matrices}
\label{sec:basalgmat}

Evaluation of polynomials at matrices, order of an invertible matrix.


% this is a part of the habilitation thesis of Max Neunhoeffer

\chapter{Composition trees}

After lots of preparations in the previous chapters
in this chapter the main part of the whole book begins. We begin to talk
about the recognition of matrix groups. We start by formulating
the concept of ``constructive recognition'' and explain the reasoning
behind it in Section~\ref{constrrecog}. In this section we develop
a preliminary formulation of the main problem which will be refined
in the next Section~\ref{recapproach}. There we explain the fundamental
approach to achieve
constructive recognition by means of a ``composition tree''.
The refinement of the formulation of the main
problem there is necessary to allow for an efficient recursive solution.
We chose to ``develop'' the problem formulation in this way in the hope
to provide both a gentle introduction as well as a precise final
formulation with good reasons for its complex structure.

In the following sections we develop a
framework for group recognition that is not only suitable for matrix
groups and projective groups but also allows for the implementation of
the asymptotically best algorithms to handle permutation groups. In
this framework we can switch between different representations of
groups within one composition tree which allows for example to use 
permutation group methods during matrix group recognition, provided
we find some set our matrix group is acting upon.

The general approach to build a composition tree is not new and is
already described in \cite{MatGrpProj}. What is new in our approach
is first the abstraction to allow for different representations
intermixed in the same composition tree and secondly the fact that
we change the generating set in each node of the composition tree
which dramatically decreases the length of the resulting straight
line programs. 

The contents of this chapter stem from joint work with \'Akos Seress
and are an elaboration on the article \cite{AkosMaxISSAC}.

\section{Constructive recognition}
\label{constrrecog}

There are at least two fundamentally different ways to represent groups on
a computer. The first uses a presentation of the group and then
expresses group elements as words in a free group representing
cosets of the normal subgroup generated by the relations. The second
approach uses an ambient group whose elements can be represented and
multiplied directly on the computer, and the group is defined by giving
a list of generators. As ambient groups one can use symmetric groups, general
linear groups or projective groups, since we can store and manipulate
permutations and matrices efficiently on a computer.

In this book we concentrate on the second approach. To formalise our
problem we first write down our assumptions about the ambient group
and then formulate the fundamental problem.

\begin{Hyp}[Ambient group]
\label{ambient}
When we speak about an \emph{ambient group} we mean a finite group that can
be represented on a computer such that we can perform the following tasks:
\begin{itemize}
\item Store and compare group elements using a bounded amount of memory
per element.
\item Multiply group elements.
\item Invert group elements.
\item Compute the order of a group element.
\end{itemize}
\end{Hyp}

\begin{Rem}
All finite symmetric groups fulfil the hypotheses in Section~\ref{ambient}.
We call subgroups of finite symmetric groups \emph{permutation groups}.

For a prime power $q$ the groups $\GL(n,q)$ and $\PGL(n,q)$ also fulfil
the hypotheses. We call subgroups of $\GL(n,q)$ \emph{matrix groups}
and subgroups of $\PGL(n,q)$ \emph{projective groups}.
\end{Rem}

\begin{Problem}[Constructive recognition --- first formulation]
\label{ProbCR1}
Let $\GG$ be an ambient group in the sense of Hypothesis~\ref{ambient} and 
assume that we are given a generating tuple $(g_1, \ldots, g_k) \in
\GG^k$ for a group
$G := \left< g_1, \ldots, g_k \right> \le \GG$. 

We say that we have \emph{recognised $G$ constructively} if we have 
computed $|G|$ and
prepared a procedure that does the following: Given $g \in \GG$,
decide whether $g \in G$ and if so, express $g$ as a straight line program
in $(g_1, \ldots, g_k)$. This latter procedure is called \emph{constructive
membership test}.
\proofend
\end{Problem}

Note that this formulation will be refined in Sections~\ref{ProbCR2} 
and~\ref{ProbCR3}.

\smallskip
The problem as posed in Problem~\ref{ProbCR1} can be solved easily by just
enumerating the finite group $G$ completely. However, this is neither
a sensible approach nor very interesting. The crucial point missing
in Problem~\ref{ProbCR1} is that we want to do constructive recognition
\emph{efficiently}. To express what we mean by that we use complexity
theory. The class of problems is given by all possible tuples of
generators of subgroups of our ambient groups. We have to specify
what we mean by ``input size'':

\begin{Def}[Input size parameters]
\label{inputsize}
Let $\GG$ be an ambient group in the sense of Hypothesis~\ref{ambient}
and $(g_1, \ldots, g_k) \in \GG^k$. If $\GG$ is either $\GL(n,q)$ or
$\PGL(n,q)$, then the input size of our problem is measured by $n$, $k$
and $\log_2(q)$. If $\GG$ is the symmetric group on $n$ points then the
input size is measured by $n$ and $k$. We silently set $q := 1$ in that
case such that we can uniformly speak of the input size parameters $n$, $k$
and $q$ in all cases.
\end{Def}

\begin{Rem}
Note that we use $\log_2(q)$ as one of the parameters of the input size
rather than $q$ itself. The reason behind this is that one needs
$\log_2(q)$ bits of data to store one finite field element of $\F_q$.
This means that the amount of storage needed for the input matrices is
proportional to $kn^2\log_2(q)$. Of course, the fact that the discrete
logarithm problem in $\F_q$ prevents us from finding a real polynomial
time algorithm for constructive recognition stems from this decision
(compare Section~\ref{thedlp}).
\proofend
\end{Rem}

There is another reason why the formulation in Problem~\ref{ProbCR1} is not
practical. Namely, the straight line programs for elements $g \in G$
written in terms of the original generators can be extremely long. In
addition, for most generating tuples of $G$ it can be tedious if not
impossible to find a method to express group elements in $G$ as straight
programs in these generators. Usually, a recognition procedure first finds
another ``nice'' generating tuple for $G$ for which a good constructive
membership testing procedure is available. To solve the original problem,
such recognition procedures remember, how they got the nice generating
tuple from the original generating tuple by means of a single, rather long
straight line program.

With the above notion of efficiency and these arguments in mind we can now 
formulate a preliminary version of the constructive recognition problem:

\begin{Problem}[Constructive recognition --- preliminary formulation]
\label{ProbCR2}
Let $\GG$ be an ambient group in the sense of Hypothesis~\ref{ambient} and 
assume we are given a generating tuple $(g_1, \ldots, g_k) \in
\GG^k$ for a group $G := \left< g_1, \ldots, g_k \right> \le \GG$. 

We say that we have \emph{recognised $G$ constructively} if we have 
computed $|G|$ and a
generating tuple $( g'_1, \ldots, g'_l )$ for $G$, for which we have
prepared a procedure that does the following: Given $g \in \GG$,
decide whether $g \in G$ and if so, express $g$ as a straight line program
in $(g'_1, \ldots, g'_l)$. The new generating tuple may or may not be the
same as the original one.

We assume that we have factorised all integers $q^i-1$ for $1 \le i \le n$
and that we can solve the discrete logarithm problem in all fields
$\F_{q^i}$ for $1 \le i \le n$ efficiently. 

We call the first phase until $G$ is
recognised constructively the \emph{recognition phase} and the second
phase the \emph{(constructive) membership test phase}.

The aim is that the algorithms for both phases have a runtime that is proved
to be bounded by a fixed polynomial in the input size parameters $n$, $k$
and $q$ in the sense of Definition~\ref{inputsize}.
\proofend
\end{Problem}

Note that this formulation will be refined in Problem~\ref{ProbCR3}.

\begin{Rem}[Randomised algorithm]
We are content with using Las Vegas or Monte Carlo algorithms (see
Section~\ref{montevegas}) provided that we can later on verify our results
deterministically, even if this takes longer than the randomised
recognition step. However, also the verification phase should have a
runtime that is proved to be bounded by a fixed polynomial in the input
size parameters $n$, $k$ and $q$.
\end{Rem}

\section{A recursive approach --- reductions}
\label{recapproach}

Traditionally, the constructive recognition problem for permutation
groups is solved by computing a stabiliser chain and a set of strong
generators using the Schreier-Sims method (see~\cite{Si} and \cite{Ser}). 

Although an analogous method can be used for matrix and projective groups,
this becomes infeasible already for very small values of $n$ since the
occurring orbit lengths can be very large. Also for large base permutation
groups (for a formal definition see Section~\ref{permgrps}) like the
symmetric and alternating groups in their natural representation there
are asymptotically better methods because the stabiliser chains are very long.

Therefore one seeks \emph{reductions} in these cases. A reduction is
a surjective group homomorphism $\varphi : G \to H$ that can be computed
explicitly and for which $H$ is again a subgroup of a possibly different
ambient group. In addition, the homomorphism $\varphi$ must either have
a non-trivial kernel or the group $H$ must be ``easier to handle'' in some
sense. This can for example mean that one of the values $n$, $k$ and
$q$ is smaller for $H$ than for $G$ while the others remain unchanged,
or that the expression $kn^2\log_2(q)$ becomes smaller such that we need
less memory to store the generators for $H$ than those for $G$.
Another possibility is that for a matrix group $G$ the homomorphic image
$H$ is a permutation group.

Having found a reduction $\varphi : G \to H$ one first tries to recognise
$H$ constructively, then computes generators for the kernel $N$ of
$\varphi$, recognises $N$ constructively and then puts everything together
to achieve a constructive recognition of $G$ from all other intermediate
results. This procedure can be repeated recursively for the image $H$ and
the kernel $N$ such that we end up with a tree (see
Figure~\ref{comptreefig}), where the nodes correspond
to subfactors of the original $G$ and where we have to recognise
the groups in the leaf nodes constructively by other means without
a further reduction.

\begin{figure}
\begin{center}
\includegraphics[width=2.5in]{comptree}
\end{center}
\caption{A composition tree}
\label{comptreefig}
\end{figure}

To make this recursion work we have to say quite carefully what has to
happen at a non-leaf node $G$ such that we actually have solved the
constructive recognition problem for the group $G$ at this node provided
we have solved the corresponding problem for the homomorphic image
$H=\varphi(G)$ and the kernel $N = \ker(\varphi)$. Note that the problem
in our first formulation in Problem~\ref{ProbCR1} could be solved
recursively in this way. However, for our formulation in Problem~\ref{ProbCR2}
this is not possible since we can only express group elements in the nice
generators and not in the original ones. This is the reason why we have to
refine the formulation once more. This time we present the final
formulation:

\begin{Problem}[Constructive recognition --- final formulation]
\label{ProbCR3}
Let $\GG$ be an ambient group in the sense of Hypothesis~\ref{ambient} and 
assume that we are given a generating tuple $(g_1, \ldots, g_k) \in
\GG^k$ for a group $G := \left< g_1, \ldots, g_k \right> \le \GG$. 

We say that we have \emph{recognised $G$ constructively} if we have 
\begin{itemize}
\item computed $|G|$ and
\item a generating tuple $( g'_1, \ldots, g'_l )$ for $G$, for which we have
\item 
prepared a procedure that does the following: Given $g \in \GG$,
decide whether $g \in G$ and if so, express $g$ as a straight line program
in $(g'_1, \ldots, g'_l)$. Also, we have
\item prepared a procedure that computes, given preimages $(p_1, \ldots,
p_k)$ of $(g_1, \ldots, g_k)$ under some (surjective) homomorphism $\psi
: \hat G \to G$ for some group $\hat G$, preimages under $\psi$ of the
nice generators $(g'_1, \ldots, g'_l)$.
\end{itemize}
The new generating tuple may or may not be the same as the original one.
The last point can for example be achieved by storing a straight line
program that expresses $(g'_1, \ldots, g'_l)$ in terms of $(g_1, \ldots,
g_k)$.

We assume that we have factorised all integers $q^i-1$ for $1 \le i \le n$
and that we can solve the discrete logarithm problem in all fields
$\F_{q^i}$ for $1 \le i \le n$ efficiently. 

We call the first phase until $G$ is
recognised constructively the \emph{recognition phase} and the second
phase the \emph{(constructive) membership test phase}.

The aim is that the algorithms for both phases have a runtime that is proved
to be bounded by a fixed polynomial in the input size parameters $n$, $k$
and $q$ in the sense of Definition~\ref{inputsize}.

In this book we refer to this problem as the \emph{constructive recognition
problem (CRP)}.
\proofend
\end{Problem}

We claim that this final formulation of the constructive recognition
problem has the property that if $\varphi : G \to H$ is a reduction, then
having solved the constructive recognition problem for $H$ and $N :=
\ker(\varphi)$ in fact suffices to solve the constructive recognition
problem for $G$. To this end we have to
clearly say what the original generators and the new, ``nice'' generators
of our groups $G$, $H$ and $N$ are. This is the purpose of the following
definition.

\begin{Def}[Reduction node]
\label{reducnode}
Let $G = \left< g_1, \ldots, g_k \right> \le \GG$ be as in 
Problem~\ref{ProbCR3}.
We say that we have set up a \emph{reduction node} for $G$, if we have
performed the following steps:
\begin{itemize}
\item[(1)] Finding a surjective group homomorphism $\varphi : G \to H$ for
which we have a procedure to compute $\varphi(g)$ for all $g \in G$.
If this procedure is given an element $g \in \GG \setminus G$ then it
returns either {\fail} or an arbitrary element in the ambient group of $H$.
\item[(2)] Solving Problem~\ref{ProbCR3} for $H = \left< \varphi(g_1),
\ldots, \varphi(g_k) \right>$ with nice generators $(h_1, \ldots, h_s)$.
\item[(3)] Computing preimages $(h'_1, \ldots, h'_s) \in G^s$ under $\varphi$
of the nice generators $(h_1, \ldots, h_s)$ of $H$ using the solution
of (2).
\item[(4)] Computing generators $(n_1, \ldots, n_t) \in N^t$ with $N :=
\ker(\varphi)$ together with a straight line program expressing
$(n_1, \ldots, n_t)$ in terms of $(g_1, \ldots, g_k)$.
\item[(5)] Solving Problem~\ref{ProbCR3} for $N = \left< n_1, \ldots, n_t
\right>$ with nice generators $(n'_1, \ldots, n'_u) \in N^u$.
\end{itemize}
For a reduction node we call the tuple $(h'_1, \ldots, h'_s, n'_1, \ldots,
n'_u)$ its nice generators.
\end{Def}

\begin{Prop}[The recursion works]
Let $G = \left< g_1, \ldots, g_k \right> \le \GG$ be as in 
Problem~\ref{ProbCR3}.
If we have set up a reduction node for $G$ (in the sense of
Definition~\ref{reducnode}), then we have solved
Problem~\ref{ProbCR3} for $G$.
\end{Prop}
\proofbeg
The group order $|G|$ is equal to the product $|H| \cdot |N|$ since
$\varphi$ is surjective and $N = \ker(\varphi)$.

Recall that we call the tuple $(h'_1, \ldots, h'_s, n'_1, \ldots, n'_u)$
the nice generators for $G$. Since 
\[ (\varphi(h'_1), \ldots, \varphi(h'_s))
= (h_1, \ldots, h_s) \] 
generates the epimorphic image $H$ of $G$ and
$(n'_1, \ldots, n'_u)$ generates the kernel $N$ the tuple of nice
generators is a generating tuple for $G$.

Given $g \in \GG$, we can test membership in $G$ constructively as follows:
First we try to map $g$ with $\varphi$. If this fails, the element $g$
does not lie in $G$ by our assumptions in \ref{reducnode}.(1). Otherwise
call the image $h \in H$. Note that in this case it can still be that
$g$ is not contained in $G$, regardless whether $h$ lies in $H$ or not! 

We then try to express $h$ as a straight 
line program $s_1$ in the nice generators $(h_1, \ldots, h_s)$ of $H$. If this
fails, then $h$ is not contained in $H$ and thus $g$ is not contained in
$G$. Otherwise we evaluate $s_1$ with inputs
$(h'_1, \ldots, h'_s)$, the preimages of the nice generators of $H$ under
$\varphi$. If $g$ is contained in $G$, then the result $h'$ is a preimage 
under $\varphi$ of $h = \varphi(g)$. Thus $n := h'^{-1}\cdot g$ is contained
in the kernel $N$ of $\varphi$. We try to express $n$ as a straight
line program $s_2$ in $(n'_1, \ldots, n'_u)$. If this fails, then $n$
is not contained in $N$ and thus $g$ is not contained in $G$. Otherwise,
we put together the straight line programs $s_1$ and $s_2$ to form
one big straight line program $s$ that reaches $h'n$ from the input 
$(h'_1, \ldots, h'_s, n'_1, \ldots, n'_u)$. In this case we have proved
that $g$ lies in $G$.
\proofend

\medskip
We conclude this section with a short overview over the rest of the chapter
and an outlook onto the following ones:
We have now found our final formulation of the constructive
recognition problem and we have shown a general approach for its solution
by means of reduction. In the next Section~\ref{findkernel} we explain how
to find the kernel $N$ of a homomorphism as in Definition~\ref{reducnode}
once we have constructively recognised the epimorphic image $H$. After that
in Section~\ref{findhom} we describe a generic framework to organise the
finding of reductions using a database of different methods. Finally we
explain how this whole framework can be used to implement the
asymptotically best algorithms for arbitrary permutation groups.
The following chapters cover the question how to actually find reductions
and solve the constructive recognition problem for matrix groups and
projective groups.

\section{Finding the kernel of a homomorphism}
\label{findkernel}

In this section we assume that we are in the situation described in
Problem~\ref{ProbCR3} and want to construct a reduction node in the sense
of Definition~\ref{reducnode}. We assume that we can produce evenly
distributed random elements in $G$ (see Section~\ref{randomelts}).

Assume that we have already found a homomorphism
$\varphi : G \to \HH$ into some ambient group $\HH$ in the sense of
Hypothesis~\ref{ambient}, such that we can map arbitrary group elements
$g \in G$ with $\varphi$. Then we set $H := \left< \varphi(g_1), \ldots,
\varphi(g_k) \right>$ and constructively recognise $H$ which produces
among other things nice generators $(h_1, \ldots, h_s)$ of $H$. Another
result of the recognition is that we can compute preimages $(h'_1, \ldots,
h'_s) \in G^s$ of the $(h_1, \ldots, h_s)$ under $\varphi$.

Equipped with all this data and methods we can now do the following
for an arbitrary element $g \in G$: Map it with $\varphi$ into $H$
and call $h := \varphi(g)$. Then find a straight line program $s$ reaching
$h$ from $(h_1, \ldots, h_s)$ and evaluate it on $(h'_1, \ldots, h'_s)$
to get an element $h' \in G$ with the property that
$\varphi(g)=\varphi(h') = h$, thus $h'^{-1}\cdot g$ lies in the kernel $N$
of $\varphi$. For this construction we have:

\begin{Prop}[Even distribution of kernel elements]
\label{evendistker}
If $x_1, x_2, \ldots, x_m$ are evenly distributed random elements in $G$,
then the above procedure produces evenly distributed random elements in
$N$.
\end{Prop}
\proofbeg
Since $H \cong G/N$, the elements of $H$ correspond to cosets of $N$ in
$G$. Our constructive membership test in $H$ produces a unique straight
line program for every element $h \in H$ and the evaluation of those
programs on the preimages $(h'_1, \ldots, h'_s)$ thus chooses exactly one
element of $G$ from each coset of $N$. Since the $x_i$ are evenly
distributed in the whole of $G$ their images under $\varphi$ are
distributed evenly in $H$. Multiplying $x_i$ with the inverse of the
chosen coset representative in the coset $x_iN$ amounts to mapping
$x_i N$ bijectively onto $N$. Again by the even distribution of the $x_i$
it follows, that the elements $y_i$ are evenly distributed in $N$.
\proofend

\smallskip
Proposition~\ref{evendistker} now allows to produce evenly distributed
elements in $N$. To find a generating tuple of $N$ we simply produce
a certain amount of random elements. Since every proper subgroup of $N$ 
has index at least $2$ in $N$, the probability to increase the
subgroup that is generated by the already produced elements is at least 
$1/2$ for every new element. Thus the probability to find a generating
set for the whole of $N$ is very high provided we produce enough random
elements.

\medskip
Sometimes, we not only recognise $H$ constructively in the sense of
Problem~\ref{ProbCR3} but also find a presentation in terms of the 
nice generators. This happens mostly but not exclusively if $H$ is an 
almost simple group. In that case, we have the following result:

\begin{Prop}[Kernel if a presentation of $H$ is known]
\label{kernelpres}
Assume that in the above situation $H$ is isomorphic to the
finitely presented group generated by generators $(h_1, \ldots, h_s)$
subject to relations $(r_1, \ldots, r_m)$ given as a straight line
program $s$ in terms of the $(h_1, \ldots, h_s)$. Let $(r'_1, \ldots,
r'_m)$ be preimages of the $(r_1, \ldots, r_m)$ under $\varphi$. 
Let further $x_i$ for $1 \le i \le k$ be elements in $\left< h'_1, \ldots, 
h'_s \right>$ with $\varphi(x_i) = \varphi(g_i)$ obtained by the procedure 
described directly before Proposition~\ref{evendistker}, and set 
$y_i := x_i^{-1} \cdot g_i \in N$. Then the kernel $N$ of
$\varphi$ is the normal closure in $G$ of the group generated by
$(y_1, \ldots, y_k, r'_1, \ldots, r'_m)$.
\end{Prop}
\proofbeg
Clearly, all the elements $y_i$ and $r'_i$ lie in $N$. Since $N$ is a
normal subgroup of $G$, the normal closure $\tilde N$ of the group generated
by $(y_1, \ldots, y_k, r'_1, \ldots, r'_m)$ is contained in $N$.

On the other hand, every element $n$ of $N$ is a product of the generators
$g_i$. Thus, setting $n = g_{i_1} \cdots g_{i_t}$ for some numbers
$i_j \in \{ 1, \ldots, k \}$, we get:
\[ n = x_{i_1} x_{i_1}^{-1} g_{i_1} \cdots x_{i_t} x_{i_t}^{-1} g_{i_t}
     = x_{i_1} y_{i_1} \cdots x_{i_t} y_{i_t}
     = \tilde n \cdot x \]
where $\tilde n$ is a product of some $G$-conjugates of the $y_{i_j}$
and $x$ is a product of the $h'_i$ that lies in $N$. Since the $h'_i$
are preimages of the generators $h_i$ of $H$ and $r'_i$ are preimages
of the relations $r_i$ the element $x$ lies in the normal closure
of the subgroup generated by the $r'_i$ and thus $\tilde N = N$.
\proofend

\begin{Rem}[Computing the kernel with a given presentation of $H$]
Using the results of the constructive recognition of $H$ together we
can explicitly compute preimages $r'_i$ of the relations $r_i$ since
the latter are given as straight line programs in terms of the nice
generators of $H$. Thus we can explicitly compute all the generators in
Proposition~\ref{kernelpres}. The normal closure can be computed using
the methods in \cite[Chapter 2]{Ser}.
% FIXME: add reference
\end{Rem}

In this section we have shown that to set up a reduction node only needs
an explicitly computable homomorphism $\varphi$ from $G$ into some
ambient group $\HH$. Given this, one can constructively recognise the
image, compute generators for the kernel and constructively recognise it,
thereby fulfilling all requirements for a reduction node.


\section{Finding homomorphisms and nice generators}
\label{findhom}

During the recursive recognition procedure we have to solve
Problem~\ref{ProbCR3} at every node of the composition tree, either
by finding a reduction and setting up a reduction node or by solving 
the constructive recognition problem directly. In this section we describe
a generic framework to organise an algorithm to achieve this.

We have to explore every group $G$ as in Problem~\ref{ProbCR3} 
occurring in our tree to find out whether we 
can solve the constructive recognition 
problem, or what kind of homomorphism can be 
applied to it. To this end, the framework holds a collection
of so-called ``find homomorphism'' methods in stock. A find homomorphism
method's objective is either to find a homomorphism $\varphi: G \to H$
onto some subgroup $H$, thereby setting up a reduction node and 
creating a new non-leaf node, or to
solve the constructive recognition 
problem directly, which can but does not always have to find
an isomorphism to some known group.

For each type of groups (permutation groups, matrix groups, and black-box
groups), the system has a database of find homomorphism methods.
We call the procedure that decides, which methods to try and in which order 
the ``method selection''. For this purpose we define a very simple and yet 
versatile algorithm, which we will describe now. It is usable independently
from group recognition, but we will explain it here in the context of
our generic recognition procedure. Note that this new method selection
procedure is not to be confused with the {\GAP} (see \cite{GAP4}) method
selection. 

The group recognition procedure for a node just calls the generic method
selection procedure with the database of find homomorphism methods
corresponding to the type of the group.

The methods in each database are ranked, thereby defining a total
order. The method selection procedure calls the methods one after another,
starting with high ranks. A find homomorphism method reports back to
the generic procedure by returning one of the four values in
Table~\ref{methselresults}.

\begin{table}[ht]
\begin{tabular}{lp{4in}}
\texttt{true}: &
   The method was successful and has either set up a
   leaf or a non-leaf node. For details see below. \\
\texttt{fail}: &
   The method has failed to find a homomorphism or
   to solve the constructive recognition problem, at least temporarily. \\
\texttt{false}: &
   The method has failed and will do so always for
   the group $G$ in question, such that there is no point in trying
   this method again for the group $G$. \\
\texttt{NotApplicable}: &
   The method is currently not applicable
   but it might become applicable later, provided new knowledge is
   found out about the group $G$.
\end{tabular}
\caption{Possible results of a find homomorphism method}
\label{methselresults}
\end{table}

The first case is the only one that terminates the recognition procedure
for the current node in the composition tree.
All other cases make it necessary to try other methods. The difference
between these latter three cases lies in the fact, how the generic
procedure chooses the next method called. If a method returns
\texttt{NotApplicable}, then the method selection just calls the next
method in the database. In the other two cases \texttt{false} and 
\texttt{fail}, the method selection again starts with the highest ranked
method, but skipping all methods that have already been tried and
have failed by returning either \texttt{false} or \texttt{fail}.

When all available methods either have declared themselves
\texttt{NotApplicable} or have failed, then the method selection 
starts all over again, now calling methods again that have returned
\texttt{fail} once but of course skipping methods that have returned
\texttt{false}. This whole process is repeated until each method has
failed a configurable number of times, when the method selection
finally gives up.

We hope that this design is simple enough to keep an overview of what is
tried in which order to prove the whole algorithm to work correctly,
and versatile enough to implement a wide range of different algorithms,
in which ``trying different methods'' is involved. We now explain
the rationale behind some of the features of this procedure.

The idea behind the fact that after a method having returned
\texttt{fail} or \texttt{false} the method selection starts again
with the highest ranked method is that even a failed method might
have found out new information about the group, thereby making
higher ranked methods, that have refused to work earlier by
returning \texttt{NotApplicable}, newly applicable.

In the {\GAP} system, information about a group $G$ is collected in
so-called {\em attributes} (for example,
a permutation group object may store an attribute whether it is transitive
or not), but we may also acquire the information that 
$G$ is simple, or that it is solvable, or we may know $|G|$, etc. 
Note that at any given time, the value of an attribute for a given group
object can already be computed or not, and the group object can learn new
information about itself during its lifetime. This feature is already
part of the {\GAP} system library. 

Further attributes may be
computed when the method selection tries to apply
different find homomorphism methods while processing the current group $G$.
Therefore, a find homomorphism method can just look whether or not
a certain attribute is already known and then decide if it starts
to work or declares itself \texttt{NotApplicable}. With new attributes
being computed, this decision might be changed. By convention, a method
should never use much computation time to find out that it is not
applicable.

The idea behind a method returning \texttt{fail} is that often
randomized algorithms are used that have the potential to fail,
but still may succeed when tried again. The ranking and the failure 
probabilities of course have to be tuned carefully to assemble
a sensible recognition system.

Once a method returns \texttt{true} it reports whether it has found a
reduction or has solved the constructive recognition problem by other
means. In the first case the generic machinery sets up a recognition node
as described in Sections~\ref{recapproach} and \ref{findkernel}. In the
other case a leaf in the composition tree is built and the find
homomorphism method that returned \texttt{true} is responsible for the
data and methods to do constructive membership testing in the group 
of the node. In this way, a composition tree is built recursively using the
database of find homomorphism methods together with the generic code
organising the recursive framework.

Another important feature of this framework is the following. Quite often,
a find homomorphism method finds out information about the group in
question that can be used further down in the composition tree most notably
in the homomorphic image or the kernel of the group homomorphism found.
For example it might already know that it has set up an isomorphism such
that the kernel is trivial or it might know that the homomorphic image of a
matrix group is a permutation group. To communicate such information to the
methods used further down in the tree there is an infrastructure to hand
down so-called ``hints''. These hints might for example consist of additional
knowledge about the group such that certain find homomorphism methods
are particularly well suited. Having this type of information available
further down in the tree helps choosing the best find homomorphism method
first. Furthermore, even a failed method can store already 
obtained data about the group in the data structure of the node it tried
to work on. In this way methods that are called later in the method
selection process can profit from this additional information. In the
chapters to come in this book we describe find homomorphism methods
and mention such hints that can help other methods along with the
description of the methods.

We illustrate the generic method selection described in this section
in the next Section~\ref{permgrps} where we explain how it is used to
glue together the asymptotically best algorithms currently known for
the handling of permutation groups to set up a constructive recognition
algorithm for the case that the ambient group is a symmetric group.


\section{Asymptotically best algorithms for permutation groups}
\label{permgrps}

Traditionally, the constructive recognition problem for permutation
groups is solved by computing a stabiliser chain and a set of strong
generators using the Schreier-Sims method (see~\cite{Si} and \cite{Ser}). 

However, this method does not work very efficiently for so-called
``large base groups''. We start by developing the necessary language
from the complexity theory of permutation groups.

\begin{Def}[Base of a permutation group]
Let $G \le S_n$ be a subgroup of the symmetric group on $\{1,\ldots,n\}$.
A tuple $B \in \{ 1, \ldots, n\}^l$ is called a \emph{base of $G$ with
length $l$}, if only the identity of $G$ fixes every point in $B$: 
\[ \{ g \in G \mid b_i^g = b_i \mbox{ for all } 1 \le i \le l \} = \{ \id \}. \]
The \emph{stabiliser chain} belonging to a base $B$ is the chain of
subgroups
\[ G \ge G_{b_1} \ge G_{b_1,b_2} \ge \cdots \ge G_{b_1,\ldots,b_k} =
\{\id\} \]
where $G_{b_1, \ldots, b_j} =
\{ g \in G \mid b_i^g = b_i \mbox{ for all } 1 \le i \le j \}$.
\end{Def}

Obviously, the length of every stabiliser chain for a permutation group 
$G$ is at least as big as the length of a shortest base of $G$.
This is the reason why computing stabiliser chains is not efficient
for groups without a ``reasonably short'' base. Of course, the
Schreier-Sims method does not necessarily find a shortest possible base.
However, one can implement the method such that in the resulting
stabiliser chain all inclusions are proper.
Therefore, we consider not the shortest possible base length but
an upper bound for the length of a base:

\begin{Prop}[Maximal base length]
If $G \le S_n$ then every base of $G$, for which all inclusions in its
stabiliser chain are proper, has length at most $\lceil \log_2(|G|) \rceil$.
\end{Prop}
\proofbeg
The index of one stabiliser in the previous one in the stabiliser chain is
at least $2$.
\proofend

To formulate precise complexity statements we have to talk about families
of permutation groups:

\begin{Def}[Families of small and large base groups]
Let $\calF$ be a family of permutation groups, all embedded into
possibly different symmetric groups $S_n$ and given by generating tuples. 
For a $G \in \calF$ we denote 
by $n(G)$ the $n$ of the symmetric group $S_n$ into which $G$ is embedded
and by $k(G)$ the number of generators by which $G$ is defined.

The family $\calF$ is called \emph{a family of small base groups} if there
is a positive constant $c$ such that $\log_2(|G|) \le \log^c_2(n(G))$ for all
$G \in \calF$. If there is no such constant then $\calF$ is called
\emph{a family of large base groups}.
\end{Def}

\begin{Rem}[A single permutation group]
Note that the terms ``large base group'' and ``small base group'' are
\emph{not defined} for a single permutation group $G$ although one is
sometimes tempted to call a member of a family of large base groups
a ``large base group''. However, complexity statements about algorithms
only make sense for families of inputs and thus it is only sensible to
talk about ``small'' or ``large'' bases in the context of families of
permutation groups.
\end{Rem}

The motivation for the terms ``family of large/small base permutation
groups'' is the following result:

\begin{Theo}[Complexity of the Schreier-Sims method]
Let $\calF$ be a family of small base groups. Then a randomised version
of the Schreier-Sims method computes a base and strong generators
for any $G \in \calF$ in nearly-linear time in the input size $k(G)\cdot
n(G)$, that is the runtime is bounded by a function in
$O(k(G)n(G) \cdot \log^c_2(n(G)))$ for some constant $c$.
\end{Theo}
\proofbeg See \cite{nearlylin} or \cite[Theorem 4.5.5]{Ser}. \proofend

\medskip
If a permutation group $G \le S_n$ is given by a list of generators, it
is at first glance not clear, whether it has a short base. Of course,
we want to avoid to compute a base by an application of the Schreier-Sims
method if there is no short base.

To solve Problem~\ref{ProbCR3} for a permutation group we use the method
selection procedure described in Section~\ref{findhom}. In the following we
describe the methods that are used in this process. An overview of these 
methods is given in Table~\ref{permdb}. The basic idea is to recognise
and handle alternating composition factors without computing a stabiliser 
chain.

\begin{table}[ht]
\begin{center}
\begin{tabular}{|c|l|l|c|}
\hline
Rank & Name & Action & Hom/Leaf \\
\hline
\hline
50 & \texttt{NonTransitive} & Restrict to orbit & Hom \\
\hline
40 & \texttt{Giant} $A_n/S_n$ & Find standard generators & Leaf \\
\hline
30 & \texttt{Imprimitive} & Find block system & Hom \\
\hline
20 & \texttt{Jellyfish} & Find standard generators & Leaf \\
\hline
10 & \texttt{StabilizerChain} & Compute a base and strong generators & Leaf \\
\hline
\end{tabular}
\end{center}
\caption{Find homomorphism methods for permutation groups}
\label{permdb}
\end{table}

The method with the highest rank $50$ is the reduction method called
\texttt{NonTransitive} for intransitive groups. It first tests whether
$G$ is transitive, and if not constructs an explicitly
computable homomorphism by restricting the action to one of the orbits in the
permutation domain. If $G$ turns out to be transitive, then the
\texttt{NonTransitive} method immediately returns \texttt{false}
indicating, that it can never be successful for the group $G$.

The next method \texttt{Giant} with rank $40$ tries to decide, whether the
group $G \le S_n$ is a ``giant'', that is, it is either equal to
$A_n$ or to $S_n$ itself. In these two cases, there are better methods than
computing a stabiliser chain to solve the constructive recognition problem.
These methods are described in detail in \cite[Section~10.2]{Ser}, see
in particular \cite[Section~10.2.4]{Ser}. The algorithm described there
is one-sided Las Vegas in the following sense:  If it recognises $A_n$ or
$S_n$, the result is guaranteed to be correct. If $G$ is not a giant, it
fails very quickly. With a small error probability it can fail even if
$G$ is equal to $A_n$ or $S_n$. Thus, the method \texttt{Giant} returns
\texttt{fail} in case of failure meaning that it could be called later
on to try again. However, in the setup described here this will not
happen, since later methods will always succeed, even if they take a
very long time.

The third method \texttt{Imprimitive} with rank $30$ tries to find a block
system using the algorithm described in \cite[Section~5.5.1]{Ser}. If it 
finds one, it constructs a homomorphism onto the action on the set of
blocks. Since the algorithm is deterministic, it returns \texttt{false}
if no block system is found and the group $G$ is proved to be primitive.

Note that in the case that the \texttt{Imprimitive} method finds a block
system, it hands the information about the blocks down to the node that is
set up for the kernel. This allows firstly to use a different method to
create generators for the kernel since one needs often much more elements
to generate a subgroup of a repeated direct product than usual. Secondly,
the system can use a special find homomorphism
method for the kernel that produces a balanced composition tree, which 
is more efficient both during the recognition phase and the membership
testing phase. Here, the infrastructure in our framework to hand down 
hints to kernel and homomorphic image of a homomorphism is used 
extensively.

The fourth find homomorphism method \texttt{Jellyfish} with rank $20$
handles another special case in which the Schreier-Sims method does
not work efficiently. This case is that an alternating or symmetric
group on $\{1,\ldots,m\}$ acts on $n = {m \choose k}^r$ $r$-tuples
of $k$-subsets of $\{1,\ldots,m\}$. Of course, the action is not
given in this way but simply as an action on the set $\{1,\ldots,n\}$
and the problem is to identify each number in $\{1,\ldots,n\}$ with
an $r$-tuple of $k$-subsets of $\{ 1, \ldots, m\}$. In fact, the
algorithm does not solve this latter problem but tries to guess
$m$, $k$ and $r$ and directly find standard generators for $A_m$ or
$S_m$. A deterministic algorithm to solve this problem is described
in \cite[Section~4]{fastmanag} and a faster randomised one in
\cite{Jellyfish}.

Finally, if everything that has previously been tried has failed, we
compute a base and strong generators using the Schreier-Sims method. This
is done by the method called \texttt{StabilizerChain} with rank $10$.

This setup tries the right things in the right order to handle groups
with small as well as with large bases and thus implements a framework to
handle all permutation groups with the asymptotically best algorithms
known.


% this is a part of the habilitation thesis of Max Neunhoeffer

\chapter{Leaves of the composition tree}
\label{chap:leaves}
\index{leaf}%

This chapter is about constructive recognition (see
Problem~\ref{ProbCR3}) of the leaves of a
composition tree (see Section~\ref{recapproach}). By the arguments in
Chapter~\ref{chap:findhom} this means that we have to be
able to solve Problem~\ref{ProbCR3} for the subgroups $G \le \GL(n,q)$
that lie in classes \DD8 and \DD9 (see Sections~\ref{descD8} and
\ref{descD9}). Because of the homomorphism from $\GL(n,q) \to
\PGL(n,q)$ and our setup of the composition tree we only have
to work with the projective version $\bar G \le \PGL(n,q)$.

We do not want to describe the details of the methods used here but
instead give an overview and refer the reader to the literature.
One reason for this is that the current state of the art does not seem
to be final, another is that the author has not contributed to the
development of algorithms in this area up to now.

This chapter is structured as follows. We begin by describing
methods based on permutation groups, which we call ``direct methods
for constructive recognition'' in Section~\ref{solvedirect}. The
next Section~\ref{nonconstructive} explains the concept of
``non-constructive recognition'' which means to determine
the isomorphism type of a given group. Once this is known, the
concept of standard generators applies, which is introduced in
Section~\ref{standardgens}. These concepts in turn are usually needed to apply
more specialised methods for constructive recognition, of which an
overview is given in Section~\ref{solveD8} for classical groups in
their natural representation and in
Section~\ref{solveD9} for almost simple groups.

\section{Direct methods for constructive recognition}
\label{solvedirect}

For ``small groups'' one can use permutation group methods to solve
Problem~\ref{ProbCR3} for a group $\bar G \le \PGL(n,q)$. The basic idea is
to find a permutation action. This immediately gives a homomorphism
into a permutation group which will be an isomorphism if the group is
simple. Even if there is a non-trivial kernel the composition tree setup
will take care of this.

Matrix groups and projective groups of course act on their natural
module and thus on vectors and subspaces. So, finding some
action is not difficult. Achieving good performance however can be
tricky. To this end we want to find short orbits. Heuristic methods
for this for matrix groups can be found in \cite{shortorbits}.

Another possibility is low index methods.
Here one would guess the point stabiliser of a point with a short
orbit, restrict the natural module to it and use the MeatAxe to find a
proper invariant subspace. The orbit of this subspace would then be
relatively short. However, although this approach looks promising and
occasionally works, it is not clear how to find generators of such a
point stabiliser, even with random methods.

Once we have an action and thus a homomorphism into a permutation 
group we can either use the methods described in Section~\ref{permgrps}
or immediately compute a stabiliser chain for the projective or matrix
group using the Schreier-Sims method (see \cite{nearlylin} or
\cite{Ser}). Both approaches solve the constructive recognition
problem \ref{ProbCR3} even if the group is not a simple group.

Note that for larger dimensions and in particular for classical groups
we do not even want to try these direct methods because any orbit we
can possible find would be prohibitively large.


\section{Non-constructive recognition}
\label{nonconstructive}
\index{non-constructive recognition}%

The term ``non-constructive recognition'' for a group means finding
the isomorphism type. In situations where direct methods for constructive 
recognition (see the previous Section~\ref{solvedirect}) fail, the usual
approach is to determine the isomorphism type of the group first and
then use additional knowledge about the group in question to do the
constructive part of the recognition. In particular to use the
technique of standard generators (see the next
Section~\ref{standardgens}), the isomorphism type has to be known in
advance.

For the non-constructive recognition problem we can for example use
element order statistics. We produce uniformly distributed elements in
the group and compute their orders. If we see an element order that
does not occur in a group of a certain isomorphism type, we can
immediately rule out this type. If we fail to see an element order that
occurs very frequently in a certain isomorphism type after some tries,
we can rule out this type with a known error probability.

See Sections~\ref{solveD8} and \ref{solveD9} for an overview of the
methods for non-constructive recognition for classical
groups in their natural representation (class \DD8) and for groups in class \DD9
respectively.

\section{Standard generators}
\label{standardgens}
\index{standard generators}%

In this section we give a definition of the term
``standard generators''. As in this whole chapter we do not want to go
into too much detail but instead give the reader an idea of the
concept.

Once the isomorphism type of a group $G = \left< g_1, \ldots,
g_k\right>$ is known (after successful non-con\-struc\-tive recognition as in
Section~\ref{nonconstructive}), one wants to find an explicit
isomorphism of $G$ to a ``standard copy $\hat G$''. Note that such an
isomorphism is not automatically ``computable'' in the sense that 
we can map group elements back and forth, as we will see below.
However, in the end we strive to use previously acquired and stored
knowledge about $\hat G$ and transfer it over to $G$ via that
isomorphism to eventually solve the constructive recognition problem (see
\ref{ProbCR3}) for $G$.

The concept of standard generators serves this purpose.

\begin{Def}[Standard generators]
\index{standard generators}%
    Let $G$ be a finite group and $\Aut(G)$ its automorphism group. Then
    $\Aut(G)$ acts on tuples of elements of $G$ componentwise. We choose
    one orbit of this action that contains tuples whose entries
    generate $G$ as a group, and call exactly those tuples in this
    orbit \emph{standard generators} for $G$. This choice is done once
    and forever for every isomorphism type of finite group and the
    chosen orbit is described by giving a set of properties of the
    tuples that uniquely determines the orbit.
\end{Def}

\begin{Rem}
Note that this choice has to be done individually for every
isomorphism type of finite groups and the properties have to be
determined intelligently, such that finding a tuple of standard
generators is possible efficiently (see Section~\ref{goodstandgens}).
\end{Rem}

This rather vague definition can best be filled with life by an
example:

\begin{Exa}[Standard generators for the sporadic simple Mathieu group
    $M_{11}$]
    \label{ExaM11}
    This description is taken from \cite[$M_{11}$ page]{WWWAtlas} and
\index{WWW-Atlas of group representations}%
    is derived in \cite[Example~11]{standgens}.

    Standard generators of $M_{11}$ are $(a,b)$ where $a$ has order $2$, 
$b$ has order $4$, $ab$ has order $11$ and $ababababbababbabb$ has 
order $4$. Note that it is a theorem that these properties uniquely
determine an orbit of $\Aut(M_{11})$ on pairs of elements of $M_{11}$.

\smallskip
To find standard generators for $M_{11}$:
\begin{enumerate}
        \setlength{\parskip}{0pt}
    \item Find an element of order $4$ or $8$. This powers up to $x$ of order 
        $2$ and $y$ of order $4$.

      [The probability of success at each attempt is $3$ in $8$.]
\item Find a conjugate $a$ of $x$ and a conjugate $b$ of $y$ such that $ab$ 
    has order $11$.

      [The probability of success at each attempt is $16$ in $165$.]
  \item If $ababbabbb$ has order $3$, then replace $b$ by its inverse.
  \item Now $ababbabbb$ has order $5$, and standard generators of $M_{11}$ 
      have been obtained.
\end{enumerate}
It is a theorem that this procedure produces standard generators and the
probabilities can be read off the character table of $M_{11}$. Of course 
we assume uniformly distributed random elements for these
probabilities to be correct.
Note that choosing random conjugates of $x$ that are uniformly distributed in the
conjugacy class of $x$ can be achieved by conjugating $x$ with a random element
that is uniformly distributed in the group.
\end{Exa}

\begin{Prop}[The virtue of standard generators]
\index{standard generators}%
If $(s_1, \ldots, s_m) \in G^m$ and $(t_1, \ldots, t_m) \in G^m$ are 
both standard generators for a group $G$, then the equations
\[ \varphi (s_i) = t_i \qquad\mbox{for all } 1 \le i \le m \]
uniquely define an automorphism $\varphi$ of $G$.

That is, if we have a tuple of standard generators $(s_1, \ldots, s_m)$ for $G$
and $\hat G$ is a standard isomorphic copy of $G$ for which we know
a tuple of standard generators $(u_1, \ldots, u_m)$, then the
equations
\[ \psi (s_i) = u_i \qquad\mbox{for all } 1 \le i \le m \]
uniquely define an explicit isomorphism $\psi$ from $G$ to $\hat G$.
\end{Prop}
\proofbeg
Since by definition both tuples lie in the same orbit under $\Aut(G)$
and both tuples generate $G$, there is exactly one automorphism
$\varphi \in \Aut(G)$ mapping $s_i$ to $t_i$ for all $1 \le i \le m$.
The hypothesis that $\hat G$ is isomorphic to $G$ takes care of the
second statement.
\proofend

\begin{Rem}[The problem of mapping elements]
Note that even if we have found standard generators in $G$, the above
definition of the explicit isomorphism $\psi$ to $\smash{\hat G}$ does in fact 
\emph{not}
enable us to map arbitrary elements of $G$ via $\psi$, because for
this we would have to express an arbitrary element of $G$ as a
straight line program in $(s_1, \ldots, s_m)$, which is exactly the
\index{straight line program}\index{SLP}%
constructive recognition problem we want to solve!

However, if we want to store certain elements or subgroups of $\hat
G$ beforehand, we can store them as straight line programs in $(u_1,
\ldots, u_m)$ and can then evaluate these straight line programs
in $(s_1, \ldots, s_m)$ to actually get their images under $\psi^{-1}$
in $G$. This fact helps to transfer previously acquired knowledge from
$\hat G$ to $G$.
\end{Rem}

\begin{Rem}[Good standard generators]
\label{goodstandgens}
\index{standard generators}%

We want to comment only briefly on this topic. Basically, the choice
of the standard generators for an isomorphism type of group, that is 
the choice of the $\Aut(G)$-orbit 
in the tuples of elements of $G$, is ``good'', if it is relatively
easy to find a tuple of standard generators by random methods. The
example in Section~\ref{ExaM11} exhibits this. The probabilities to find
the right elements in the algorithm presented there are quite good,
such that very few random elements will usually lead to success.

A large collection of such good choices of standard generators together 
with algorithms to find them can be found on the WWW-Atlas of group
\index{WWW-Atlas of group representations}%
representations, see \cite{WWWAtlas}.
\end{Rem}

\begin{App}[Storing hints for stabiliser chains]
\label{hintsstabchains}
\index{hints for stabiliser chains}%
One immediate application of standard generators is the following. If
a group has a subgroup $U$ with a relatively low index, then we can store
generators for this subgroup as a straight line program in standard
generators. Once we have recognised the isomorphism type of $G$
using non-constructive recognition and have found standard generators
in $G$, we can evaluate this straight line program, get a
generating set of a subgroup $U$ of $G$ with this low index, restrict
the natural module to $U$ and find a proper invariant subspace. Provided
that $G$ acts irreducibly on its natural module, the
$G$-orbit of this subspace in the natural action on subspaces then
gives a permutation action of $G$ which is isomorphic to the one on
cosets of $U$. Note that the existence of such a proper invariant subspace 
of course depends not only on the isomorphism type of $G$ but rather
also on the actual representation $G$ comes in.
Thus, using standard generators in this way, we can collect hints
to find ``good'' actions for the different absolutely irreducible
matrix representations of a group.

Eamonn O'Brien and Robert Wilson have for example done exactly this
for the sporadic simple groups. Their hints data is available in the
{\MAGMA} system (see \cite{Magma}) and will be used in the composition tree
implementation in the {\GAP} system as well.
\end{App}
\index{standard generators}%

\section{The classical case in natural representation: \DD8}
\label{solveD8}

The seminal paper by Neumann and Praeger \cite{neumann-praeger} which
presents an algorithm to decide whether a given group $G \le \GL(n,q)$
contains the special linear group was the starting point of a whole
industry of papers concerned with non-constructive and constructive
recognition of groups.

An algorithm to recognise classical groups in their
natural representation non-constructively is given in
\cite{classicalnonconstructive}. Once this is done, other algorithms
apply: In \cite{slrecogconstr} an algorithm to recognise $\SL(n,q)$
in its natural representation constructively is presented with
effective cost $O(n^4q)$. For arbitrary classical groups in their
natural representation the results in \cite{peteconstructiveclassical}
give algorithms to solve the constructive recognition problem with
effective cost $O(n^5 \log^2 q)$, however these algorithms need an
$\SL(2,q)$ oracle, that is, they rely on a solution of the constructive
recognition problem for $\SL(2,q)$. Recently a new preprint (see
\cite{recogclassicalodd}) has appeared which handles the case of odd
characteristic.

The basic idea of all these constructive recognition algorithms is to find
a tuple of standard generators and then perform a base change
such that linear algebra methods can be used to express arbitrary
elements as straight line programs in the standard generators.


\section{The almost simple case: \DD9}
\label{solveD9}
\index{almost simple}%

This part of the whole group recognition project is probably the one
for which the currently known methods are least satisfying. In particular
because many isomorphism types of groups have to be dealt with in a case by
case fashion a lot of work still needs to be done.
In this section we try to give an overview of the known methods by
means of references to the literature. We intentionally leave out
complexity results for the sake of brevity and because the last word on
these does not yet seem to be spoken. A more detailed account of
the state of the art can be found in \cite{OB}.

We begin with a discussion of the problem with the ``almost'' in
``almost simple''.

If a group $G \le \GL(n,q)$ with $Z := G \cap Z(\GL(n,q))$ 
is contained in class \DD9, there is a
non-abelian simple group $\bar N$ and a group $T$ with $\bar N
\leq T \leq \Aut(\bar N)$ and $G/Z \cong T$. The first problem for
both constructive and non-constructive recognition is that $G/Z$ is
itself not necessarily simple, after all, $G/Z$ is only ``almost simple''.
However, the Schreier conjecture, which follows from the classification
of finite simple groups, says that $\Aut(\bar N) / \bar N$ is solvable
for all finite simple groups~$\bar N$. 

In fact, this outer automorphism
group is rather small for groups occurring in practice. 
If $\bar N$ is alternating or sporadic, then
$|\Out(\bar N)| = 2$ except for $|\Out(A_6)| = 4$.
In \cite[Lemma 1.4]{LucchiniMorigi} it is shown
that for a simple group $\bar N$ of Lie type in cross-characteristic, 
$|\Out(\bar N)| \le \beta \log n$ for some global constant
$\beta$. If $\bar N$ is a simple group of Lie type in its defining
characteristic, then $|\Out(\bar N)| \le 2(n+1)\log q$ by
\cite[Proof of Lemma 1.3]{LucchiniMorigi}.

So the first step for recognition is to find the simple subgroup
isomorphic to $\bar N$ of $G/Z$, we denote this by $N/Z$ in the following. 
One approach is to go down the derived series
until a group is reached which is ``probably perfect''. To test for
the latter, one uses the algorithm by Leedham-Green and O'Brien in
\cite[5.3]{RecogTensInd}. This algorithm estimates the order of an element in a
factor group if the group and the normal subgroup are given.
This technique produces generators for $N$ efficiently.

Most publications in this area seem to concentrate on the
non-constructive and constructive recognition of simple groups.
However, to completely solve Problem~\ref{ProbCR3} for all groups in
\DD9 one has to be able to do constructive recognition for all
almost simple groups. There seems to be no method known to get hold of
the factor group $G/N$ in general since in most cases
the restriction of the natural module to $N$ is absolutely
irreducible.  We will ignore this problem here, in particular since $G/N$
is very small in most cases, such that solving the constructive recognition
problem for $N$ together with a few coset case distinctions can solve the
problem in $G$ satisfactorily.

Next we try to give an overview of the known techniques for
non-constructive recognition of simple groups. Alternating groups and
sporadic groups can be recognised non-constructively by looking at
element orders of random elements. For Lie type groups there is an
algorithm for non-constructive recognition in \cite{blackboxlienonconstr}.
One outstanding case in this paper is covered by methods in
\cite{altseimer}. However, both algorithms need to know the defining
characteristic of the Lie type group in advance. To determine this,
there are three concurrent methods, one is described in
\cite{primpowgraphs}, a newer one in \cite{findingcharlie} and thirdly there
is unpublished work by Seress which uses statistics about the two largest 
projective element orders occurring.

In 2001 Malle and O'Brien developed a practical implementation of all 
the algorithms for non-constructive recognition known at the time,
which is distributed with the {\MAGMA} system.

Assuming from now on that the non-constructive recognition is achieved we
conclude this section with a list of references to work that has been done
to solve the constructive recognition problem for classes of simple groups.
Note that many of these solutions include the corresponding almost simple
groups thereby solving specific cases of \DD9 groups completely.

An algorithm to recognise the alternating group $A_k$ and the symmetric
group $S_k$ constructively in an arbitrary 
representation is described in \cite{bbsymaltconstr}. An alternative
algorithm was developed in \cite{bratuspak}.

For classical groups in another than their natural representation (see
Section~\ref{solveD8} for the \DD8 case) there is an algorithm in
\cite{bbclassical}. The particularly important case of the special linear
group $\SL(2,q)$ of rank $2$ in non-natural but defining characteristic
representation is dealt with in \cite{classicallargefield}
and \cite{psl2qconstr} in the sense that an algorithm is described to find
an explicit computable epimorphism to $\PSL(2,q)$ in its natural
representation. Further work to improve the algorithms for simple classical 
groups in arbitrary representations can be found in \cite{bbomega},
\cite{bbunitary}, \cite{bbpsldq}, \cite{computingmatrix} and
\cite{bbortho}. Recently the preprint \cite{smalldegreegl} appeared which
deals with recognising small degree representations of general linear
groups specifically.

For Lie type groups work on constructive recognition has appeared in
\cite{recogSL3}, \cite{rybaid} and \cite{suzukiconstr}. An article about 
constructive recognition of exceptional groups of Lie type by Kantor and Magaard is
in preparation. However, this area is still work in progress.

For the sporadic simple groups the data collected by O'Brien and Wilson
about subgroup chains (see Section~\ref{hintsstabchains}) can be used to
solve the constructive recognition problem after non-con\-struc\-tive
recognition using the direct methods mentioned in
Section~\ref{solvedirect}. 

We finish this section by mentioning two other approaches to the
constructive part of the recognition problem.

One is published
in the preprint \cite{bbconstrmember}. It uses involution centralisers
and the fact that generators for them can be computed efficiently (see
\cite{BrayInv}). One instance of the constructive membership test problem
for a group $G$ is translated into three instances of the corresponding
problem in the centralisers of certain involutions which come up during the
run of the algorithm.

Finally we want to mention the paper \cite{gensift}, which introduces a
generic framework for constructive membership testing in a group whose
isomorphism type is known. It relies on standard generators (see
Section~\ref{standardgens}) and prepared subgroup chains stored as straight
line programs in the standard generators.
\index{straight line program}\index{SLP}%

% REFERENCE: Kantor, Seress: Black box classical groups???


\appendix

% this is a part of the habilitation thesis of Max Neunhoeffer

\chapter{Addendum}

to be written


%\mbox{}
\thispagestyle{fancy}

\chapter{Notations}

\label{NotationIndex}
\begin{longtable}{|lll|}
\hline\endfoot
\hline\endhead
\hline
\multicolumn{3}{|l|}{Symbols:}\\
\hline
$\emptyset$             & the empty set
                        & \\
$\subseteq$             & is contained in
                        & \\
$\supseteq$             & contains
                        & \\
$\subsetneq$            & is contained properly
                        & \\
$\supsetneq$            & contains properly
                        & \\
$\circ$                 & concatenation of mappings
                        & \ref{conventions} \\
$\circlearrowleft$      & commutative diagram 
                        & \\
$\hookrightarrow$       & injective mapping
                        & \\
$\twoheadrightarrow$    & surjective mapping
                        & \\
$\cong$                 & isomorphism
                        & \\
$\equiv$                & congruence
                        & \\
$(-|-)$                 & bilinear form
                        & \\
$\displaystyle\prod$    & product
                        & \\
$\displaystyle\coprod$  & coproduct
                        & \\
$\displaystyle\bigoplus$ & direct sum of modules
                        & \\
$\left< - \right>_A$    & $A$-span
                        & \\
$\oplus$                & direct sum
                        & \\
$\triangleleft$         & is ideal in 
                        & \\
$\varphi^{-1}(N)$       & full preimage of the set $N$ under $\varphi$
                        & \\
$\mathbb{C}$            & set of complex numbers
                        & \\
$\det M$                & determinante of the matrix $M$
                        & \\
$E_d$                   & $d \times d$ identity matrix
                        & \\
$\End_G(M)$             & set of endomorphisms of the $G$-module $M$
                        & \\
$\F_q$                  & field with $q$ elements
                        & \\
$\GL_n(q)$              & group of invertible matrices in $\F_q^{n \times n}$
                        & \\
$\Hom_G(M,N)$           & set of homomorphism between the $G$-modules $M$
                          and $N$
                        & \\
$\id_M$                 & identity mapping $M \to M$
                        & \\
$\Ima f$                & image of the mapping $f$
                        & \\
$\ker f$                & kernel of the mapping $f$
                        & \\
$\N$                    & set of natural numbers, $0 \notin \N$
                        & \\
$\Q$                    & set of rational numbers
                        & \\
$\R$                    & set of real numbers
                        & \\
$\rad(R)$, $\rad(M)$    & Jacobson-radical of a ring $R$ or module $M$
                        & \ref{Konventionen} \\
$S_n$                   & symmetric group on $n$ points
                        & \\
$\soc(M)$               & socle of a module $M$
                        & \\
$\Z$                    & set of rational integers 
                        & \\
\hline
\end{longtable}

\phantomsection{}
\label{NotationIndexEnd}

\clearpage
\markboth{Appendix C: List of figures}{Appendix C: List of figures}

\newcommand{\friss}[1]{}
\newcommand{\myitem}[2]{\rlap{#1}\hspace*{\cftchapnumwidth}{#2}}

\phantomsection{}
\addcontentsline{toc}{chapter}{\myitem{C}{List of figures}}
\vspace*{10mm}
{\huge\bf Appendix C}

\listoffigures

\phantomsection{}
\addcontentsline{toc}{chapter}{\myitem{D}{List of tables}}
\vspace*{28mm}
{\huge\bf Appendix D}

\listoftables

\stepcounter{chapter}
\stepcounter{chapter}

\markboth{Appendix C: List of figures}{Appendix C: List of figures}

%\newpage\mbox{}
%\clearpage\mbox{}\thispagestyle{fancy}

\bibliographystyle{alpha}
\bibliography{habil}

\printindex

\end{document}

