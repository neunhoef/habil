% This is the main file of the Habilitation thesis of Max Neunhoeffer

\documentclass[openany,11pt,british]{book}

\usepackage[latin1]{inputenc}
\usepackage{amssymb}
\usepackage{makeidx}
\usepackage[british]{babel}
%\usepackage{stmaryrd}
\usepackage[mtbold,subscriptcorrection,mtpluscal]{mathtime}
%\usepackage[heavybold,uprightgreek,subscriptcorrection,mtpluscal]{mathtime}
\usepackage{theorem}
\usepackage[all,ps]{xy}
%\usepackage{showkeys}
\usepackage{ifthen}
%\usepackage[bf,center]{caption}
%\renewcommand{\captionfont}{\sffamily}
\usepackage{calc}
\usepackage{tocloft}
\setlength{\cftchapnumwidth}{8mm}
\setlength{\cftsecnumwidth}{12mm}
\setlength{\cftfignumwidth}{12mm}
\setlength{\cfttabnumwidth}{12mm}
%\usepackage{titlesec}
\usepackage{fancyhdr}
\usepackage{color}
\usepackage{longtable}
%\usepackage[dvipdfm]{graphicx}
\usepackage{graphicx}
\usepackage{pgf}
\usepackage[numbib,numindex,notlot,notlof]{tocbibind}

%\usepackage[colorlinks=true,backref=page,dvipdfm,%hypertex,
\usepackage[colorlinks=true,backref=page,pdftex,
%\usepackage[colorlinks=true,backref=page,dvips,
            linkcolor=MyBlue,urlcolor=MyRed,citecolor=MyGreen,
            pdftitle={Constructive Recognition of Finite Groups},
            pdfauthor={Max Neunhoeffer},
            pdfsubject={Matrix and projective groups},
            pdfkeywords={constructive recognition matrix group projective group composition tree Aschbacher polynomial time}]{hyperref}
\usepackage{myhyppdf}

%\definecolor{RoyalBlue}{rgb}{0.0236,0.0894,0.6179}
%\definecolor{RoyalGreen}{rgb}{0.0236,0.6179,0.0894}
%\definecolor{RoyalRed}{rgb}{0.6179,0.0236,0.0894}
\definecolor{MyBlue}{rgb}{0.01,0.05,0.5}
\definecolor{MyGreen}{rgb}{0.01,0.4,0.05}
\definecolor{MyRed}{rgb}{0.7,0.01,0.05}

% The paper format:
\usepackage[a4paper,
%\usepackage[letterpaper,  % for letter: make it possible but not optimized
            height=8.30in,width=5.87in,
            vmarginratio=1:1,
            % this is 71% of a4 in both dimensions ==> page usage 50%
            % use default: hmarginratio=2:3,vmarginratio=2:3,
            % showframe,   % for verification purposes
            headheight=1\baselineskip,verbose]{geometry}
%So zum Korrekturlesen mit breitem rechtem Rand:
%%%\usepackage[a4paper,
%%%%\usepackage[letterpaper,  % for letter: make it possible but not optimized
%%%            height=8.30in,width=5.87in,
%%%            vmarginratio=1:1,
%%%            % this is 71% of a4 in both dimensions ==> page usage 50%
%%%            % use default: hmarginratio=2:3,vmarginratio=2:3,
%%%            %showframe,   % for verification purposes
%%%            headheight=1\baselineskip,verbose,twoside=false,left=10mm]{geometry}

\usepackage{algorithm}
\usepackage{algorithmic}
% Dirty hack to make hyperref happy:
\makeatletter
\def\theHALC@line{\thealgorithm-\theALC@line}
\def\theHALC@rem{\thealgorithm-\theALC@rem}
\makeatother


% Bemerkung: showkeys und hyperref funktionieren nicht zusammen!

% The page headings:
\fancyhead[LO]{\nouppercase{\rightmark}}
\fancyhead[RE]{\nouppercase{\leftmark}}
\fancyhead[LE,RO]{\thepage}
\fancyhead[FCO,FCE]{}
\pagestyle{fancy}

\setcounter{tocdepth}{2}

% Change arrows in xypic:
\SelectTips{cm}{11}
\UseTips

\makeindex

\parindent0pt

% Mathematische Operatoren wie z.B. rad:
\makeatletter
\newcommand{\maop}[1]{%
\ensuremath{\mathop{\operator@font #1}\nolimits}}
\newcommand{\maopl}[1]{%
\ensuremath{\mathop{\operator@font #1}\limits}}
\makeatother

\newcommand{\rad}{\maop{rad}}
\newcommand{\soc}{\maop{soc}}
\newcommand{\myimplies}{\ensuremath{\Longrightarrow}}
\newcommand{\myiff}{\ensuremath{\iff}}
\newcommand{\Hom}{\maop{Hom}}
\newcommand{\End}{\maop{End}}
\newcommand{\Tr}{\maop{Tr}}
\newcommand{\id}{\maop{id}}
\newcommand{\Ima}{\maop{Im}}
\newcommand{\Char}{\maop{char}}
\newcommand{\GL}{\maop{GL}}
\newcommand{\GGL}{\maop{\Gamma L}}
\newcommand{\PGL}{\maop{PGL}}
\newcommand{\SL}{\maop{SL}}
\newcommand{\PSL}{\maop{PSL}}
\newcommand{\Sp}{\maop{Sp}}
\newcommand{\PSp}{\maop{PSp}}
\newcommand{\GSp}{\maop{GSp}}
\newcommand{\SU}{\maop{SU}}
\newcommand{\PSU}{\maop{PSU}}
\newcommand{\SO}{\maop{SO}}
\newcommand{\PSO}{\maop{PSO}}
\newcommand{\POmega}{\maop{P\Omega}}
\newcommand{\tens}[1][]{\ifthenelse{\equal{#1}{}}{\maopl{\otimes}}%
{\mathop{\raisebox{0.4mm}{$\scriptstyle\maopl{\otimes}_{#1}$}}}}
\newcommand{\smtens}[1][]{\ifthenelse{\equal{#1}{}}{\maopl{\otimes}}%
{\mathop{\raisebox{0.15mm}{$\scriptscriptstyle\maopl{\otimes}_{#1}$}}}}
\newcommand{\myle}{\leqslant}
\newcommand{\myge}{\geqslant}
\newcommand{\Span}{\maop{Span}}
\newcommand{\Cmpl}{\maop{Cmpl}}
\newcommand{\ord}{\mathrm{ord}}
\newcommand{\ann}{\mathrm{ann}}
\newcommand{\len}{\mathrm{length}}
\newcommand{\lc}{\mathrm{lc}}
\newcommand{\lcm}{\mathrm{lcm}}
\newcommand{\opspd}{\mbox{\sc OpsPD}}
\newcommand{\opsgcd}{\mbox{\sc OpsGcd}}
\newcommand{\Prob}{\mathrm{Prob}}
\newcommand{\GG}{\mathcal{G}}
\newcommand{\HH}{\mathcal{H}}
\newcommand{\calF}{\mathcal{F}}
\newcommand{\calY}{\mathcal{Y}}
\newcommand{\rsp}{\mathrm{RowSp}}
\newcommand{\la}{\left<}
\newcommand{\ra}{\right>}
\newcommand{\ve}{\varepsilon}
\newcommand{\one}{\mathbf{1}}
\newcommand{\fail}{\textsc{Fail}}
\newcommand{\Gal}{\maop{Gal}}
\newcommand{\Aut}{\maop{Aut}}
\newcommand{\Out}{\maop{Out}}
\newcommand{\Inn}{\maop{Inn}}
\newcommand{\Syl}{\maop{Syl}}
\newcommand{\Pro}{\maop{Prob}}

\newcommand{\proofbeg}{\noindent\textbf{Proof:}\ }
\newcommand{\proofof}[1]{\noindent\textbf{Proof of #1:}}
\newcommand{\proofend}{\hfill$\Box$}

% Einige Definitionen fuer haeufig vorkommende Buchstaben:

\newcommand{\F}{\ensuremath{\mathbb{F}}}
\let\ll=\l   % fuer alle Faelle!
\renewcommand{\l}{\ensuremath{\ell}}
\newcommand{\C}{\ensuremath{\mathbb{C}}}
\newcommand{\N}{\ensuremath{\mathbb{N}}}
\newcommand{\Q}{\ensuremath{\mathbb{Q}}}
\newcommand{\R}{\ensuremath{\mathbb{R}}}
\newcommand{\Z}{\ensuremath{\mathbb{Z}}}
\newcommand{\CC}[1]{\ensuremath{\mathcal{C}_{#1}}}
\newcommand{\DD}[1]{\ensuremath{\mathcal{D}_{#1}}}
\newcommand{\T}{\ensuremath{\mathrm{T}}}
% Ich will spaeter noch umkonfigurieren:
\newcommand{\Emph}[1]{{\boldmath\textbf{#1}}}
\newcommand{\ba}[1]{\overline{\rule{0mm}{1.2ex}#1}}

% Einige Gimmicks:
\newcommand{\Proof}{\noindent\textbf{Proof:} }
\newcommand{\ProofEnd}{\hfill$\Box$\par}
\newcommand{\GAP}{\textsf{GAP}}
\newcommand{\MAGMA}{\textsc{MAGMA}}
\newcommand{\cvec}{\textsf{cvec}}

\newenvironment{compactlist}{\begin{list}{}{\setlength{\itemsep}{0pt}%
\setlength{\labelwidth}{1cm}\addtolength{\labelsep}{3mm}%
\addtolength{\leftmargin}{3mm}}}{\end{list}}

\renewcommand{\thechapter}{\Roman{chapter}}
\renewcommand{\thesubsection}{(\arabic{section}.\arabic{subsection})}
\newcommand{\mychapter}[1]{\chapter{#1}}
\makeatletter
\let\@oldrefstepcounter=\refstepcounter
\def\refstepcounter#1{\@oldrefstepcounter{#1}%
\ifthenelse{\equal{#1}{subsection}}%
{\gdef\@currentlabel{\thechapter.\thesubsection}}{}}
\makeatother

\theoremstyle{changebreak} 
\theoremheaderfont{\bfseries\boldmath}
\newtheorem{Prop}{Proposition}[section]
\newtheorem{Theo}[Prop]{Theorem}
\newtheorem{Lemm}[Prop]{Lemma}
\newtheorem{Cor}[Prop]{Corollary}
\newtheorem{Conj}[Prop]{Conjecture}
\theorembodyfont{\upshape}
\newtheorem{Def}[Prop]{Definition}
\newtheorem{DefProp}[Prop]{Definition/Proposition}
\newtheorem{Rem}[Prop]{Remark}
\newtheorem{Rems}[Prop]{Remarks}
\newtheorem{Not}[Prop]{Notation}
\newtheorem{Hyp}[Prop]{Hypothesis}
\newtheorem{Exa}[Prop]{Example}
\newtheorem{Problem}[Prop]{Problem}
\newtheorem{Obs}[Prop]{Observation}
\newtheorem{Expl}[Prop]{Explanation}
\newtheorem{App}[Prop]{Application}

\newcommand{\ruecke}{\mbox{\phantom{\rm\bf\thesubsection\ }}}

% Some things for the subfield paper:
\newcommand{\by}{\times}
\newcommand{\tp}{\otimes}
\newcommand{\idiv}{\ {\textrm{div}\ }}

\newenvironment{proof}{\normalsize {\noindent\bf Proof}:}%
{{\hfill $\Box$\par\medskip}}

%\includeonly{}

\begin{document}

\thispagestyle{empty}
\begin{center}
    {\Huge Constructive Recognition of Finite Groups}\\[3cm]
\large
Der Fakult\"at f\"ur\\[5mm]
Mathematik, Informatik und Naturwissenschaften\\[5mm]
der Rheinisch-Westf\"alischen Technischen Hochschule Aachen\\[5mm]
vorgelegte Habilitationsschrift\\[5mm]
zur Erlangung der Venia legendi\\[4cm]
von\\[5mm]
Dr. rer. nat. Max Neunh\"offer\\[5mm]
geboren in Heidelberg\\[4cm]
November 2008
\end{center}
\newpage

%\title{Computing with matrix groups}
%\author{Max Neunh\"offer}
%\maketitle

% this is a part of the habilitation thesis of Max Neunhoeffer

\chapter*{Preface}
\addcontentsline{toc}{chapter}{Preface}

Ultimate goal: Computing with matrix groups, mention projective.

History: Permutation groups, stabiliser chains, base and strong generators,
many highly efficient algorithms, complexity theory as tool, large
base groups, do different things in different situations, use
randomisation!

First goal: Constructive recognition (compare to sifting)
Matrix group recognition project. A bit of history.

Basic approach, rough sketch.

This book gives an overview over the current state of the art and
shows some contributions of the author. Collaborations.
Implementations.

Work to do (in particular C9 groups).

Not mentioned: New methods for matrices over finite fields using
floating point numbers, details on C9, Mark Stather's work, further
algorithms building on top of constructive recognition. Black box
groups.

Structure of this book. Go through chapters. Explain authors.

Acknowledgements. Thanks.

In particular:

RWTH, Gerhard Hiss, Alice Cheryl, UWA, Akos, Ohio State, St Andrews, Colva, Jon,
Steve. Proofreading: Colva, Papa, Anja.



\newpage\enlargethispage{-2\baselineskip}
\tableofcontents
\thispagestyle{fancy}

% this is a part of the habilitation thesis of Max Neunhoeffer

\chapter{Introduction}
\label{chap:intro}

This chapter covers a few fundamental concepts and algorithms which
will be used in the rest of the book. We start talking about the
complexity of algorithms, go on with the concept of straight line
programs in groups and finish by mentioning the idea of randomisation
in algorithms and a way to produce nearly uniformly distributed
elements in a finite group.

\section{Some notes on complexity theory}
\label{sec:complexity}

\index{complexity theory}%
Already in this chapter, but all the more in the rest of the book, we
talk about the analysis of algorithms. In this section we want to
discuss briefly what we mean by this at all.

An algorithm is usually designed to solve a whole family of problems of
different sizes. Obviously, for a single, particular instance of a
computational problem one can simply store the answer and look it up
when needed in nearly no time. But this is usually not what we intend
to do when we develop an algorithm. Furthermore, it is clear that a
small instance of a computational problem usually will need less time
to solve than a big instance of the same problem. It takes for example longer 
to multiply two $10000\times 10000$-matrices with entries in the finite 
field $\F_q$ than to multiply two $100\times 100$-matrices
with entries in the same finite field $\F_q$,
even if these two situations are instances of the same
computational problem ``matrix multiplication over $\F_q$''.

Therefore, when we talk about analysing an algorithm that solves a
certain family of computational problems, we assign each instance of
this problem a ``size'' and ask how the runtime of the algorithm grows
in comparison to the size of different instances. This is in vague terms
what is meant by ``complexity of an algorithm'', and ``complexity
\index{complexity}%
theory'' is the study of the complexity of algorithms. The size of an
instance of the above mentioned matrix multiplication problem of two $n
\times n$-matrices could for example be measured by the value $n$.

Of course, different computers are running at different speeds, so
what we usually do is to count the number of steps in the algorithm
needed to complete a particular instance of the computational problem
as a function of the size of the problem instance.
We try to keep the different steps comparable. We might for example
count the number of elementary field operations (addition,
\index{elementary field operation}%
subtraction, multiplication, inversion) during the execution of a
matrix multiplication. The standard simple-minded approach to matrix
multiplication would then need $(2n-1)\cdot n^2$ elementary field
operations since each entry of the result is a sum of $n$ products of
numbers in $\F_q$, that is, we need $n$ products and $n-1$ additions
for each of the $n^2$ matrix entries. The complexity of this algorithm
in terms of the size $n$ of the input would then be $(2n-1)\cdot n^2$
elementary field operations.

If the growth rate of the complexity of an algorithm A is slower
\index{growth rate}%
than the one for another algorithm $B$ solving the same problem, then 
we would consider A to be ``better'' than B. Note that it is
possible that the complexity of B is in fact smaller than
the one of A for small problem sizes. In that case we might want to
implement both algorithms and use B for smaller instances and A for
bigger instances. In this way complexity theory helps to come up with
better implementations. If we had for example an algorithm B to multiply
two $n\times n$-matrices over $\F_q$ using $n^4/20$ elementary field
operations, it would be worthwhile to use it as opposed to the
standard algorithm A for small matrices, since
$n^4/20 \le (2n-1)\cdot n^2$ for $n \le 39$. However, for big
matrices A would outperform B dramatically.

Counting the steps in an algorithm can be very tedious and much more
difficult if there are decisions and case distinctions during the
execution. Randomisation as described in Section~\ref{montevegas}
makes this even more difficult. Therefore we are usually happy to come
up with an upper bound for the number of steps necessary. In addition,
when we are only interested in the growth rate of the complexity, we
\index{growth rate}%
are only interested in the type of function expressing this upper
bound in terms of the problem size and are actually not interested in
the constants. 

So, to determine that the growth rate of the complexity
\index{growth rate}%
of the example algorithm A above is smaller than the one of B, we do
not need the exact number of steps $(2n-1)\cdot n^2$ and $n^4/20$
respectively, but we only need to know that the first behaves ``like
$n^3$'' and the second ``like $n^4$''. The higher exponent $4$ in the
second function eventually beats the smaller constant $1/20$.
Note however that without knowing the constants we would in fact not
know that the break-even point of where we should start using A in
favour of B is at $n=39$. Having noticed this, we would like to comment
that knowing the constants in the complexity of algorithms can be very
interesting for coming up with good implementations.

Different parts of this book go differently about this. Whenever
possible we have tried to include constants in the complexity
estimates. However, sometimes this would have been too tedious and
would have complicated things dramatically. In these cases we are
content with determining the growth rate. For this purpose, we adopt
\index{growth rate}%
the standard big-$O$ notations, which we will briefly repeat here.
It is used to formulate statements about the asymptotic behaviour of
functions.

\begin{Def}[Capital-$O$-notation]
    \label{capO}\index{O-notation@$O(-)$-notation}%
    Let $\R^+$ be the set of positive real numbers and
    $g : \R^+ \to \R$.
    We say that a function $f : \R^+ \to \R$ is $O(g)$ if there are
    two positive real constants $C$ and $D$ such that 
    $|f(x)| \le C \cdot |g(x)|$ for all $x > D$.
\end{Def}

This will be used in the analysis of algorithms by saying for
example: Algorithm A above has complexity $O(n^3)$ whereas algorithm B
\index{complexity}%
above has complexity $O(n^4)$. Obviously it has to be clear from the
context, which family of computational problems the algorithm deals
with and how the size $n$ of an instance of the problem is measured.

There is a certain sloppiness in this usage of Definition~\ref{capO}.
Strictly speaking the following statement is true as well: 
``The function $n \mapsto (2n-1)\cdot n^2$ is $O(n^5)$.'' This follows
from the fact that the function $n \mapsto n^3$ is $O(n^5)$. However,
this statement is not as strong as possible. Obviously, we are
interested in the fact that the exponent $3$ is the \emph{least
possible $e$} for which the statement ``the complexity of algorithm A is
$O(n^e)$'' is true. Sometimes we can prove that our statement is best
possible and sometimes we cannot.

Up to now we have concentrated on the ``time complexity'' of
\index{time complexity}%
algorithms, that is, the asymptotic behaviour of the runtime with
growing problem sizes. Although this is usually the most interesting
aspect, it is of course also necessary to keep an eye on other
computational resources like memory usage. We speak of ``space
complexity'' in this case and use a similar setup and language. This
\index{space complexity}%
only plays a minor role in this book.

The point of view we adopt in this book is that the complexity analysis of
algorithms is a tool to come up with good implementations. Complexity
results tell us, which algorithms are suited best for which problem
classes and possibly which problem sizes, and they tell us, what is
considered to be a ``satisfactory'' algorithm for a problem class and
what is considered to be ``lacking''. In general we would like to
devise algorithms which have a polynomial as an asymptotic bound of their
time complexity in terms of the problem size. We call these algorithms
``polynomial-time algorithms''.


\section{Straight line programs}
\label{slp}

\index{straight line program}\index{SLP}%
Let $G$ be a group that is given by a tuple of generators $(g_1,
\ldots, g_k)$. That is, every element in $G$ can be expressed as a
finite product of powers of the $g_i$ and their inverses. However,
since $G$ is not necessarily abelian, the generators may occur more
than once and the products can be quite long. We call such a product a 
\emph{word in the generators $(g_1, \ldots, g_k)$}. 
\index{word in generators}%
For finite groups we do not need the
inverses of the generators since they all have finite order and can
thus be expressed by positive powers.

Quite often in applications we want to encode rather long words in
generators on a computer. One reason for this is that we want to store certain
elements in a known group in terms of a generating tuple (see in
particular Section~\ref{standardgens}). Another reason is for example that to
evaluate a group homomorphism $\varphi:G \to H$ on arbitrary elements
$g \in G$, if we only know the images $\varphi(g_i)$ of a tuple of
generators for $G$, we need to express $g$ explicitly in terms of
the generators $(g_1, \ldots, g_k)$. Finally, the problem of constructive
recognition (see Problem~\ref{ProbCR3}) involves sometimes rather
\index{constructive recognition}%
long words in a tuple of generators, too.

To store and evaluate such words efficiently is the purpose of
\emph{straight line programs (SLP)}.

In general, straight line programs are programs that have no branches
or loops, their execution follows a ``straight line'' under all
circumstances. For the purpose of storing and evaluating words in
generators in a group, we further restrict this to the following:

\begin{Def}[Straight Line Program (SLP)]
    \label{defslp}\index{straight line program}\index{SLP}%
    \index{straight line program!definition of}\index{SLP!definition of}%
    A \emph{straight line program} is a procedure that takes as input a
    $k$-tuple $(g_1, \ldots, g_k)$ of group elements in a 
    common group and consists of a finite 
    sequence of steps, which are each one of the following:
\begin{itemize}
    \item Compute the product of two stored elements, or
    \item compute the inverse of a stored group element.
\end{itemize}
    An SLP starts with the elements $g_1, \ldots, g_k$,
    stores all intermediate results and returns one or more of the 
    group elements collected during its execution. The number of
    steps in the SLP is called its \emph{length}.
\end{Def}

\begin{Rem}
A straight line program of length $l$ can encode words in its input
of length up to $2^l$. Obviously, it cannot encode all words of that
length.
\end{Rem}

Implementations and data structures for straight line programs are
available in the {\GAP} (see \cite{GAP4}) and {\MAGMA} (see
\cite{Magma})
computer algebra systems. The
WWW-Atlas of group representations uses straight line programs to
\index{WWW-Atlas of group representations}%
store generators for maximal subgroups of groups. Note that the
current implementations of straight line programs in these systems
provide an even more compact storage by allowing arbitrary finite
products of powers of previously stored group elements in each step.


\section{Randomised algorithms}
\label{montevegas}
\index{randomised algorithm}\index{algorithm!randomised}%

Traditionally, an algorithm is a completely deterministic procedure to
achieve a certain goal. Whenever it is executed, it performs the same
steps and thus behaves in the same way when called twice with the same input.

However, there is a certain limitation in this paradigm. In particular
in situations, in which we want to find some result that can later
be verified to be correct easily, randomised methods excel. By
randomised
methods we mean algorithms that involve a certain amount of random
choices. That is, the sequence of steps performed by a randomised
algorithm depends on certain random choices done during the algorithm.
Of course, in practice we will usually use pseudo random numbers to
do these random choices.

We do not want to go into too much detail here, but there are many
examples in which randomised algorithms can greatly outperform
deterministic algorithms. However, how do we measure or analyse
the performance of a randomised algorithm, given that it does
different things on different calls with the same input, and thus
has different runtimes on different occasions?

One possibility for performance analysis is to look at the average
or the expected runtime. Although this is a good and
interesting thing to look at, this type of analysis often stays a
bit unsatisfactory, since one never knows, how long the algorithm will
take at most in a particular instance.

Therefore the most common approach for randomised algorithms is to do a 
worst-case analysis. However, clearly the absolutely worst case is
that by incredible bad luck all random choices turn out to be wrong
and the algorithm does not succeed even after a very long time. To get
rid of this problem we have to allow our algorithms to fail in some
way, most commonly simply by giving up with \textsc{Fail} as answer
after a certain time.
Using this exit route, we can devise algorithms that are
guaranteed to terminate after a certain number of steps or a given
amount of time. To be useful in practice, we of course want to have a
bound for the probability with which this failure occurs. Optimally, we
want to prescribe an upper bound for the failure probability
beforehand. The guaranteed upper bound for the runtime then might
depend on the choice of the failure probability bound.

In general we distinguish between so-called ``Monte Carlo'' and
``Las Vegas'' algorithms as defined in the following.

\begin{Def}[Monte Carlo algorithm]
\index{randomised algorithm}\index{algorithm!randomised}%
\index{Monte Carlo algorithm}%
    A \emph{Monte Carlo algorithm with failure probability $\epsilon$}
    is a randomised algorithm that is guaranteed to terminate after
    a finite amount of time with some result, if the probability for
    returning a wrong result is bounded by $\epsilon$.
\end{Def}

A bit more satisfying is the following.

\begin{Def}[Las Vegas algorithm]
\index{randomised algorithm}\index{algorithm!randomised}%
\index{Las Vegas algorithm}%
    A \emph{Las Vegas algorithm with failure probability $\epsilon$}
    is a randomised algorithm that is guaranteed to terminate after
    a finite amount of time with either the correct result or
    \textsc{Fail} indicating failure, if the probability for failure
    is bounded by $\epsilon$.
\end{Def}

The two concepts are sometimes related by the following.

\begin{Rem}[Upgrading Monte Carlo to Las Vegas by verification]
\index{verification}%

Assume that there is an efficient way to verify the correctness of the output
of a Monte Carlo algorithm. Then we can immediately upgrade the
algorithm to be of Las Vegas type by following it with a verification
step that returns \textsc{Fail}, if the result was incorrect in the first
place. ``Efficient'' here means that the verification does not take
much longer than the Monte Carlo computation in the first place.
\end{Rem}

We will use the terms ``Monte Carlo algorithm'' and ``Las Vegas
algorithm'' in this sense throughout this book.

\section{Random elements in finite groups}
\label{randomelts}

Randomised algorithms in group theory need random elements in groups.
Moreover, to allow a proper analysis of the behaviour of such
algorithms one needs to know quite a lot about the distribution of
the random elements in the group. Usually, the best with respect to
analysis is to have a source of uniformly distributed random elements.

For most applications we are content with pseudo randomness, that is, with a
deterministic procedure which produces from some initial seed a sequence of 
elements with a good uniform distribution. Choosing different seeds (or
maybe actually choosing the seed at random) then leads to a different
behaviour of the algorithm in each call.

There are well-known methods to produce good uniformly distributed
pseudo random numbers (see for example \cite[Chapter~3]{AOCP2}).
Building on these, there is a method to produce pseudo random elements
in a finite group given by generators, which works astonishingly
well in the sense that the distribution of the elements is very close
to uniform. In the sequel we describe this method briefly but refer
for proofs to the literature. After this we discuss some of the
advantages and limitations of this method.

\begin{algorithm}
\caption{$\quad$ \sc RattleStep}
\label{rattlestep}
\index{Rattle@\textsc{Rattle}}%
\begin{algorithmic}
\STATE \textbf{Input:} A pair $(a,(h_1, \ldots, h_n))$ with $a \in G$
and $G = \left< h_1, \ldots, h_n \right>$.
\STATE \textbf{Output:} A modified pair $(a,(h_1, \ldots, h_n))$ with $a \in G$
and $G = \left< \smash{h_1, \ldots, h_n} \right>$.
\vspace*{2mm}
\STATE $i := \textsc{Random}(\{ 1,2,\ldots,n\}$
\STATE $j := \textsc{Random}(\{1,2,\ldots,n-1\}$
\STATE $b := \textsc{Random}(\{1,2\}$
\IF {$j = i$}
    \STATE $j := j + 1$
\ENDIF
\IF {$b = 1$}
    \STATE $h_i := h_i \cdot h_j$
\ELSE
    \STATE $h_i := h_j \cdot h_i$
\ENDIF
\STATE $a := a \cdot h_i$
\STATE \textbf{Return} modified $(a,(h_1,\ldots,h_n))$
\end{algorithmic}
\end{algorithm}

\begin{Def}[Random elements with \textsc{Rattle}]
    \label{rattle}
\index{Rattle@\textsc{Rattle}}%
Let $G$ be a group given by a tuple $(g_1, \ldots, g_k)$ of generators
and $n, N \in \N$ with $n \ge k$. The \textsc{Rattle} method to
produce random elements in $G$ is the following procedure:

It uses a variable $(a,(h_1,\ldots,h_n)) \in G \times G^n$
which is changed during the runtime by calls to
Algorithm~\ref{rattlestep}.

It first initialises $(a,(h_1,\ldots,h_n))$
by $a := \mathbf{1}_G$ and $h_i := g_i$ for $1 \le i \le k$ and $h_i
:= \mathbf{1}_G$ for $k < i \le n$.

Then it calls Algorithm~\ref{rattlestep} $N$
times with $(a,(h_1, \ldots, h_n))$ as argument
thereby changing it and
finally returns the last value of $a$ as a random element $a_0$ in $G$.

After this initialisation phase it produces a sequence of random
elements $(a_i)_{i \in \N}$ by calling 
Algorithm~\ref{rattlestep} repeatedly with 
$(a,(h_1, \ldots, h_n))$ as argument thereby changing it and
assigning the value of $a$ to $a_i$ after call number $i$.
\end{Def}

\begin{Rem}[Variant of product replacement]
\index{Rattle@\textsc{Rattle}}%
The \textsc{Rattle} method described above is a variant of the
``product replacement algorithm'', because the main part of a step replaces 
an element by a product of it with another one.
\end{Rem}

\begin{Rem}[Comments on the implementation of \textsc{Rattle}]
\index{Rattle@\textsc{Rattle}}%
It is not completely clear how to choose the parameters $n$ and $N$. In
principal $n$ could be chosen equal to $k$. However, what a good length
$N$ of the initialisation phase is depends on the group $G$ and
on the generating tuple $(g_1, \ldots, g_k)$. In practice one chooses $n$
slightly bigger than $k$ and $N$ around $100$ unless $k$ is very big.
For large $k$ the value of $N$ has to be chosen bigger. It is clear
that a constant value for $N$ will not work well for $|G| \to \infty$.
\end{Rem}

\begin{Prop}[\textsc{Rattle} converges to the uniform distribution]
    \label{proprattle}
\index{Rattle@\textsc{Rattle}}%
    The distribution of the element $a_0$ in the \textsc{Rattle} procedure (see
    Definition~\ref{rattle}) converges for $N \to \infty$ to the
    uniform distribution.
\end{Prop}
\proofbeg
See \cite[Section~4]{LGMurray}.
\proofend

\subsubsection{A brief discussion of \textsc{Rattle}}

\index{Rattle@\textsc{Rattle}}%
Although it can be proved that for every finite group $G$ and every
generating tuple $(g_1, \ldots, g_k)$ the distribution of the element
$a_0$ for $N \to \infty$ tends to the uniform distribution (see
Proposition~\ref{proprattle}), there is
not much known about the rate of convergence. So, picking the value for
$N$ in practice is difficult, in particular if we do not know anything
about $G$.

If we use the sequence produced by the \textsc{Rattle} method
after initialisation as a sequence of random elements in $G$, 
subsequent elements seem to be uniformly distributed, but adjacent
elements in the sequence are clearly not distributed independently.
Obviously the next element in the sequence depends heavily on the
previous state $(a,(h_1,\ldots,h_n))$ and the previous element is
equal to the $a$ value of this state. However, the state contains of
course more information than simply the value $a$.

Despite these obvious deficiencies, the algorithm
works surprisingly well in practice. The computational cost for the 
initialisation is 200 multiplications and after that 2 more
multiplications for every further element in the sequence.
The memory requirements are minimal and the produced sequence of
random elements is good enough for most purposes. Analysing algorithms
that use \textsc{Rattle}
\index{Rattle@\textsc{Rattle}}%
with the assumption that it produces uniformly distributed random
elements in the group provides good predictions on how
well these algorithms work in practice.

Throughout this book the \textsc{Rattle} method is used to produce
random elements in group and the above assumption is made.
\index{Rattle@\textsc{Rattle}}%


% this is a part of the habilitation thesis of Max Neunhoeffer

\chapter{Matrices over finite fields}

\section{Implementing matrices over finite fields}

\index{Matrix}

\section{Matrix arithmetic}

\section{Basic algorithms for matrices}


% Have renamed:
%  env{Lem} -> env{Lemm}
%  env{remark} -> env{Rem}

%\title{Computing Minimal Polynomials of Matrices}
%
%\author{Max Neunh\"offer}
%\email{max.neunhoeffer@math.rwth-aachen.de}
%%\homepage{http://www.math.rwth-aachen.de/~Max.Neunhoeffer}
%\address{Lehrstuhl D f\"ur Mathematik, Templergraben 64, 52062 Aachen,
%Germany}
%\author{Cheryl E.~Praeger}
%\email{praeger@maths.uwa.edu.au}
%%\homepage{http://www.maths.uwa.edu.au/~praeger/}
%\address{University of Western Australia, School of Mathematics and
%Statistics\\(M019), 35 Stirling Highway, Crawley 6009, Western Australia}
%
%\begin{abstract}
%We present and analyse a Monte-Carlo algorithm to compute the minimal 
%polynomial of an $n\times n$ matrix over a finite field that requires
%$O(n^3)$ field operations and $O(n)$ random vectors, and is well suited for
%successful practical implementation. The algorithm, 
%and its complexity analysis, use standard algorithms for polynomial 
%and matrix operations. We compare features of the algorithm with 
%several other algorithms in the literature. 
%In addition we present a deterministic verification procedure
%which is similarly efficient in most cases but has a worst-case
%complexity of $O(n^4)$. Finally, we report the results of practical experiments
%with an implementation of our algorithms in comparison with the
%current algorithms in the {\sf GAP} library.
%\end{abstract}
%
%\classification{Primary: 15A21; Secondary: 15A15}
%
%\keywords{Minimal polynomial, Frobenius normal form, Monte-Carlo algorithm,
%randomisation, matrix, finite field}

% \received{...}
% \revised{...}
% \accepted{...}
%
%\maketitle

\chapter{Computing characteristic and minimal polynomials}

This chapter is about the computation of characteristic and minimal
polynomials of matrices over finite fields. The contents of this
chapter until Section~\ref{performance} are joint work with Cheryl
Praeger and are already published as \cite{minpolypaper}.
%FIXME

\section{Introduction}

Let $\F$ be a finite field and $M \in \F^{n \times n}$ a matrix. This paper
presents and analyses a Monte Carlo  algorithm to compute the minimal 
polynomial of $M$, that is,
the monic polynomial $\mu \in \F[x]$ of least degree, such that
$\mu(M) = 0$. 
Determining the minimal polynomial is one of the fundamental computational
problems for matrices and has a wide range of applications. As well as 
revealing information about the Frobenius
normal form of $M$, the minimal polynomial also elucidates the structure 
of $\F^n$ viewed as $\F[x]$-module, where $x$ acts by multiplication with $M$. 
In addition the order of $M$ modulo scalars is often found by first 
determining the minimal polynomial. Apart from these applications it has
important practical utility, for example
in the context of the matrix group recognition project~\cite{OB}.

For these and other reasons a number of algorithms to determine the 
minimal polynomial may be found in the literature. We discuss some 
of them below. Our primary objective 
was to provide a simple and practical algorithm that could be 
implemented easily and would work well over small finite fields. In
particular we did not want to produce matrices with entries in 
larger fields or polynomial rings as intermediate results, 
and we preferred to restrict ourselves to using only row operations 
(rather than a combination of row and column operations). 
In addition we wished to use standard field and polynomial arithmetic,
and we wished to give  an explicit worst-case upper bound
for the number of elementary field operations needed, and not only an
asymptotic complexity statement.
Our Monte Carlo algorithm adheres to these requirements for matrices 
over finite fields $\F_q$.

\begin{Theo}\label{main}
For a given matrix $M \in \F_q^{n \times n}$ and a positive 
real number $\epsilon < 1/2$, Algorithm~\ref{algminpolymc}
computes the minimal polynomial of $M$ with probability at least $1-\epsilon$.
For sufficiently large $n$ and fixed $\epsilon$, the number of elementary 
field operations required is less than $7n^3$ plus the costs of 
factorising a degree $n$ polynomial over $\F_q$ and constructing at most $n$
random vectors in $\F^n$.
\end{Theo}

%For more details see Section~\ref{minpoly}. 
The novel aspect of our algorithm 
is the introduction of randomisation in our characteristic polynomial computation. While
not necessary for this computation it underpins our proof of the Monte Carlo 
nature of our minimal polynomial algorithm. In addition to the Monte Carlo
minimal polynomial algorithm
we present and analyse in Section~\ref{verify} a deterministic verification procedure to
be run after Algorithm~\ref{algminpolymc} that has a similar asymptotic
complexity in many cases, but is $O(n^4)$ in the worst-case scenario.
Our motivation for giving concrete upper bounds for the costs of 
various component procedures was that, in practical implementations, 
these assist us to compare different algorithms in order to 
decide which to use in different situations. At the end of the paper we discuss
a practical implementation and tests of the algorithms in the {\sf GAP}
system~\cite{GAP4}.

\subsection{Other algorithms in the light of our requirements}

There are several interesting and asymptotically efficient
minimal polynomial algorithms of $n\times n$ matrices in the literature. The most  
asymptotically efficient deterministic algorithm is due to Storjohann 
\cite{Stor01} in 2001. It is nearly optimal, `requiring about the same number of 
field operations as required for matrix 
multiplication'\footnote{\cite[Abstract, p368]{Stor01}}. 
It involves a divide-and-conquer strategy that produces matrices 
with entries in polynomial rings as intermediate results. 
Changing the scalars to a larger field or polynomial ring is 
something we wished to avoid as it creates additional complications in practical 
applications within a computer algebra system used for group and 
matrix algebra computations.

Storjohann's earlier deterministic algorithm~\cite{Stor98} in 1998 uses classical 
field arithmetic and requires $O(n^3)$ field operations. 
It first reduces the matrix to `zig-zag form', using a mix of row 
and column operations, then produces the Smith normal form 
as a matrix with polynomial entries, and finally the Frobenius normal form.
In systems such as {\sf GAP}, matrices over small finite fields 
are stored in a compressed form that
makes row operations simple, but column operations difficult.
Restricting to one of these types of operations was one of our criteria.

A Monte Carlo minimal polynomial algorithm of Giesbrecht~\cite{Gie95} from 1995
that runs in `nearly optimal time' contains some 
features we find desirable for practical implementation, namely 
his algorithm first constructs a `modular cyclic decomposition' 
using random vectors, similar to our characteristic polynomial 
computation in Section~\ref{charpoly}. However, further steps include a 
modification of the `divide-and-conquer' 
Keller-Gehrig algorithm \cite{KelG85} and lead to a 
Las Vegas algorithm that computes a Frobenius form over an extension 
field and then the minimal polynomial.
The field size over which the given matrix is written is assumed 
to be greater than $n^2$, and if this is not the case it is 
suggested that an embedding into a larger extension field be used.
Several of these features were undesirable for us.

In \cite[Section 4]{AC97} Augot and Camion propose a deterministic algorithm
to compute the minimal polynomial of a matrix which is to some extent
similar to our algorithm. It is deterministic with
complexity $O(n^3 + m^2 \cdot n^2)$ field operations, 
where $m$ is the number of blocks
in the shift Hessenberg form. They prove that the complexity is $O(n^3)$ in the
average case. However, in the worst case it is $O(n^4)$, and 
no constants are provided in the complexity estimates. Although the
principal approach of their algorithm is similar to ours, the details
differ very much from our algorithm and analysis.

An interesting commentary on various algorithms, together with some new 
algorithms is given by Eberly~\cite{Eb00}. Eberly (see Theorem 4.2
in \cite{Eb00}) gives in particular a randomised algorithm for matrices 
over small fields that produces output from which (amongst other things)
the minimal polynomial can be computed, at a cost of $O(n^3)$.
The papers \cite{Eb00,Gie95,Stor98,Stor01} contain references to other 
minimal polynomial algorithms. In all of the algorithms mentioned 
the asymptotic complexity statements give no information about 
the constants involved. 

\subsection{Outline of the paper}
In Section~\ref{notation} we introduce our notation, in
Section~\ref{complexity} we cite a few complexity bounds for basic
algorithms. The next Section~\ref{ordpoly} introduces order polynomials
and derives a few results about them. Then we turn to the computation
of the characteristic polynomial in Section~\ref{charpoly}, since this
is the first step in our minimal polynomial algorithm, which is described
and analysed in Section~\ref{minpoly}. We explain and modify the 
well-known algorithm to compute characteristic polynomials by introducing
some randomisation, because this is later needed in the analysis of our
main Monte Carlo algorithm. In Section~\ref{probest} we give some
probability estimates that are also used later in the analysis.
The second last Section~\ref{verify}
covers the deterministic verification of the results of our
Monte Carlo algorithm. We describe in detail cases in which this
verification is efficient and when it has a worse complexity.
Finally, in Section~\ref{performance} we report on the performance
of an implementation of our algorithm, including runtimes in 
realistic applications.  We compare these
times with the current implementation for minimal polynomial computations
in the {\sf GAP} library (see \cite{GAP4}). We show that our algorithm performs
much better in important cases, and that our bounds on the computing
cost are reflected in practical experiments.


\section{Notation}
\label{notation}

Throughout the paper $\F$ will be a fixed field. Although we envisage
$\F$ to be a finite field for our applications, this is not necessary
for most of our results. However in the later sections we use some probability 
estimates from Section~\ref{probest} that are only valid for 
finite fields.

In all our runtime bounds we will assume that one 
field operation (that is, one addition, difference, multiplication or
inversion) takes a fixed amount of time and we simply count
the number of such operations occurring in our algorithms.

We denote the set of $(m \times n)$-matrices over $\F$ by $\F^{m \times n}$
and the set of row vectors of length $m$ by $\F^m$. For a vector
$v \in \F^m$ we write $v_i$ for its $i$-th component and for a matrix 
$M \in \F^{m \times n}$ we denote its $i$-th row, which is
a row vector of length $n$, by $M[i]$. We use ``row vector
times matrix'' operations, and in general right modules throughout.
If $V$ is a vector space over $\F$ and $W$ is a subspace, the
quotient space is denoted by $V/W$ and its cosets by
$v+W$ for $v \in V$. The $\F$-linear span of the vectors
$v^{(1)}, \ldots, v^{(k)} \in V$ is denoted by 
$\left< v^{(1)}, \ldots, v^{(k)}\right>_\F$.

If $M \in \F^{n \times n}$ is a matrix and $V = \F^n$, we have a
natural action of $M$ as an endomorphism of $V$ by right multiplication.
The same holds for every $M$-invariant subspace $W < V$ and for
the corresponding quotient space $V/W$. We describe such a situation
by saying that ``the matrix $M$ induces an action on the $\F$-vectorspace''
$V, W, V/W$ respectively.

Throughout, $\F[x]$ denotes
the polynomial ring over $\F$ in an indeterminate $x$. For a square matrix $M$ and
a polynomial $p \in \F[x]$ we denote the evaluation of $p$ at $M$
by $p(M)$.

Whenever a matrix $M$ induces an action on a vector space $U$, we
will view $U$ as a right $\F[x]$-module by letting $x$ act like $M$,
that is $v \cdot x := v\cdot M$ in the above examples. We denote the
characteristic polynomial of this action by $\chi_{M,U}$. That is,
$\chi_{M,U}$ is the characteristic polynomial 
of the $(\dim_\F(U) \times \dim_\F(U))$-matrix given by choosing 
a basis of $U$ and writing the
action of $M$ induced on $U$ as a matrix with respect to that basis.
We use the same convention analogously for the corresponding minimal
polynomial $\mu_{M,U}$. Furthermore, we denote the $\F[x]$-submodule of
$U$ generated by the vectors $u_1, \ldots, u_n$ by $\left< u_1, \ldots,
u_n \right>_M$.

We use the two functions 
\begin{equation}\label{si}
s^{(1)}(a,b) := \sum_{i=b+1}^a i\quad \mbox{and}\quad
s^{(2)}(a,b) := \sum_{i=b+1}^a i^2
\end{equation}
for complexity expressions.
Note that for $a > b > c$ we have $s^{(j)}(a,c) = s^{(j)}(a,b) + s^{(j)}(b,c)$
for $j \in {1,2}$ and 
\begin{eqnarray}
\label{formels1}
s^{(1)}(n,0) &=& s^{(1)}(n,-1) = \frac{n(n+1)}{2}
\qquad\mbox{and} \\
\label{formels2}
s^{(2)}(n,0) &=& s^{(2)}(n,-1) = \frac{n(n+1)(2n+1)}{6}.
\end{eqnarray}

For later complexity estimates we note the following inequalities.

\begin{Lemm}[Some upper bounds]
\label{estimates}
If $n = \sum_{i=1}^k d_i$ for some $d_i \in \N \setminus\{0\}$ and
$s_j := \sum_{i=1}^j d_i$ we have
\[ \sum_{j=1}^k s_j \le \frac{n(n+1)}{2} \quad \mbox{and} \quad
   \sum_{j=1}^k s_j(s_j+1) 
   \le \frac{n(n+1)(n+2)}{3}. \]
\end{Lemm}
\proofbeg
We claim that for fixed $n$ both expressions are maximal if and only if
all $d_i$ are equal to one. Namely, if we replace $d_j$ in some sequence
$(d_i)_{1 \le i \le k}$ by the two numbers $a$ and $d_j-a$ resulting
in the new sequence $(d'_1, \ldots, d'_{k+1}) := 
(d_1, d_2, \ldots, d_{j-1}, a, d_j-a, d_{j+1}, \ldots, d_k)$ 
of length $k+1$, the following happens: For $s'_u := \sum_{i=1}^u d'_i$
we observe, that $s'_u=s_u$ for $1\le u\le j-1$, $s'_j=s_{j-1}+a$, and 
$s_u'=s_{u-1}$ for $j+1\leq u\leq k+1$. Thus we get
\[ \sum_{i=1}^{k+1} s'_i = \left(\sum_{i=1}^{j-1} s_i \right)
   + s_{j-1}+a + \left(\sum_{i=j}^k s_i\right)
   = \left(\sum_{i=1}^k s_i\right) + s_{j-1} + a
   > \sum_{i=1}^k s_i \]
and
\[ \left( \sum_{i=1}^{j-1} s_i(s_i+1) \right)
   + (s_{j-1}+a)(s_{j-1}+a+1)
   + \left( \sum_{i=j}^k s_i(s_i + 1) \right)
   > \sum_{i=1}^k s_i(s_i + 1) \]
respectively. 

Since every sequence $(d_i)_{1 \le i \le k}$ can
be refined to the constant sequence $(1)_{1 \le i \le n}$ the
expressions in the lemma can be bounded as follows:
\[ \sum_{j=1}^n j = \frac{n(n+1)}{2} \quad\mbox{and}\quad
   \sum_{j=1}^n j(j+1)
   = \frac{n(n+1)(2n+1)}{6} + \frac{n(n+1)}{2} \]
respectively using Formulae~(\ref{formels1}) and (\ref{formels2}) 
proving all claims.
\proofend

\section{Complexity bounds for basic algorithms}
\label{complexity}

In some algorithms presented in later sections we use greatest common divisors
of univariate polynomials. To analyse these algorithms we use the 
following bounds which arise from standard polynomial computation. 
We take this approach because the standard algorithms for polynomials
are good enough for our complexity estimates in applications and we do not 
need the asymptotically best algorithms, discussion of which may be found 
conveniently in \cite{vzG}. By an \emph{elementary field operation} 
we mean addition, subtraction, multiplication or division of two field 
elements.

%\begin{Not}[Number of steps of polynomial division with remainder]
%\label{opsnot}
%We denote by $\opspd(n,m)$ the number of elementary field operations
%necessary to divide a polynomial $f \in \F[x]$ by a polynomial $g \in \F[x]$
%with remainder where $\deg f = n \ge m = \deg g$. 

%We denote by $\opsgcd(n,m)$ the number of elementary field operations
%necessary to compute the greatest common divisor of two polynomials
%$f,g \in \F[x]$ where $\deg f = n \ge m = \deg g$.

%Of course, these depend on the algorithms used.
%\end{Not}

\begin{Prop}[Complexity of standard greatest common divisor algorithm]
\label{standardgcd}\mbox{}

Let $f,g \in \F[x]$ with $n := \deg f \ge \deg g =: m$, and $f = qg + r$
with $q,r \in \F[x]$ such that $r=0$ or $\deg r < \deg g$. Then there is an
algorithm to compute $q$ and $r$ that needs less than $2(m+1)(n-m+1)$ 
elementary field operations. 

Furthermore, there is an
algorithm to compute $\gcd(f,g)$ that needs less than
$2(m+1)(n+1)$ elementary field operations.
\end{Prop}

\begin{Rem}{We intentionally give bounds here which are not best possible,
because we want the bound for the $\gcd$ computation to be symmetrical in
$m$ and $n$.}
\end{Rem}

\proofof{Proposition \ref{standardgcd}}
Use polynomial division and the standard {\sc Gcd} algorithm
and count. See \cite[Section 2.4 and Section 3.3]{vzG} for
smaller bounds that imply our symmetric bounds.
%The standard polynomial division algorithm repeatedly subtracts a shifted 
%scalar multiple of $g$ from $f$. Since $\deg g = m$, such a subtraction
%needs one scalar division and $2m$ elementary field operations and we have to
%repeat this at most $n-m+1$ times resulting in a maximal number
%of elementary field operations of $(2m+1)(n-m+1)$ which is smaller than
%$2(m+1)(n-m+1)$ as claimed.
%
%The standard Euclidean algorithm to compute the greatest common divisor
%first divides $f$ by $g$ and then repeatedly divides the current remainder
%by the previous one. Thus, the maximal number of elementary field
%operations needed is
%\begin{eqnarray*}
%\opspd(n,m) &+& \sum_{i=1}^{m-1} \opspd(i+1,i)
%   \le 2(m+1)(n-m+1) + \sum_{i=1}^{m-1} 4(i+1) \\
%   &=& 2(m+1)(n-m+1) + 4\cdot \frac{(m-1)(m+2)}{2} \\
%   &<& 2(m+1)(n-m+1) + 2(m+1)m \\
%   &=& 2(m+1)(n+1).
%\end{eqnarray*}
\proofend

\subsection{Polynomial factorisation}\label{polyfactn}

Some of our algorithms return partially factorised polynomials
which facilitate later factorisation into
irreducible factors. However, since the extent of this partial factorisation
is difficult to estimate, we use in our
analyses the complexity of finding the complete 
factorisation of a polynomial over a finite field as a product of 
irreducibles. We need such factorisations in our main algorithm.
In keeping with our other methods we make use of standard
polynomial factorisation procedures.

Details can be found in Knuth~\cite[4.6.2]{knuth} of a deterministic 
polynomial factorisation algorithm inspired by an idea of Berlekamp. 
Its cost is polynomial in both the degree $n$ and field size 
$|\F|=q$, as it requires $O(q)$ computations
of greatest common divisors. Thus it works well only for $q$ small. 
%However, 
%the first part of the procedure can be used as a test for irreducibility, 
%and the cost for this part depends only on $n$ and $\log_2q$, 
%rather than on $q$, and so can be used for all values of $q$.
When $q$ is large there is available a randomised (Las Vegas) 
version of the procedure 
which will always return accurately the number $r$ of irreducible factors 
of $f(x)\in\F[x]$, but for which there is a small nonzero probability that it will 
fail to find all the irreducible factors.
It involves the procedure {\sc RandomVector}, which is discussed further in Subsection~\ref{random}, to 
produce independent uniformly distributed random elements of an 
$n$-dimensional vector space over $\F$  for 
which a basis is known. Throughout the paper logarithms are always taken to base $2$.

\begin{Rem}[\sc PolynomialFactorisation]\label{rem:polyfactn}
Suppose $f(x)\in \F[x]$ of degree $n\geq 1$ with $r$ irreducible 
factors (counting multiplicities) and, if $q$ is large, suppose that 
we are given a real number $\ve$ such that $0<\ve<1/2$. The  number 
fact$(n,q)$ of elementary field operations required to factorise $f$ is at most

\smallskip
\begin{tabular}{lp{1.8in}}
$8n^3 + (3qr+17\log_2q)n^2$&to find the complete set of irreducible 
factors, if $q$ is small; \\
$O\big((\log \,\ve^{-1})(\log n)(\xi_n $ $+ n^2\log^3 q) +n^3\log^2q\big)$& 
to find the complete set of irreducible factors, if $q$ is large;\\
\end{tabular}

\noindent where $\xi_n$ is an upper bound for the cost of one run of\/
{\sc RandomVector} on $\F^n$.
If $q$ is large the second algorithm may fail, but with probability less than $\ve$.
\end{Rem}





\section{Order polynomials}
\label{ordpoly}

Let $M$ be a matrix in $\F^{n \times n}$ that induces an action
on an $\F$-vector space $V$.

We briefly recall the definition of the term ``order polynomial'':

\begin{Def}[Order polynomial $\ord_M(v)$ and relative order polynomial]
{\rm
The \emph{order polynomial} $\ord_M(v)$ of a vector $v \in V$ is the
monic polynomial $p \in \F[x]$ of smallest degree such that $v \cdot
p(M) = 0 \in V$. In particular $\ord_M(0)=1$.

For an $M$-invariant subspace  $W < V$, the \emph{relative order polynomial}
$\ord_M(v+W)$ (of $v$ relative to $W$) is the order polynomial of the element
$v+W \in V/W$ with respect to the induced action of $M$ on $V/W$.
}
\end{Def}

\begin{Rem}
If we consider $V$ as an $\F[x]$-module as in 
Section~\ref{notation}, then $p$ is the monic generator
of the annihilator $\ann_{\F[x]}(v)$ of $v$ in $\F[x]$.
\end{Rem}

%\smallskip
The following observation follows immediately from the definition above.

\begin{Lemm}[Relative order polynomials]
\label{relorderpol}
For an $M$-invariant subspace  $W < V$ and $v\in V$, 
$\ord_M(v+W)$ is the monic polynomial $p \in \F[x]$ of smallest degree such 
that $v \cdot p(M) \in W$.
\end{Lemm}
%\proofbeg This is clear because a coset $v+W \in V/W$ is zero
%if and only if $v \in W$. \proofend

%\smallskip
We now turn to the question of how one computes the order polynomial of
a vector $v \in V$. The basic idea is to apply the matrix
$M$ to the vector repeatedly computing a sequence $v, vM, vM^2, \ldots, vM^d$
until $vM^d$ is a linear combination
\[ 
vM^d = \sum_{i=0}^{d-1} a_i vM^i, 
\]
with $a_i \in \F$, for $0 \le i < d$. If $d$ is minimal such that
this is possible, we have
\[ \ord_M(v) = x^d - \sum_{i=0}^{d-1} a_i x^i. 
\]
Although this procedure is simple and well-known, we 
present it in order to make explicit the number of elementary field 
operations needed.
To this end we describe in detail the computation of solutions for
the systems of linear equations involved. 

\begin{Def}[Row semi echelon form]
{\rm
    A non-zero matrix $S = (S_{i,j}) \in \F^{m \times n}$ is in 
    \emph{row semi echelon form} if there are positive 
    integers $r \le m$ and $j_1,\ldots,j_r \le n$ such that,
    for each $i \le r$, $S_{i,j_i} = 1$ and $S_{k,j_i} = 0$ for all $k > i$, 
    and also $S_{k,j} = 0$ whenever $k > r$. For $i \le r$, column 
    $j_i$ is called the \emph{leading column of row $i$}, and we write
    $\lc(i) = j_i$. A sequence of vectors $u_1,\ldots,u_m \in F^n$ is 
    said to be in semi echelon form if the matrix with rows 
    $u_1,\ldots,u_m$ is in row semi echelon form.
}
\end{Def}

\begin{Def}[Semi echelon data sequence]
{\rm    Let $Y \in \F^{m \times n}$ be a matrix with $m \le n$ and of rank $m$. A 
    \emph{semi echelon data sequence for $Y$} is a tuple $\calY=
(Y,S,T,l)$, where
    $S \in \F^{m \times n}$ is in row semi echelon form with
    leading column indices $l = (\lc(1), \ldots, \lc(m))$, and
    $T \in \GL(m,\F)$ with $TY=S$. Further, $T$ is a lower triangular
    matrix, that is, for $T = (T_{i,j})$ we have
    $T_{i,j} = 0$ for $i < j$. For a semi echelon data sequence $\calY$
    we call the number $m$ its \emph{length}, sometimes denoted
    $\len(\calY)$.

A semi echelon data sequence $\calY'=(Y',S',T',l')$ is said to 
\emph{extend} $\calY$
if $\len(\calY')>\len(\calY)$, the first $\len(\calY)$ rows of $Y', S'$ form
the matrices $Y, S$ respectively, and the first $\len(\calY)$ entries of $l'$
form the sequence $l$. 
}
\end{Def}

\begin{Rem}\label{rem:seds}
(a) The idea of this concept is that for a matrix 
$S \in \F^{m\times n}$ in row semi echelon form it is relatively
cheap to decide whether a given vector $v \in \F^n$ lies in the row
space of $S$, and if so, to write it as a linear combination of the
rows of $S$, that is, to find a vector $a \in \F^m$ such that
$v = aS = aTY$ (see Algorithm~\ref{clean}). Thus, the vector
$v$ is expressed as a linear combination of the rows of $Y$
using the vector $aT$ as coefficients.

(b) It is convenient to introduce, for each $n$, a \emph{trivial semi echelon data 
sequence} $\calY_0=(Y,S,T,l)$ of length zero, in which we regard $Y, S$ as 
empty matrices in $\F^{0\times n}$, $T$ an empty $0\times 0$ matrix, and 
$l$ an empty sequence of length $0$. By convention we take the row spaces of
the empty matrices $Y, S$ to be the zero subspace of $\F^n$.  
\end{Rem}
%we need to make clear what is assumed about the trivial $\calY$, and 
%modify Alg 1 to cover it.

\medskip
We now present Algorithm~\ref{clean}, which is one step in the
computation of a semi echelon data sequence for a matrix $Y$. We
denote by $S[i]$ the $i$-th row of the matrix $S$, and by $\rsp(S)$ the 
row space of $S$.

\begin{algorithm}
\caption{$\quad$ \sc CleanAndExtend}
\label{clean}
\begin{algorithmic}
\STATE \textbf{Input:} A semi echelon data sequence $\calY=(Y,S,T,l)$ with 
         $Y,S \in \F^{m \times n}$  (possibly $m=0$), $v \in \F^n$.
\STATE \textbf{Output:} A triple $(c,\calY',a')$ where $c$ is {\sc True} 
or {\sc False}, $\calY'$ equals or extends $\calY$, and $a'\in\F^{\len
(\calY')}$.
\vspace*{2mm}
\STATE $w := v$
\STATE $a := 0 \in \F^m$, or $a:=\emptyset$ if $m=0$ \hspace*{2mm}
\COMMENT{note that $w=v-aS$ if $m\ne0$}
\FOR {$i = 1$ to $m$}
    \STATE $a_i := w_{l_i}$
    \STATE $w := w - a_i \cdot S[i]$
\ENDFOR\hspace*{6mm}
\COMMENT{still $w=v-aS$ if $m\ne0$} 
\IF {w = 0}
    \STATE \textbf{return} $(\mbox{\sc True},(Y,S,T,l),a)$
\ELSE
    \STATE $j := $ index of first non-zero entry in $w$
    \STATE $a' := [ a \ \ w_j ]$,\quad $l' := \left[ l \ \ j \right]$, $\quad$
    \STATE $Y' := \left[ 
       \begin{array}{c} Y \\ v \end{array} \right]$, $\quad$
           $S' := \left[ 
       \begin{array}{c} S \\ w_j^{-1} \cdot w \end{array} \right]$, $\quad$
           $T' := \left[ 
       \begin{array}{cc} T & 0 \\ -w_j^{-1}\cdot aT & w_j^{-1} 
       \end{array} \right]$
       or $T'=\left[ w_j^{-1}\right]$ if $m=0$.
    \STATE \textbf{return} $(\mbox{\sc False}, (Y',S',T',l'), a' )$
\ENDIF
\end{algorithmic}
\end{algorithm}

\begin{Prop}[Correctness and complexity of Algorithm~\ref{clean}:
\label{PropCleanAndExtend}{\sc CleanAndExtend}]
If Algorithm~\ref{clean} returns $(${\sc True}, $\calY,a)$, then $v\in\rsp
(Y)$, and either $m=0$ with $v=0$ and $a$ an empty sequence, or $m>0$ and $v=aS$. If 
Algorithm~\ref{clean} returns $(${\sc False}, $\calY',a')$, then $v\not
\in\rsp(Y)$, $\calY'$ has length $m+1$ and extends $\calY$, and $v=a'S'$.
Algorithm~\ref{clean} requires at most $2mn$ field operations
if $v\in\rsp(Y)$, and
$(2m+1)n + (m+1)^2 + 1$ field operations otherwise.
\end{Prop}

\begin{Rem}\label{rem:cae}
(a) Given a semi echelon data sequence $(Y,S,T,l)$ with $Y,S \in \F^{m
\times n}$ and a vector $v \in \F^n$, Algorithm~\ref{clean} tries to
write $v$ as a linear combination of the rows of $S$. If this is not
possible, it constructs an extended semi echelon data sequence.

\medskip\noindent
(b) For the case of finite fields a simple and useful
optimisation is to reduce, where possible, the number of operations for vectors and matrices, 
for example, where a vector is multiplied by the zero scalar and the
result is added to some other vector. This can reduce the number of operations for
sparse vectors and matrices. Our estimates for the numbers of field operations then become
over-estimates.
\end{Rem}

\smallskip
\proofof{Proposition~\ref{PropCleanAndExtend}}
Suppose first that $m=0$. If $v=0$ then the algorithm returns  $(${\sc True}, $\calY,a)$, 
with $a$ an empty sequence, and by convention $v\in \rsp(Y)$. On the other hand if 
$v\ne 0$, then the algorithm returns  $(${\sc False}, $\calY',a')$, with $a'=[v_j], S'=[v_j^{-1}v]$ 
where $j$ is minimal such that the $j$-entry of $v$ is non-zero. Thus $v\not\in\rsp(Y)$ (by convention)
and $v=a'S'$ as asserted. Also $T'Y'=[v_j^{-1} v]=S'$ so $\calY'$ extends $\calY$. 

\medskip
We may therefore assume that $m\geq1$. During the \textbf{for} loop the algorithm `cleans out the positions in the copy
$w$ of $v$ corresponding to leading columns of the rows of $S$'. That is to say, at the end of the loop,
$a=(v_{l_1},\dots,v_{l_m})$, and $w=v-\sum_{i=1}^m a_i S[i] =v-aS$ has a zero entry in position $l_i$,
for each $i\leq m$. 
If at this stage $w=0$, then we have $v=aS\in\rsp(S)=\rsp(Y)$, and the algorithm returns $(${\sc True}, $\calY,a)$. Suppose then that 
$w\ne0$ after the \textbf{for} loop. Then the leftmost position
$j$ in $w$ such that $w_j\ne0$ is found, and $w$ is `divided' by the scalar $w_j$. 
The resulting vector $w' := w_j^{-1} \cdot(v - aS)$ becomes  
row $m+1$ of $S'$ with $j$ the leading column of row $m+1$. Since $v$ is the
new last row of $Y'$ we have $T'Y' = S'$, because 
$-w_j^{-1} \cdot aTY + w_j^{-1} v = w_j^{-1} \cdot (v-aS) = w'$.
By construction, $T'$ is an invertible lower triangular matrix and
$S'$ is in row semi echelon form. Thus $\calY'=(Y',S',T',l')$ is a
semi echelon data sequence of length $m+1$ extending $\calY$. 
The algorithm returns  $(${\sc False}, $\calY',a')$, and $v=w+aS= a'S'$.

The \textbf{for} loop needs $2mn$ field operations if we count both multiplications
and additions. If $v\in\rsp(Y)$ then the algorithm terminates after this loop.
On the other hand, if  $v\not\in\rsp(Y)$, then 
Algorithm~\ref{clean} needs one inversion
of the scalar $w_j$ plus $2 \cdot \sum_{i=1}^m i = m(m+1)$
field operations for the vector times matrix multiplication
$aT$, because $T$ is a lower triangular matrix. This is altogether $m(m+1)+1$ operations. 
Finally, the scalar negation of $w_j^{-1}$ and the multiplication of $aT$ by 
$-w_j^{-1}$ needs another $m+1$ field operations, and a further $n$ operations 
are needed for the computation of $w_j^{-1} w$ in $S'$. Thus the
total number of field operations is at most $2mn + (m+1)^2 + 1 + n$.
\proofend

\smallskip
Having Algorithm~\ref{clean} at hand we can now present
Algorithm~\ref{algordpoly}, which computes relative order polynomials.
Since a (non-relative) order polynomial may be regarded as a relative order
polynomial with respect to the zero subspace, Algorithm~\ref{algordpoly}
can also be used to compute order polynomials, starting with the trivial semi
echelon data sequence, (see Remark~\ref{rem:seds}~(b)).

\begin{algorithm}
\caption{$\quad$ \sc RelativeOrdPoly}
\label{algordpoly}
\begin{algorithmic}
\STATE \textbf{Input:} A semi echelon data sequence $\calY=(Y,S,T,l)$ with 
$Y,S \in \F^{m \times n}$ (possibly $m=0$),
\STATE \hspace*{0mm} \phantom{\textbf{Input:}}% 
$v \in \F^n$, 
and $M \in \F^{n \times n}$ such that $W:=\rsp(Y)$ is $M$-invariant.
\STATE \textbf{Output:} A triple $(p,\calY',b)$ where $p\in\F[x]$, $\calY'$ equals or 
extends $\calY$, and $b\in\F^{\len(\calY')}$.

\vspace*{2mm}
\STATE $(Y',S',T',l') := (Y,S,T,l)$ \hspace*{5mm}
\COMMENT{the primed variables will change during the algorithm}
\STATE $v' := v$
\STATE $m' := m$ \hspace*{3.5cm}  \COMMENT{can be zero!}
\LOOP
    \STATE $(c,(Y',S',T',l'),a) := \mbox{\sc CleanAndExtend}((Y',S',T',l'),v')$  \hspace*{1cm} 
\COMMENT{$T'Y'=S'$, $v'=aS'$}
    \IF { $ c = \mbox{\sc True} $ }
        \STATE \textbf{leave loop}
    \ENDIF
    \STATE $v' := v' \cdot M$
     \STATE $m' := m' +1$	
\ENDLOOP \hspace*{3.5cm}  \COMMENT{at this stage $c=$ {\sc True}, $v'=aS'$, $m'=\len(\calY')$}
\STATE $d := m'-m$
\STATE $b := a\cdot T'$
\STATE $p :=x^d-\sum_{i=0}^{d-1} b_{m+1+i} x^i$
\STATE \textbf{return} $(p, (Y',S',T',l'),b)$
\end{algorithmic}
\end{algorithm}

\begin{Prop}[Correctness and complexity of Algorithm~\ref{algordpoly}:
{\sc RelativeOrdPoly}]
\label{proprelorderpol}
Let $\calY=(Y,S,T,l)$ be a semi echelon data sequence with $Y,S \in \F^{m \times
n}$ (possibly $m=0$), $v \in \F^n$, and $M \in \F^{n \times n}$
such that $W:=\rsp(Y)$ is $M$-invariant. 
Then  Algorithm~\ref{algordpoly} returns a triple  $(p,\calY',b)$ 
consisting of the relative order polynomial $p := \ord_M(v+W)$ of degree $d$, a 
semi echelon data sequence $\calY'=(Y',S',T',l')$ of length $m+d$ equal to or extending $\calY$, 
and a vector $b \in \F^{m+d}$ such that $vM^d = bY'$. If $d>0$, then
rows $m+1, \ldots, m+d$ of $Y'$ are equal to $v,vM,\ldots,vM^{d-1}$
respectively. Algorithm~\ref{algordpoly} requires at most
\begin{eqnarray*}
2dn^2 &+& (n+2)d +2(m+d)n 
+ 2(n+1)s^{(1)}(m+d-1,m-1) + \cdots \\
\cdots &+& s^{(2)}(m+d-1,m-1) 
+ 2s^{(1)}(m+d,0)
\end{eqnarray*}
elementary field operations where $s^{(1)}$ and $s^{(2)}$ are the
functions defined in (\ref{si}).
\end{Prop}

\begin{Rem}
Note that, if $d=0$ then $S'=S$, so $v=b'Y\in W$, and in this case $p=1$. 
Algorithm~\ref{algordpoly} successively considers the vectors $v+W,
vM+W, \ldots, vM^d+W$  (those are the successive values of
$v'$) until $vM^d+W$ lies in the subspace of $V/W$ 
generated by the vectors $v+W, vM+W, \ldots, vM^{d-1}+W$. 
The given matrix $S$ together with Algorithm~\ref{clean} defines a
direct sum decomposition of $\F$-vector spaces $V := \F^n = W \oplus W'$
where $W'$ is the subspace of vectors having $0$ in all positions
occurring in the list $l$. Since $W' \cong V/W$, 
Algorithm~\ref{algordpoly} effectively computes in $V/W$ by always `cleaning
out' vectors using $S$ first. 
\end{Rem}

\proofof{Proposition~\ref{proprelorderpol}} 
By Proposition~\ref{PropCleanAndExtend}, after each run of 
{\sc CleanAndExtend} in the \textbf{loop}, the tuple 
$(Y',S',T',l')$ is a semi echelon 
data sequence equal to or extending $\calY$, so $T'Y'=S'$, and $v'=aS'$. 
Moreover, at the beginning of the $i^{th}$ run of the  \textbf{loop} (since $c=$ 
{\sc False} at the end of each previous run), we have
$m'=m+i-1, v'=vM^{i-1}$,  and $Y'=Y$ if $i=1$ and otherwise $Y'$ is $Y$ 
followed by additional rows $v,vM,\dots,vM^{i-2}$. Thus the  \textbf{loop}
runs $d+1$ times (possibly $d=0$) and at the end of the $(d+1)^{st}$ run 
we have $c=$ {\sc True}, $Y'$ is as claimed, and $v'=vM^d=aS'$. Now $b=aT'=(b_1,\dots,b_{m+d})$ satisfies
\[
vM^d=aS'=aT'Y'=bY'=b'Y+\sum_{i=0}^{d-1}b_{m+i+1}vM^i,
\]
that is, $vp(M)=b'Y\in W$, where $b'=(b_1,\dots,b_m)$. Moreover, $d$ is minimal 
such that an expression of this type exists, so $p=\ord_M(v+W)$.  

As for the number of elementary field operations, Algorithm~\ref{algordpoly}
calls {\sc CleanAndExtend} exactly $d+1$ times with the lengths 
of the input semi echelon data sequences being $m,m+1, \ldots, m+d$. Each but the last call
returns {\sc False} and the last call returns {\sc True}. Thus, by
Proposition~\ref{PropCleanAndExtend}, the number of steps needed for this is at most
\[
\left(\sum_{i=m}^{m+d-1} ((2i+1)n+(i+1)^2+1)\right)  +  2(m+d)n  \]
\[
   = s^{(2)}(m+d-1,m-1) + (2n+2)s^{(1)}(m+d-1,m-1) + (n+2)d +2(m+d)n.
\]
In addition, we have to do $d$ multiplications of $v'$ with $M$, which require
$2n^2$ elementary field operations each, and finally the computation of $b$
requires $2s^{(1)}(m+d,0)$ elementary field operations, again since 
$T'$ is a lower triangular matrix. Summing up gives the number in 
the statement. 
\proofend

\smallskip
We conclude this section with two lemmas that are used to compute
absolute order polynomials using relative ones. We again view $V$ as
an $\F[x]$-module by letting $x$ act like $M$. For $\{v_1,\dots,v_m\}
\subseteq V$, we denote by $\la v_1,\dots,v_m\ra_M$ the $\F[x]$-submodule 
of $V$ generated by $\{v_1,\dots,v_m\}$, that is, the smallest $M$-invariant
subspace containing  $\{v_1,\dots,v_m\}$. If $m=1$ then $\la v_1\ra_M$ 
is the $\F$-span of the set $\{v_1,v_1M,\dots, v_1M^{n-1}\}$. We refer to 
$\la v_1\ra_M$ as a \emph{cyclic subspace} 
relative to $M$.

\begin{Lemm}[Order polynomials in cyclic subspaces]
\label{ordpolcyclic}
Let $v\in V$, $W = \left< v \right>_M < V$, and
$p := \ord_M(v)$ with $d := \deg(p)$. 
Then for each $w \in W$, there is a unique polynomial
$q \in \F[x]$ of degree less than $d$ such that $w = vq(M)$.
Moreover,
\[ \ord_M(w) = \frac{p}{\gcd(p,q)}. \]
\end{Lemm}
\proofbeg
The existence and uniqueness of $q$ follows from identifying $V$ as an 
$\F[x]$-module. Let $f := \ord_M(w)$ and $r := \gcd(p,q)$, so that $wf(M)=0$, and 
$p=rp', q=rq'$ for some $p',q'\in\F[x]$ such that $\gcd(p',q')=1$. 
Then  $w p'(M) = v q(M) p'(M) = v p(M) 
q'(M) = 0$, and hence $f$  divides $p'$.

Assume first that $q$ is a divisor of $p$, so that $r = q$ and $p=qp'$. 
Then since $0=wf(M)=vq(M)f(M)$, it follows
that $p = \ord_M(v)$ divides $qf$. Thus $p'$ divides $f$, and so
$f=p'$ in this case, as required.

In the general case we have $a,b \in \F[x]$ such that 
$r = qa+pb$. Then since $wa(M) = vq(M)a(M) = vr(M) - vp(M)b(M) = vr(M)$,
it follows that $\ord_M(vr(M))$ divides $f$. However, $\ord_M(vr(M)) = p'$ 
by the previous paragraph, and hence $f = p'$.
\proofend

%\textbf{Or this one:}
%
%\proofbeg The cyclic module $v\F[x]$ is isomorphic
%to $\F[x]/(p\F[x])$ because $p\F[x] = \ann_{\F[x]}(v)$. Under this isomorphism
%$w$ is mapped to $q+p\F[x]$. Thus 
%$\ord_M(w)\F[x] = \ann_{\F[x]}(q+p\F[x])$. Since $\lcm(p,q) = qp/\gcd(p,q)$
%in the unique factorisation domain $\F[x]$ the statement follows
%as all occurring polynomials here are monic.
%\proofend

\begin{Lemm}[Absolute and relative order polynomials]
\label{absordpoly}
Let $W$ be an $M$-invariant subspace of $V$, $v \in V$ and 
$q := \ord_M(v+W) \in \F[x]$. Then
\[ \ord_M(v) = q \cdot \ord_M(vq(M)). \]
\end{Lemm}
\proofbeg
Let $p := \ord_M(v) \in \F[x]$. Then $vp(M) \in W$ (since $vp(M)=0$),
and hence by Lemma~\ref{relorderpol}, $q$ divides $p$. If $q=p$
then $vq(M)=0$ so $\ord_M(vq(M)) =1$ and the result follows.
Otherwise $\deg(q)< \deg(p)$, $w:=vq(M)\in\la v\ra_M$, and so by
Lemma~\ref{ordpolcyclic}, $p=\ord_M(w)\cdot \gcd(p,q)=\ord_M(w)\cdot q$.
% On the
% other hand, $p$ divides $q \cdot \ord_M(vq)$ since the latter is a
% polynomial annihilating $v$. Letting now $r \in \F[x]$ such that $p=qr$,
% we get that $r$ divides $\ord_M(vq)$ and since $0 = vqr$
% that $\ord_M(vq)$ divides $r$. Since both polynomials are monic it
% follows, that they are equal thus proving the claim.
\proofend

\section{Computing the characteristic polynomial}

\label{charpoly}

In this section we present a version of a standard algorithm for computing
the characteristic polynomial of a matrix together with its analysis. 
It differs from the standard version in its use of randomisation. 

\subsection{Random vectors}\label{random}

Our characteristic polynomial algorithm, and later ones, 
make use of the algorithms {\sc RandomVector} and 
{\sc RandomVector*} that produce independent uniformly 
dis\-tri\-bu\-ted random vectors, and independent uniformly distributed
random non-zero vectors,             
respectively, in a given finite vector space for which a basis is known.    
The algorithms are invoked for spaces
$\F^s$, for $s \in \N$, and for subspaces of $V$ of the form
%\[
%V(\neg\, l)=\{ v\,|\, v_{j}=0\quad\mbox{for}\quad  j\ne l_i, 1\leq i\leq m\}
%\quad\mbox{where}\quad l=(l_1,\dots,l_m).
%\]
\[
V(l)=\{ v\,|\, v_{l_i}=0\quad\mbox{for}\quad  1\leq i\leq m\}
\quad\mbox{where}\quad l=(l_1,\dots,l_m).
\]
If $l$ is the empty sequence then $V(l)=V$.
For a semi echelon data sequence $\calY = (Y,S,T,l)$, the 
vector space $V$ is the sum $V=V(l) \oplus \rsp(S)$. 

If $b=\mbox{\sc RandomVector}(\F^{\len(\calY)})$, then $bS$ is a
uniformly distributed random vector of $\rsp(S)$. 
Moreover we assume that for the disjoint spaces 
$\F^{\len(\calY)}$ and $V(l)$ the 
algorithms {\sc RandomVector} and {\sc RandomVector*} are applied 
independently so that in particular, 
if $a=\mbox{\sc RandomVector*}(V(l))$ 
then the sum 
$a + bS$ is a uniformly distributed random vector of $V\setminus \rsp(S)$. 

{\sc RandomVector} and, if we neglect the possibility of obtaining
the zero vector, also {\sc RandomVector*}, could proceed by selecting
independent uniformly distributed random field elements as coefficients
of the basis vectors. For the subspace $V(l)$, we could put zeros into
the entries occurring in $l$ and make random selections of elements from
$\F$ for each entry not in $l$.
For this reason we denote by $\xi_{r}$ an upper bound 
for the cost of {\sc RandomVector} or {\sc RandomVector*} applied to an 
$r$-dimensional space for one of these cases.
If $r<s$ then $\xi_r\leq\xi_s$ and $\xi_{r_1}+\xi_{r_2} \leq \xi_{r_1+r_2}$, and we would 
expect $\xi_r$ to vary linearly with $r$. In 
practical implementations the cost is much less than the cost of the field operations involved 
in the algorithm below.


\subsection{Characteristic polynomial algorithm}


The characteristic polynomial algorithm below would terminate successfully without 
making random selections of vectors. However, the use of 
randomisation is key to our application of this algorithm for finding minimal polynomials. 
As in previous sections, let $M$ be a matrix in $F^{n \times n}$ 
acting naturally on $V := \F^n$. 




\begin{algorithm}
\caption{$\quad$ \sc CharPoly}
\label{algcharpoly}
\begin{algorithmic}
\STATE \textbf{Input:} $M \in \F^{n\times n}$
\STATE \textbf{Output:} a tuple $(k, (p^{(j)})_{1\leq j\leq k}, \calY, (b^{(j)})_{1\leq j\leq k})$ 
where each $p^{(j)}\in\F[x]$, $b^{(j)}\in\F^n$, and $\calY$ is a semi 
\STATE \hspace*{0mm} \phantom{\textbf{Output:}}% 
echelon data sequence of length $n$.
\vspace*{2mm}
\STATE $i := 0$
\STATE $\calY^{(0)}:=$ a trivial semi echelon data sequence
\WHILE {$\len(\calY^{(i)}) < n$}
    \STATE $i := i + 1$
    \STATE $a := \mbox{\sc RandomVector}(\F^{\len(\calY^{(i-1)})})$
    \STATE $c := \mbox{\sc RandomVector*}(V(l^{(i-1)}))$
    \STATE $v^{(i)} := aS^{(i-1)} + c$
	\STATE 
       \hspace*{10mm} \COMMENT{ $v^{(i)}\not\in\rsp(S^{(i-1)})$ where $\calY^{(i-1)}=(Y^{(i-1)},S^{(i-1)},
                              T^{(i-1)},l^{(i-1)})$}
    \STATE $(p^{(i)},\calY^{(i)},b^{(i)}) :=$
   % \STATE \hspace*{3cm}
             $\mbox{\sc RelativeOrdPoly}(\calY^{(i-1)},v^{(i)},M)$
\STATE  \hspace*{10mm} \COMMENT{$b^{(i)}\in\F^{\len(\calY^{(i)})}$; we add $n-\len(\calY^{(i)})$ zeros to make $b^{(i)}\in\F^{n}$}
\ENDWHILE
\STATE $k := i$
\STATE \textbf{return} $(k,(p^{(j)})_{1 \le j \le k}, 
                       \calY^{(k)}, 
(b^{(j)})_{1 \le j \le k})$
\end{algorithmic}
\end{algorithm}

\begin{Prop}[Correctness and complexity of Algorithm~\ref{algcharpoly}]
%: {\sc CharPoly}]
\label{propcharpoly}\mbox{}
Given a matrix $M \in \F^{n \times n}$, Algorithm~\ref{algcharpoly} returns
a tuple $(k,(p^{(i)})_{1 \le i \le k}, (Y,S,T,l),(b^{(i)})_{1 \le i \le k})$ 
with the following features: $\prod_{i=1}^k p^{(i)} = \chi_{M,V}$ is the 
characteristic polynomial
of $M$ in its action on $V$. The tuple $(Y,S,T,l)$ is a semi echelon data
sequence for the invertible matrix $Y \in \F^{n \times n}$ whose rows are
\[ v\!^{(1)}, v\!^{(1)}M, \ldots, v\!^{(1)}M^{d_1-1}, 
v\!^{(2)}, v\!^{(2)}M, \ldots, v\!^{(2)}M^{d_2-1},
   \ldots, v\!^{(k)}, v\!^{(k)}M, \ldots, v\!^{(k)} M^{d_k-1} 
\]
where $d_i := \deg(p^{(i)})$ for $1 \le i \le k$. Further, for $1 \le i \le k$,
$v^{(i)}$ is a uniformly distributed random element of\/ $V\setminus W_{i-1}$, 
$v^{(i)} M^{d_i} = b^{(i)} Y$ and $p^{(i)} = \ord_M(v^{(i)} + W_{i-1})$,
where $W_{i-1} := \left< v^{(1)}, \ldots, v^{(i-1)}\right>_M$ (for $i>1$), 
an $M$-invariant subspace of $V$ of dimension $s_{i-1}:=\sum_{j=1}^{i-1} d_j$, and 
$W_0=0$ of dimension $s_0=0$.
Moreover, Algorithm~\ref{algcharpoly} requires at most
%\[ \frac{10}{3}n^3
%   +\frac{3}{2}n^2
%   +\frac{7}{6}n
%   +2n\sum_{i=1}^k s_{i-1}
%   +2n\sum_{i=1}^k s_i
%   +\sum_{i=1}^k s_i(s_i+1) \]
%elementary field operations, where  $s_i := \sum_{j=1}^i d_j$ (so $s_0=0$), 
\[
\frac{33}{6}n^3+4n^2+\frac{3}{2}n 
\]
elementary field operations, plus
$k\xi_{n}$ for the $k$ calls to {\sc RandomVector*} and {\sc RandomVector}. Neglecting 
the latter cost this  is less than $6n^3$ elementary field operations, for sufficiently large $n$.
\end{Prop}

\begin{Rem}
We denote the semi echelon data sequences 
$\calY^{(i)}$ in the algorithm using indices to enable us to speak 
more easily about the intermediate results. However in practice
we have only one variable $\calY=(Y,S,T,l)$, the entries of which are 
growing during the execution of the algorithm.
\end{Rem}
%The vector $b^{(i)}$ comes into existence as a vector of length
%$s_i$, which is the number of rows of $Y^{(i)}$.
%The statement $v^{(i)}M^{d_i} = b^{(i)} Y$ in the proposition is to be 
%understood by padding $b^{(i)}$ with zeros up to length $n$.

%\emph{Remark 2:} Since $S^{(i-1)}$ is in row semi echelon form, choosing
%a random $v^{(i)}$ not contained in $W_{i-1}$ is easily achieved by
%putting zeros into the positions occurring in $l^{(i-1)}$ --- which are
%the leading columns of the rows of $S^{(i-1)}$ --- and randomising the
%other entries, such that the resulting vector is non-zero.

\begin{Rem}
Note that we do not multiply together the factors of 
$\chi_{M,V}$ because in our application of Algorithm~\ref{algcharpoly} 
we do not need the product itself.
\end{Rem}

\proofof{Proposition~\ref{propcharpoly}}
Most statements in the proposition follow immediately from 
Proposition~\ref{proprelorderpol}, since in the $i^{th}$ run of the 
`while' loop, Algorithm~\ref{algcharpoly}
chooses a  vector $v^{(i)}$ that is a uniformly distributed random element 
of $V\setminus \rsp(S^{(i-1)})$
% = V \setminus \rsp(Y^{(i-1)})$, 
and applies Algorithm~\ref{algordpoly}.
In particular, this immediately establishes all statements about
$(Y,S,T,l)$ including the one about the invertibility and the
rows of $Y$. Also it is clear that $p^{(i)} = \ord_M(v^{(i)} + W_{i-1})$.

Next we show that $\prod_{i=1}^k p^{(i)} = \chi_{M,V}$. 
This follows by considering
the matrix $YMY^{-1}$, which has the same characteristic polynomial
as $M$. Considering the action of $M$ with respect to the 
ordered basis $\F^n$ given by the rows of $Y$, it follows from
the construction that $YMY^{-1}$ 
(written with respect to the standard basis) is equal to
\[ \left[\begin{array}{cccc}
 C_1       &   0  & \cdots & 0 \\
 B^{(2)}_1 &  C_2 & \ddots & \vdots \\
 \vdots    &\ddots& \ddots & 0 \\
 B^{(k)}_1 &\cdots& B^{(k)}_{k-1} & C_k
\end{array} \right] \]
where the matrix $C_i$ is the companion matrix of the polynomial $p^{(i)}$,
that is, for $p^{(i)} = x^{d_i} - \sum_{j=0}^{d_i-1} a_j x^j$,
\[ C_i = \left[ \begin{array}{ccccc}
  0      & 1      & 0      & \cdots    & 0 \\
  0      & 0      & 1      & \ddots    & \vdots \\
  \vdots & \vdots & \ddots & \ddots    & 0 \\
  0      & 0      & \cdots & 0         & 1 \\
  a_0    & a_1    & \cdots & a_{d_i-2} & a_{d_i-1}
    \end{array} \right] \in \F^{d_i \times d_i}. \]
The $B^{(i)}_j$, for $2 \le i \le k$ and $1 \le j \le i-1$, are matrices
in $\F^{d_i \times d_j}$
with one non-zero row at the bottom and all other rows zero.
If $b^{(i)} = (b^{(i)}_1,\dots,b^{(i)}_n)$, then the bottom row of 
$B^{(i)}_j$ is $(b^{(i)}_{s_{j-1}+1},\dots,b^{(i)}_{s_j})$.

With this format at hand it is clear that the characteristic polynomial
of $YMY^{-1}$ is equal to the product $\prod_{i=1}^k p^{(i)}$ because
the $C_i$ are companion matrices.

Finally we derive the statement about the number of elementary field operations
needed by Algorithm~\ref{algcharpoly}.

In the $i^{th}$ run of the {\bf while} loop, the cost of constructing the
random vectors $a$ and $c$ is at most 
\[ \xi_{n-\len(\calY^{(i-1)})}+\xi_{\len(\calY^{(i-1)})}
\leq \xi_{n}, 
\]
(see Subsection~\ref{random}). The cost to compute $v^{(i)}$ is 
at most $2s_{i-1}n$ elementary field operations,
where $s_0=0$, and for $i\geq1$, $s_{i}=\sum_{j=1}^{i}d_j$ with 
$d_j=\deg p^{(j)}$. The cost 
of applying Algorithm {\sc RelativeOrdPoly}
is, by Proposition~\ref{proprelorderpol}, at most
\[ 
2d_in^2 + (n+2)d_i +2s_{i}n + 2(n+1)s^{(1)}(s_i-1,s_{i-1}-1) 
+ s^{(2)}(s_i-1,s_{i-1}-1) + 2s^{(1)}(s_i,0) 
\]
elementary field operations, noting that the value 
of `$d$' is $d_i$, the value of `$m$' is $s_{i-1}$, $s_{i-1}+d_i=s_i$,
and $s^{(1)}$, $s^{(2)}$ are the functions defined in (\ref{si}).

We consider the different terms
one by one, summing each over $i$ from $1$ to $k$. 
The total cost of constructing the random vectors is at most $k\xi_{n}$.
Summing the terms $2 s_{i-1} n$ gives $2n\sum_{i=1}^{k} s_{i-1}$, and
summing the terms $2d_in^2$ gives $2n^3$ since $\sum_{i=1}^k d_i=n$. 
Similarly, summing the terms
$(n+2)d_i$ gives $(n+2)n$. From the terms $2s_in$ we
get a contribution of 
$2n \sum_{i=1}^{k} s_i$.
The next two expressions involving the functions $s^{(1)}$ and $s^{(2)}$
sum to $2(n+1)s^{(1)}(n-1,0) = n(n+1)(n-1)$ and $s^{(2)}(n-1,0) = 
\frac{(n-1)n(2n-1)}{6}$ respectively, using (\ref{formels1}) and
(\ref{formels2}) and the properties noted
above it. 
Finally, the terms $2s^{(1)}(s_i,0)$ sum to 
$2\sum_{i=1}^k s^{(1)}(s_i,0) 
= \sum_{i=1}^k s_i(s_i+1)$.
Thus in total we obtain $k\xi_n$ plus
\[ 2n^3
   +n(n+1)(n-1)
   +\frac{(n-1)n(2n-1)}{6}
   +n(n+2)
   +2n\sum_{i=1}^k s_{i-1}
   +2n\sum_{i=1}^k s_i
   +\sum_{i=1}^k s_i(s_i+1) 
\]
elementary field operations. The first four of these terms sum to $\frac{10}{3}n^3
   +\frac{3}{2}n^2
   +\frac{7}{6}n$. 
Using Lemma~\ref{estimates},
\[ 
   2n\sum_{i=1}^k s_{i-1}
   +2n\sum_{i=1}^k s_i
   +\sum_{i=1}^k s_i(s_i+1)\leq 2n^2(n+1)+\frac{n(n+1)(n+2)}{6} \]
so the total cost is at most 
\[
\frac{33}{6}n^3+4n^2+\frac{3}{2}n +k\xi_n.
\]
For sufficiently large $n$ this is less than $6n^3+k\xi_n$.
\proofend


\section{Probability estimates using the structure theory for modules}
\label{probest}

The basic idea of our minimal polynomial Algorithm~\ref{algminpolymc}
is to compute the order polynomials of a few
random vectors under the action of a given matrix $M$ and to prove that, 
with high probability, their least common multiple is
equal to the minimal polynomial of $M$.
The purpose of this section is to use the structure theory 
of $V = \F^n$ as an $\F[x]$-module to derive probability
estimates to be used in that proof.

First suppose that the characteristic polynomial of $M$ is written as a product
$\chi_{M,V} = \prod_{i=1}^t q_i^{e_i}$ with pairwise distinct
irreducible polynomials $q_i \in \F[x]$ and positive integer
multiplicities $e_i$.

Using \cite[Theorem 3.12]{Jacob1} we can then write the $\F[x]$-module $V$
as a direct sum of primary cyclic modules
\begin{equation} \label{primary}
V \cong \bigoplus_{i=1}^t \bigoplus_{j=1}^{m_i} w_{i,j} \F[x] 
\end{equation}
such that $\ord_M(w_{i,j}) = q_i^{f_{i,j}}$ with
$e_i \ge f_{i,1} \ge f_{i,2} \ge \cdots \ge f_{i,m_i} \ge 1$
and $\sum_{j=1}^{m_i} f_{i,j}=e_i$ for $1 \le i \le t$.

The minimal polynomial $\mu_{M,V}$ is the least
common multiple of the order polynomials of the vectors 
$(w_{i,j})_{1 \le i \le t, 1 \le j \le m_i}$, and hence is 
$\mu_{M,V} = \prod_{i=1}^t (q_i)^{f_{i,1}}$.

We use this structural description to derive the
first probability bound for the case where $\F = \F_q$ is a finite field
with $q$ elements.

\begin{Prop}[Probability that a %given 
$q_i$ has equal multiplicity in $\mu_{M,V}$
and $\ord_M(v)$]
\label{ProbOneMult}
Let $\F = \F_q$ be a finite field with $q$ elements, let $V=\F^n$, let
$U$ be a (possibly zero) $M$-invariant subspace such that the multiplicity of
$q_i$ in $\mu_{M,U}$ is strictly smaller than in $\mu_{M,V}$, and let 
$v$ be a uniformly distributed random element of $V\setminus U$. Then 
the multiplicity of $q_i$ is the same in $\ord_M(v)$ and $\mu_{M,V}$ 
with probability greater than
$1-q^{-\deg q_i}$.
\end{Prop}
\proofbeg
By assumption the multiplicity of $q_i$ in $\mu_{M,U}$ 
is less that its multiplicity $f:=f_{i,1}$ in $\mu_{M,V}$. 
Let $w:=w_{i,1}$, with $w_{i,1}$ as in (\ref{primary}), so that $V=X\oplus Y$ with
$X, Y$ invariant under $M$ and $X=\la w\ra_M$. Then $\mu_{M,X}=
q_i^{f}$. We may identify the primary cyclic $\F[x]$-module 
$X$ with  $w\F[x]$, which is
isomorphic to the module $\F[x]/(q_i^{f_{i,1}}\F[x])$,
and in turn this is uniserial with composition series
\[ 
0 <   \frac{q_i^{f-1}\F[x]}{q_i^{f}\F[x]}
     <   \frac{q_i^{f-2}\F[x]}{q_i^{f}\F[x]} <
\cdots <  \frac{q_i\F[x]}{q_i^{f}\F[x]} 
       < \frac{\F[x]}{q_i^{f}\F[x]}. 
\]
Thus, $X$ has a unique maximal $\F[x]$-submodule, namely  $X':=\la w q_i(M)\ra_M$,
and $X'$ has codimension $r:=\deg(q_i)$ in $X$. 

As discussed above, each vector $v\in V$ has a 
unique expression as $v=x+y$ with $x\in X, y\in Y$. Moreover
$\ord_M(v)$ is the least common multiple of $\ord_M(x)$ and $\ord_M(y)$.
In particular, if $x\not\in X'$, then $\ord_M(x)=q_i^{f}$ and hence
the multiplicity of $q_i$ in $\ord_M(v)$ and $\mu_{M,V}$ is the same.
The number of vectors $v=x+y$ with $x\not\in X'$ is 
\[
|X\setminus X'|\cdot |Y|=(1-\frac{1}{q^r})|X|\cdot|Y|=(1-\frac{1}{q^r})q^n.
\]
Each of these vectors $v$ lies in $V\setminus U$ since the multiplicity of
$q_i$ in $\mu_{M,U}$ is less than $f$. Thus the probability, for
a uniformly distributed random $v\in V\setminus U$, that the multiplicity 
of $q_i$ in $\ord_M(v)$ and $\mu_{M,V}$ is the same is at least
\[
(1-\frac{1}{q^r})\frac{q^n}{|V\setminus U|} > 1-\frac{1}{q^r}.
\]
\proofend

\begin{Rem}
If for some irreducible factor $q_i$ we have $m_i > 1$
and $f_{i,1} = f_{i,2}$, then the above probability is even higher,
because we can apply the above argument independently to two or more summands
$w_{i,1}\F[x]$ and $w_{i,2}\F[x]$.
\end{Rem}


\smallskip
We now give a second probability bound which will be crucial in our
Monte Carlo algorithm to compute the minimal polynomial. In 
that algorithm we choose  a sequence of vectors 
$v^{(1)}, \dots v^{(u)}$ such that $v^{(1)}$ is a uniformly distributed
random element of $V\setminus\{0\}$, and for $i\geq2$ we choose
$v^{(i)}$ as a uniformly distributed random element of 
$V \setminus U$, where $U=\left< v^{(1)}, \ldots, v^{(i-1)} \right>_M$.
We hope to find $\mu_{M,V}$ as the least common multiple of the
orders of these vectors.


\begin{Prop}[Probability that a least common multiple of order polynomials is
equal to $\mu_{M,V}$]
\label{ProbAllMult}
Let\/ $\F = \F_q$ be a finite field with $q$ elements.
Suppose a sequence of vectors $v^{(1)}, \ldots, v^{(u)} \in V$ is chosen
as follows: $v^{(1)}$ is a uniformly
distributed random element of $V\setminus\{0\}$, and for $i>1$,
$v^{(i)}$ is a  uniformly distributed random element of  
$V \setminus \left< v^{(1)}, \ldots, v^{(i-1)} \right>_M$. 
Let
\[ 
f := \lcm( \ord_M(v^{(1)}), \ord_M(v^{(2)}), \ldots, 
\ord_M(v^{(u)}) ). 
\]
Then the probability that  $f = \mu_{M,V}$  is greater than
\[ 1-\sum_{i=1}^t q^{-u\deg q_i}. \]
\end{Prop}
\proofbeg
Consider the random experiment described in the statement. We first
examine one irreducible factor $q_i$. Let $E_i$ denote the event
that the multiplicity of $q_i$ in $f$ is strictly smaller than 
the multiplicity $f_{i,1}$ of $q_i$ in $\mu_{M,V}$. Furthermore, for
$1 \le j \le u$, let $F_j$  be the event that the multiplicity of $q_i$ in
$\ord_M(v^{(j)})$ is strictly smaller than $f_{i,1}$. 

Note that
the $F_j$ are not stochastically independent since we choose
$v^{(j)}$ outside of $\left< v^{(1)}, \ldots, v^{(j-1)}\right>_M$.
However, $E_i = F_1 \cap F_2 \cap \cdots \cap F_u$ because $f$ is the
least common multiple of the order polynomials of the $v^{(j)}$.
By Proposition~\ref{ProbOneMult} applied with $U=\{0\}$, 
the probability $\Prob(F_1)$
is less than $q^{-\deg q_i}$. Moreover, in the situation that 
$F_1 \cap \cdots \cap F_j$ holds and $j<u$,
we apply Proposition~\ref{ProbOneMult} with the
subspace $U := \left< v^{(1)}, \ldots, v^{(j)} \right>_M$ to conclude
that the conditional probability $\Prob(F_{j+1} | F_1 \cap \cdots \cap F_j)$
is less than $q^{-\deg q_i}$.
Thus we have
\begin{eqnarray*}
\Prob(E_i) &=& \Prob(F_1)\cdot \Prob(F_2 | F_1) \cdot \Prob(F_3 | F_1 \cap F_2)
   \cdot \cdots \cdot \Prob(F_u | F_1 \cap \cdots \cap F_{u-1}) \\
   &<& q^{-u\deg q_i}.
\end{eqnarray*}
Finally we consider all the different irreducible factors $q_i$. 

Even though
the events $E_1,\ldots,E_t$ may not be stochastically independent, we have
\[ \Prob( E_1 \cup \cdots \cup E_t ) \le \sum_{i=1}^t \Prob(E_i)
   < \sum_{i=1}^t q^{-u\deg q_i} \]
as claimed.
\proofend

\section{Computing minimal polynomials}
\label{minpoly}

Our minimal polynomial algorithm runs Algorithm~\ref{algcharpoly} as its
first step. So assume, from now on,
that we have already run Algorithm~\ref{algcharpoly}
and obtained all the output it produces, in particular
the basis given by the rows of the matrix $Y$ (as in Proposition~\ref{propcharpoly}),
\[ 
(v^{(1)}, v^{(1)}M, \ldots, v^{(1)} M^{d_1-1}, \ldots, v^{(k)}, 
v^{(k)} M, \ldots, v^{(k)} M^{d_k-1}) 
\]
the relative order polynomials $p^{(i)} = \ord_M(v^{(i)}
+ W_{i-1})$, and the vectors $b^{(i)}$ for $1 \le i \le k$.
Also assume that we have factorised all the polynomials $p^{(i)}$
as products $p^{(i)} = \prod_{j=1}^t q_j^{e_{i,j}}$ of irreducible 
polynomials $(q_j)_{1 \le j \le t}$.

The matrices $M$ and $YMY^{-1}$ have the same characteristic
and minimal polynomials. Also the order polynomials $\ord_M(v)$ and
$\ord_{YMY^{-1}}(vY^{-1})$ are equal for every $v \in V$ and thus also
the order polynomials $\ord_M(vY)$ and $\ord_{YMY^{-1}}(v)$ are equal for every
$v \in V$.

For the convenience of the reader we display the matrix
$YMY^{-1}$ in Figure~\ref{bigmat}. Note in particular that the matrix is
sparse, provided that the degrees $d_i$ are not too small.
Due to the special form of $YMY^{-1}$ it is much more efficient to compute
the images of vectors under $YMY^{-1}$, than under $M$. 
This is crucial in the analysis of our algorithms. 
Therefore we will from now on do all computations of order polynomials
with respect to $YMY^{-1}$.

Set $M' := YMY^{-1}$ and $W'_i := W_i Y^{-1}$ for $1 \le i \le k$.
Note that we have $v^{(i)} = e^{(s_{i-1}+1)} Y$ for $1 \le i \le k$
where $e^{(1)}, \ldots, e^{(n)}$ is the standard basis of $\F^n$. (That is,  
 $e^{(i)}$ contains exactly one $1$ in position $i$ and otherwise zeros. Recall
that $s_i = \sum_{j=1}^i d_j$ with $s_0 = 0$.) Furthermore, for $1 \le i \le k$,
the space $W_i = \left< v^{(1)}, \ldots, v^{(i)} \right>_M$ is equal to the space
$\{ vY \mid v \in \F^n \mbox{ with } v_j = 0 \mbox{ for } j > s_i \}$. 
Thus, the space $W'_i$ is the $\F$-linear
span $\left< e^{(1)}, e^{(2)}, \ldots, e^{(s_i)}\right>_\F$ and we have a
filtration
\[ 0 = W'_0 < W'_1 < W'_2 < \cdots < W'_k = V \]
such that each quotient $W'_i/W'_{i-1}$ is an $M'$-cyclic space generated 
by the coset represented by the standard basis vector $e^{(s_{i-1}+1)}$.

We begin by presenting Algorithm~\ref{algordpolabs} which computes the
absolute order polynomial of a vector with respect to the matrix $YMY^{-1}$,
using all the data acquired during Algorithm~\ref{algcharpoly}. 
We will apply this later
in the minimal polynomial algorithm to the first few of the vectors 
$e^{(s_{i-1}+1)}$ 
produced during a run of Algorithm~\ref{algcharpoly}. Note that for 
the analysis it is crucial that a number $i$ such that
the vector $v$ lies in $W'_i$ is given as input to the algorithm.
We denote the $j^{th}$ entry of $v$ by $v_j$.

\begin{figure}
\caption{Overview over the matrix $YMY^{-1}$}
\label{bigmat}
\[ \left[ \begin{array}{cccccc|cccccc|c|cccccc}
\cline{1-6}
  0 & 1 &      &      &   &   \\
    & 0 & 1    &      & 0 &   \\
    &   &\ddots&\ddots&   &   \\
    & 0 &      &   0  & 1 &   \\
    &   &      &      & 0 & 1 \\
  {}* & * &   *  &   *  & * & * \\
\cline{1-12}
  &&&&&& 0 & 1 &      &      &   &   \\
  &&&&&&  & 0 & 1    &      & 0 &   \\
  &&&&&&  &   &\ddots&\ddots&   &   \\
  &&&&&&  & 0 &      &   0  & 1 &   \\
  &&&&&&  &   &      &      & 0 & 1 \\
  {}*&*&*&*&*&*&* & * &   *  &   *  & * & * \\
\cline{1-13}
  &&&\vdots&&& &&&\vdots&&& \ddots \\
\cline{1-19}
  &&&&&& &&&&&& & 0 & 1 &      &      &   &   \\
  &&&&&& &&&&&& & & 0 & 1    &      & 0 &   \\
  &&&&&& &&&&&& & &   &\ddots&\ddots&   &   \\
  &&&&&& &&&&&& & & 0 &      &   0  & 1 &   \\
  &&&&&& &&&&&& & &   &      &      & 0 & 1 \\
  {}*&*&*&*&*&*& *&*&*&*&*&*&* &*& * &   *  &   *  & * & * \\
\hline
\end{array} \right] \]
\end{figure}

\begin{algorithm}
\caption{$\quad$ \sc OrdPoly}
\label{algordpolabs}
\begin{algorithmic}
\STATE \textbf{Input:} $M$, $k$, $(Y,S,T,l)$, $(p^{(j)})_{1 \le j \le k}$,
$(b^{(j)})_{1 \le j \le k}$ as returned by {\sc CharPoly},
\STATE \hspace*{0mm} \phantom{\textbf{Input:}}%
$i$ with $1 \le i \le k$ and 
$v \in W'_i$, and the factorisation  
$p^{(j)} = \prod_{r=1}^t q_r^{e_{j,r}}$ for all $j\leq k$
\vspace*{2mm}
\STATE $f := 1 \in \F[x]$
\REPEAT
    \STATE $h := \sum_{j=1}^{d_i} v_{s_{i-1}+j} x^{j-1}$
    \IF {$h \neq 0$}
        \STATE $g := p^{(i)}/\gcd(h,p^{(i)})$
        \STATE $f := f \cdot g$
        \IF {$i > 1$}
            \STATE $v := v \cdot g(YMY^{-1})$ \hspace*{1cm} 
            \COMMENT{see Proposition~\ref{propordpol} for this action}
        \ENDIF
    \ENDIF
    \STATE $i := i - 1$
\UNTIL {$i=0$}
\STATE \textbf{return} $f$
\end{algorithmic}
\end{algorithm}

\begin{Prop}[Correctness and complexity of Algorithm~\ref{algordpolabs}:
{\sc OrdPoly}]
\label{propordpol}
Let $\F = \F_q$ be a field with $q$ elements.
Given the output of Algorithm~\ref{algcharpoly}, 
the factorisation of all $p^{(j)} = \prod_{r=1}^t q_r^{e_{j,r}}$
for $1 \le j \le k$, 
a natural number $i$ 
with $1 \le i \le k$, and a vector $v \in W'_i$, 
Algorithm~\ref{algordpolabs} computes the order
polynomial $\ord_{YMY^{-1}}(v)$. The number of elementary field operations
required is at most
\[ 
\sum_{j=1}^i \left( 7 d_j^2 + s_j + d_j s_j + 2d_j \sum_{r=1}^j s_r \right)
\ \leq\ (i+10)s_i^2 + is_i 
\]
where $d_j=\deg p^{(j)}$, $s_i=\sum_{j=1}^i d_i$ for $i\geq1$ and $s_0=0$;
and this is less than $n^3$ for $n$ sufficiently large.
\end{Prop}
\proofbeg
Since we are computing an order polynomial with respect to the matrix
$M' = YMY^{-1}$ we can always use the form of this matrix as displayed
in Figure~\ref{bigmat}.

The basic idea of Algorithm~\ref{algordpolabs} is to use
Lemmas~\ref{ordpolcyclic} and \ref{absordpoly} applied to the filtration
\[ 0 = W'_0 < W'_1 < W'_2 < \cdots < W'_k = V. \]

Starting in the space $W'_i/W'_{i-1}$, Algorithm~\ref{algordpolabs} inductively
computes the relative order polynomial $g := \ord_{M'}(v+W'_{i-1})$. This assertion follows from
Lemma~\ref{ordpolcyclic} noting that,
by our discussion above, $p^{(i)} = \ord_M(v^{(i)}+W_i)= \ord_{M'}(e^{(s_{i-1}+1)}+W'_{i-1})$.
Next $v g(M')$ is evaluated, which lies in $W'_{i-1}$ by
Lemma~\ref{relorderpol}, and the induction can go on with $i$
replaced by $i-1$. The polynomial $f$ returned is the product of all 
the relative order polynomials computed in the repeat loop, and this is equal to
$\ord_M(v)$, by Lemma~\ref{absordpoly}.

To count the number of elementary field operations is a bit complicated
here. Note first that by assumption we already know a factorisation of all the
$p^{(i)} = \prod_{j=1}^t q_j^{e_{i,j}}$ into
irreducible factors. Now  $\gcd(h,p^{(i)})$ is equal to the product of
the greatest common divisors $\gcd(h,q_j^{e_{i,j}})$, for $j\leq t$. Since the degrees of the
polynomials  $q_j^{e_{i,j}}$ sum up to the degree $d_i$ of $p^{(i)}$, finding these gcd's, by 
Proposition~\ref{standardgcd}, requires at most
\[ 
2(\deg(h)+1) \cdot \sum_{j=1}^t \left(\deg(q_j) e_{i,j} + 1\right) 
\] 
field operations, which is at most $4d_i^2$ since
$\deg(q_j) e_{i,j} + 1 \le 2 \deg(q_j) e_{i,j}$. Note that this is a rather
crude estimate. At this stage we know all multiplicities of the
$q_j$ in $\gcd(h,p^{(i)})$ and thus in $g := p^{(i)}/\gcd(h,p^{(i)})$.
With at most
another $3d_i^2$ elementary field operations we can compute the product
$g$ since multiplying two polynomials of degrees $a$ and $b$ respectively
needs at most $2(a+1)(b+1)$ elementary field operations, and the degrees 
of all intermediate results and all factors are smaller than $d_i$. Thus all the
$\gcd$ calculations and together need at most $7\sum_{j=1}^i d_j^2$
elementary field operations.

Note that we do not count the multiplications of the polynomials $f$ and $g$ in 
the loop since we are content with returning the result in factorised
form. However Algorithm~\ref{algordpolabs} is formulated with these
multiplications to simplify the presentation.


Now we have to
discuss the number of operations needed to evaluate $v g(M')$.
Due to the sparseness of $M'$, an application of $M'$ needs only a shift
(which we neglect here) and an addition of a multiple of the non-zero
part of $b^{(j)}$ for $1 \le j \le i$ needing $2s_j$ operations each.
Thus, the evaluation of $v g(M')$ needs at most
\[ 
s_i + \deg(g) \cdot \left( s_i + \sum_{r=1}^i 2s_r \right) 
\]
elementary field operations. Since the degree of $g$ is at most
$d_i$, this number is bounded above by
\[ 
s_i + d_i \cdot s_i + 2d_i \cdot \sum_{r=1}^i s_r. 
\]
Thus, Algorithm~\ref{algordpolabs} needs at most
\[ 
\sum_{j=1}^i \left( 7d_j^2 + s_j + d_j s_j + 2d_j \sum_{r=1}^j s_r \right) 
\]
elementary field operations, as claimed in the proposition.

To find a simpler upper bound we look at the terms one by one. The
last and most important term can be bounded by
\begin{eqnarray*}
 2\sum_{j=1}^i \left( d_j \sum_{r=1}^j s_r \right)
   &=& 2 \sum_{r=1}^i s_r \sum_{j=r}^i d_j
   = 2 \sum_{r=1}^i s_r (s_i - s_{r-1}) \\
   &=& 2 \sum_{r=1}^i s_r (s_i - s_r) + 2 \sum_{r=1}^i s_r d_r
   \le (i+2) \cdot s_i^2
\end{eqnarray*}
since for all $r$ either $s_r$ or $s_i-s_r$ is less than or equal to
$s_i/2$.

The third term $\sum_{j=1}^i d_j s_j$ is bounded by $s_i^2$,
the second term $\sum_{j=1}^i s_j$ is bounded by $is_i$, and the term
$7 \sum_{j=1}^i d_j^2$ by $7s_i\sum_{j=1}^i d_j = 7s_i^2$.

Altogether this amounts to a bound of $(i+10)s_i^2 + is_i$ as claimed.
Asymptotically, this is bounded above by $n^3$ in the worst case as
$n \to \infty$.

\proofend

\smallskip
Now we present our main Algorithm~\ref{algminpolymc}. 

\begin{algorithm}
\caption{$\quad$ \sc MinPolyMC}
\label{algminpolymc}
\begin{algorithmic}
\STATE \textbf{Input:} $M \in \F_q^{n \times n}$, $0 < \epsilon < 1/2$.
\STATE \hspace*{0mm} \phantom{\textbf{Input:}}%
       $((p^{(j)})_{1 \le j \le k},(Y,S,T,l),(b^{(j)})_{1 \le j \le k})
       := \mbox{\sc CharPoly}(M)$
\vspace*{2mm}
\STATE Factorise all $p^{(j)} = \prod_{r=1}^t q_r^{e_{j,r}}$ 
\STATE Determine the least $u \in \N$ such that
 $\sum_{r=1}^t q^{-u\deg q_r} \le \epsilon$
\STATE $u := \min\{ u,k \}$
\STATE $f := \lcm(p^{(1)}, \ldots, p^{(k)})$ 
\FOR {$i = 2$ to $u$}
    \STATE $f := \lcm(f,\mbox{\sc OrdPoly}
           (M,k,(Y,S,T,l),(p^{(j)})_{1 \le j \le k}, 
           (b^{(j)})_{1 \le j \le k},i,e^{(s_{i-1}+1)}))$
\ENDFOR
\IF {$u = k$ or $\deg f = n$}
    \STATE \textbf{return} $(\mbox{\sc True},f)$
\ELSE
    \STATE \textbf{return} $(\mbox{\sc Uncertain},f)$
\ENDIF
\end{algorithmic}
\end{algorithm}

\begin{Prop}[Correctness and complexity of Algorithm~\ref{algminpolymc}:
{\sc MinPolyMC}]\label{propminpoly}
Given a matrix $M \in \F_q^{n \times n}$ and a number
$\epsilon$ with $0 < \epsilon < 1/2$, Algorithm~\ref{algminpolymc}
returns a tuple $(b,f)$, where $b$ is either {\sc True} or {\sc Uncertain}
and $f \in \F_q[x]$ is a polynomial. With probability at least $1-\epsilon$
the polynomial $f=\mu_{M,\F_q^n}$, and if
\/ $b = \mbox{\sc True}$ then
$f = \mu_{M,\F_q^n}$ is guaranteed. Moreover, 
if\/ $f \ne \mu_{M,\F_q^n}$, then 
$f$ is a proper divisor of $\mu_{M,\F_q^n}$ and
every irreducible factor of $\mu_{M,\F_q^n}$ divides $f$.

The number of elementary field operations needed by 
Algorithm~\ref{algminpolymc} is bounded above by
\[ 
{\rm char}(n,q) + {\rm fact}(n,q) + \sum_{i=1}^u \left( (i+10)s_i^2 + is_i \right) 
\]
where ${\rm char}(n,q)$ is an upper bound for the number of elementary field operations needed to
compute the characteristic polynomial (see
Proposition~\ref{propcharpoly}), ${\rm fact}(n,q)$ is an upper bound for 
the number of elementary field
operations needed to factorise each of a set of polynomials over $\F_q$ whose degrees sum to $n$
(see Subsection~{\rm\ref{polyfactn}}). Moreover either $u=k$, or $u<k$ and
$\sum_{j=1}^t q^{-u\deg q_j} \le \epsilon$.

For $n$ sufficiently large and fixed $\epsilon$, this is less than
\[ 6n^3 + {\rm fact}(n,q) + 
  2 \lceil\frac{\log n-\log \epsilon}{\log q}\rceil^2 \cdot n^2 
\]
which is less than $7n^3 + {\rm fact}(n,q)$ (plus the cost of computing at most $n$ random 
vectors in Algorithm~\ref{algcharpoly}).
\end{Prop}


\begin{Rem}
Note that if we use a randomised polynomial factorisation algorithm
(necessary for large $q$), then the algorithm can be modified to allow
for a possible failure of factorisation of the `Factorise' step (line
2). Thus Theorem~\ref{main} follows from Proposition~\ref{algminpolymc}.
An upper bound for the term {\rm fact}$(n,q)$ in the complexity
bound is given in Remark~{\rm\ref{rem:polyfactn}}, and this yields an
upper bound in Proposition~{\rm\ref{propminpoly}} of $O(n^3\log^3 q)$
for $n$ sufficiently large and fixed $\epsilon$.
\end{Rem}

\proofof{Proposition~\ref{propminpoly}}
Algorithm~\ref{algminpolymc} first computes the characteristic polynomial
of $M$ in its action on $\F_q^n$ and its factorisation. This computation provides
firstly the irreducible factors $q_j$ of the minimal polynomial that allow us 
to determine $u$, and secondly the input needed for running Algorithm~\ref{algordpolabs}
to compute the order polynomials of $v^{(2)}, \ldots, v^{(u)}$. Note that
$p^{(1)} = \ord_M(v^{(1)})$. By Proposition~\ref{propcharpoly}, the vector $v^{(j)}$ 
is a uniformly distributed random element of $V\setminus\{0\}$ if $j=1$, or
$V\setminus\left< v^{(1)}, \ldots, v^{(j-1)}\right>_M$ if $j>1$.
Hence, by Propositions~\ref{ProbAllMult} and~\ref{propordpol}, the probability that
$f$ after termination of Algorithm~\ref{algminpolymc} is equal to
$\mu_{M,\F_q^n}$ is at least $1-\epsilon$. 

  From the discussion at the beginning of Section~\ref{probest},
$\mu_{M,\F_q^n}$ is the least common multiple of the $k$ polynomials
$\ord_M(v^{(1)}), \dots,\ord_M(v^{(k)})$, and hence if $u=k$ then
$f = \mu_{M,\F_q^n}$. This also implies, since the initial value
of $f$ is $\lcm(p^{(1)}, \ldots, p^{(k)})$, that the returned
polynomial $f$ divides $\mu_{M,\F_q^n}$ and every irreducible factor of
$\mu_{M,\F_q^n}$ divides $f$. In particular, if $\deg f =n$ then we must
have $f=\chi_{M,\F_q^n}=\mu_{M,\F_q^n}$. Thus if $(\mbox{\sc True},f)$
is returned then $f = \mu_{M,\F_q^n}$ is guaranteed.


The number of elementary field operations needed follows from 
Propositions~\ref{propcharpoly} and~\ref{propordpol} and summing. Note that,
after the factorisations computed in line 2 of the algorithm, we neglect
the forming of least common multiples and the products here, because
all results from Algorithm~\ref{algordpolabs} come already factorised
into irreducible factors. We can thus compute the least common multiples
by taking maximums of multiplicities. Hence the first displayed upper
bound is proved.

For the asymptotic complexity bound we have to consider the initial value of the number $u$,
namely the least integer $u$ such that $\sum_{j=1}^t q^{-u \deg q_j} \le \epsilon$.
The largest value of this sum occurs when all the $q_j$ have degree 1, and as there are 
then at most $n$ such polynomials,  $\sum_{j=1}^t q^{-u \deg q_j}\le nq^{-u}$. 
Thus $u$ is at most the least integer such that $nq^{-u}\le \epsilon$, namely 
\[
u_0:=\lceil \frac{\log n + \log (\epsilon^{-1})}{\log q}\rceil
\]
and the value of $u$ used in the algorithm is at most $\min(\{k,u_0\})\leq u_0$.
By Proposition~\ref{propcharpoly}, the asymptotic value of char$(n,q)$ is
less than $6n^3$ for $n$ sufficiently large, (plus the cost $k\xi_n$ 
of making $k$ random selections of vectors). 
By Proposition~\ref{propordpol},  the number of elementary
field operations used for the computation of the  $u\le u_0$ order polynomials
is at most
\[
\sum_{i=1}^u\left( (i+10)s_i^2+is_i\right) \leq u^2(s_u^2+s_u)+10us_u^2
\le u_0^2(n^2+n)+10u_0 n^2.
\]
which, for sufficiently large $n$ and fixed $\epsilon$, is less than $2u_0^2n^2<n^3$.
\proofend

\section{Deterministic verification}
\label{verify}

In this section we explain how the probabilistic result of our Monte Carlo
algorithm can be verified deterministically. We begin by discussing cases
that can be handled rather cheaply, before we present a more general
result, which unfortunately has a nasty worst-case scenario.

All notation from previous sections remains in force. The first result 
follows immediately from Proposition~\ref{propminpoly}.

\begin{Prop}[Cases, in which the result is already proven to be correct]
If the output polynomial of Algorithm~\ref{algminpolymc} is  $\chi_{M,\F_q^n}$,
then the output is $(${\sc True}, $\chi_{M,\F_q^n})$
and is correct.
%If $\mu_{M,\F_q^n} = \chi_{M,\F_q^n}$ then either the output of 
%Algorithm~\ref{algminpolymc} is incorrect, or the output is 
%$(${\sc True}, $\mu_{M,\F_q^n})$.
%That is, if the correct minimal polynomial is computed, this is
%automatically proved in the process.
\end{Prop}

For the next result observe that, if Algorithm~\ref{algminpolymc} 
is modified so that the {\bf for} loop is run $k$ times, then the resulting 
polynomial $f$ is guaranteed to be the minimal polynomial, giving a deterministic
algorithm with proven result. (Proof of correctness 
is given in the proof of Proposition~\ref{propminpoly}.)

\begin{Prop}[Case of few random vectors chosen during computation of
$\chi_{M,\F_q^n}$]
If $k \le \sqrt{n}$,  and the {\bf for} loop in 
Algorithm~\ref{algminpolymc} is run $k$ times, then the output polynomial
is $\mu_{M,\F_q^n}$. The 
%total cost for this loop is bounded by
%\[ n^3 + 10 n^{5/2} + n^2 \]
%elementary field operations such that the overall cost is less than
overall cost of this modification of Algorithm~\ref{algminpolymc} is at most
\[ {\rm char}(n,q) + {\rm fact}(n,q) + n^3 + 10 n^{5/2} + n^2 \]
elementary field operations.
\end{Prop}

\proofbeg The only change 
to the complexity estimate is for the number of elementary field operations 
in the second last line of the proof of Proposition~\ref{propminpoly}:
\[ \sum_{i=1}^k((i+10)s_i^2 + is_i)\le
k^2(s_k^2+s_k) +10 ks_k^2 
  = k^2 (n^2 + n) + 10 k n^2 \le n^3 + 10 n^{5/2} + n^2. \]
The rest follows from Proposition~\ref{propminpoly}.
\proofend

\medskip
For the following discussion we need a lemma:

\begin{Lemm}[Cost of evaluation of a polynomial at a matrix]
\label{costpolyeval}
Let $M \in \F^{n \times n}$ be a matrix and $f \in \F[x]$ a polynomial 
with degree $d < n$. Then the evaluation $f(M)$ can be computed using
at most $2dn^3$ elementary field operations.
\end{Lemm}

\proofbeg We take $2n^3$ elementary field operations as an upper bound for a matrix
multiplication. The computation of the powers $M^2, M^3, \ldots, M^d$
needs at most $2(d-1)n^3$ elementary field operations. The
multiplication, for each $i=1,\dots,d$, of $M^i$ by a coefficient of $f$
and addition of the result to the already computed matrix (the sum of previous terms) 
needs another $2dn^2$ elementary field operations. Finally, the
constant term of $f$ has to be added along the diagonal, which is yet
another $n$ elementary field operations. Since $d+1\le n \le 2 n^2$, 
this is altogether at most $2dn^3$
as claimed.
\proofend

\smallskip
Of course, this immediately implies:

\begin{Cor}[Small degree minimal polynomial]
If $\deg \mu_{M,\F_q^n} < n$, then the output of Algorithm~\ref{algminpolymc} 
can verified by evaluation using at most 
\[ 2\cdot n^3 \cdot \deg \mu_{M,\F_q^n} \] 
elementary field operations.
\end{Cor}

\begin{Rem}
Note that using \cite[Theorem 2]{AC97} we could lower the complexity
in Lemma~\ref{costpolyeval} to $O(\sqrt d n^3)$ provided we 
stored $O(\sqrt d)$ matrices in memory at the same time. However, since
storing a matrix in $\F^{n \times n}$ needs $O(n^2)$ of memory, this
approach would often become impractical before  a
concrete problem would become intractable because of time constraints. 
We use our estimates in Lemma~\ref{costpolyeval} because of
these practical considerations.
However, in some practical situations, an improved polynomial evaluation
algorithm using more memory may be suitable.
\end{Rem}

We now present Algorithm~\ref{algminpolyverify} that can be run after
Algorithm~\ref{algminpolymc} to verify the correctness of the resulting polynomial
deterministically.

\begin{algorithm}
\caption{$\quad$ \sc MinPoly verification}
\label{algminpolyverify}
\begin{algorithmic}
\STATE \textbf{Input:} $M \in \F^{n \times n}$, $\chi_{M,V} = \prod_{i=1}^t q_i^{e_i}$ (factorised), 
and a candidate $\prod_{i=1}^t q_i^{f_i}$ for $\mu_{M,\F_q^n}$ (factorised),
\STATE \hspace*{0mm} \phantom{\textbf{Input:}}%
all data from Algorithm~\ref{algminpolymc}
\FOR {$i=1$ to $t$}
    \IF {$f_i < e_i$}
        \STATE $M' := q_i(YMY^{-1})^{f_i}$
        \STATE $d := \dim_\F( \ker(M') )$
        \IF {$d < \deg(q_i) \cdot e_i$} 
            \STATE \textbf{return} $i$
        \ENDIF
    \ENDIF
\ENDFOR
\STATE \textbf{return} {\sc True}
\end{algorithmic}
\end{algorithm}

\begin{Prop}[Deterministic minimal polynomial verification]
\label{propverify}\mbox{}

If Algorithm~\ref{algminpolyverify} is called with candidate minimal polynomial
$\prod_{i=1}^t q_i^{f_i}$ from Algorithm~\ref{algminpolymc}, 
then it either returns {\sc True} or a 
positive integer~$j$.
In the former case, $\mu_{M,\F_q^n}=\prod_{i=1}^t q_i^{f_i}$,
while in the latter case the
multiplicity of $q_j$ in  $\mu_{M,\F_q^n}$ is greater than $f_j$.
The number of elementary field operations required by 
Algorithm~\ref{algminpolyverify} is at most
\[ 
%\sum_{i=1}^t \left( 2r_in^3 + 2\log(f_i) \, n^3 + n^3 \right) =
 n^3 \cdot \sum_{i=1}^t \left( 2r_i + 2\lceil \log f_i \rceil+1 \right).
\] 
\end{Prop}

\proofbeg Let $r_i := \deg q_i$ for $i = 1, \ldots, t$.
We again view $\F^n$ as $\F[x]$-module as in Section~\ref{probest} by
letting $x$ act as right multiplication by $M$. By 
\cite[Theorem~3.12]{Jacob1}, it is isomorphic to a direct sum of 
primary cyclic $\F[x]$-modules
\[ 
\F^n \cong \bigoplus_{i=1}^t \bigoplus_{j=1}^{m_i} w_{i,j} \F[x], 
\]
such that $\ord_M(w_{i,j}) = q_i^{f_i,j}$ with 
$e_i \ge f_{i,1} \ge f_{i,2} \ge \cdots \ge f_{i,m_i} \ge 1$ and
$\sum_{j=1}^{m_i} f_{i,j} = e_i$. Thus, for each $i$,
$q_i$ occurs in  $\mu_{M,\F_q^n}$ with multiplicity $f_{i,1}$,
and so in particular $f_i\le f_{i,1}$.
The element $q_i^{f_i}$ acts
invertibly on all direct summands $w_{i',j} \F[x]$ with $i' \neq i$
since $q_i$ is irreducible and every order polynomial of a non-zero vector
in such a direct summand is a power of $q_{i'}$, by 
Lemma~\ref{ordpolcyclic}. For $i' = i$ however, the dimension of the kernel 
of the action of $q_i^{f_i}$ on $w_{i,j} \F[x]$ is 
$r_i \cdot \min\{f_i,f_{i,j}\}$.
Thus the dimension of the kernel of the action of $q_i^{f_i}$ on the
whole of $\F^n$ is equal to 
\[
r_i\sum_{j=1}^{m_i}\min\{f_i,f_{i,j}\}
\le r_i\sum_{j=1}^{m_i}f_{i,j}=r_ie_i
\]
with equality if and only if $f_i\ge f_{i,1}$. 
Since $f_i\le f_{i,1}$, equality holds above if and only if $f_i$
is equal to the multiplicity $f_{i,1}$ of $q_i$ in  $\mu_{M,\F_q^n}$.
Therefore, Algorithm~\ref{algminpolyverify} always returns the
result as stated in the Proposition.

As to the cost, Algorithm~\ref{algminpolyverify} evaluates $q_i$ at 
$YMY^{-1}$ which needs at most $2r_i n^3$ elementary field operations
by Lemma~\ref{costpolyeval}. It then takes the result to the
$f_i^{\mathrm{th}}$ power, which can be done by repeated squaring
with at most $2n^3\lceil \log f_i \rceil$ elementary field operations, and 
finally computes the dimension of a null space, which can be done with at most
$n^3$ elementary field operations (compute a semi echelon basis
of the row space of the matrix). Note that we are not using the 
sparseness of $YMY^{-1}$ here.
\proofend

\begin{Rem}
The cost in Proposition~\ref{propverify} is much smaller than $n^4$ in
many cases. One of the worst cases is that $\chi_{M,\F^n}$ contains lots of
different factors of degree $1$ each occurring with multiplicity $3$, and that
all $f_i$ are equal to $2$. Then Algorithm~\ref{algminpolyverify} has to
square about $n/3$ matrices and compute the null spaces of the results. 
This amounts to about
$2n^4/3$ elementary field operations. This is about twice as fast as 
directly evaluating the minimal polynomial. Note that even in this case
only about every sixth entry of\/ $YMY^{-1}$ is different from zero. 
\end{Rem}

\section{Performance in practice}
\label{performance}

In this section we give some experimental evidence about our 
new algorithm in comparison to the current implementation in the
{\sf GAP} library (see \cite{GAP4}). Other computer algebra systems 
perform similarly.

All computations have been done on a machine with a Pentium 4 processor
running at 3,2 GHz with 4 GB of main memory and 1 MB of second level
cache. 

\subsection{Guide to the test data}
The timing results
are in Figure~\ref{timings}, all times are in seconds. 
The column marked ``$n$'' contains the
dimension of the matrix, the column marked ``$q$'' the number of elements
of the base field. The columns marked ``Lib'' and ``Old'' contain the times
needed for one run of the old deterministic algorithm
in the {\sf GAP} library and of a new implementation of the same algorithm
respectively. The column ``MC'' contains the total time for our 
Monte Carlo algorithm as presented in Algorithm~\ref{algminpolymc}.
The next three columns marked ``Spinup'', ``Fact'' and ``OrdPols''
contain the times for the three phases of this algorithm respectively, 
namely the first phase to compute the characteristic polynomial via
relative order polynomials, the second phase to factor all factors of
the characteristic polynomial and count multiplicities, and the third
phase to compute some absolute order polynomials to guess the minimal
polynomial. Finally, the last column marked ``Ver.'' contains
the time for the deterministic verification via 
Algorithm~\ref{algminpolyverify}.
The maximal error probability for our Monte Carlo algorithm was
$\epsilon = 1/100$ for all runs.

\subsection{The test matrices}
Next, we describe the matrices $M_1, \ldots, M_{10}$ we used.

(a) \quad The matrices $M_1$ and $M'_1$ were purely random matrices from
$\F_3^{1000\times 1000}$ with all entries
chosen with uniform distribution from the field $\F_3$. Such matrices
are with very high probability cyclic, that is, their characteristic and
minimal polynomials are equal. Usually, Algorithm~\ref{algcharpoly} only
has to pick very few random vectors for such matrices. The {\bf for} loop
of Algorithm~\ref{propverify} quickly checks whether the least common multiple
of the relative order polynomials (which is the input candidate polynomial)
already has degree $n$. It turned out that $M_1'$ was cyclic but not $M_1$, 
and this explains the
big differences in the runtimes for these matrices.

(b)\quad The matrix $M_2$ is one coming from actual applications. Namely, it is
the matrix $a+b+ab$ where the two matrices $a,b \in \F_2^{4370 \times
4370}$ describe the action of two standard generators of the Baby monster
sporadic simple group on its smallest faithful simple module over $\F_2$.
The matrices $a$ and $b$ were downloaded from the WWW Atlas of group
representations (see \cite{WWWAtlas}). The matrix $a+b+ab$ is
interesting because it is one of the algebra words that is used
in the {\sc MeatAxe} (see \cite{MeatAxeParker} and \cite{MeatAxeHoltRees})
to compute composition series of modules and
we could very well imagine using the minimal polynomial instead of
the characteristic polynomial in some places in the {\sc MeatAxe}.

The reason why the standard algorithm for the minimal polynomial
performed rather badly on this matrix is that its characteristic polynomial
has an irreducible factor of degree about $n/2$ and many repeated linear
factors. Therefore the standard algorithm spins up a large subspace
many times.

(c)\quad The matrices $M_3$ -- $M_7$ were constructed in the following way:
In the language of $\F[x]$-modules we chose the order polynomials of
the generators of their primary cyclic submodules, that is we chose the
minimal polynomials on the primary cyclic submodules. For irreducible factors
of degree one this amounts to choosing the sizes and numbers of the
Jordan blocks occurring in the Jordan normal form of the matrix.
After writing down the corresponding normal form of the matrix we
conjugated it with a random element of the general linear group to
get a dense matrix with the same normal form.

For $M_3 \in \F_5^{600 \times 600}$ we chose one cyclic summand with
minimal polynomial $(x-\zeta_5)^{300}$ plus 300 summands with minimal
polynomial $x-\zeta_5$, where $\zeta_5 \in \F_5$ is a primitive root.
This is a typical case in which our Monte Carlo algorithm and
the deterministic verification both perform very well in comparison
to older techniques. The reason for this is that the high dimensional
cyclic subspace is spun up many times in the standard minimal polynomial
algorithm as for the matrix $M_2$.

For $M_4 \in \F_3^{1200 \times 1200}$ we chose 400 cyclic summands with
minimal polynomial $(x-\zeta_3)^2$ plus 400 cyclic summands with
minimal polynomial $(x-\zeta_3)$, where again $\zeta_3 \in \F_3$ is
a primitive root. In this case our algorithms performed very well but were not
that much faster than the older techniques, since the standard algorithm
does not spin up many large subspaces as for $M_3$.

For $M_5 \in \F_{251}^{600 \times 600}$ we chose 200 different linear
factors $x-\alpha$ and for each added one cyclic space with minimal polynomial
$(x-\alpha)^2$ and one with $x-\alpha$. This is one of the worst case
scenarios for our deterministic verification as can be seen from the
horrible runtime.

For $M_6 \in \F_2^{2391 \times 2391}$ we chose the irreducible polynomial
$q = x^3+x^2+1 \in \F_2[x]$ of degree $3$ and added cyclic spaces with 
respective minimal polynomials $q^{400}$, $q^{200}$, $q^{100}$,
$q^{50}$, $q^{25}$, $q^{12}$, $q^6$, $q^3$ and $q$.

For $M_7 \in \F_{3^4}^{220 \times 220}$ we chose an irreducible polynomial
$q \in \F_{10}[x]$ of degree $10$ and added cyclic spaces with
respective minimal polynomials $q^{10}$, $q$, $q^2$, $q^3$, $q$, $q^2$ and
$q^3$.

(d)\quad The matrices $M_8$ and $M_9$ were standard generators of 
${\rm GL}(400,17)$, 
conjugated by the pseudo-random element $M_{10}$ of the same group.
Note that $M_8$ had
order 16 while $M_9$ and $M_{10}$ had very high order and were cyclic 
matrices. We chose these examples because they may be typical of 
difficult cases in an application of the minimal polynomial algorithm
for computing the projective order of a matrix. 

Our algorithm %ran into the case that it 
very quickly discovered that the least common multiple of the relative order
polynomials was already equal to the characteristic polynomial.

\begin{figure}
\caption{Timings for minimal polynomial computation}
\label{timings}
\begin{center}
\begin{tabular}{|c|r|r|r|r|r|r|r|r|r|}
\hline
Matrix & $q$ & $n$ & Lib & Old & MC & Spinup & Fact & OrdPols & Ver. \\
\hline
\hline
$M_1$  & 3   & 1000 & 4.24$^*$ & 1.74 & 10.1 & 0.92 & 9.20 & 0 & 0 \\
$M_1'$ & 3   & 1000 & 3.65$^*$ & 2.14 & 1.13 & 1.13 & 0 & 0 & 0 \\
$M_2$  & 2   & 4370 & 25613 & 7696 & 10.9 & 7.64 & 1.96 & 1.28 & 9.22 \\
$M_3$  & 5   &  600 & 121 & 50.2 & 0.45 & 0.28 & 0.02 & 0.15 & 0.58 \\
$M_4$  & 3   & 1200 & 3.70$^*$& 1.19 & 1.20 & 1.03 & 0.15 & 0.02 & 0.27 \\
$M_5$  & 251 &  600 & 8.5 &16.0& 10.26 & 8.69 & 0.73 & 0.83 & 636 \\ 
$M_6$  & 2   & 2391 & 31.5& 8.2& 4.6  & 2.30 & 0.35 & 1.95 & 2.46 \\
$M_7$  & 243 &  220 & 2.52 & 3.15& 1.06 & 1.01 & 0.03 & 0.018 & 0.80 \\
$M_8$  & 17  &  400 & 0.980 & 0.450  & 0.069 & 0.035 & 0.027 & 0.007 & 0.064 \\
$M_9$  & 17  &  400 & 0.611 & 0.500 & 0.581 & 0.581 & 0 & 0 & 0 \\
$M_{10}$& 17  &  400 & 0.607 & 0.504 & 0.583 & 0.583 & 0 & 0 & 0 \\
\hline
\end{tabular}

\medskip
$^*$ averaged over 10 runs
\end{center}
\end{figure}

%\bibliographystyle{alpha}
%\bibliographystyle{jcm}

% \begin{thebibliography}{MtxHR}
% 
% \bibitem[AC97]{AC97}
% D. Augot, P. Camion, On the Computation of Minimal Polynomials, Cyclic
% Vectors, and Frobenius Forms, \emph{Linear Algebra and its
% Applications} {\bf 260} (1997), 61--94.
% 
% \bibitem[Eb00]{Eb00}
% Wayne Eberly, 
% Asymptotically Efficient Algorithms for the Frobenius Form
% \emph{Department of Computer Science, University of Calgary
% Technical Report}, 2000. 
% {\small\verb+(http://pages.cpsc.ucalgary.ca/~eberly/Research/+\\ \hspace*{1cm}
% \verb+publications.php)+}
% 
% \bibitem[GAP4]{GAP4}
%   The GAP~Group, \emph{GAP -- Groups, Algorithms, and Programming, 
%   Version 4.4.9}; 
%   2006,
%   \verb+(http://www.gap-system.org)+.
% 
% \bibitem[Gie95]{Gie95}
% M. Giesbrecht, Nearly optimal algorithms for canonical matrix forms,
% \emph{SIAM Journal on Computing} {\bf 24} (1995), 948-969.
% 
% 
% \bibitem[MtxHR]{MeatAxeHoltRees}
% D.~F. Holt, S.~Rees, Testing modules for irreducibility, J. Austral. Math. Soc.
%   Ser. A 57~(1) (1994) 1--16.
% 
% \bibitem[Jac74]{Jacob1}
% Nathan Jacobson.
% \newblock {\em Basic Algebra I.}
% \newblock W.~H.~Freeman and Company, San Francisco, 1974.
% 
% \bibitem[KG85]{KelG85}
% W. Keller-Gehrig, Fast algorithms for the characteristic polynomial,
% \emph{Theor. Computer Science} {\bf 36} (1985), 309--317.
% 
% \bibitem[Knu98]{knuth}
% Donald E. Knuth, \emph{The Art of Computer Programming (Vol. 2):
% Seminumerical Algorithms},
% 3rd edn., Addison Wesley, Reading Mass., 1998.
% 
% \bibitem[OB]{OB}
% E. A. O'Brien, Towards effective algorithms for linear groups. In \emph{Finite
% geometries, groups and computation} (Eds: A. Hulpke, R. Liebler, T. Penttila and \'A. Seress.) de Gruyter, Berlin, 2006. pp. 163--190.
% 
% 
% \bibitem[MtxP]{MeatAxeParker}
% R.~A. Parker, The computer calculation of modular characters (the meat-axe),
%   in: Computational group theory (Durham, 1982), Academic Press, London, 1984,
%   pp. 267--274.
% 
% \bibitem[St98]{Stor98}
% A. Storjohann, An $O(n^3)$ algorithm for Frobenius normal form, \emph{Proceedings of the International Symposium on Symbolic and Algebraic Computation (ISSAC'98)}, ACM Press, 1998, pp. 101-104.
% 
% \bibitem[St01]{Stor01}
%  A. Storjohann, Deterministic computation of the Frobenius form, 
% \emph{Foundations of Computer Science, 2001. Proceedings. 42nd IEEE Symposium}, IEEE Computer Society Press, 
% 2001, pp. 368 - 377 
% 
% \bibitem[WAtlas]{WWWAtlas}
%   Robert Wilson, Peter Walsh, Jonathan Tripp, Ibrahim Suleiman, Stephen
% Ro\-gers, Richard Parker, Simon Norton, Simon Nickerson, Steve Linton,
% John Bray and Rachel Abbott. \\ \emph{The WWW Atlas of Finite Group 
% Representations}, 1999. \\
%   \verb+(http://brauer.maths.qmul.ac.uk/Atlas/)+.
% 
% \bibitem[vzG03]{vzG}
% J. von zur Gathen. \emph{Modern Computer Algebra}, Cambridge
% University Press, 2nd edition, 2003.
% 
% \end{thebibliography}

% is is a part of the habilitation thesis of Max Neunhoeffer

\chapter{Further linear algebra algorithms}
\label{chap:linalg}

In this chapter we collect for the sake of completeness some
linear algebra algorithms together with their complexity analysis
that are used in later chapters. At the end we comment on two major
open problems, namely the discrete logarithm problem in finite fields and
integer factorisation, which come up in the analysis of matrix group
algorithms.

\section{Order and projective order of a matrix}
\label{sec:orders}

This section is about the computation of the order and the projective
order of a matrix $M \in \F_q^{n \times n}$. After defining these 
terms we describe the 
ideas of the method in \cite{CellLeedOrder} in particular to show
two things: First that computing the minimal polynomial of a matrix
\index{minimal polynomial}%
is a crucial step in the computation of its order, and second that
integer factorisations of some of the numbers $q^i-1$ for $1 \le i \le n$ 
can be needed in the process (see Section~\ref{intfact}).

\begin{DefProp}[Order and projective order]
\index{order of a matrix}\index{projective order of a matrix}%
Let $\F$ be a field and $M \in \F^{n \times
n}$ an invertible matrix. The \emph{order} of $M$ is the least natural
number $o$ such that $M^o$ is equal to the identity matrix\/ $\one$.
The \emph{projective order} of $M$ is the least natural number $p$
such that $M^p$ is a scalar matrix, that is, a scalar multiple of the
identity matrix. It follows immediately by division with remainder in
the exponent that $o$ divides every natural number $k$ for which 
$M^k$ is equal to\/ $\one$ and that $p$ divides every natural number $k$
for which $M^k$ is a scalar multiple of\/ $\one$. Thus $p$ divides in
particular $o$ and it follows immediately that, if $M^p = \lambda \cdot
\one$ with $\lambda \in \F$, then $o$ is $p$ times the order of the
scalar $\lambda$. \proofend
\end{DefProp}

For a monic polynomial $f \in \F[X]$ with non-zero constant term
we define the \emph{order} (\emph{projective order} respectively) of 
$X + f\F[X]$ in $\F[X]/f\F[X]$ modulo $f$ as the least natural 
number $p$ such that $X^p$ is congruent 
to $1$ (a scalar respectively) modulo $f$ or equivalently, that $X^p-1$ 
($X^p-\lambda$ respectively) is
divisible by $f$ (for some $\lambda \in \F$). 
The order and projective order
of a matrix $M$ as above and that of $X$ modulo the minimal polynomial 
of $M$ are linked by the following lemma.

\enlargethispage{1\baselineskip}
\begin{Lemm}[Orders and projective orders]
Let\/ $\F$ be a field and $M \in \F^{n \times n}$ an invertible matrix.
Then the (projective) order of $M$ is equal to the (projective) order of
$X + \mu_M \F[X]$ modulo the minimal polynomial $\mu_M$ of $M$.
\end{Lemm}
\proofbeg 
The polynomial $X^p-\lambda$ is divisible by $\mu_M$ if and
only if $M^p = \lambda \one$ for all $\lambda \in \F$.
\proofend

In the following we use this lemma to switch between matrices and
polynomials as seems appropriate for the argumentation.

Now we want to discuss the computation of both the order and the
projective order of a matrix or its minimal polynomial respectively.

Let $f \in \F[X]$ be a monic polynomial with non-zero constant term. 
By the Chinese Remainder theorem the factor ring
$\F[X]/f\F[X]$ is isomorphic to
\[ \F[X]/f\F[X] \cong
   \prod_{i=1}^k \F[X]/f_i^{e_i} \F[X] \]
where $f = \prod_{i=1}^k f_i^{e_i}$ is the factorisation of $f$ into
its pairwise distinct irreducible factors $f_i$. Using this
isomorphism the (projective) order of $X+f\F[X]$ is equal to the least 
common multiple of the (projective) orders of the $X + f_i^{e_i}
\F[X]$ in $\F[X]/f_i^{e_i}\F[X]$. Thus, as a first step we factorise
$f$ completely and now determine the (projective) order of
$X+g^e\F[X]$ for an irreducible monic polynomial $g$ of degree $d$ 
with non-zero constant term.

From now on we switch back to matrices. Let $C \in \F^{d \times d}$ be 
the companion matrix of $g$ and $N$ the $(de \times de)$-block matrix with 
$C$ along the main block diagonal, $(d \times d)$-identity matrices
along the block diagonal directly above the main diagonal and zero blocks
elsewhere:
\[ N = \left[ \begin{array}{ccccc}
    C      & \one   & 0      & \cdots & 0 \\
    0      & C      & \one   & \ddots & \vdots \\
    \vdots & \ddots & \ddots & \ddots & 0 \\
    \vdots & \ddots & \ddots & C      & \one \\
    0      & \cdots & \cdots & 0      & C
\end{array} \right]. \]
The minimal polynomial $\mu_N$ of $N$ over $\F$ is $g^e$ 
which can be seen as follows:
Let $K$ be the splitting field $K$ of $g$ over $\F$. Since $g$ is
irreducible and thus has no multiple roots, the 
matrix $C$ is similar to a diagonal matrix over $K$, that is, there is an
invertible $T \in K^{d \times d}$ such that $TCT^{-1}$ is diagonal
with pairwise disjoint eigenvalues.
Thus, conjugating $N$ with the block matrix having $T$ along the
main block diagonal and permuting rows and columns suitably shows that
the Jordan normal form of $N$ over $K$ consists of $d$ blocks of
size $e$, thus every eigenvalue of $C$ occurs in the minimal
polynomial of $N$ with multiplicity $e$. Thus the minimal polynomial
of $N$ over $K$ is $g^e$ and thus also $\mu_N = g^e$.

We assume from now on that $\F$ is the finite field $\F_q$ with $q$
elements.
Writing $N = \tilde C + \tilde \one$ where $\tilde C$ is the matrix
with only $C$ along the main block diagonal and $\tilde \one$
accordingly we have $\tilde C \cdot \tilde \one = \tilde \one \cdot \tilde C$
and thus $N^k = \sum_{i=0}^k {k \choose i} \tilde C^{k-i} \tilde
\one^i$. This immediately implies that the $(d \times d)$-block
of $N^k$ in position $(j,j+i)$ is ${k \choose i}C^{k-i}$ for all
$1 \le j \le d-i$. 

Therefore the (projective) order of $N$ can be determined in the
following way: The number $\lceil \log_p(e)\rceil$ is the number
of factors $p$ that have to occur in $k$, such that all binomial
coefficients ${k \choose i} \equiv 0 \ (\mbox{mod } p)$ for $1 \le i < e$ 
(where we set ${k
\choose i} = 0$ for $i > k$). Let $l$ be the (projective) order of
$X+g\F[X]$ in the field $\F[X]/g\F[X]$. Then the (projective) order of
$N$ is the product $p^{\lceil \log_p(e) \rceil} \cdot l$. Note that
determining the number $l$ involves computing in the field extension
$\F[X]/g\F[X]$.

In practice one takes $q^d-1$ (respectively $\frac{q^d-1}{q-1}$) 
as an upper bound
for the (projective) order of $X+g\F[X]$ and then uses the ``bounded order
algorithm'' described in \cite[Section 2]{CellLeedOrder} to get the
actual (projective) order. However, this algorithm depends on the
integer factorisation (see Section~\ref{intfact}) 
of $q^d-1$ since it uses a divide-and-conquer
approach using this factorisation.

Altogether we can conclude the following result.

\begin{Prop}[Computing the (projective) order of a matrix]
\index{order of a matrix}\index{projective order of a matrix}%
Let $M$ be a matrix in\/ $\F_q^{n \times n}$. Assume that the 
minimal polynomial of $M$ is known and completely factorised into
\index{minimal polynomial}%
irreducible factors, and that all integer factorisations of $q^i-1$
for $1 \le i \le n$ are known (see Section~\ref{intfact}). 

Then the (projective) order of $M$ can be
computed in $O(n^3 \cdot \log_2(q) \cdot \log_2(n\log_2(q)))$
field operations.
\end{Prop}
\proofbeg Use the same arguments as in the proof of 
\cite[\textsc{Order Algorithm}]{CellLeedOrder}. 
\proofend

\begin{Rem}
This shows that every improvement of the algorithm to compute the
minimal polynomial also helps to compute the (projective) order.
\index{minimal polynomial}%
\end{Rem}

\section{Solving systems of linear equations}
\label{sec:syslineq}
\index{system of linear equations}%

This section describes and analyses the problem of solving a system
of linear equations. This can be formulated as follows:

Let $\F$ be a field, $Y \in \F^{m \times n}$ and $b \in \F^{1 \times n}$.
Find all $x \in \F^{1 \times m}$ with $xY = b$.

Sometimes the problem has to be solved for more than one $b$ and sometimes
it is enough to find one solution $x$. In particular the homogeneous case 
$b=0$ is important, here the set of solutions is called the \emph{nullspace}
of $M$.
\index{nullspace}%

The general approach to this problem is standard, thus we are mostly
interested in the complexity. The basic procedure is
Algorithm~\ref{semiechelonise}: By running
all rows of $Y$ successively through Algorithm~\ref{clean} we
find three things in the process: First a subsequence
$t=(t_1,t_2, \ldots, t_{r})$ of $(1,2,\ldots, m)$ with
$1 \le t_1 < t_2 < \cdots < t_r \le m$ such that the matrix
$Y' \in \F^{r \times n}$ consisting of the rows with numbers 
$t_1, t_2, \ldots, t_r$ of $Y$ has rank $r$. Secondly we get a
semi echelon data sequence $\calY = (Y',S,T,l)$ for $Y'$ as defined
in Definition~\ref{semiecheseq}. Thirdly, for every row of $Y$ that
is a linear combination of the rows above it in $Y$, we find a linear 
relation of the rows of $Y$ and all these relations span the nullspace
of $Y$ in the end.

We leave the details and the proof that this procedure does what
we claim to the reader. The following proposition analyses the cost
of Algorithm~\ref{semiechelonise}.

\begin{algorithm}
\caption{$\quad$ \sc SemiEchelonise}
\index{SemiEchelonise@\textsc{SemiEchelonise}}%
\label{semiechelonise}
\begin{algorithmic}
\STATE \textbf{Input:} A matrix $Y \in \F^{m \times n}$.
\STATE \textbf{Output:} Indices $1 \le t_1 < t_2 < \cdots < t_r$,
a matrix $Y' \in \F^{r \times n}$ with semi echelon data sequence
\STATE \mbox{}\phantom{\textbf{Output:}}
$\calY = (Y',S,T,l)$ and a matrix $N \in \F^{(m-r) \times n}$
whose rows span the nullspace of $Y$
\STATE $\calY := $ empty semi echelon sequence for rows with length $n$
\STATE $N := $ empty matrix for rows with length $n$
\STATE $t := $ empty sequence
\FOR {$i=1$ to $m$}
    \STATE $(f,\calY,a) := \textsc{CleanAndExtend}(\calY,Y[i])$
    \IF {$f = \textsc{True}$}
        \STATE Append a row to $N$ using $a$
    \ELSE
        \STATE Append $i$ to sequence $t$
    \ENDIF
\ENDFOR
\STATE \textbf{Return:} $(t,\calY,N)$
\end{algorithmic}
\end{algorithm}

\begin{Prop}[Complexity of Algorithm~\ref{semiechelonise}]
\label{semiechelon}
\index{SemiEchelonise@\textsc{SemiEchelonise}}%
If $r$ is the rank of $M$, Algorithm~\ref{semiechelonise} needs
at most
\[ \frac{r(r+1)(2r+1)}{6} + nr^2 + 2rn(m-r) + r \]
elementary field operations. This is $O(nm^2)$.
\end{Prop}
\proofbeg
We add up the maximal number of elementary field operations needed
by the calls to \textsc{CleanAndExtend} using
Proposition~\ref{PropCleanAndExtend}. We start with the calls that
extend the semi echelon data sequence:
\begin{eqnarray*}
\sum_{i=0}^{r-1} \left( (2i+1)n + (i+1)^2 + 1 \right) 
  &=& \frac{r(r-1)}{2} \cdot 2n + rn + \frac{r(r+1)(2r+1)}{6} + r \\
  &=& nr^2 + \frac{r(r+1)(2r+1)}{6} + r,
\end{eqnarray*}
where we use Formulas \ref{formels1} and \ref{formels2}. In the worst
case that the first $r$ rows of $Y$ are linearly independent the
number of operations used to clean the rows that do not extend
the semi echelon data sequence is at most $(m-r)\cdot 2rn$.
For the asymptotic statement we use $r \le \min\{ m,n \}$.
\proofend

\medskip
After Algorithm~\ref{semiechelonise} has completed, the rows of the
matrix $N$ span the solution space of the system of linear equations
$xY = 0$. For a given $0 \neq b \in \F^{1 \times n}$ we have to run
the ``cleaning'' part of Algorithm~\ref{clean} once.
The result is either that the system has no solutions or a vector
$a \in \F^{1 \times r}$ such that $b = aS = aTY'$ for the matrix $S$ in
$\calY$ that is in row semi echelon form. Since $Y'$ consists of a 
selection of the rows of $Y$, the vector $aT$ contains all the necessary
information to put together a solution $x$ with $b = xY$. The set
of all solutions is then equal to $\{ x+n \mid n \in \rsp(N) \}$.

\enlargethispage{-1\baselineskip}
\begin{Cor}[Solving a system of linear equations]
\label{solvelinsys}
\index{system of linear equations}%
The set of solutions of the system of linear equations $xY=0$ can be
read off from the output of Algorithm~\ref{semiechelonise}. For
the system $xY=b$ it can be derived using
$2rn + r(r+1)$ elementary field operations.

Thus such a system can be solved in $O(nm^2)$ elementary field operations.
\end{Cor}
\proofbeg
The above method needs one call to \textsc{CleanAndExtend} using at most
$2rn$ elementary field operations and the vector matrix multiplication
$aT$ with the lower triangular matrix $T$ needs at most
$2\cdot \frac{r(r+1)}{2}$ elementary field operations.
\proofend

\section{Inverting matrices}
\label{sec:invert}
\index{inverting a matrix}%

Inverting a matrix $M \in \F^{n \times n}$ is the same as solving the
system of linear equations $xM = e_i$ for all the standard basis
vectors $e_i \in \F^{1 \times n}$. By Proposition~\ref{semiechelon}
and Corollary~\ref{solvelinsys} we get the following result.

\begin{Prop}[Inverting a matrix]
\index{inverting a matrix}%
Let $M \in \F^{n \times n}$ be an invertible matrix. Then the inverse
can be computed using at most
\[ \frac{n(n+1)(2n+1)}{6} + 4n^3 + n^2 + n \]
elementary field operations, which is asymptotically less than $5n^3$
and this is $O(n^3)$.
\end{Prop}
\proofbeg
We first run Algorithm~\ref{semiechelonise} using at most
\[ \frac{n(n+1)(2n+1)}{6} + n^3 + n \]
elementary field operations by Proposition~\ref{semiechelon}. Then
we solve $xM = e_i$ for $1 \le i \le n$ using a total of
\[ n(2n^2+n(n+1)) = 3n^3+n^2 \]
elementary field operations. Summing up gives the claim.
\proofend

\section{The discrete logarithm problem}
\label{thedlp}
\index{discrete logarith problem}\index{DLP}%

\begin{Problem}[The discrete logarithm problem in finite fields]
\index{discrete logarith problem}\index{DLP}%
    Let $q = p^k$ be a power of a prime $p$. Then the multiplicative
    group of the finite field $\F_q$ is cyclic, say $\F_q^\times =
    \left< \zeta \right>$ for a fixed element $\zeta$ of order $q-1$.

    The discrete logarithm problem (DLP) in the finite field $\F_q$ is
    the following:

    \begin{center}\fbox{\parbox{3.5in}{\hfill
    Given $y \in \F_q^\times$, find $a \in \Z$ such that $y =
    \zeta^a$.\hfill\mbox{}}}
\end{center}
\end{Problem}

\subsection*{Computing discrete logarithms}
\index{discrete logarith problem}\index{DLP}%

Obviously this problem can be solved easily using an algorithm that
simply tries all powers of~$\zeta$. However, this simplistic approach
has complexity $O(q)$ in the worst case, which very quickly
becomes impractical as $q$ grows.

Better methods are the Shanks and Pollard methods (see 
\index{Shanks method}\index{Pollard method}%
\cite[Section~3]{odlyzkodlp}) with complexity $O(q^{1/2})$, 
the Index Calculus Method due to
\index{Index Calculus Method}%
Kraitchik (see \cite{McCurley}) and the more recent Number Field Sieve
methods (see \cite[Section~4]{odlyzkodlp}).
\index{Number Field Sieve}%

An overview of the current state of the art to solve this problem
including lots of references
is given in \cite{odlyzkodlp}. For some values of $q$ (in particular
for ``small $p$'') there are known
algorithms with an asymptotic complexity of
\[ \exp( (C+o(1)) (\log q)^{1/3} (\log \log q)^{2/3} ) \]
for some constant $C$,
but it is still an active area of research to come up with methods
with this complexity for all values of $q$.

Currently there seems to be no hope for algorithms that have
polynomial complexity in $\log q$ on classical computers. The development
of quantum computers might change this in the future.

For practical purposes in matrix group algorithms 
we can safely assume that the discrete logarithm
problem does not pose any significant difficulties as long as $q$ is
smaller or equal to $2^{32}$.
\index{discrete logarith problem}\index{DLP}%

\section{Integer factorisation}
\label{intfact}
\index{integer factorisation}%

\begin{Problem}[The integer factorisation problem]
\index{integer factorisation}%
    Let $n \in \N$ be an integer.
    The integer factorisation problem is the following:

    \begin{center}\fbox{\parbox{5in}{\hfill
        Find the unique expression of $n$ as a product of prime
        powers $n = \prod_{i=1}^k p_i^{e_i}$. 
        \hfill\mbox{}}}
\end{center}
\end{Problem}

\subsection*{Computing integer factorisations}
\index{integer factorisation}%

Obviously this problem can be solved by trial division in time
$O(n^{1/2} \log^2 n)$, however, like for the discrete logarithm problem, 
there is currently no known
algorithm to solve this problem with complexity that is polynomial
in $\log n$.

An overview of the current state of the art in integer factorisation
is given in \cite{gabifaktorisierung}.

For numbers $n$ of the special form $a^k \pm 1$, which includes the
cardinalities $p^k - 1$ of the multiplicative groups of finite fields,
extensive tables for ``small'' values of $a$ and $k$ have been
collected. They can be found on the site \cite{brentfact} and are
built into computer algebra systems to speed up integer factorisation.
In particular the factorisations of all numbers $a^k \pm 1$ which are
smaller than $10^{255}$ are provided there.

For practical purposes in matrix group algorithms 
we can safely assume that the integer factorisation
problem for $n = q^i - 1$ 
does not pose any significant difficulties as long as $q^i$ is
smaller than $10^{255}$.
\index{integer factorisation}%



%% this is a part of the habilitation thesis of Max Neunhoeffer

\chapter{The Meat-Axe}
\label{chap:meataxe}

This chapter basically describes, what the Meat-Axe can do for us.

\section{Composition series}

\section{Homomorphisms}

\section{Socles and radicals}

\section{Complexity analysis of the Meat-Axe}

% this is a part of the habilitation thesis of Max Neunhoeffer

\chapter{Composition trees}

After lots of preparations in the previous chapters
in this chapter the main part of the whole book begins. We begin to talk
about the recognition of matrix groups. We start by formulating
the concept of ``constructive recognition'' and explain the reasoning
behind it in Section~\ref{constrrecog}. Then we explain the fundamental
approach to achieve
constructive recognition by means of a ``composition tree'' in
Section~\ref{recapproach}. We develop a
framework for group recognition that is not only suitable for matrix
groups and projective groups but also allows for the implementation of
the asymptotically best algorithms to handle permutation groups. In
this framework we can switch between different representations of
groups within one composition tree which allows for example to use 
permutation group methods during matrix group recognition, provided
we find some set our matrix group is acting upon.

The general approach to build a composition tree is not new and is
already described in \cite{MatGrpProj}. What is new in our approach
is first the abstraction to allow for different representations
intermixed in the same composition tree and secondly the fact that
we change the generating set in each node of the composition tree
which dramatically decreases the length of the resulting straight
line programs. 

The contents of this chapter stem from joint work with \'Akos Seress
and are an elaboration on the article \cite{AkosMaxISSAC}.

\section{Constructive recognition}
\label{constrrecog}

There are at least two fundamentally different ways to represent groups on
a computer. The first uses a presentation of the group and then
expresses group elements as words in a free group representing
cosets of the normal subgroup generated by the relations. The second
approach uses an ambient group whose elements can be represented and
multiplied directly on the computer and defines the group by giving
a generating set. As ambient groups one can use symmetric groups, general
linear groups or projective groups, since we can store and manipulate
permutations and matrices efficiently on a computer.

In this book we concentrate on the second approach. To formalise our
problem we first write down our assumptions about the ambient group
and then formulate the fundamental problem.

\begin{Hyp}[Ambient group]
\label{ambient}
When we speak about an \emph{ambient group} we mean a finite group that can
be represented on a computer such that we can perform the following tasks:
\begin{itemize}
\item Store and compare group elements.
\item Have available the identity element.
\item Multiply group elements.
\item Invert group elements.
\item Compute the order of a group element.
\end{itemize}
\end{Hyp}

\begin{Rem}
All finite symmetric groups fulfil the hypotheses in Section~\ref{ambient}.
We call subgroups of finite symmetric groups \emph{permutation groups}.

For a prime power $q$ the groups $\GL(n,q)$ and $\PGL(n,q)$ also fulfil
the hypotheses. We call subgroups of $\GL(n,q)$ \emph{matrix groups}
and subgroups of $\PGL(n,q)$ \emph{projective groups}.
\end{Rem}

\begin{Problem}[Constructive recognition --- first formulation]
\label{ProbCR1}
Let $\mathcal{G}$ be an ambient group in the sense of \ref{ambient} and 
assume we are given a generating tuple $(g_1, \ldots, g_k) \in
\mathcal{G}^k$ for a group
$G := \left< g_1, \ldots, g_k \right> \le \mathcal{G}$. 

We say that we have \emph{recognised $G$ constructively} if we have 
computed $|G|$ and a
generating tuple $( g'_1, \ldots, g'_l )$ for $G$, for which we have
prepared a procedure that does the following: Given $g \in \mathcal{G}$,
decide whether $g \in G$ and if so, express $g$ as a straight line program
in $(g'_1, \ldots, g'_l)$. The new generating tuple may or may not be the
same as the original one.
\end{Problem}

\begin{Rem}
The problem as posed in \ref{ProbCR1} can be solved easily by just
enumerating the finite group $G$ completely. However, this is neither
a sensible approach nor very interesting. The crucial point missing
in \ref{ProbCR1} is that we want to do constructive recognition
\emph{efficiently}. 
\end{Rem}

\section{A recursive approach}
\label{recapproach}

\section{Finding homomorphisms and nice generators}
\label{findhom}

\section{Expressing group elements in terms of the nice generators}
\label{expressslp}

\section{Finding the kernel of a homomorphism}
\label{findkernel}

\section{Asymptotically best algorithms for permutation groups}
\label{permgrps}


% this is a part of the habilitation thesis of Max Neunhoeffer

\chapter{Finding homomorphisms}
\label{chap:findhom}

In the previous Chapter~\ref{chap:comptree} we have formulated the problem
of constructive recognition of groups in Problem~\ref{ProbCR3} and have
explained what a reduction is. This chapter describes how to find
reductions for matrix groups and projective groups and how to prove
that a certain collection of methods will for any given
matrix group or projective group either find a reduction
or show that a situation is on hand, in which the constructive 
recognition problem can be solved efficiently by other means. By the 
arguments in Chapter~\ref{chap:comptree} this solves
Problem~\ref{ProbCR3} whenever the ambient group is $\GL(n,q)$ or
$\PGL(n,q)$.

We use two fundamental theoretical tools. The first is Aschbacher's theorem
about the subgroup structure of the classical groups, which we describe
in detail for the case of $\GL(n,q)$ in Section~\ref{sect:aschbacher},
and the second is the classification of finite simple groups. We are
content with the statements in Aschbacher's theorem about the general
linear group since they suffice for the purposes of constructive
recognition and the statements are quite a bit simpler than for the
other classical groups. For details see \cite{aschbacher}.

Aschbacher's theorem states that every subgroup of $\GL(n,q)$
is either a member of at least one of $7$ concretely given classes 
\CC1 to \CC7 of 
subgroups, or it contains a classical group in its natural representation, 
or it is an almost-simple group. IS THAT RIGHT?
% FIXME

All the classes \CC1 to \CC7 are somehow defined in a geometric way (see
Sections~\ref{descC1} to \ref{descC7}) and thus promise some kind of
reduction. The two other cases are covered by two further classes \CC8
and \CC9, which are described in Sections~\ref{descC8} and \ref{descC9}.
For members of the latter two classes one will have to solve the
constructive recognition problem without further reduction.

The idea is to provide efficient algorithms for all the classes \CC1 to
\CC7 to recognise whether a given matrix group lies in the class, and if
so, to find a reduction using this information. If none of these algorithms
succeeded, Aschbacher's theorem shows that the group must be a member of
\CC8 or \CC9. In that case the constructive recognition problem has to be
solved by different means, usually by first finding out which classical
or almost simple group it is and then using this information to do the
constructive recognition in a special case, for example using standard
generators (see Sections~\ref{solveC8} and \ref{solveC9}).

The purpose of this chapter is to explain the statement of Aschbacher's 
theorem for $\GL(n,q)$ in detail and to give an overview over the known
methods to deal with the different classes together with references into
the literature. An algorithm to recognise and handle classes \CC3 and \CC5
provided that the group does not lie in class \CC1 is described in detail
in Chapter~\ref{chap:subsemi}.

\section{Aschbacher's theorem}
\label{sect:aschbacher}

\begin{Not}
For this section we fix $n \in \N$ and $q=p^e$ for a prime $p$ and
talk about the group $\GG := \GL(n,q)$. We denote the vector space $\F_q^{1
\times n}$ by $V$ and note that $\GG$ acts from the right on $V$ by
vector-matrix multiplication.
\end{Not}

\begin{Theo}[Aschbacher, specialised to $\GL(n,q)$]
\label{Asch}
Let $G$ be a subgroup of\/ $\GG = \GL(n,q)$.
Then $G$ is contained in a member of at least one of the
classes \CC1 to \CC9 of subgroups described in Sections~\ref{descC1}
to \ref{descC9}.
\end{Theo}
\proofbeg See \cite{aschbacher} and \cite{kleilieb}.
\proofend

\begin{Rem}
In our description of the classes \CC1 to \CC9 we follow
\cite{kleilieb}. Kleidman and Liebeck change the definition 
slightly in comparison to Aschbacher but argue that Theorem~\ref{Asch}
remains true with their definitions (see \cite[Chapter~4]{kleilieb}).
\end{Rem}

\subsection{Description of Class \CC1: ``reducible''}
\label{descC1}

A subgroup $G < \GG$ is a member of $\CC1$ if there is a subspace
$W < V$ that is stabilised by $G$, that is, $Wg = W$ for all $g \in G$.

\subsection{Description of Class \CC2: ``imprimitive''}
\label{descC2}

\subsection{Description of Class \CC3: ``semi-linear''}
\label{descC3}

\subsection{Description of Class \CC4: ``tensor-decomposable''}
\label{descC4}

\subsection{Description of Class \CC5: ``subfield''}
\label{descC5}

\subsection{Description of Class \CC6: ``extraspecial''}
\label{descC6}

\subsection{Description of Class \CC7: ``tensor-induced''}
\label{descC7}

\subsection{Description of Class \CC8: ``classical''}
\label{descC8}

\subsection{Description of Class \CC9: ``almost-simple''}
\label{descC9}

\section{Finding a reduction in the reducible case: \CC1}
\label{solveC1}

\section{Finding a reduction in the semi-linear or subfield case: \CC3/\CC5}
\label{solveC3C5}

\section{Finding a reduction in the extraspecial case: \CC6}
\label{solveC6}

\section{Finding a reduction in the imprimitive case: \CC2}
\label{solveC2}

\section{Finding a reduction in the tensor-decomposable case: \CC4}
\label{solveC4}

\section{Finding a reduction in the tensor-induced case: \CC7}
\label{solveC7}




% this is a part of the habilitation thesis of Max Neunhoeffer

\chapter{Recognising subfield and semilinear}
\label{chap:subsemi}

to be written

% this is a part of the habilitation thesis of Max Neunhoeffer

\chapter{Leaves of the composition tree}
\label{chap:leaves}

\section{The classical case: C8}
\label{solveC8}

\section{The almost simple case: C9}
\label{solveC9}


%% this is a part of the habilitation thesis of Max Neunhoeffer

\chapter{Further algorithms}
\label{chap:furtheralg}



\appendix

%% this is a part of the habilitation thesis of Max Neunhoeffer

\chapter{Addendum}

to be written


%\mbox{}
\thispagestyle{fancy}

\chapter{Notation and Symbols}

\label{NotationIndex}
\begin{longtable}{|lll|}
\hline\endfoot
\hline\endhead
\hline
\multicolumn{3}{|l|}{Symbols:}\\
\hline
$\emptyset$             & the empty set
                        & \\
$\subseteq$             & is contained in
                        & \\
$\supseteq$             & contains
                        & \\
$\subsetneq$            & is contained properly
                        & \\
$\supsetneq$            & contains properly
                        & \\
$\circ$                 & concatenation of mappings (left after right)
                          or central product
                        & %\ref{conventions} 
                        \\
$\cong$                 & isomorphism
                        & \\
$\lceil x \rceil$       & smallest integer $\ge x$
                        & \\
$\lfloor x \rfloor$     & greatest integer $\le x$
                        & \\
$(-|-)$                 & bilinear form
                        & \\
$\displaystyle\sum$     & sum
                        & \\
$\displaystyle\prod$    & product
                        & \\
$\displaystyle\bigoplus$ & direct sum of modules
                        & \\
$\left< - \right>_A$    & $A$-span by $-$
                        & \\
$\left< - \right>$      & group generated by $-$
                        & \\
$\left< - \right>^G$    & normal closure in $G$ of $-$
                        & \\
$\oplus$                & direct sum
                        & \\
$\otimes$               & tensor product
                        & \\
$\times$                & direct product or cartesian product
                        & \\
$\F^\times$             & multiplicative group of the field $\F$
                        & \\
$\unlhd$                & is normal subgroup in 
                        & \\
$\varphi^{-1}(N)$       & full preimage of the set $N$ under $\varphi$
                        & \\
$G \wr H$               & wreath product of $G$ with $H \le S_n$
                        & \\
$[x,y]$                 & commutator $x^{-1}y^{-1}xy$
                        & \\
$A_n$                   & alternating group on $n$ points
                        & \\
$\ann_R(v)$             & annihilator in $R$ of $v$
                        & \\
$\Aut(G)$               & automorphism group of $G$
                        & \\
$\CC1$ to $\CC9$        & Aschbacher classes, see Sections~\ref{descD1} to
                          \ref{descD9} 
                        & \\
$C_G(H)$                & centraliser in $G$ of $H$
                        & \\
$\DD1$ to $\DD9$        & modified Aschbacher classes, see 
                          Sections~\ref{descD1} to \ref{descD9} 
                        & \\
$\deg f$                & degree of the polynomial $f$
                        & \\
$\det M$                & determinante of the matrix $M$
                        & \\
$\dim_{\F}(V)$          & $\F$-dimension of $V$
                        & \\
$E(X)$                  & expected value of the random variable $X$
                        & \\
$\End_G(M)$             & set of endomorphisms of the $G$-module $M$
                        & \\
$\exp(x)$               & exponential function of $x$ (i.e. $e^x$)
                        & \\
$F^{n \times m}$        & $n \times m$-matrices with entries in $F$
                        & \\
$\F_q$                  & field with $q$ elements
                        & \\
$F[G]$                  & set of linear combinations of the elements of
                          the matrix group $G$
                        & \\
$FG$                    & group algebra of $G$ over $F$
                        & \\
$|G|$                   & order of the group $G$
                        & \\
$G'$                    & derived subgroup of the group $G$
                        & \\
$\gcd(a,b)$             & greatest common divisor of $a$ and $b$
                        & \\
$\GL(n,q)$              & group of invertible matrices in $\F_q^{n \times n}$
                        & \\
$\GGL(n,q)$             & group of semilinear transformations of 
                          $\F_q^n \to \F_q^n$
                        & \\
$\Hom_G(M,N)$           & set of homomorphism between the $G$-modules $M$
                          and $N$
                        & \\
$\id$                   & identity mapping
                        & \\
$\Ima(f)$               & image of the mapping $f$
                        & \\
$\Inn(G)$               & inner automorphism group of the group $G$
                        & \\
$\ker(f)$               & kernel of the mapping $f$
                        & \\
$\lcm(a,b)$             & least common multiple of $a$ and $b$
                        & \\
$\log_b(x)$             & logarithm of $x$ with respect to the basis $b$
                          ($= e$ if omitted)
                        & \\
$|M|$                   & cardinality of the set $M$
                        & \\
$\min(M)$               & minimum of the set $M$
                        & \\
$\mu_{M,V}$             & minimal polynomial of $M$ in its action on $V$
                        & \\
$N_G(H)$                & normaliser in $G$ of $H$
                        & \\
$\N$                    & set of natural numbers, $0 \notin \N$
                        & \\
$\ord_M(v)$             & order polynomial of $M$ acting on $v$
                        & \\
$O(g)$                  & capital-O notation, see Section~\ref{capO}
                        & \\
$\mathrm{O}^{\epsilon}(n,q)$   & orthogonal group in $\GL(n,q)$, see 
                          \cite[Section 2.4]{ATLAS}
                        & \\
$\Omega^{\epsilon}(n,q)$ & $\mathrm{O}^{\epsilon}(n,q)'$ in
most cases, see \cite[Section 2.4]{ATLAS} for details
                        & \\
$\Out(G)$               & outer automorphism group of the group $G$
                        & \\
$\PGL(n,q)$             & full projective group ($\GL(n,q)$ modulo scalars)
                        & \\
$\Phi(G)$               & Frattini subgroup of $G$ 
                        & \\
$\POmega^{\epsilon}(n,q)$   & simple projective orthogonal group (see
                          \cite[Section 2.4]{ATLAS})
                        & \\
$\Prob(E)$              & probability of the event $E$
                        & \\
$\Prob(E|F)$            & probability of the event $E$ under the condition $F$
                        & \\
$\PSL(n,q)$             & special projective group ($\SL(n,q)$ modulo scalars)
                        & \\
$\PSp(n,q)$             & projective symplectic group ($\Sp(n,q)$ modulo 
                          scalars)
                        & \\
$\PSU(n,q)$             & projective special unitary group ($\SU(n,q)$ modulo scalars)
                        & \\
$\R$                    & set of real numbers
                        & \\
$\rsp(Y)$               & row space of the matrix $Y$
                        & \\
$\rho\!\uparrow^{\otimes G}$ & tensor induced (projective) representation
                          see Section~\ref{tensorinduction}
                        & \\
$S_n$                   & symmetric group on $n$ points
                        & \\
$\SL(n,q)$              & special linear group, matrices of determinant $1$
                          in $\GL(n,q)$
                        & \\
$\Sp(n,q)$              & symplectic group in $\GL(n,q)$
                        & \\
$\SU(n,q)$              & special unitary group in $\GL(n,q^2)$
                        & \\
$\Syl(G,p)$             & Sylow $p$-subgroup of $G$
                        & \\
$\mathrm{U}(n,q)$       & unitary group in $\GL(n,q^2)$
                        & \\
$V|_N$                  & module $V$ restricted to $N$
                        & \\
$\chi_{M,V}$            & characteristic polynomial of $M$ in its action on $V$
                        & \\
$\Z$                    & set of rational integers 
                        & \\
$Z(G)$                  & centre of the group $G$
                        & \\
\hline
\end{longtable}

\phantomsection{}
\label{NotationIndexEnd}

\clearpage
\markboth{Appendix B: List of figures}{Appendix B: List of figures}

\newcommand{\friss}[1]{}
\newcommand{\myitem}[2]{\rlap{#1}\hspace*{\cftchapnumwidth}{#2}}

\phantomsection{}
\addcontentsline{toc}{chapter}{\myitem{B}{List of figures}}
%\vspace*{10mm}
{\huge\bf Appendix B}

\listoffigures

\phantomsection{}
\addcontentsline{toc}{chapter}{\myitem{C}{List of tables}}
\vspace*{18mm}
{\huge\bf Appendix C}

\listoftables

\stepcounter{chapter}
\stepcounter{chapter}

\markboth{Appendix C: List of figures}{Appendix C: List of figures}

%\newpage\mbox{}
%\clearpage\mbox{}\thispagestyle{fancy}

{\sloppy
\bibliographystyle{alpha}
\bibliography{habil}
}

\printindex

\end{document}

