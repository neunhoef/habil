% This is the main file of the Habilitation thesis of Max Neunhoeffer

\documentclass[openany,11pt,british]{book}

\usepackage[latin1]{inputenc}
\usepackage{amssymb}
\usepackage{makeidx}
\usepackage[british]{babel}
%\usepackage{stmaryrd}
\usepackage[mtbold,subscriptcorrection,mtpluscal]{mathtime}
%\usepackage[heavybold,uprightgreek,subscriptcorrection,mtpluscal]{mathtime}
\usepackage{theorem}
\usepackage[all,ps]{xy}
\usepackage{showkeys}
\usepackage{ifthen}
\usepackage[bf,center]{caption}
%\renewcommand{\captionfont}{\sffamily}
\usepackage{calc}
\usepackage{tocloft}
\setlength{\cftchapnumwidth}{8mm}
\setlength{\cftsecnumwidth}{12mm}
%\usepackage{titlesec}
\usepackage{fancyhdr}
\usepackage{color}
\usepackage{longtable}
\usepackage{graphicx}
\usepackage[colorlinks=true,backref=page,dvipdfm,%hypertex,
            linkcolor=MyBlue,urlcolor=MyRed,citecolor=MyGreen,
            pdftitle={Computing with Matrix Groups},
            pdfauthor={Max Neunh\string\�ffer},
            pdfsubject={Matrix Groups},
            pdfkeywords={constructive recognition}]{hyperref}
\usepackage{myhyppdf}
\usepackage{algorithm}
\usepackage{algorithmic}

\usepackage[numbib,numindex,notlot,notlof]{tocbibind}

%\definecolor{RoyalBlue}{rgb}{0.0236,0.0894,0.6179}
%\definecolor{RoyalGreen}{rgb}{0.0236,0.6179,0.0894}
%\definecolor{RoyalRed}{rgb}{0.6179,0.0236,0.0894}
\definecolor{MyBlue}{rgb}{0.01,0.05,0.5}
\definecolor{MyGreen}{rgb}{0.01,0.4,0.05}
\definecolor{MyRed}{rgb}{0.7,0.01,0.05}

% The paper format:
\usepackage[a4paper,
%\usepackage[letterpaper,  % for letter: make it possible but not optimized
            height=8.30in,width=5.87in,
            % this is 71% of a4 in both dimensions ==> page usage 50%
            % use default: hmarginratio=2:3,vmarginratio=2:3,
            %showframe,   % for verification purposes
            headheight=1\baselineskip,verbose]{geometry}
%So zum Korrekturlesen mit breitem rechtem Rand:
%%%\usepackage[a4paper,
%%%%\usepackage[letterpaper,  % for letter: make it possible but not optimized
%%%            height=8.30in,width=5.87in,
%%%            % this is 71% of a4 in both dimensions ==> page usage 50%
%%%            % use default: hmarginratio=2:3,vmarginratio=2:3,
%%%            %showframe,   % for verification purposes
%%%            headheight=1\baselineskip,verbose,twoside=false,left=5mm]{geometry}

% Bemerkung: showkeys und hyperref funktionieren nicht zusammen!

% The page headings:
\fancyhead[LO]{\nouppercase{\rightmark}}
\fancyhead[RE]{\nouppercase{\leftmark}}
\fancyhead[LE,RO]{\thepage}
\fancyhead[FCO,FCE]{}
\pagestyle{fancy}

\setcounter{tocdepth}{2}

% Change arrows in xypic:
\SelectTips{cm}{11}
\UseTips

\makeindex

\parindent0pt

% Mathematische Operatoren wie z.B. rad:
\makeatletter
\newcommand{\maop}[1]{%
\ensuremath{\mathop{\operator@font #1}\nolimits}}
\newcommand{\maopl}[1]{%
\ensuremath{\mathop{\operator@font #1}\limits}}
\makeatother

\newcommand{\rad}{\maop{rad}}
\newcommand{\soc}{\maop{soc}}
\newcommand{\myimplies}{\ensuremath{\Longrightarrow}}
\newcommand{\myiff}{\ensuremath{\iff}}
\newcommand{\Hom}{\maop{Hom}}
\newcommand{\End}{\maop{End}}
\newcommand{\Tr}{\maop{Tr}}
\newcommand{\id}{\maop{id}}
\newcommand{\Ima}{\maop{Im}}
\newcommand{\Char}{\maop{char}}
\newcommand{\GL}{\maop{GL}}
\newcommand{\GGL}{\maop{\Gamma L}}
\newcommand{\PGL}{\maop{PGL}}
\newcommand{\SL}{\maop{SL}}
\newcommand{\PSL}{\maop{PSL}}
\newcommand{\Sp}{\maop{Sp}}
\newcommand{\PSp}{\maop{PSp}}
\newcommand{\GSp}{\maop{GSp}}
\newcommand{\tens}[1][]{\ifthenelse{\equal{#1}{}}{\maopl{\otimes}}%
{\mathop{\raisebox{0.4mm}{$\scriptstyle\maopl{\otimes}_{#1}$}}}}
\newcommand{\smtens}[1][]{\ifthenelse{\equal{#1}{}}{\maopl{\otimes}}%
{\mathop{\raisebox{0.15mm}{$\scriptscriptstyle\maopl{\otimes}_{#1}$}}}}
\newcommand{\myle}{\leqslant}
\newcommand{\myge}{\geqslant}
\newcommand{\Span}{\maop{Span}}
\newcommand{\Cmpl}{\maop{Cmpl}}
\newcommand{\ord}{\mathrm{ord}}
\newcommand{\ann}{\mathrm{ann}}
\newcommand{\len}{\mathrm{length}}
\newcommand{\lc}{\mathrm{lc}}
\newcommand{\lcm}{\mathrm{lcm}}
\newcommand{\opspd}{\mbox{\sc OpsPD}}
\newcommand{\opsgcd}{\mbox{\sc OpsGcd}}
\newcommand{\Prob}{\mathrm{Prob}}
\newcommand{\GG}{\mathcal{G}}
\newcommand{\HH}{\mathcal{H}}
\newcommand{\calF}{\mathcal{F}}
\newcommand{\calY}{\mathcal{Y}}
\newcommand{\rsp}{\mathrm{RowSp}}
\newcommand{\la}{\left<}
\newcommand{\ra}{\right>}
\newcommand{\ve}{\varepsilon}
\newcommand{\one}{\mathbf{1}}
\newcommand{\fail}{\textsc{Fail}}
\newcommand{\Gal}{\maop{Gal}}
\newcommand{\Aut}{\maop{Aut}}
\newcommand{\Out}{\maop{Out}}
\newcommand{\Inn}{\maop{Inn}}
\newcommand{\Syl}{\maop{Syl}}
\newcommand{\Pro}{\maop{Prob}}

\newcommand{\proofbeg}{\noindent\textbf{Proof:}\ }
\newcommand{\proofof}[1]{\noindent\textbf{Proof of #1:}}
\newcommand{\proofend}{\hfill$\blacksquare$}


% Einige Definitionen fuer haeufig vorkommende Buchstaben:

\newcommand{\F}{\ensuremath{\mathbb{F}}}
\let\ll=\l   % fuer alle Faelle!
\renewcommand{\l}{\ensuremath{\ell}}
\newcommand{\C}{\ensuremath{\mathbb{C}}}
\newcommand{\N}{\ensuremath{\mathbb{N}}}
\newcommand{\Q}{\ensuremath{\mathbb{Q}}}
\newcommand{\R}{\ensuremath{\mathbb{R}}}
\newcommand{\Z}{\ensuremath{\mathbb{Z}}}
\newcommand{\CC}[1]{\ensuremath{\mathcal{C}_{#1}}}
\newcommand{\T}{\ensuremath{\mathrm{T}}}
% Ich will spaeter noch umkonfigurieren:
\newcommand{\Emph}[1]{{\boldmath\textbf{#1}}}
\newcommand{\ba}[1]{\overline{\rule{0mm}{1.2ex}#1}}

% Einige Gimmicks:
\newcommand{\Proof}{\noindent\textbf{Proof:} }
\newcommand{\ProofEnd}{\hfill$\square$\par}
\newcommand{\GAP}{\textsf{GAP}}
\newcommand{\MAGMA}{\textsc{MAGMA}}
\newcommand{\cvec}{\textsf{cvec}}

\newenvironment{compactlist}{\begin{list}{}{\setlength{\itemsep}{0pt}%
\setlength{\labelwidth}{1cm}\addtolength{\labelsep}{3mm}%
\addtolength{\leftmargin}{3mm}}}{\end{list}}

\renewcommand{\thechapter}{\Roman{chapter}}
\renewcommand{\thesubsection}{(\arabic{section}.\arabic{subsection})}
\newcommand{\mychapter}[1]{\chapter{#1}}
\makeatletter
\let\@oldrefstepcounter=\refstepcounter
\def\refstepcounter#1{\@oldrefstepcounter{#1}%
\ifthenelse{\equal{#1}{subsection}}%
{\gdef\@currentlabel{\thechapter.\thesubsection}}{}}
\makeatother

\theoremstyle{changebreak} 
\theoremheaderfont{\bfseries\boldmath}
\newtheorem{Prop}{Proposition}[section]
\newtheorem{Theo}[Prop]{Theorem}
\newtheorem{Lemm}[Prop]{Lemma}
\newtheorem{Cor}[Prop]{Corollary}
\newtheorem{DefProp}[Prop]{Definition/Proposition}
\newtheorem{Conj}[Prop]{Conjecture}
\theorembodyfont{\upshape}
\newtheorem{Def}[Prop]{Definition}
\newtheorem{Rem}[Prop]{Remark}
\newtheorem{Rems}[Prop]{Remarks}
\newtheorem{Not}[Prop]{Notation}
\newtheorem{Hyp}[Prop]{Hypothesis}
\newtheorem{Exa}[Prop]{Example}
\newtheorem{Problem}[Prop]{Problem}

\newcommand{\ruecke}{\mbox{\phantom{\rm\bf\thesubsection\ }}}

% Some things for the subfield paper:
\newcommand{\by}{\times}
\newcommand{\tp}{\otimes}
\newcommand{\idiv}{\ {\textrm{div}\ }}

\newenvironment{proof}{\normalsize {\noindent\sc Proof}:}%
{{\hfill $\Box$\par\medskip}}

%\includeonly{}

\begin{document}

\title{Computing with matrix groups}
\author{Max Neunh\"offer}
\maketitle

% this is a part of the habilitation thesis of Max Neunhoeffer

\chapter*{Preface}
\addcontentsline{toc}{chapter}{Preface}
\markboth{Preface}{The Goal}

\renewcommand{\thechapter}{0}

%This chapter gives a brief overview of the mathematical area and the 
%content of the present book and finishes with the acknowledgements.

\section{The Goal}

The mathematical area of the present book is computational group
theory. The ultimate goal of this theory is to be able to do
computations in and with groups. Traditionally there are several ways
to implement groups on computers. One way is to work with finitely
presented groups, that is by specifying generators and relations.
Another is to use group actions to actually store,
multiply, invert and compare group elements in groups. Using 
actions on sets leads to the study of permutation groups, using linear
actions leads to the study of matrix groups and using projective
actions leads to the study of projective groups, by which we mean
groups in which the elements are invertible matrices modulo scalars.
Considering other actions to represent groups on a computer is
possible (see for example \cite{Kohl}) but up to now not studied much.

For finite permutation groups there are highly efficient
algorithms to compute the order of a group, a composition series,
centralisers of elements, stabilisers, $p$-Sylow subgroups, to find
normal subgroups, maximal subgroups, group homomorphisms, to test
membership of group elements and the like.
Standard techniques use stabiliser chains, base and strong generating
set methods, for a good account of the known methods see \cite{Ser}. 
These algorithms are implemented in computer algebra
systems like {\GAP} (see \cite{GAP4}) and {\MAGMA} (see \cite{Magma})
and the status of this area can justifiably be called satisfactory.

A few interesting experiences have been made from this area. One is that 
complexity theory,
\index{complexity theory}%
which is the study of the asymptotic behaviour of the runtime of algorithms 
as the problem size tends to infinity\footnote{For a brief discussion see 
Section~\ref{sec:complexity}.}, has been a very important tool to
devise efficient algorithms and to implement them. In particular,
worst-case analysis of the complexity of permutation group 
algorithms has shown that
\index{complexity}%
it is in practice worthwhile to first exclude a few nasty cases ---
for which special methods apply --- before applying standard methods,
which would be disastrously slow in these nasty cases. Large base
groups (see Section~\ref{SmallLargeBase}) are an example for such
nasty cases for stabiliser chain methods.
Another interesting
experience from the study of algorithms for finite permutation groups 
is that randomised algorithms, followed by some kind of
deterministic verification of the results, have been very successful.
\index{verification}%

The situation for matrix groups over finite fields, which are subgroups 
of some general linear group $\GL(n,q)$ --- the group of all
invertible $n \times n$-matrices with entries in the finite field
$\F_q$ with $q$ elements with matrix multiplication as product ---
is less satisfactory. Roughly speaking, if one enters a few
elements $g_1, \ldots, g_k$ of some $\GL(n,q)$ into a computer algebra 
system like {\GAP} or {\MAGMA}, it is in general quite limited what
these systems can tell you about the subgroup of $\GL(n,q)$ that is
generated by the elements $g_i$.

So the ultimate goal in this subarea of computational group theory
is to be able to do as much as possible with these matrix groups over
finite fields, guided by what is already possible efficiently for
finite permutation groups. The general idea is to use the complexity
analysis of algorithms as a tool to come up with ``efficient''
algorithms. Generally an algorithm is considered to be ``good'' if
its runtime is bounded from above by a polynomial in the size of the
input.

A first step in this direction is set out by the so-called matrix group
recognition project (see \cite{MatGrpProj} and \cite{OB}). This project
came to live after the seminal paper \cite{neumann-praeger} by Peter Neumann 
and Cheryl Praeger in 1992, in which they present an algorithm which decides,
whether or not a group $G = \left< g_1, \ldots, g_k\right> \le \GL(n,q)$
given by a set of generators contains the special linear group $\SL(n,q)$.
The matrix group recognition project in the meantime has become an ongoing 
collaborative effort of many researchers, and its aim
is to devise efficient algorithms to perform the constructive recognition
of a matrix group $G \le \GL(n,q)$ which is given by a set of generators. 
With the term ``constructive recognition'' we mean
\begin{itemize}
\setlength{\itemsep}{0pt}\setlength{\parskip}{0pt}
\item to find the group order $|G|$, and
\item to set up a procedure to 
\begin{itemize}
\setlength{\itemsep}{0pt}\setlength{\parskip}{0pt}
\item test whether an element $g \in \GL(n,q)$ is contained in $G$, and if so,
\item express $g$ explicitly as a word in the given generators.%
\footnote{For more details on the constructive recognition problem see
Section~\ref{constrrecog}.}
\end{itemize}
\end{itemize}
One good reason, why it is sensible to solve this problem for matrix groups
first, is that the extremely successful method of stabiliser chains and
bases and strong generating sets solves exactly
this problem for permutation groups and is one of the fundamental
algorithms for them. Therefore it is conceivable that an efficient solution
to the constructive recognition problem for matrix groups will provide a
similar fundament for the solution of further computational problems for
matrix groups.

The basic approach for the constructive recognition of matrix groups is
to compute a so-called composition tree for $G$. The nodes of this binary 
\index{composition tree}%
tree are groups together with a homomorphism to another group, the two
descendants of a node are the kernel and image of this homomorphism. The
leaves of the tree are simple groups, in which the constructive recognition
\index{leaf}%
problem has to be solved by other means, and the root of the tree is $G$
itself. The infrastructure of the tree is organised such that a solution to
the constructive recognition problem for $G$ can be put together
by recursively traversing the tree and finally using the solutions to the
constructive recognition problems in the leaves. This description of the
composition tree here is
necessarily very rough, for details see \cite{MatGrpProj} or \cite{OB}
or Section~\ref{recapproach}.

\section{This Book}

The matrix group recognition project has produced many
results but is not yet fully completed.
This book gives an overview of the current state of the art and
shows some contributions of the author. The collaborative nature of the
whole project lends itself to joint papers by more than one person.
Therefore, some of the author's contributions to the project, which can
be found in Chapters~\ref{chap:charminpoly}, \ref{chap:comptree} and
\ref{chap:subsemi}, are in fact collaborations with other authors.

A second major part of the contributions of the author to the project lies
in actual implementations. Together with \'Akos Seress we have started an
implementation of the best known algorithms for group recognition in the
{\GAP} computer algebra system (see \cite{GAP4}). This implementation comes
in the form of two {\GAP} packages \textsf{recogbase} (see
\index{recog package@\textsf{recog} package}%
\index{recogbase package@\textsf{recogbase} package}%
\cite{recogbase}) and \textsf{recog} (see \cite{recog})
which are due to be published soon. The first provides a generic framework
to implement composition trees in {\GAP} for arbitrary types of groups. 
\index{composition tree}%
One interesting feature of the generic framework is that the type of group
can vary within a single composition tree. The
second package tries to collects the best known methods for group
recognition for permutation groups, matrix groups and projective groups.
Thus, everybody who contributes implementations will be an author of the
\textsf{recog} package. 

This book describes a major part of the algorithms used in both
packages. For some methods, mainly for the leaves of the composition tree,
the reader is however referred to the literature.
\index{leaf}%

Since the matrix group project is not yet finished, the present work
cannot give a complete description of a solution to the constructive
recognition problem. In particular for the leaves of the composition tree
a lot of work and improvement of algorithms still has to be done, as is
explained in some detail in Chapter~\ref{chap:leaves}.
\index{leaf}%

We have intentionally left out a number of topics which are related, mostly
because research on them is not completed or not even in a state for a
satisfactory description. One is
the whole field of algorithms for matrix groups that build on constructive
recognition. Derek Holt and Mark Stather have recently published a paper
\cite{HoltStather} pointing in that direction and Mark Stather's PhD thesis
\cite{StatherPhd} and \cite{StatherSylow} also lie in this area. 
Another area left out is the
constructive recognition of black box groups, because we wanted to
concentrate on matrix and projective groups. We only cover the
verification phase very briefly, which is however necessary to eventually
have computational proofs of the results. The reason for this is that in
fact very little work has been done on the actual implementation of
verification routines. What is needed for this is basically good
presentations for the groups occurring in the leaves. 
\index{leaf}%
Finally a global analysis of the
whole procedure of building a composition tree is still needed. This seems
achievable using estimates on the length of a composition series but is not
yet done.

Now we turn to the things that are contained in this book and
outline its structure. 

Chapter~\ref{chap:intro} introduces briefly some concepts from computer
science which apply to computational group theory and describes a
way to produce nearly uniformly distributed random elements in a
finite group. Chapter~\ref{chap:cmats} describes a new method to
implement matrices over finite fields on a computer. Having an efficient
implementation of the finite field arithmetic and linear algebra
routines is obviously an indispensable foundation for implementing
matrix group algorithms. The contents of this chapter are not yet
published elsewhere.

Chapter~\ref{chap:charminpoly} contains a new randomised method to
compute the minimal polynomial of a square matrix over a finite field.
It is basically a copy of the paper \cite{minpolypaper} jointly
written by Cheryl Praeger and the author. Computing minimal
polynomials of invertible matrices is an important ingredient to compute the
order and projective order of such matrices, which are computations
that are used throughout nearly all matrix group algorithms.

Chapter~\ref{chap:linalg} completes the description of the
infrastructure for implementing matrix group algorithms by explaining
how to compute the order and projective order of a matrix and how to
perform higher level linear algebra computations like solving systems
of linear equations and inverting matrices. In addition the two
major obstacles for polynomial time algorithms, the discrete logarithm
problem and integer factorisation, are introduced. This chapter is
nothing new but is needed for the sake of completeness.

In Chapter~\ref{chap:comptree} we explain the basic problem attacked
by the matrix group recognition project, namely the constructive
recognition problem. We give a gentle introduction starting with a
rough formulation of the problem followed by two refinements. Then the
fundamental approach using composition trees and a generic framework
for group recognition are explained. We explain in detail the idea and
purpose of reduction homomorphisms. The
contents of this chapter are a variation and extension of the paper
\cite{AkosMaxISSAC} by \'Akos Seress and the author, in which the
composition tree approach is refined by allowing for a change in the generating
set of the group to be recognised. This refinement can dramatically
increase the performance because the resulting straight line programs
are much shorter than in the traditional version. Finally, the chapter
closes with a description how the currently best known algorithms for
the constructive recognition of permutation groups fit in nicely with
the proposed framework.

The next Chapter~\ref{chap:findhom} explains a variant of Aschbacher's
theorem on subgroups of classical groups restricted to the general linear
group case. A relatively short complete proof is given.
Our variant changes the definition of the occurring
classes of subgroups slightly to make it easier to devise algorithms
for finding reduction homomorphisms for groups in some of these classes.
This approach already seems to bear fruit in the last part of the
chapter where we present an overview of algorithms to find reduction
homomorphisms for groups in the different classes. For the new classes
\DD2, \DD4 and \DD7 new ideas to tackle groups in these classes are
given as well as references to the literature for the currently best
known methods.

Chapter~\ref{chap:subsemi} is a copy of the preprint \cite{subfieldpaper}
which is joint work of the author with Jon Carlson and Colva
Roney-Dougal. It presents new completely analysed randomised algorithms
to find a reduction for the case that $G \le \GL(n,q)$ acts irreducibly
on its natural module and lies in at least one of the semilinear or
subfield Aschbacher classes \CC3 and \CC5.

Chapter~\ref{chap:leaves} finally tries to summarise the state of the
art for algorithms to do constructive recognition for the leaves of
the composition tree, that is for groups in the Aschbacher classes
\CC8 and \CC9. We do not try to explain any of the best known
methods there because the final word on them seems not to be spoken
as of this writing. Rather, we explain the concepts of non-constructive
recognition and standard generators and give references to the
literature.

\section{Acknowledgements}

First and foremost I would like to thank Gerhard Hi\ss{} for a multitude
of things. He was not only my PhD supervisor but also gave me the
opportunity to be employed at the RWTH Aachen for 10 years in research
and teaching positions. He holds the chair of Lehrstuhl D f\"ur
Mathematik and thereby provided a superb environment for me to develop
from being a student of mathematics to be a professional mathematician.
I greatly enjoyed my time in Aachen and Gerhard Hi\ss{} has always been a
great advisor and good friend. I also thank the RWTH Aachen and in
particular all the people at Lehrstuhl D for the wonderful time I had
in Aachen.

I thank Alice Niemeyer and Cheryl Praeger for inviting me
repeatedly to the University of Western Australia in Perth for
research visits. During one of these visits I met \'Akos Seress, whom
I thank for motivating me to start working in the area of
computational matrix group theory, for successfully collaborating with
me over several years and for inviting me repeatedly to Columbus, Ohio
for research visits.

Furthermore I thank my coauthors Jon Carlson, Cheryl Praeger, 
Colva Roney-Dougal and \'Akos Seress for the interesting and fruitful
collaboration on a major part of the work that went into the present book.

I thank the University of St Andrews and in particular Steve Linton
for employing me since 2007 and giving me the opportunity to complete
the work on this book as part of my research job.

A big thank-you also goes to Colva Roney-Dougal who helped a lot in
proofreading and who agreed to constantly correct my English.

I wholeheartedly thank my parents for their permanent support and
for allowing me to have the long mathematics education I had, which
gave me the opportunity to get into my dream job.

Last but not least I would like to thank my wonderful wife Anja for
her constant support and encouragement during the completion of this
book.

\renewcommand{\thechapter}{\Roman{chapter}}


\tableofcontents
\thispagestyle{fancy}

% this is a part of the habilitation thesis of Max Neunhoeffer

\chapter{Introduction}
\label{chap:intro}

This chapter covers a few fundamental concepts and algorithms which
will be used in the rest of the book. We start talking about the
complexity of algorithms, go on with the concept of straight line
programs in groups and finish by mentioning the idea of randomisation
in algorithms and a way to produce nearly uniformly distributed
elements in a finite group.

\section{Some notes on complexity theory}
\label{sec:complexity}

\index{complexity theory}%
Already in this chapter, but all the more in the rest of the book, we
talk about the analysis of algorithms. In this section we want to
discuss briefly what we mean by this at all.

An algorithm is usually designed to solve a whole family of problems of
different sizes. Obviously, for a single, particular instance of a
computational problem one can simply store the answer and look it up
when needed in nearly no time. But this is usually not what we intend
to do when we develop an algorithm. Furthermore, it is clear that a
small instance of a computational problem usually will need less time
to solve than a big instance of the same problem. It takes for example longer 
to multiply two $10000\times 10000$-matrices with entries in the finite 
field $\F_q$ than to multiply two $100\times 100$-matrices
with entries in the same finite field $\F_q$,
even if these two situations are instances of the same
computational problem ``matrix multiplication over $\F_q$''.

Therefore, when we talk about analysing an algorithm that solves a
certain family of computational problems, we assign each instance of
this problem a ``size'' and ask how the runtime of the algorithm grows
in comparison to the size of different instances. This is in vague terms
what is meant by ``complexity of an algorithm'', and ``complexity
\index{complexity}%
theory'' is the study of the complexity of algorithms. The size of an
instance of the above mentioned matrix multiplication problem of two $n
\times n$-matrices could for example be measured by the value $n$.

Of course, different computers are running at different speeds, so
what we usually do is to count the number of steps in the algorithm
needed to complete a particular instance of the computational problem
as a function of the size of the problem instance.
We try to keep the different steps comparable. We might for example
count the number of elementary field operations (addition,
\index{elementary field operation}%
subtraction, multiplication, inversion) during the execution of a
matrix multiplication. The standard simple-minded approach to matrix
multiplication would then need $(2n-1)\cdot n^2$ elementary field
operations since each entry of the result is a sum of $n$ products of
numbers in $\F_q$, that is, we need $n$ products and $n-1$ additions
for each of the $n^2$ matrix entries. The complexity of this algorithm
in terms of the size $n$ of the input would then be $(2n-1)\cdot n^2$
elementary field operations.

If the growth rate of the complexity of an algorithm A is slower
\index{growth rate}%
than the one for another algorithm $B$ solving the same problem, then 
we would consider A to be ``better'' than B. Note that it is
possible that the complexity of B is in fact smaller than
the one of A for small problem sizes. In that case we might want to
implement both algorithms and use B for smaller instances and A for
bigger instances. In this way complexity theory helps to come up with
better implementations. If we had for example an algorithm B to multiply
two $n\times n$-matrices over $\F_q$ using $n^4/20$ elementary field
operations, it would be worthwhile to use it as opposed to the
standard algorithm A for small matrices, since
$n^4/20 \le (2n-1)\cdot n^2$ for $n \le 39$. However, for big
matrices A would outperform B dramatically.

Counting the steps in an algorithm can be very tedious and much more
difficult if there are decisions and case distinctions during the
execution. Randomisation as described in Section~\ref{montevegas}
makes this even more difficult. Therefore we are usually happy to come
up with an upper bound for the number of steps necessary. In addition,
when we are only interested in the growth rate of the complexity, we
\index{growth rate}%
are only interested in the type of function expressing this upper
bound in terms of the problem size and are actually not interested in
the constants. 

So, to determine that the growth rate of the complexity
\index{growth rate}%
of the example algorithm A above is smaller than the one of B, we do
not need the exact number of steps $(2n-1)\cdot n^2$ and $n^4/20$
respectively, but we only need to know that the first behaves ``like
$n^3$'' and the second ``like $n^4$''. The higher exponent $4$ in the
second function eventually beats the smaller constant $1/20$.
Note however that without knowing the constants we would in fact not
know that the break-even point of where we should start using A in
favour of B is at $n=39$. Having noticed this, we would like to comment
that knowing the constants in the complexity of algorithms can be very
interesting for coming up with good implementations.

Different parts of this book go differently about this. Whenever
possible we have tried to include constants in the complexity
estimates. However, sometimes this would have been too tedious and
would have complicated things dramatically. In these cases we are
content with determining the growth rate. For this purpose, we adopt
\index{growth rate}%
the standard big-$O$ notations, which we will briefly repeat here.
It is used to formulate statements about the asymptotic behaviour of
functions.

\begin{Def}[Capital-$O$-notation]
    \label{capO}\index{O-notation@$O(-)$-notation}%
    Let $\R^+$ be the set of positive real numbers and
    $g : \R^+ \to \R$.
    We say that a function $f : \R^+ \to \R$ is $O(g)$ if there are
    two positive real constants $C$ and $D$ such that 
    $|f(x)| \le C \cdot |g(x)|$ for all $x > D$.
\end{Def}

This will be used in the analysis of algorithms by saying for
example: Algorithm A above has complexity $O(n^3)$ whereas algorithm B
\index{complexity}%
above has complexity $O(n^4)$. Obviously it has to be clear from the
context, which family of computational problems the algorithm deals
with and how the size $n$ of an instance of the problem is measured.

There is a certain sloppiness in this usage of Definition~\ref{capO}.
Strictly speaking the following statement is true as well: 
``The function $n \mapsto (2n-1)\cdot n^2$ is $O(n^5)$.'' This follows
from the fact that the function $n \mapsto n^3$ is $O(n^5)$. However,
this statement is not as strong as possible. Obviously, we are
interested in the fact that the exponent $3$ is the \emph{least
possible $e$} for which the statement ``the complexity of algorithm A is
$O(n^e)$'' is true. Sometimes we can prove that our statement is best
possible and sometimes we cannot.

Up to now we have concentrated on the ``time complexity'' of
\index{time complexity}%
algorithms, that is, the asymptotic behaviour of the runtime with
growing problem sizes. Although this is usually the most interesting
aspect, it is of course also necessary to keep an eye on other
computational resources like memory usage. We speak of ``space
complexity'' in this case and use a similar setup and language. This
\index{space complexity}%
only plays a minor role in this book.

The point of view we adopt in this book is that the complexity analysis of
algorithms is a tool to come up with good implementations. Complexity
results tell us, which algorithms are suited best for which problem
classes and possibly which problem sizes, and they tell us, what is
considered to be a ``satisfactory'' algorithm for a problem class and
what is considered to be ``lacking''. In general we would like to
devise algorithms which have a polynomial as an asymptotic bound of their
time complexity in terms of the problem size. We call these algorithms
``polynomial-time algorithms''.


\section{Straight line programs}
\label{slp}

\index{straight line program}\index{SLP}%
Let $G$ be a group that is given by a tuple of generators $(g_1,
\ldots, g_k)$. That is, every element in $G$ can be expressed as a
finite product of powers of the $g_i$ and their inverses. However,
since $G$ is not necessarily abelian, the generators may occur more
than once and the products can be quite long. We call such a product a 
\emph{word in the generators $(g_1, \ldots, g_k)$}. 
\index{word in generators}%
For finite groups we do not need the
inverses of the generators since they all have finite order and can
thus be expressed by positive powers.

Quite often in applications we want to encode rather long words in
generators on a computer. One reason for this is that we want to store certain
elements in a known group in terms of a generating tuple (see in
particular Section~\ref{standardgens}). Another reason is for example that to
evaluate a group homomorphism $\varphi:G \to H$ on arbitrary elements
$g \in G$, if we only know the images $\varphi(g_i)$ of a tuple of
generators for $G$, we need to express $g$ explicitly in terms of
the generators $(g_1, \ldots, g_k)$. Finally, the problem of constructive
recognition (see Problem~\ref{ProbCR3}) involves sometimes rather
\index{constructive recognition}%
long words in a tuple of generators, too.

To store and evaluate such words efficiently is the purpose of
\emph{straight line programs (SLP)}.

In general, straight line programs are programs that have no branches
or loops, their execution follows a ``straight line'' under all
circumstances. For the purpose of storing and evaluating words in
generators in a group, we further restrict this to the following:

\begin{Def}[Straight Line Program (SLP)]
    \label{defslp}\index{straight line program}\index{SLP}%
    \index{straight line program!definition of}\index{SLP!definition of}%
    A \emph{straight line program} is a procedure that takes as input a
    $k$-tuple $(g_1, \ldots, g_k)$ of group elements in a 
    common group and consists of a finite 
    sequence of steps, which are each one of the following:
\begin{itemize}
    \item Compute the product of two stored elements, or
    \item compute the inverse of a stored group element.
\end{itemize}
    An SLP starts with the elements $g_1, \ldots, g_k$,
    stores all intermediate results and returns one or more of the 
    group elements collected during its execution. The number of
    steps in the SLP is called its \emph{length}.
\end{Def}

\begin{Rem}
A straight line program of length $l$ can encode words in its input
of length up to $2^l$. Obviously, it cannot encode all words of that
length.
\end{Rem}

Implementations and data structures for straight line programs are
available in the {\GAP} (see \cite{GAP4}) and {\MAGMA} (see
\cite{Magma})
computer algebra systems. The
WWW-Atlas of group representations uses straight line programs to
\index{WWW-Atlas of group representations}%
store generators for maximal subgroups of groups. Note that the
current implementations of straight line programs in these systems
provide an even more compact storage by allowing arbitrary finite
products of powers of previously stored group elements in each step.


\section{Randomised algorithms}
\label{montevegas}
\index{randomised algorithm}\index{algorithm!randomised}%
\enlargethispage{1\baselineskip}

Traditionally, an algorithm is a completely deterministic procedure to
achieve a certain goal. Whenever it is executed, it performs the same
steps and thus behaves in the same way when called twice with the same input.

However, there is a certain limitation in this paradigm. In particular
in situations, in which we want to find some result that can later
be verified to be correct easily, randomised methods excel. By
randomised
methods we mean algorithms that involve a certain amount of random
choices. That is, the sequence of steps performed by a randomised
algorithm depends on certain random choices done during the algorithm.
Of course, in practice we will usually use pseudo random numbers to
do these random choices.

We do not want to go into too much detail here, but there are many
examples in which randomised algorithms can greatly outperform
deterministic algorithms. However, how do we measure or analyse
the performance of a randomised algorithm, given that it does
different things on different calls with the same input, and thus
has different runtimes on different occasions?

One possibility for performance analysis is to look at the average
or the expected runtime. Although this is a good and
interesting thing to look at, this type of analysis often stays a
bit unsatisfactory, since one never knows, how long the algorithm will
take at most in a particular instance.

Therefore the most common approach for randomised algorithms is to do a 
worst-case analysis. However, clearly the absolutely worst case is
that by incredible bad luck all random choices turn out to be wrong
and the algorithm does not succeed even after a very long time. To get
rid of this problem we have to allow our algorithms to fail in some
way, most commonly simply by giving up with \textsc{Fail} as answer
after a certain time.
Using this exit route, we can devise algorithms that are
guaranteed to terminate after a certain number of steps or a given
amount of time. To be useful in practice, we of course want to have a
bound for the probability with which this failure occurs. Optimally, we
want to prescribe an upper bound for the failure probability
beforehand. The guaranteed upper bound for the runtime then might
depend on the choice of the failure probability bound.

In general we distinguish between so-called ``Monte Carlo'' and
``Las Vegas'' algorithms as defined in the following.

\begin{Def}[Monte Carlo algorithm]
\index{randomised algorithm}\index{algorithm!randomised}%
\index{Monte Carlo algorithm}%
    A \emph{Monte Carlo algorithm with failure probability $\epsilon$}
    is a randomised algorithm that is guaranteed to terminate after
    a finite amount of time with some result, if the probability for
    returning a wrong result is bounded by $\epsilon$.
\end{Def}

A bit more satisfying is the following.

\begin{Def}[Las Vegas algorithm]
\index{randomised algorithm}\index{algorithm!randomised}%
\index{Las Vegas algorithm}%
    A \emph{Las Vegas algorithm with failure probability $\epsilon$}
    is a randomised algorithm that is guaranteed to terminate after
    a finite amount of time with either the correct result or
    \textsc{Fail} indicating failure, if the probability for failure
    is bounded by $\epsilon$.
\end{Def}

The two concepts are sometimes related by the following.

\begin{Rem}[Upgrading Monte Carlo to Las Vegas by verification]
\index{verification}%

Assume that there is an efficient way to verify the correctness of the output
of a Monte Carlo algorithm. Then we can immediately upgrade the
algorithm to be of Las Vegas type by following it with a verification
step that returns \textsc{Fail}, if the result was incorrect in the first
place. ``Efficient'' here means that the verification does not take
much longer than the Monte Carlo computation in the first place.
\end{Rem}

We will use the terms ``Monte Carlo algorithm'' and ``Las Vegas
algorithm'' in this sense throughout this book.

\section{Random elements in finite groups}
\label{randomelts}

Randomised algorithms in group theory need random elements in groups.
Moreover, to allow a proper analysis of the behaviour of such
algorithms one needs to know quite a lot about the distribution of
the random elements in the group. Usually, the best with respect to
analysis is to have a source of uniformly distributed random elements.

For most applications we are content with pseudo randomness, that is, with a
deterministic procedure which produces from some initial seed a sequence of 
elements with a good uniform distribution. Choosing different seeds (or
maybe actually choosing the seed at random) then leads to a different
behaviour of the algorithm in each call.

There are well-known methods to produce good uniformly distributed
pseudo random numbers (see for example \cite[Chapter~3]{AOCP2}).
Building on these, there is a method to produce pseudo random elements
in a finite group given by generators, which works astonishingly
well in the sense that the distribution of the elements is very close
to uniform. In the sequel we describe this method briefly but refer
for proofs to the literature. After this we discuss some of the
advantages and limitations of this method.

\begin{algorithm}
\caption{$\quad$ \sc RattleStep}
\label{rattlestep}
\index{Rattle@\textsc{Rattle}}%
\begin{algorithmic}
\STATE \textbf{Input:} A pair $(a,(h_1, \ldots, h_n))$ with $a \in G$
and $G = \left< h_1, \ldots, h_n \right>$.
\STATE \textbf{Output:} A modified pair $(a,(h_1, \ldots, h_n))$ with $a \in G$
and $G = \left< \smash{h_1, \ldots, h_n} \right>$.
\vspace*{2mm}
\STATE $i := \textsc{Random}(\{ 1,2,\ldots,n\}$
\STATE $j := \textsc{Random}(\{1,2,\ldots,n-1\}$
\STATE $b := \textsc{Random}(\{1,2\}$
\IF {$j = i$}
    \STATE $j := j + 1$
\ENDIF
\IF {$b = 1$}
    \STATE $h_i := h_i \cdot h_j$
\ELSE
    \STATE $h_i := h_j \cdot h_i$
\ENDIF
\STATE $a := a \cdot h_i$
\STATE \textbf{Return} modified $(a,(h_1,\ldots,h_n))$
\end{algorithmic}
\end{algorithm}

\begin{Def}[Random elements with \textsc{Rattle}]
    \label{rattle}
\index{Rattle@\textsc{Rattle}}%
Let $G$ be a group given by a tuple $(g_1, \ldots, g_k)$ of generators
and $n, N \in \N$ with $n \ge k$. The \textsc{Rattle} method to
produce random elements in $G$ is the following procedure:

It uses a variable $(a,(h_1,\ldots,h_n)) \in G \times G^n$
which is changed during the runtime by calls to
Algorithm~\ref{rattlestep}.

It first initialises $(a,(h_1,\ldots,h_n))$
by $a := \mathbf{1}_G$ and $h_i := g_i$ for $1 \le i \le k$ and $h_i
:= \mathbf{1}_G$ for $k < i \le n$.

Then it calls Algorithm~\ref{rattlestep} $N$
times with $(a,(h_1, \ldots, h_n))$ as argument
thereby changing it and
finally returns the last value of $a$ as a random element $a_0$ in $G$.

After this initialisation phase it produces a sequence of random
elements $(a_i)_{i \in \N}$ by calling 
Algorithm~\ref{rattlestep} repeatedly with 
$(a,(h_1, \ldots, h_n))$ as argument thereby changing it and
assigning the value of $a$ to $a_i$ after call number $i$.
\end{Def}

\begin{Rem}[Variant of product replacement]
\index{Rattle@\textsc{Rattle}}%
The \textsc{Rattle} method described above is a variant of the
``product replacement algorithm'', because the main part of a step replaces 
an element by a product of it with another one.
\end{Rem}

\begin{Rem}[Comments on the implementation of \textsc{Rattle}]
\index{Rattle@\textsc{Rattle}}%
It is not completely clear how to choose the parameters $n$ and $N$. In
principal $n$ could be chosen equal to $k$. However, what a good length
$N$ of the initialisation phase is depends on the group $G$ and
on the generating tuple $(g_1, \ldots, g_k)$. In practice one chooses $n$
slightly bigger than $k$ and $N$ around $100$ unless $k$ is very big.
For large $k$ the value of $N$ has to be chosen bigger. It is clear
that a constant value for $N$ will not work well for $|G| \to \infty$.
\end{Rem}

\begin{Prop}[\textsc{Rattle} converges to the uniform distribution]
    \label{proprattle}
\index{Rattle@\textsc{Rattle}}%
    The distribution of the element $a_0$ in the \textsc{Rattle} procedure (see
    Definition~\ref{rattle}) converges for $N \to \infty$ to the
    uniform distribution.
\end{Prop}
\proofbeg
See \cite[Section~4]{LGMurray}.
\proofend

\subsubsection{A brief discussion of \textsc{Rattle}}

\index{Rattle@\textsc{Rattle}}%
Although it can be proved that for every finite group $G$ and every
generating tuple $(g_1, \ldots, g_k)$ the distribution of the element
$a_0$ for $N \to \infty$ tends to the uniform distribution (see
Proposition~\ref{proprattle}), there is
not much known about the rate of convergence. So, picking the value for
$N$ in practice is difficult, in particular if we do not know anything
about $G$.

If we use the sequence produced by the \textsc{Rattle} method
after initialisation as a sequence of random elements in $G$, 
subsequent elements seem to be uniformly distributed, but adjacent
elements in the sequence are clearly not distributed independently.
Obviously the next element in the sequence depends heavily on the
previous state $(a,(h_1,\ldots,h_n))$ and the previous element is
equal to the $a$ value of this state. However, the state contains of
course more information than simply the value $a$.

Despite these obvious deficiencies, the algorithm
works surprisingly well in practice. The computational cost for the 
initialisation is 200 multiplications and after that 2 more
multiplications for every further element in the sequence.
The memory requirements are minimal and the produced sequence of
random elements is good enough for most purposes. Analysing algorithms
that use \textsc{Rattle}
\index{Rattle@\textsc{Rattle}}%
with the assumption that it produces uniformly distributed random
elements in the group provides good predictions on how
well these algorithms work in practice.

Throughout this book the \textsc{Rattle} method is used to produce
random elements in group and the above assumption is made.
\index{Rattle@\textsc{Rattle}}%


% this is a part of the habilitation thesis of Max Neunhoeffer

\chapter{Matrices over finite fields}

This chapter covers the implementation of the basic operations for matrices 
over finite fields. We begin with a description of a new concrete 
representation of such matrices on nowadays computers in 
Section~\ref{sec:ffematrices}, both in main memory and on storage.
We then analyse the performance and complexity of matrix arithmetic 
for this new representation in Section~\ref{sec:matarith} and compare
it to previous implementations. In Section~\ref{sec:basalgmat}, we give
an overview over the most basic algorithms for finite field matrices,
before we conclude this chapter with the description of a new method
to compute minimal polynomials of matrices over finite fields in
Section~\ref{sec:charminpoly}, which is used to compute projective orders
of matrices.

%FIXME

\section{Representing vectors and matrices over finite fields}
\label{sec:ffematrices}

If you already know something about compact vectors and matrices over 
finite fields you can safely skip the next subsection and directly
proceed to Section~\ref{ssec:cvec}. We first explain the basic idea.

\subsection{The idea}

If $p$ is a prime then elements of the finite field $\F_p$ with $p$
elements can be represented by the integers $0, 1, \ldots, p-1$. Thus,
storing one such element on a computer needs only $\lceil \log_2(p)
\rceil$ bits. The finite extension field  $\F_q$ with $q = p^k$ elements 
is built
as quotient $\F_p[x]/c_k \F_p[x]$ using the Conway polynomial
$c_k$ (see \cite{Nickel} or on the web at \cite{ConwayFL}). Since the
Conway polynomials are monic by definition, an element
of\, $\F_q$ can be represented by one polynomial over\, $\F_p$ of degree 
less then $k$ and thus by storing $k$ elements of\, $\F_p$ using $\lceil
\log_2(p^k) \rceil$ bits.

Since linear algebra operations over finite fields can be performed
rather efficiently by modern microprocessors, the limiting factor 
for such computations is memory access. Therefore, it is performance
critical to store vectors and matrices using as little memory as possible,
and to choose the data structure in a way that allows for fast memory
access. We call such memory efficient data structures ``compact''.

This idea is quite old (see ADDREFERENCE) and has already been used
extensively (see for example \cite{CMeatAxe} or \cite{GAP4} or
various other systems).

There is a fundamental difference between the characteristic $p=2$ case
and all other characteristics. The reason for this is that in
characteristic $2$ the addition of vectors can be implemented by using
the XOR (exclusive or) operation available in every microprocessor
instruction set. In odd characteristic, the available instructions
do not fit so well to finite field arithmetic. Therefore, previous
implementations mostly rely on table lookups to perform arithmetic
operations. Since the memory space available for lookup tables is
limited, this method is limited to relatively small fields (usually up to
fields with $256$ elements) and has to use single byte accesses
as opposed to processor word accesses, for which modern machines
seem to be optimised. These limitations seem to become more serious
as word lengths in standard microprocessors increase.

The main novelty in the approach presented here is to overcome this
problem by choosing the data structures in a way that allows to use
processor word operations for all arithmetic operations in all
characteristics. Also this idea has been used before (see
\cite{EssenLinAlg}), however, no implementation of this seems to
be available any more and, more importantly, the approach still
insisted on storing whole elements of\, $\F_q$ in one machine word.
We will pack as many prime field elements as possible into a machine
word and distribute the various prime field coefficients of a single
element of\, $\F_q$ into distinct machine words to allow for better
memory usage in the case of extension fields.


\subsection{Compact vectors}
\label{ssec:cvec}

We first describe in detail how a vector of length $n$ over the field
$\F_q$ with $q=p^k$ elements is stored on a machine with word length $32$ 
or $64$ bits respectively. The ideas for the step from $32$ to $64$ bits
can be applied analogously for future microprocessors with even larger
word length. We ignore architectures with funny word lengths not being
a multiple of $32$ bits.

Let $e$ be $\lceil \log_2(2p-1)\rceil$ for $p > 2$ and $1$ for $p=2$. 
This is the number of bits
necessary to store an integer in the range $0,1, \ldots, 2p-2$ except
for $p=2$, where it is $1$. This number $e$ is the number of bits we 
reserve for every prime field element we want to store. For $p=2$ this
is evidently best possible, whereas for odd $p$, we seem to waste
one bit per prime field element, since we seem to need only store
numbers in the range $0,1,\ldots p-1$. This additional bit in the
data structure allows us to represent a sum of two numbers in the
range $0,1,\ldots, p-1$ using $e$ bits in odd characteristic.

We start with the prime field case $q=p$.

We pack as many prime field elements as possible into a machine
word, that is, $w := \lfloor 32/e \rfloor$ prime field elements per word
on a machine with word length $32$ bit. On machines with $64$ bits
we pack $w := 2 \lfloor 32/e \rfloor$ prime field elements into one word.
Note that we do not use $\lfloor 64/e \rfloor$ elements which can be one
more, since then the conversion between the different data formats for
different word lengths becomes too expensive and awkward.

We always imagine the least significant bit in a machine word as being 
on the right hand side. To store a vector, we start filling machine words 
from right to left, always using $e$ bits for one prime field element and
packing $w$ prime field elements into a word.

We illustrate this layout in the following example for $p=5$ and thus
$e=4$ on a machine with $32$ bit word length:

\begin{verbatim}
  bit             3322|2222|2222|1111|1111|1100|0000|0000
  number:         1098|7654|3210|9876|5432|1098|7654|3210
  prime field         |    |    |    |    |    |    |
  element number: 7777|6666|5555|4444|3333|2222|1111|0000
\end{verbatim}

Within each block of $e$ bits, we represent prime field elements
by the binary representation of a number in the range $0,1,\ldots,p-1$
with the least significant bit of this binary representation on the
right hand side. Thus, in our example $1+1+1+1 \in \F_5$ would be
represented by the bit sequence \texttt{0100} and $1+1+1$ by \texttt{0011}.
Note that the most significant bit in this representation is always $0$.

The first $w$ prime field elements in a vector are stored in the first
machine word of the vector, the next $w$ in the next word and so on.

We proceed now to the extension field case $q=p^k$. Let $e$ and $w$ be
exactly the same values as above. We now have to store $k$ prime field
elements for every element of $\F_q = \F_p[x]/c_k\F_p[x]$, namely the 
coefficients of the unique residue class representative of degree smaller 
than $k$, where $c_k$ is the Conway polynomial used to construct the 
finite field extension $\F_q$ over $\F_p$ (for details see \cite{Nickel}
or on the web \cite{ConwayFL}). Namely, if $a \in \F_q$ is represented by
the polynomial $\sum_{i=0}^{k-1} a_i x^i$, then we have to store the prime 
field elements $a_0, a_1, \ldots, a_{k-1}$. 

In a compact vector of length $n$ of elements of $\F_q$ we distribute
those numbers in the following way: The first $k$ machine words in the
memory representation of the vector are used to store the first $w$
elements of the vector. The first machine word holds all the coefficients
$a_0$ of those $\F_q$ elements, the second the coefficients $a_1$ and so
on. Since one machine word can hold up to $w$ prime field elements this
fills the first $k$ words rather satisfactorily. The second $k$ machine words
in the vector then hold the vector elements with indices 
$w+1, w+2, \ldots, 2w$ in exactly the same way.

Note again that for machines with a word length of $64$ bit we choose the
value $w$ only twice as big as for $32$ bit machines even if one more
prime field element would fit into the machine word.

The only natural limit of this implementation is that 
$\lceil \log_2(2p-1) \rceil$ must be smaller or equal to the word length
of the microprocessor.

We can now explain, how we can add vectors in the above representation
only by doing a series of word operations.

\subsection{Adding compact vectors}

The basic addition formula for finite field elements is rather simple:
For prime field elements we just have to add two numbers in the range 
from $0$ to $p-1$, and
subtract $p$, if the sum is greater or equal to $p$. The extension
field elements are done component-wise. However, we have to solve
the problem of doing this simple operation using word operations and 
thus doing this for $w$ prime field elements at the same time.
This is easy for $p=2$, since we can use the standard XOR (exclusive or)
operation.

The idea to overcome this problem for $p > 2$ is the following. 
Assume $a$ and $b$
are two integers in the range from $0$ to $p-1$. By adding
$2^{e-1}-p$ to the sum $a+b$ we get $t := a+b+2^{e-1}-p$, which
has the property, that $t \ge 2^{e-1}$ if and only if $a+b \ge p$
(remember $2p-1 \le 2^e$ and thus $p-1 < 2^{e-1}$). That is, if
the number $t$ is represented using binary expansion with $e$ bits,
then the most significant bit is set if and only if $a+b \ge p$.

This idea is now used for two words $a$ and $b$, containing $w$
prime field elements each. Every prime field element uses exactly $e$ bits
in its word and we call these sections of $e$ adjacent bits in a word 
``components'' for the moment. 

We prepare an ``offset'' word $o$ that contains in each component the
number $2^{e-1}-p$ and a ``mask'' word $m$ that contains in each component
the number $2^{e-1}$ meaning, that in each component only the most significant
bit is set and all others are zero. In addition, we keep a word $n$
containing the number $p$ in each component.

The finite field sum $c$ of $a$ and $b$ is now computed in the following way:
First $s := a+b$ and $t := a+b+o$ are computed. We then use an AND 
operation for words to extract exactly the most significant bits of
those components, in which the sum was greater or equal to $p$.
This is done by computing $r := t \ \&\  m$ (we use the \& symbol to
indicate bit-wise AND operations). Bit-shifting
the word $r$ by $e-1$ bits to the right (we use the notation
$r \gg (e-1)$ for this) and subtracting the
result from $r$ now results in a word $u := r - (r \gg (e-1))$
having the number $2^{e-1}-1$
in those components, in which the sum was greater or equal to $p$ and
$0$ in the others. Finally, doing a bitwise AND operation of $u$ with
$n$ results in exactly the right word to subtract from $s$ to get
the correct result.

Thus, the complete formula is
\[ a+b - \Big(\big(r - (r \gg (e-1))\big) \ \&\ n \Big)
   \qquad \mbox{where}\quad r = (a+b+o) \ \&\  m \]

We illustrate this by an example for $p=3$ on a machine with
$32$ bit word length. In this case, $e = 3$ and $w = 10$. We want to
add the words shown below in the rows depicted by \texttt{a} and \texttt{b}.
We also show the prepared words $o$, $m$, and $n$. The bits marked
with \texttt{X} are not used and are all equal to zero.

\begin{verbatim}
  bit             33|222|222|222|211|111|111|110|000|000|000
  number:         10|987|654|321|098|765|432|109|876|543|210
                    |   |   |   |   |   |   |   |   |   |
  a:              XX|000|010|001|000|010|001|000|010|001|000
  b:              XX|000|010|010|010|001|001|001|000|000|000
  o:              XX|001|001|001|001|001|001|001|001|001|001
  m:              XX|100|100|100|100|100|100|100|100|100|100
  n:              XX|011|011|011|011|011|011|011|011|011|011
\end{verbatim}

In the following table we show some intermediate results and repeat
the input values for easier verification:

\begin{verbatim}
  a+b:            XX|000|100|011|010|011|010|001|010|001|000
  a+b+o:          XX|000|101|100|011|100|011|010|011|010|001
  r:              XX|000|100|100|000|100|000|000|000|000|000
  u:              XX|000|011|011|000|011|000|000|000|000|000
  u&n:            XX|000|011|011|000|011|000|000|000|000|000
  result:         XX|000|001|000|010|000|010|001|010|001|000
  a:              XX|000|010|001|000|010|001|000|010|001|000
  b:              XX|000|010|010|010|001|001|001|000|000|000
\end{verbatim}

Of course, it is a coincidence here that $u$ is equal to 
$u \ \&\ n$, since $p = 2^{e-1}-1$.

This means, that the addition of two words containing $w$ prime field
elements can be done in $7$ word operations for $p > 2$. For the
$p=2$ case, we only need one XOR operation. Thus we have proved:

\begin{Prop}[Addition of compact vectors]
\label{addvec}
We assume a machine with $32$ bit word length. For $2 < p < 2^{31}$,
two compact vectors of length $n$ over the field of $q=p^k$
elements can be added using $7k\cdot \lceil n/w \rceil$ word operations
(plus memory fetches and stores), where $w = \lfloor 32/e \rfloor$
and $e = \lceil \log_2(2p-1) \rceil$. 

For $p=2$, only $k \cdot \lceil n/32 \rceil$ word operations are needed.

For machines with $64$ bit word length, the number of word operations
is halved in both cases and we can work with primes $p < 2^{63}$.
\end{Prop}
\Proof See above. \ProofEnd

\begin{Rem}
Note that since the amount of memory needed for a vector of length $n$
is $k \cdot \lceil n/w \rceil$ for $p > 2$ and $\lceil n/32 \rceil$ for
$p=2$ this means that the addition needs $7$ word operations for each
word of a vector for $p>2$ and $1$ word operation for $p=2$.

Assuming a microprocessor
with a long enough instruction pipeline, we can conclude that all this
can be done as fast as accessing the main memory to fetch $a$ and $b$ and
store the result somewhere, such that the number $7$ does not hurt at all. 
Compare Section~\ref{ssec:discussion} and see Section~\ref{sec:cache}
for some additional comments on processor caches.
\end{Rem}

Next we consider multiplication of vectors by scalars.

\subsection{Multiplication by scalars}

To explain the method, we restrict our attention to the case that one
compact vector shall be multiplied by one scalar and the result 
shall be stored in some other memory location.

Let $e$ and $w$ be defined as in Section~\ref{ssec:cvec}.

We start by discussing the prime field case $\F_p$.
Since a scalar $s \in \F_p$ is represented by an integer in the range $0$ 
to $p-1$ in its binary expansion, we can multiply a compact vector $v$
by $s$ by repeatedly adding vectors to themselves starting with $v$ and
adding up those multiples
whose corresponding bits in the binary expansion of $s$ are set. That is,
if $s = \sum_{i=0}^{e-2} s_i 2^i$ with $s_i \in \{0,1\}$ and again
$e = \lceil \log_2(2p-1) \rceil$, we compute $s\cdot v$ by computing
$v, 2^1 \cdot v, 2^2 \cdot v, \ldots, 2^{e-2} \cdot v$ and then summing
$\sum_{i=0}^{e-2} s_i \cdot (2^i \cdot v)$.
All this can be done with at most $2(e-2)$ vector additions.

We now proceed to the extension field case $\F_q = \F_p[x]/c_k \F_p[x]$ 
with the Conway polynomial $c_k$ and $q = p^k$.
Here again a scalar $s \in \F_q$ is represented by an expansion
$s = \sum_{i=0}^{k-1} s_i x^i + c_k \cdot \F_p[x]$. For the rest of this
section we omit the ``$+ c_k \cdot \F_p[x]$'' and denote cosets
by their representing polynomials of degree less than $k$.

We reduce the problem to scalar multiplications of vectors with
prime field elements. To this end, we have to be able to multiply
a vector with the primitive root $x$.

Considering a single scalar $t = \sum_{i=0}^{k-1} t_i x^i \in \F_q$, 
we see that 
\[ xt = \sum_{i=1}^{k-1} (t_{i-1}x^i) 
+ \sum_{i=0}^{k-1} t_{k-1} \cdot (x^k - c_k) \in \F_q \] 
(remember that we compute in $\F_p[x]$ modulo $c_k$).
Thus, the multiplication by $x$ can be achieved by a shift, one
multiplication of $x^k - c_k$ with the prime field element $t_{k-1}$,
and an addition.

Since we distribute the prime field elements belonging to a single
extension field element in our vector into adjacent words, we can 
do the shift basically for free by accessing a shifted memory location.
But we still have to deal with the fact, that we have $w$ possibly different
highest coefficients $t_{k-1}$ stored together in one word. However,
since we have to multiply all of them with the same prime field element
coming from the expansion of $x^k - c_k = \sum_{i=0}^{k-1} b_i x^i$ 
we can use the method described
for the prime field case above to compute every word to be added to the
shifted vector. That is, for each $k$ adjacent words 
$(a_{jk},a_{jk+1},\ldots,a_{(j+1)k-1})$ in our vector we have to
add the words shifted by one $(0,a_{jk},a_{jk+1},\ldots,a_{(j+1)k-2})$
to the words $(a_{(j+1)k-1} \cdot b_0, \ldots, a_{(j+1)k-1} \cdot b_{i-1})$.
Therefore, the total cost is exactly the same as for one multiplication of
a vector by a prime field scalar and one addition of vectors.

For the full computation of $s \cdot v$, we have to perform this multiplication
by $x$ altogether $k-1$ times, multiply each intermediate result by the
prime field scalar $s_i$ and sum everything up. Thus, the total cost
of the multiplication $s \cdot v$ is the same as $k$ multiplications
of a vector by a prime field scalar and $k-1$ additions of vectors.
Thus, we have proved the following Proposition:

\begin{Prop}[Multiplication of vectors by scalars]
\label{multvec}
We assume a machine with $32$ bit word length.
Let $p$ be a prime, $q = p^k$, and $w$ and $e$ as above:
$w = \lfloor 32/e \rfloor$ and $e = \lceil \log_2(2p-1) \rceil$.

For $2 < p < 2^{31}$, the total cost of a multiplication of a compact 
vector $v$ of length $n$ over $\F_q$ by a scalar $s \in \F_p$ is at most 
$14k(e-2)\cdot \lceil 32/w \rceil$ word operations (plus memory fetches
and stores). For $p=2$, the scalar
can only be $0$ or $1$ and therefore no computation is necessary at all.

For $2 < p < 2^{31}$, the total cost of a multiplication of a compact
vector $v$ of length $n$ over $\F_q$ by a scalar $s \in \F_q$ is at most
$7k^2(2e-3)\cdot \lceil 32/w \rceil$. For $p = 2$, the total cost is
at most $2k(k-1) \cdot \lceil n/32 \rceil$ word operations.
\end{Prop}

\Proof See above and just add up the number of word operations for up
to $k$ multiplications of a vector by a prime field scalar and $k-1$
additions of vectors. Note for $p=2$ that the vector $n$ consists of
$k \cdot \lceil n/32 \rceil$ words and that we have to count one
vector addition for the ``multiplication with a scalar'', since 
the word $a_{(j+1)k-1}$ has to be XORed to those words whose corresponding
bit is set in $x^k - c_k$.
\ProofEnd

\begin{Rem}
Since a vector of length $n$ needs $k \cdot \lceil n/w \rceil$ words
for $p > 2$, this means, that the number of word operations per word of
the vector needed for one multiplication with a scalar is at most 
$7k(2e-3)$ for $p > 2$.
\end{Rem}

\subsection{Memory throughput in real implementations}

In this section we present timing results in real implementations. We 
compared two implementations: The first is an implementation of the
ideas presented in this chapter in the {\sf GAP} package {\sf cvec}
(see \cite{cvec}) by the author, and the second is the standard implementation
of compact vectors in the {\sf GAP} kernel written by Steve Linton
(see \cite{GAP4}).
The latter uses byte-oriented table lookup for fields $\F_q$ with 
$3 \le q \le 256$. We compared vector additions over $\F_2$ and $\F_7$, 
and multiplications of vectors by scalars over $\F_7$ in both implementations.
Finally, we tested multiplications of vectors by scalars over $\F_{3^k}$
for $1 \le k \le 5$ in the new implementation.
Note that the {\sf GAP} library does not offer direct access to the
operation $z := v+w$ without memory allocation. Therefore this 
measurement is missing.

We used three relatively new machines with popular micro processors, 
namely two different machines
with an Intel Pentium 4 with 1024 kB second level cache running at 3.2 GHz, 
and a machine with a AMD Athlon 64 X2 Dual Core Processor 3800+ with 512 kB
second level cache running at 2.0 GHz. The two Pentium 4 machines were
equipped with memory modules with the same specifications (PC 3200, CL3). 

In all runs, no other processes
were consuming significant amounts of CPU time such that nothing should
have interfered with the caches. Strangely enough, the two nearly identical
machines show different performance. 
We cannot explain the differences in memory throughputs.
Therefore we have presented both
results to show that such measurements have a certain fluctuation
obviously involving parameters which cannot easily be determined.
For a discussion of these results see below.

To demonstrate the effect of second level caches, we used different 
lengths of vectors. We used vectors using $32 \cdot 10^6$ bytes each in
order to make sure that none of the data is in the second level cache,
and vectors using $125000$ and $62500$ bytes each to make sure that
most accesses should lead to cache hits in the second level cache.
Of course, all operations are repeated many times to get a high accurracy
of the measured time for one operation by averaging.

We tested three different operations. The first, which we denote by 
``$z := v+w$'', is addition of two different vectors and writing the
result to some other memory location. The time for memory allocation
was not considered. The second operation, which we denote by 
``$v := v+w$'', is also addition of two different vectors, but the result
is written into the same memory location as one of the summands.
The third operation is denoted by ``$w := sw$'' and is multiplication
of a vector by a (non-zero) scalar in place, that is, the result overwrites
the original vector. We chose the scalar $6 + 7\Z \in \F_7$, since its
binary expansion is $110$ and it is thus one of the most ``expensive'' scalars
for multiplication. Note that for the case of table lookups the chosen
scalar does not matter.

For vector times scalar computations over $\F_{3^k}$ we always chose as
scalar the coset of a polynomial with no zero coefficients, which is
the worst with respect to performance.

All results are memory throughput values in megabytes per second. 
That is, a result of $1000$ means that altogether $1000$ times
$1024^2$ bytes of memory have been accessed. ``Altogether'' means, that for
the operation $z := v+w$ both read operations (for $v$ and $w$) and the
write operation into $z$ are counted, that is, for vectors of length
$125000$ one such addition needs to access $375000$ bytes in memory,
namely reading $250000$ and writing $125000$. The same holds for
$v := v + w$, whereas the operation $w := sw$ only reads the memory once
and writes it again, which means that one such operation has to access
$250000$ bytes of memory.

All our results are shown in Table~\ref{memthrough}. The columns marked
with ``C'' are results obtained using the {\sf cvec} package and those 
marked with ``L'' are results obtained using the {\sf GAP} library.

In the next section we discuss some aspects of these results.

\begin{table}[ht]
\begin{center}
\begin{tabular}{|l|r|r|r|r|r|r|}
\hline
Test            & C, P4$_1$ & C, P4$_2$ & C, Ath & 
                  L, P4$_1$  & L, P4$_2$  & L, Ath \\
\hline
\hline
$\F_2$, vectors $32\cdot 10^6 B$, $z := v+w$ 
& 2764 & 2684 & 1940 & ---  & --- & --- \\
\hline                                                    
$\F_2$, vectors $32\cdot 10^6 B$, $v := v+w$ 
& 3629 & 2965 & 2643 & 3577 & 2812 & 2677 \\
\hline                                                    
$\F_2$, vectors $125\,000 B$, $z := v+w$     
& 6839 & 4249 & 8499 & ---  & --- & --- \\
\hline                                                    
$\F_2$, vectors $125\,000 B$, $v := v+w$     
& 6167 & 3791 &12076 & 4818 & 3298 & 12116 \\
\hline                                                    
$\F_2$, vectors $62\,500 B$, $z := v+w$      
& 5643 & 4379 & 7560 & ---  & --- & --- \\
\hline
$\F_2$, vectors $62\,500 B$, $v := v+w$      
& 6458 & 3969 &11237 & 5109 & 3406 &11635 \\
\hline                                                    
\hline
$\F_7$, vectors $32\cdot 10^6 B$, $z := v+w$ 
& 2380 & 1691 & 1902 & ---  & --- & --- \\
\hline
$\F_7$, vectors $32\cdot 10^6 B$, $v := v+w$ 
& 2624 & 1749 & 2515 & 1088 & 708 & 488  \\
\hline                                                    
$\F_7$, vectors $32\cdot 10^6 B$, $w := sw$  
& 2289 & 1504 & 2267 & 1205 & 870 & 422  \\
\hline                                                    
$\F_7$, vectors $125\,000 B$, $z := v+w$     
& 2671 & 1761 & 5745 & ---  & --- & --- \\
\hline
$\F_7$, vectors $125\,000 B$, $v := v+w$     
& 2843 & 1816 & 5356 & 1102 & 713 & 496  \\
\hline                                                    
$\F_7$, vectors $125\,000 B$, $w := sw$      
& 2388 & 1490 & 4273 & 1375 & 876 & 427  \\
\hline                                                    
$\F_7$, vectors $62\,500 B$, $z := v+w$      
& 2765 & 1742 & 5472 & ---  & --- & --- \\
\hline
$\F_7$, vectors $62\,500 B$, $v := v+w$      
& 3080 & 1878 & 5529 & 1170 & 737 & 498 \\
\hline                                                    
$\F_7$, vectors $62\,500 B$, $w := sw$       
& 2386 & 1565 & 5138 & 1342 & 898 & 434 \\
\hline
\hline
$\F_3$, vectors $32\cdot 10^6 B$, $w := sw$
& 2297 & 1471 & 2219 & --- & --- & --- \\
\hline
$\F_9$, vectors $32\cdot 10^6 B$, $w := sw$
& 153 & 98 & 336 & --- & --- & --- \\
\hline
$\F_{27}$, vectors $32\cdot 10^6 B$, $w := sw$
& 99 & 68 & 232 & --- & --- & --- \\
\hline
$\F_{81}$, vectors $32\cdot 10^6 B$, $w := sw$
& 88 & 58 & 204 & --- & --- & --- \\
\hline
$\F_{243}$, vectors $32\cdot 10^6 B$, $w := sw$
& 70 & 47 & 170 & 1166 & 867 & 422 \\
\hline
\end{tabular}
\end{center}
\caption{Memory throughput for vector operations, C: {\sf cvec} package, 
L: {\sf GAP} library}
\label{memthrough}
\end{table}

\subsection{Discussion of results}
\label{ssec:discussion}

The results for the $\F_2$ case show the throughput for raw memory access
since the XOR operations for addition within the processor registers cost
nearly nothing. 

We first look at the top $\F_2$ part of Table~\ref{memthrough}.

We observe, that both implementations perform very similar, which is
only to be expected, since they use the exact same method. For
main memory access we seem to be nearly identical throughputs. For 
second level cache accesses we see some differences, which can be
explained for the given architectures by looking at the C code and the
produced assembler code, which would lead too far here. However, 
the general picture is the same.

One can clearly see in the $\F_2$ part of
Table~\ref{memthrough} that the second level cache seems to help noticably
when the vectors involved are shorter. This is to be expected, but shows
only a factor of between $1.5$ and $3$ for the Pentium~4 and up to 
a factor of $5$ for the Athlon processor.

Theoretically, one could expect a higher speedup factor, but modern
processors and memory interfaces seem to be highly optimised for 
linear memory accesses using so-called bursts. This means that the
processor has a special method of accessing large amounts of consecutive
memory words, which we seem to use here.

We turn now to the middle part of the table concerning the field $\F_7$.
Here one can see two important things: The first is that the $7$ word
operations per word stated in Proposition~\ref{addvec} can be done as
fast as main memory access (compare main memory throughputs for $p=2$ and
$p=7$ on all architectures). If nearly all accesses lead to hits in 
the second level cache, then the $7$ operations reduce the memory
throughput, but only by a factor of about $2$ to $3$, depending on
the processor architecture.

The second thing is that the byte oriented table lookup is noticably
slower than the word accesses. When working in the second level cache,
this can cripple the memory throughput by a factor of up to $10$, at least on
the Athlon architecture. Note that the vector sizes are chosen small
enough, such that not only the three vectors but also the lookup tables
should fit together completely in the second level cache.
The same observation holds for the multiplication with a scalar. 

Of course, for other finite fields $\F_q$ with $q \le 256$ the situation
is different. Among these fields the number $7k(2e-3)$ of word
operations per vector word is at most $105$. The experiments shown in
the third part of Table~\ref{memthrough} concerning fields $\F_{3^k}$ 
indicate, that
if this number is assumed, the performance of multiplication of a vector
by a scalar is much slower than with table lookup. It follows that there 
is no method that is better in all situations.

However, in the light of the grease method explained in
Section~\ref{ssec:vecmat} computing too many vector times scalar 
operations can often be avoided, which renders the new method
superior in many real life examples.


\index{Matrix}

\section{Matrix arithmetic}
\label{sec:matarith}

Traditionally, matrices are implemented in the {\sf GAP} system as
lists of (row-) vectors. This means, that the list object only
holds references to the subobjects, which are the rows of the matrix.
Therefore, one can for example permute rows very efficiently without
actually touching the bulk of the data in the rows. Instead, one can
just redirect references. In addition, different matrices can share
rows, which leads to the fact that the modification of one matrix
can in fact modify the other matrix. Although this can give rise to
nasty bugs, it can also help to increase efficiency both with respect
to memory usage and with respect to performance. We call matrices
implemented with this approach ``row list matrices''.

An alternative way would be to actually embed the row data into the
matrix object itself. A row would then no longer be a {\sf GAP} object
in its own right, but only a part of one. The advantage here is that
we need fewer memory allocations (only one per matrix) and can better
control cache issues, since we know better, where in memory our row data
is stored. We call matrices implemented with this approach ``flat
matrices''.

When devising efficient algorithms dealing with matrices one has to know
which type of matrix one is using, because one has to know, when the
system actually copies data and when it only passes references.

For the {\cvec} package we chose the row list matrix approach since it fits
better into the existing {\GAP} library and allows for easier code reusage.
We also discuss this approach further in this section. We call a list
of compact vectors of the same length over the same field a ``compact
matrix''.

Note however, that Beth Holmes and Richard Parker are currently working
on an implementation of flat matrices in the {\GAP} system.


\subsection{Addition and multiplication with scalars}

This section is extremely short, since to add matrices or
to multiply a matrix by a scalar simply means to add corresponding
row vectors or multiply all rows by the scalar respectively.
Thus we immediately get from Propositions~\ref{addvec} and
\ref{multvec}:

\begin{Cor}[Matrix addition and multiplication of matrices by scalars]
We assume a machine with $32$ bit word length. Let $M$ and $M'$ be 
two compact matrices with $m$ rows and $n$ columns over $\F_q$, where $q
= p^k$.

For $2 < p < 2^{31}$, the matrices $M$ and $M'$
can be added using $7km\cdot \lceil n/w \rceil$
word operations
(plus memory fetches and stores), where $w = \lfloor 32/e \rfloor$
and $e = \lceil \log_2(2p-1) \rceil$. 
For $p=2$, only $mk \cdot \lceil n/32 \rceil$ word operations are needed.

For $2 < p < 2^{31}$, the total cost of a multiplication of
$M$ by $s \in \F_p$ is at most 
$14km(e-2)\cdot \lceil 32/w \rceil$ word operations (plus memory fetches
and stores). For $p=2$, the scalar
can only be $0$ or $1$ and therefore no computation is necessary at all.

For $2 < p < 2^{31}$, the total cost of a multiplication of $M$
by $s \in \F_q$ is at most
$7k^2m(2e-3)\cdot \lceil 32/w \rceil$.
 For $p = 2$, the total cost is
at most $2k(k-1)m \cdot \lceil n/32 \rceil$ word operations.

For machines with $64$ bit word length, the number of word operations
is halved in both cases and we can work with primes $p < 2^{63}$.
\end{Cor}
\Proof This is immediate from Propositions~\ref{addvec} and \ref{multvec}.
\ProofEnd

\subsection{Vector-matrix multiplication}
\label{ssec:vecmat}

This section is about the multiplication of a row vector 
$v \in \F_q^{1 \times m}$ by a matrix $M \in \F_q^{m \times n}$. The result
is a vector $vM \in \F_q^{1 \times n}$. Since by convention we are working 
mainly with row vectors, this is the only operation between vectors
and matrices we have to consider. The left multiplication of a column
vector by a matrix can be simulated using the transposed matrix.

The multiplication $vM$ can be understood as taking a linear combination
of the rows of $M$, the vector $v$ just contains the coefficients. This
kind of reasoning already leads to the most efficient way to implement
this operation: We just multiply the $i$-th row of $M$ by the scalar
$v_i$ and add the product vector to the final result for $1 \le i \le m$.
Thus this operation can be done using at most $m$ multiplications
of a vector by a scalar plus $m-1$ vector additions.

If we are considering only one vector-matrix multiplication this is basically
all one can say. However, in a situation in which we apply one given
matrix repeatedly to lots of vectors (for example during matrix 
multiplication (see below) or when enumerating orbits), there is a
trick called ``grease'', which was invented by Richard Parker and 
probably others.

The idea is that if the base field is small, there are not so many
different linear combinations of a finite number of vectors. Thus,
we can distribute the rows of our matrix into small groups, say of
$\ell$ adjacent rows each, and precompute all possible linear combinations of
all vectors in each group. If we now want to multiply a vector $v$ and the 
matrix, we extract the entries of $v$ corresponding to each group,
look up the linear combination and add it to the result. The number
$\ell$ of elements in each group is called ``grease level''.

\begin{figure}[ht]
\begin{center}
\input{grease.pstex_t}
\end{center}
\caption{Illustration for Vector-matrix multiplication and grease}
\label{grease}
\end{figure}

This technique is illustrated pictorially in Figure~\ref{grease}. There
one can see a vector-matrix multiplication. The vector is divided into
sections of length $\ell$, beginning from the left, leaving a section of
possibly less than $\ell$ elements at the right hand side. Each such section
corresponds to $\ell$ (or less at the end) adjacent rows in the matrix, we 
call such a submatrix a ``grease block''.
To do grease level $\ell$, we compute all possible linear combinations of
each block of $\ell$ rows and store them. If we have computed and stored
all this data, we call the matrix ``greased''.
Then a vector-matrix multiplication
with this matrix only has to add $\lceil m/\ell-1 \rceil$ vector additions
of looked up vectors. More precisely, we have the following result:

\begin{Theo}[Grease: expected cost and gain]
\label{theogrease}
Let $M$ be a matrix with $m$ rows and $n$ columns over the field\/ $\F_q$
for $q = p^k$, and let $\ell \in \Z_{> 0}$.

Then greasing the matrix $M$ with grease level $\ell$ multiplies the amount
of memory needed for $M$ by at most $q^\ell$ and costs at most 
$\lceil m/\ell \rceil \cdot (q^\ell-k\cdot \ell -1)$ row vector additions 
and $m \cdot (k-1)$ multiplications of a row vector with a primitive
root of $\F_q$.

If $M$ is greased, a vector $v$ can be multiplied from the right with $M$
using only $\lceil m/\ell -1 \rceil$ row vector additions plus the 
cost of extracting the finite field elements from the vector $v$, which
is usually neglectible.

The standard approach for a vector-matrix multiplication needs 
approximately $m(q-1)/q$ multiplications of a row vector by a non-zero
scalar from $\F_q$ plus $(m-1)(q-1)/q$ row vector additions, assuming that 
about every $q$-th element of $v$ is equal to zero.
\end{Theo}
\Proof We first discuss the precomputation step to compute the greased
matrix. Since there are $q^\ell$ different linear combinations of $\ell$ 
vectors the statement about the needed memory is clear.

We have to do the following for every grease block and thus 
all costs are multiplied by $\lceil m/\ell \rceil$. 
After multiplying each vector in the
block $k-1$ times by the primitive root of $\F_q$ represented by the
polynomial $x \in \F_p[x]$ and storing the intermediate results, 
we can compute all $\F_q$-linear combinations
of the vectors in the block by computing all $\F_p$-linear combinations
of the vectors we already have. Thus, by Lemma~\ref{alllinkomb} below we can
compute all those linear combinations with 
$p^{k\cdot \ell} - k\cdot \ell - 1 = 
q^\ell - k \cdot \ell - 1$ row vector additions, which is the cost in the
theorem. Note that we compute here the scalar multiplications by the
primitive root of $\F_q$ as full multiplication although it can be done
more efficiently since the element represented by the polynomial $x$
is sparse, such that the scalar multiplication with it is faster.

After having greased the matrix, a multiplication of a vector with
the matrix can be done by adding appropriate vectors. We have to add
one for every grease block, leading to $\lceil m/\ell - 1\rceil$ additions.
\ProofEnd

\begin{Rems}
\begin{itemize}
\item The value $(2m-1)(q-1)/q$ for the standard approach is a
good estimate for practical considerations about when to grease and when
not. It can be replaced by the upper bound $2m-1$ neglecting
possible zero entries in the vector.
\item It is crucial to implement the extraction of entries
from the vector $v$ as efficient as possible. If this is not done
right, the cost of extracting elements in one vector can cancel out
the benefit of fewer scalar multiplications.
\item The fact that we need very few scalar multiples in the
precomputation phase and none in the vector times matrix phase is
very useful with respect to our new implementation in the
{\cvec} package, since there additions are sometimes much cheaper
than multiplication with scalars.
\end{itemize}
\end{Rems}
 
The following lemma is used in the proof of Theorem~\ref{theogrease}:

\begin{Lemm}[Computing all\/ $\F_p$-linear combinations]
\label{alllinkomb}
Let $v_1, \ldots, v_{j}$ be $\F_p$-linearly  
independent vectors over a finite field $\F_q$ with $q = p^k$
for some $j \in \Z_{> 0}$. If
all $\F_p$-linear combinations of $v_1, \ldots, v_{j-1}$ are already
computed, then all $\F_p$-linear combinations of $v_1, \ldots, v_j$ can
be computed with $p^j - p^{j-1} - 1$ vector additions.

This implies in particular that all $\F_p$-scalar multiples of a vector
can be computed with $p-2$ vector additions, and inductively, that
all $\F_p$-linear combinations of $v_1, \ldots, v_j$ can be be computed
without assumption using $p^j-j-1$ vector additions.
\end{Lemm}
\Proof All non-zero $\F_p$-scalar multiples of $v_j$ can be computed by
successively adding $v_j$ altogether $p-2$ times. Then every $\F_p$-linear
combination of $v_1, \ldots, v_j$ is either one of the already
computed $\F_p$-linear combinations of $v_1, \ldots, v_{j-1}$ or the
non-zero multiples of $v_j$, or a sum of two such vectors. Thus, we 
can compute all missing $p^j - p^{j-1} - (p-1)$ vectors with one
vector addition each, which proves the first statement. 

This includes the statement for $j=1$, which handles the case of computing
all $\F_p$-scalar multiples of one vector. Without assumtions, we can
apply the first statement inductively for $i=1, 2, \ldots, j$ proving
the last statement, since
\[ \sum_{i=1}^j (p^i - p^{i-1} - 1) = p^j - j - 1. \]
\ProofEnd

\begin{Rem}
Thus computing all $\F_p$-linear combinations can be done
with one addition per vector that is not already given, assuming
the zero vector and the input vectors to be already there.
\end{Rem}

\subsection{Matrix multiplication}

Let $M \in \F_q^{m \times n}$ and $N \in \F_q^{n \times s}$ with $q = p^k$.
The matrix multiplication $M \cdot N$ can be computed by multiplying the
rows of $M$ from the right with the matrix $N$ and putting the $m$ resulting
rows of length $s$ into the resulting matrix in $\F_q^{m \times s}$.

There are at least three methods to improve the performance of this
computation: The first is using grease (see the previous section), which
we will analyse in detail in this section. 

The second is using
methods to arrange the same computations in a different order to 
achieve that more memory fetches and stores are handled by the second
level cache. This approach is used in some implementations and can
speed up computations. However, we do not want to go into detail here.
For some discussion of this topic see Section~\ref{sec:cache}. 

The third method is to use the methods of Strassen and Winograd to
get a lower exponent than $3$ in the complexity of matrix multiplication
of square matrices.

We now discuss grease before we analyse for which sizes of
matrices Strassen/Winograd is worthwhile.

When the matrices we want to multiply are big enough such that the number
of rows of $M$ is large in comparison to $q^\ell$
for a suitable chosen grease level $\ell$, it is not necessary to grease
the complete matrix $N$ before beginning the multiplication. Rather,
we can proceed with the whole multiplication "`by grease blocks"', that
is, we compute in a first step all possible linear combinations of the
first $\ell$ rows of $N$, then run through all rows of $M$ looking
only on the first $\ell$ entries, and add each one looked up
linear combination of the rows of $N$. After that we can forget
our precomputed linear combinations and proceed to the next grease block
in $N$ and so on.

The implementation in the {\cvec} package decides about whether it
should use this method by looking at the number of rows of $M$. It compares
the numbers and
estimating whether $m/\ell$ row vector additions

\begin{figure}[ht]
\begin{center}
\includegraphics[width=14cm]{matmulF2_all}
\end{center}
\caption{Matrix multiplication times over $\F_2$ against matrix dimension}
\end{figure}

\begin{figure}[ht]
\begin{center}
\includegraphics[width=14cm]{matmulF2_low}
\end{center}
\caption{Matrix multiplication times over $\F_2$ against matrix dimension}
\end{figure}


\subsection{Matrix inversion}

Greasing.

\subsection{Semi echelonisation}

Cleaning, spinning.

\section{Cache issues}
\label{sec:cache}

Expected performance gain, problems.

\section{Characteristic and minimal polynomial}
\label{sec:charminpoly}

\section{Basic algorithms for matrices}
\label{sec:basalgmat}

Evaluation of polynomials at matrices, order of an invertible matrix.


% Have renamed:
%  env{Lem} -> env{Lemm}
%  env{remark} -> env{Rem}
%  bibitem knuth -> AOCP2

\chapter{Computing characteristic and minimal polynomials}
\label{chap:charminpoly}
\index{characteristic polynomial}\index{minimal polynomial}%

This chapter is about the computation of characteristic and minimal
polynomials of matrices over finite fields. The contents of this
chapter are joint work with Cheryl
Praeger and are already published as \cite{minpolypaper}.

\section{Introduction}

Let $\F$ be a finite field and $M \in \F^{n \times n}$ a matrix. This paper
presents and analyses a Monte Carlo  algorithm to compute the minimal 
polynomial of $M$, that is,
the monic polynomial $\mu \in \F[x]$ of least degree, such that
$\mu(M) = 0$. 
Determining the minimal polynomial is one of the fundamental computational
problems for matrices and has a wide range of applications. As well as 
revealing information about the Frobenius
normal form of $M$, the minimal polynomial also elucidates the structure 
of $\F^n$ viewed as $\F[x]$-module, where $x$ acts by multiplication with $M$. 
In addition the order of $M$ modulo scalars is often found by first 
determining the minimal polynomial. Apart from these applications it has
important practical utility, for example
in the context of the matrix group recognition project~\cite{OB}.

For these and other reasons a number of algorithms to determine the 
minimal polynomial may be found in the literature. We discuss some 
of them below. Our primary objective 
was to provide a simple and practical algorithm that could be 
implemented easily and would work well over small finite fields. In
particular we did not want to produce matrices with entries in 
larger fields or polynomial rings as intermediate results, 
and we preferred to restrict ourselves to using only row operations 
(rather than a combination of row and column operations). 
In addition we wished to use standard field and polynomial arithmetic,
and we wished to give  an explicit worst-case upper bound
for the number of elementary field operations needed, and not only an
asymptotic complexity statement.
Our Monte Carlo algorithm adheres to these requirements for matrices 
over fields $\F_q$ of order $q$.

\begin{Theo}\label{main}
For a given matrix $M \in \F_q^{n \times n}$ and a positive 
real number $\varepsilon < 1/2$, Algorithm~\ref{algminpolymc}
computes the minimal polynomial of $M$ with probability at least $1-\varepsilon$.
For sufficiently large $n$ and fixed $\varepsilon$, the number of elementary 
field operations required is less than $7n^3$ plus the costs of 
factorising a degree $n$ polynomial over $\F_q$ and constructing at most $n$
random vectors in $\F^n$.
\end{Theo}

%For more details see Section~\ref{minpoly}. 

Our algorithm to compute the minimal polynomial first computes the
characteristic polynomial in a standard way by spinning up and then
factoring out cyclic
subspaces. However, the novel aspect in this first phase
is the introduction of randomisation. While
not necessary for the computation of the characteristic polynomial
it underpins our proof of the Monte Carlo 
nature of our minimal polynomial algorithm. In addition to the Monte Carlo
minimal polynomial algorithm
we present and analyse in Section~\ref{verify} a deterministic
verification procedure to
\index{verification}%
be run after Algorithm~\ref{algminpolymc} that has a similar asymptotic
complexity in many cases, but is $O(n^4)$ in the worst-case scenario.
Our motivation for giving concrete upper bounds for the costs of 
various component procedures was that, in practical implementations, 
these assist us to compare different algorithms in order to 
decide which to use in different situations. At the end of the paper we discuss
a practical implementation and tests of the algorithms in the {\sf GAP}
system~\cite{GAP4}.

\subsection{Other algorithms in the light of our requirements}

There are several interesting and asymptotically efficient
minimal polynomial algorithms for $n\times n$ matrices in the literature. The most  
asymptotically efficient deterministic algorithm is due to Storjohann 
\cite{Stor01} in 2001. It is nearly optimal, `requiring about the same number of 
field operations as required for matrix 
multiplication' (see \cite[Abstract, p368]{Stor01}). 
It involves a divide-and-conquer strategy that produces matrices 
with entries in polynomial rings as intermediate results. 
Changing the scalars to a larger field or polynomial ring is 
something we wished to avoid as it creates additional complications in practical 
applications within a computer algebra system used for group and 
matrix algebra computations.

Storjohann's earlier deterministic algorithm~\cite{Stor98} in 1998 uses classical 
field arithmetic and requires $O(n^3)$ field operations. 
It first reduces the matrix to `zig-zag form', using a mix of row 
and column operations, then produces the Smith normal form 
as a matrix with polynomial entries, and finally the Frobenius normal form.
In systems such as {\sf GAP}, matrices over small finite fields 
are stored in a compressed form that
makes row operations simple, but column operations difficult.
Restricting to one of these types of operations was one of our criteria.

A Monte Carlo minimal polynomial algorithm of Giesbrecht~\cite{Gie95} from 1995
that runs in `nearly optimal time' contains some 
features we find desirable for practical implementation, namely 
his algorithm first constructs a `modular cyclic decomposition' 
using random vectors, similar to our characteristic polynomial 
computation in Section~\ref{charpoly}. However, further steps include a 
modification of the `divide-and-conquer' 
Keller-Gehrig algorithm \cite{KelG85} and lead to a 
Las Vegas algorithm that computes a Frobenius form over an extension 
field and then the minimal polynomial.
The field size over which the given matrix is written is assumed 
to be greater than $n^2$, and if this is not the case it is 
suggested that an embedding into a larger extension field be used.
Several of these features were undesirable for us.

In \cite[Section 4]{AC97} Augot and Camion propose a deterministic algorithm
to compute the minimal polynomial of a matrix which is to some extent
similar to our algorithm. It is deterministic with
complexity $O(n^3 + m^2 \cdot n^2)$ field operations, 
where $m$ is the number of blocks
in the shift Hessenberg form. They prove that the complexity is $O(n^3)$ in the
average case. However, in the worst case it is $O(n^4)$, and 
no constants are provided in the complexity estimates. Although the
principal approach of their algorithm is similar to ours, the details
differ very much from our algorithm and analysis.

An interesting commentary on various algorithms, together with some new 
algorithms is given by Eberly~\cite{Eb00}. Eberly (see Theorem 4.2
in \cite{Eb00}) gives in particular a randomised algorithm for matrices 
over small fields that produces output from which (amongst other things)
the minimal polynomial can be computed, at a cost of $O(n^3)$.
The papers \cite{Eb00,Gie95,Steel,Stor98,Stor01} contain references to other 
minimal polynomial algorithms.  In all of the algorithms mentioned 
the asymptotic complexity statements give no information about 
the constants involved. 

On a practical note,  the minimal polynomial algorithm implemented in the GAP library
is the one in \cite{Steel} and (although we have been unable to 
confirm this) we assume that this is the algorithm implemented 
in {\MAGMA} \cite{Magma}. 

\subsection{Outline of the paper}
In Section~\ref{notation} we introduce our notation, in
Section~\ref{complexity} we cite a few complexity bounds for basic
algorithms. The next Section~\ref{ordpoly} introduces order polynomials
and derives a few results about them. Then we turn to the computation
of the characteristic polynomial in Section~\ref{charpoly}, since this
is the first step in our minimal polynomial algorithm, which is described
and analysed in Section~\ref{minpoly}. We explain and modify the 
well-known algorithm to compute characteristic polynomials by introducing
some randomisation, because this is later needed in the analysis of our
main Monte Carlo algorithm. In Section~\ref{probest} we give some
probability estimates that are also used later in the analysis.
The second last Section~\ref{verify}
covers the deterministic verification of the results of our
\index{verification}%
Monte Carlo algorithm. We describe in detail cases in which this
verification is efficient and when it has a worse complexity.
Finally, in Section~\ref{performance} we report on the performance
of an implementation of our algorithm, including runtimes in 
realistic applications.  We compare these
times with the current implementation for minimal polynomial computations
in the {\sf GAP} library (see \cite{GAP4}), and as mentioned above, we 
believe that {\MAGMA} and {\sf GAP} are both using the algorithm in 
\cite{Steel}. We show that our algorithm performs
much better in important cases, and that our bounds on the computing
cost are reflected in practical experiments.



\section{Notation}
\label{notation}

Throughout the paper $\F$ will be a fixed field. Although we envisage
$\F$ to be a finite field for our applications, this is not necessary
for most of our results. However in the later sections we use some probability 
estimates from Section~\ref{probest} that are only valid for 
finite fields.

By an \emph{elementary field operation} 
\index{elementary field operation}%
we mean addition, subtraction, multiplication or division of two field 
elements.
In all our runtime bounds we will assume that one elementary
field operation takes a fixed amount of time and we simply count
the number of such operations occurring in our algorithms.

We denote the set of $(m \times n)$-matrices over $\F$ by $\F^{m \times n}$
and the set of row vectors of length $m$ by $\F^m$. For a vector
$v \in \F^m$ we write $v_i$ for its $i$-th component and for a matrix 
$M \in \F^{m \times n}$ we denote its $i$-th row, which is
a row vector of length $n$, by $M[i]$. We use ``row vector
times matrix'' operations, and in general right modules throughout.
If $V$ is a vector space over $\F$ and $W$ is a subspace, the
quotient space is denoted by $V/W$ and its cosets by
$v+W$ for $v \in V$. The $\F$-linear span of the vectors
$v^{(1)}, \ldots, v^{(k)} \in V$ is denoted by 
$\left< v^{(1)}, \ldots, v^{(k)}\right>_\F$.

If $M \in \F^{n \times n}$ is a matrix and $V = \F^n$, we have a
natural action of $M$ as an endomorphism of $V$ by right multiplication.
The same holds for every $M$-invariant subspace $W < V$ and for
the corresponding quotient space $V/W$. We describe such a situation
by saying that ``the matrix $M$ induces an action on the $\F$-vector space''
$V, W, V/W$ respectively.

Throughout, $\F[x]$ denotes
the polynomial ring over $\F$ in an indeterminate $x$. For a square matrix $M$ and
a polynomial $p \in \F[x]$ we denote the evaluation of $p$ at $M$
by $p(M)$.

Whenever a matrix $M$ induces an action on a vector space $U$, we
will view $U$ as a right $\F[x]$-module by letting $x$ act like $M$,
that is $v \cdot x := v\cdot M$ in the above examples. We denote the
characteristic polynomial of this action by $\chi_{M,U}$. That is,
$\chi_{M,U}$ is the characteristic polynomial 
of the $(\dim_\F(U) \times \dim_\F(U))$-matrix given by choosing 
a basis of $U$ and writing the
action of $M$ induced on $U$ as a matrix with respect to that basis.
We use the same convention analogously for the corresponding minimal
polynomial $\mu_{M,U}$. Furthermore, we denote the $\F[x]$-submodule of
$U$ generated by the vectors $u^{(1)}, \ldots, u^{(n)}$ by $\left< u^{(1)}, \ldots,
u^{(n)} \right>_M$.

We use the two functions 
\begin{equation}\label{si}
s^{(1)}(a,b) := \sum_{i=b+1}^a i\quad \mbox{and}\quad
s^{(2)}(a,b) := \sum_{i=b+1}^a i^2
\end{equation}
for complexity expressions.
Note that for $a > b > c$ we have $s^{(j)}(a,c) = s^{(j)}(a,b) + s^{(j)}(b,c)$
for $j \in {1,2}$ and 
\begin{eqnarray}
\label{formels1}
s^{(1)}(n,0) &=& s^{(1)}(n,-1) = \frac{n(n+1)}{2}
\qquad\mbox{and} \\
\label{formels2}
s^{(2)}(n,0) &=& s^{(2)}(n,-1) = \frac{n(n+1)(2n+1)}{6}.
\end{eqnarray}

For later complexity estimates we note the following inequalities.

\begin{Lemm}[Some upper bounds]
\label{estimates}
If $n = \sum_{i=1}^k d_i$ for some $d_i \in \N \setminus\{0\}$ and
$s_j := \sum_{i=1}^j d_i$ we have
\[ \sum_{j=1}^k s_j \le \frac{n(n+1)}{2} \quad \mbox{and} \quad
   \sum_{j=1}^k s_j(s_j+1) 
   \le \frac{n(n+1)(n+2)}{3}. \]
\end{Lemm}
\proofbeg
We claim that for fixed $n$ both expressions are maximal if and only if
all $d_i$ are equal to one. We leave it to the reader to check that
both totals increase if we replace $d_j$ in some sequence
$(d_i)_{1 \le i \le k}$ by the two numbers $a$ and $d_j-a$ resulting
in the new sequence $(d'_1, \ldots, d'_{k+1}) := 
(d_1, d_2, \ldots, d_{j-1}, a, d_j-a, d_{j+1}, \ldots, d_k)$ 
of length $k+1$.
% the following happens: For $s'_u := \sum_{i=1}^u d'_i$
%we observe, that $s'_u=s_u$ for $1\le u\le j-1$, $s'_j=s_{j-1}+a$, and 
%$s_u'=s_{u-1}$ for $j+1\leq u\leq k+1$. Thus we get
%\[ \sum_{i=1}^{k+1} s'_i = \left(\sum_{i=1}^{j-1} s_i \right)
%   + s_{j-1}+a + \left(\sum_{i=j}^k s_i\right)
%   = \left(\sum_{i=1}^k s_i\right) + s_{j-1} + a
%   > \sum_{i=1}^k s_i \]
%and
%\[ \left( \sum_{i=1}^{j-1} s_i(s_i+1) \right)
%   + (s_{j-1}+a)(s_{j-1}+a+1)
%   + \left( \sum_{i=j}^k s_i(s_i + 1) \right)
%   > \sum_{i=1}^k s_i(s_i + 1) \]
%respectively. 
%
%Since every sequence $(d_i)_{1 \le i \le k}$ can
%be refined to the constant sequence $(1)_{1 \le i \le n}$ the
%expressions in the lemma can be bounded as follows:
%\[ \sum_{j=1}^n j = \frac{n(n+1)}{2} \quad\mbox{and}\quad
%   \sum_{j=1}^n j(j+1)
%   = \frac{n(n+1)(2n+1)}{6} + \frac{n(n+1)}{2} \]
%respectively using Formulae~(\ref{formels1}) and (\ref{formels2}) 
%proving all claims.
\proofend

\section{Complexity bounds for basic algorithms}
\label{complexity}

In some algorithms presented in later sections we use greatest common divisors
of univariate polynomials. To analyse these algorithms we use the 
following bounds which arise from standard polynomial computation. 
We take this approach because the standard algorithms for polynomials
are good enough for our complexity estimates in applications and we do not 
need the asymptotically best algorithms, discussion of which may be found 
conveniently in \cite{vzG}. 
%\begin{Not}[Number of steps of polynomial division with remainder]
%\label{opsnot}
%We denote by $\opspd(n,m)$ the number of elementary field operations
%necessary to divide a polynomial $f \in \F[x]$ by a polynomial $g \in \F[x]$
%with remainder where $\deg f = n \ge m = \deg g$. 

%We denote by $\opsgcd(n,m)$ the number of elementary field operations
%necessary to compute the greatest common divisor of two polynomials
%$f,g \in \F[x]$ where $\deg f = n \ge m = \deg g$.

%Of course, these depend on the algorithms used.
%\end{Not}

\begin{Prop}[Complexity of standard greatest common divisor algorithm]
\label{standardgcd}
\index{Gcd@\textsc{Gcd}}%
Let $f,g \in \F[x]$ with $n := \deg f \ge \deg g =: m$, and $f = qg + r$
with $q,r \in \F[x]$ such that $r=0$ or $\deg r < \deg g$. Then there is an
algorithm to compute $q$ and $r$ that needs less than $2(m+1)(n-m+1)$ 
elementary field operations. 

Furthermore, there is an
algorithm to compute $\gcd(f,g)$ that needs less than
$2(m+1)(n+1)$ elementary field operations.
\end{Prop}

\begin{Rem}{We intentionally give bounds here which are not best possible,
since we want the bound for the $\gcd$ computation to be symmetrical in
$m$ and $n$.}
\end{Rem}

\proofof{Proposition \ref{standardgcd}}
Use polynomial division and the standard {\sc Gcd} algorithm
and count. See \cite[Section 2.4 and Section 3.3]{vzG} for
smaller bounds that imply our symmetric bounds.
%The standard polynomial division algorithm repeatedly subtracts a shifted 
%scalar multiple of $g$ from $f$. Since $\deg g = m$, such a subtraction
%needs one scalar division and $2m$ elementary field operations and we have to
%repeat this at most $n-m+1$ times resulting in a maximal number
%of elementary field operations of $(2m+1)(n-m+1)$ which is smaller than
%$2(m+1)(n-m+1)$ as claimed.
%
%The standard Euclidean algorithm to compute the greatest common divisor
%first divides $f$ by $g$ and then repeatedly divides the current remainder
%by the previous one. Thus, the maximal number of elementary field
%operations needed is
%\begin{eqnarray*}
%\opspd(n,m) &+& \sum_{i=1}^{m-1} \opspd(i+1,i)
%   \le 2(m+1)(n-m+1) + \sum_{i=1}^{m-1} 4(i+1) \\
%   &=& 2(m+1)(n-m+1) + 4\cdot \frac{(m-1)(m+2)}{2} \\
%   &<& 2(m+1)(n-m+1) + 2(m+1)m \\
%   &=& 2(m+1)(n+1).
%\end{eqnarray*}
\proofend

\subsection{Polynomial factorisation}\label{polyfactn}
\index{polynomial factorisation}\index{factorisation!of polynomials}%

Some of our algorithms return partially factorised polynomials
which facilitate later factorisation into
irreducible factors. However, since the extent of this partial factorisation
is difficult to estimate, we use in our
analyses the complexity of finding the complete 
factorisation of a polynomial over a finite field as a product of 
irreducibles. We need such factorisations in our main algorithm.
In keeping with our other methods we make use of standard
polynomial factorisation procedures.

Details can be found in Knuth~\cite[4.6.2]{AOCP2} of a deterministic 
polynomial factorisation algorithm inspired by an idea of Berlekamp. 
Its cost is polynomial in both the degree $n$ and field size 
$|\F|=q$, as it requires $O(q)$ computations
of greatest common divisors. Thus it works well only for $q$ small. 
%However, 
%the first part of the procedure can be used as a test for irreducibility, 
%and the cost for this part depends only on $n$ and $\log_2q$, 
%rather than on $q$, and so can be used for all values of $q$.
There is available a randomised (Las Vegas) version of the procedure
which (for arbitrary $q$) will always
%When $q$ is large there is available a randomised (Las Vegas) 
%version of the procedure 
%which will always 
return accurately the number $r$ of irreducible factors 
of $f(x)\in\F[x]$, but for which there is a small non-zero probability that it will 
fail to find all the irreducible factors.
It involves the procedure {\sc RandomVector}, which is discussed further in Subsection~\ref{random}, to 
produce independent uniformly distributed random elements of an 
$n$-dimensional vector space over $\F$  for 
which a basis is known. Throughout the paper logarithms are always taken to base $2$.

\begin{Rem}[\sc PolynomialFactorisation]\label{rem:polyfactn}
Suppose $f(x)\in \F[x]$ of degree $n\geq 1$ with $r$ irreducible 
factors (counting multiplicities) and, if $q$ is large, suppose that 
we are given a real number $\ve$ such that $0<\ve<1/2$. The  number 
fact$(n,q)$ of elementary field operations required to find a complete
set of irreducible factors of $f(x)$ is at most

\smallskip
\begin{tabular}{lp{1.8in}}
$8n^3 + (3qr+17\log q)n^2$& deterministic algorithm \\
$O\big((\log \,\ve^{-1})(\log n)(\xi_n $ $+ n^2\log^3 q) +n^3\log^2q\big)$& 
Las Vegas algorithm
\end{tabular}\hspace*{-2mm}

\noindent where $\xi_n$ is an upper bound for the cost of one run of\/
{\sc RandomVector} on $\F^n$. The Las Vegas algorithm may fail, but with probability
less than $\ve$.
\end{Rem}





\section{Order polynomials}
\label{ordpoly}

Let $M$ be a matrix in $\F^{n \times n}$ that induces an action
on an $\F$-vector space $V$.

We briefly recall the definition of the term ``order polynomial'':

\begin{Def}[Order polynomial $\ord_M(v)$ and relative order polynomial]
{\rm
The \emph{order polynomial} $\ord_M(v)$ of a vector $v \in V$ is the
\index{order polynomial}%
\index{relative order polynomial}\index{order polynomial!relative}%
monic polynomial $p \in \F[x]$ of smallest degree such that $v \cdot
p(M) = 0 \in V$. In particular $\ord_M(0)=1$.

For an $M$-invariant subspace  $W < V$, the \emph{relative order polynomial}
$\ord_M(v+W)$ (of $v$ relative to $W$) is the order polynomial of the element
$v+W \in V/W$ with respect to the induced action of $M$ on $V/W$.
}
\end{Def}

\begin{Rem}
If we consider $V$ as an $\F[x]$-module as in 
Section~\ref{notation}, then $p$ is the monic generator
of the annihilator $\ann_{\F[x]}(v)$ of $v$ in $\F[x]$.
\end{Rem}

%\smallskip
The following observation follows immediately from the definition above.

\begin{Lemm}[Relative order polynomials]
\label{relorderpol}
\index{relative order polynomial}\index{order polynomial!relative}%
For an $M$-invariant subspace  $W < V$ and $v\in V$, 
$\ord_M(v+W)$ is the monic polynomial $p \in \F[x]$ of smallest degree such 
that $v \cdot p(M) \in W$.
\end{Lemm}
%\proofbeg This is clear because a coset $v+W \in V/W$ is zero
%if and only if $v \in W$. \proofend

%\smallskip
We now turn to the question of how one computes the order polynomial of
a vector $v \in V$. The basic idea is to apply the matrix
$M$ to the vector repeatedly computing a sequence \[ v, vM, vM^2, \ldots, vM^d \]
until $vM^d$ is a linear combination
\[ 
vM^d = \sum_{i=0}^{d-1} a_i vM^i, 
\]
with $a_i \in \F$, for $0 \le i < d$. If $d$ is minimal such that
this is possible, we have
\[ \ord_M(v) = x^d - \sum_{i=0}^{d-1} a_i x^i. 
\]
Although this procedure is simple and well-known, we 
present it in order to make explicit the number of elementary field 
operations needed.
To this end we describe in detail the computation of solutions for
the systems of linear equations involved. 

\begin{Def}[Row semi echelon form]
    \label{rowsemiechelon}
\index{row semi echelon form}%
{\rm
    A non-zero matrix $S = (S_{i,j}) \in \F^{m \times n}$ is in 
    \emph{row semi echelon form} if there are positive 
    integers $r \le m$ and $j_1,\ldots,j_r \le n$ such that,
    for each $i \le r$, $S_{i,j_i} = 1$ and $S_{k,j_i} = 0$ for all $k > i$, 
    and also $S_{k,j} = 0$ whenever $k > r$. For $i \le r$, column 
    $j_i$ is called the \emph{leading column of row $i$}, and we write
    $\lc(i) = j_i$. A sequence of vectors $u^{(1)},\ldots,u^{(m)} \in F^n$ is 
    said to be in semi echelon form if the matrix with rows 
    $u^{(1)},\ldots,u^{(m)}$ is in row semi echelon form.

}
\end{Def}

\noindent
Note that in Definition~\ref{rowsemiechelon} we do not assume 
$j_1 < j_2 < \cdots < j_r$ which is the usual condition for an echelon
form.

\begin{Def}[Semi echelon data sequence]
    \label{semiecheseq}
\index{semi echelon data sequence}%
{\rm    Let $Y \in \F^{m \times n}$ be a matrix with $m \le n$ and of rank $m$. A 
    \emph{semi echelon data sequence for $Y$} is a tuple $\calY=
(Y,S,T,l)$, where
    $S \in \F^{m \times n}$ is in row semi echelon form with
    leading column indices $l = (\lc(1), \ldots, \lc(m))$, and
    $T \in \GL(m,\F)$ with $TY=S$. Further, $T$ is a lower triangular
    matrix, that is, for $T = (T_{i,j})$ we have
    $T_{i,j} = 0$ for $i < j$. For a semi echelon data sequence $\calY$
    we call the number $m$ its \emph{length}, sometimes denoted
    $\len(\calY)$.
A semi echelon data sequence $\calY'=(Y',S',T',l')$ is said to 
\emph{extend} $\calY$
if $\len(\calY')>\len(\calY)$, the first $\len(\calY)$ rows of $Y'$ and $S'$ form
the matrices $Y$ and $S$ respectively, and the first $\len(\calY)$ entries of $l'$
form the sequence $l$. 
}
\end{Def}

\begin{Rem}\label{rem:seds}
(a) The idea of this concept is that for a matrix 
$S \in \F^{m\times n}$ in row semi echelon form it is relatively
cheap to decide whether a given vector $v \in \F^n$ lies in the row
space of $S$, and if so, to write it as a linear combination of the
rows of $S$, that is, to find a vector $a \in \F^m$ such that
$v = aS = aTY$ (see Algorithm~\ref{clean}). Thus, the vector
$v$ is expressed as a linear combination of the rows of $Y$
using the vector $aT$ as coefficients.

(b) We call a semi echelon data sequence \emph{trivial} if $m=0$.
% It is convenient to introduce, for each $n$, a \emph{trivial semi echelon data 
% sequence} $\calY_0=(Y,S,T,l)$ of length zero, in which we regard $Y, S$ as 
% empty matrices in $\F^{0\times n}$, $T$ an empty $0\times 0$ matrix, and 
% $l$ an empty sequence of length $0$. 
In this case, by convention, we take the row spaces of
the empty matrices $Y$ and $S$ to be the zero subspace of $\F^n$, we
denote the empty sequence in $\F^0$ by $0$, and we interpret $aS$ as
the zero vector of $\F^n$.
\end{Rem}
%we need to make clear what is assumed about the trivial $\calY$, and 
%modify Alg 1 to cover it.

\medskip
We now present Algorithm~\ref{clean}, which is one step in the
computation of a semi echelon data sequence for a matrix $Y$. We
denote by $S[i]$ the $i$-th row of the matrix $S$, and by $\rsp(S)$ the 
row space of $S$.

\begin{algorithm}
\caption{$\quad$ \sc CleanAndExtend}
\index{CleanAndExtend@\textsc{CleanAndExtend}}%
\label{clean}
\begin{algorithmic}
\STATE \textbf{Input:} A semi echelon data sequence $\calY=(Y,S,T,l)$ with 
         $Y,S \in \F^{m \times n}$, $v \in \F^n$ (possibly $m=0$).
\STATE \textbf{Output:} A triple $(c,\calY',a')$ where $c$ is {\sc True} 
if $v \in \rsp(Y)$ and {\sc False} otherwise, $\calY'$ equals or 
extends $\calY$ respectively with $\len(Y') \le \len(Y)+1$, 
and $a'\in\F^{\len (\calY')}$, such that $v = a'S'$.
\vspace*{2mm}
\STATE $w := v$
\STATE $a := 0 \in \F^m$
\COMMENT{note that $w=v-aS$}
\FOR {$i = 1$ to $m$}
    \STATE $a_i := w_{l_i}$
    \STATE $w := w - a_i \cdot S[i]$
\ENDFOR\hspace*{6mm}
\COMMENT{still $w=v-aS$} 
\IF {w = 0}
    \STATE \textbf{return} $(\mbox{\sc True},(Y,S,T,l),a)$
\ELSE
    \STATE $j := $ index of first non-zero entry in $w$
    \STATE $a' := [ a \ \ w_j ]$,\quad $l' := \left[ l \ \ j \right]$, $\quad$
    \STATE $Y' := \left[ 
       \begin{array}{c} Y \\ v \end{array} \right]$, $\quad$
           $S' := \left[ 
       \begin{array}{c} S \\ w_j^{-1} \cdot w \end{array} \right]$, $\quad$
           $T' := \left[ 
       \begin{array}{cc} T & 0 \\ -w_j^{-1}\cdot aT & w_j^{-1} 
       \end{array} \right]$
       %or $T'=\left[ w_j^{-1}\right]$ if $m=0$.
    \STATE \textbf{return} $(\mbox{\sc False}, (Y',S',T',l'), a' )$
\ENDIF
\end{algorithmic}
\end{algorithm}

\begin{Prop}[Correctness and complexity of Alg.~\ref{clean}:
\label{PropCleanAndExtend}{\sc CleanAndExtend}]
\index{CleanAndExtend@\textsc{CleanAndExtend}}%
The output of Algorithm~\ref{clean} satisfies the Output
specifications.
%If Algorithm~\ref{clean} returns $(${\sc True}, $\calY,a)$, then $v\in\rsp
%(Y)$, and either $m=0$ with $v=0$ and $a$ an empty sequence, or $m>0$ and $v=aS$. If 
%Algorithm~\ref{clean} returns $(${\sc False}, $\calY',a')$, then $v\not
%\in\rsp(Y)$, $\calY'$ has length $m+1$ and extends $\calY$, and $v=a'S'$.
Moreover, Algorithm~\ref{clean} requires at most $2mn$ field operations
if $v\in\rsp(Y)$, and
$(2m+1)n + (m+1)^2 + 1$ field operations otherwise.
\end{Prop}

\begin{Rem}\label{rem:cae}
(a) Given a semi echelon data sequence $(Y,S,T,l)$ with $Y,S \in \F^{m
\times n}$ and a vector $v \in \F^n$, Algorithm~\ref{clean} tries to
write $v$ as a linear combination of the rows of $S$. If this is not
possible, it constructs an extended semi echelon data sequence.

\medskip\noindent
(b) For the case of finite fields a simple and useful
optimisation is to reduce, where possible, the number of operations for vectors and matrices, 
for example, where a vector is multiplied by the zero scalar and the
result is added to some other vector. This can reduce the number of operations for
sparse vectors and matrices. Our estimates for the numbers of field operations then become
over-estimates.
\end{Rem}

\smallskip
\proofof{Proposition~\ref{PropCleanAndExtend}}
The proof of the correctness of Algorithm~\ref{clean} is left to the
reader.
%Suppose first that $m=0$. If $v=0$ then the algorithm returns  $(${\sc True}, $\calY,a)$, 
%with $a$ an empty sequence, and by convention $v\in \rsp(Y)$. On the other hand if 
%$v\ne 0$, then the algorithm returns  $(${\sc False}, $\calY',a')$, with $a'=[v_j], S'=[v_j^{-1}v]$ 
%where $j$ is minimal such that the $j$-entry of $v$ is non-zero. Thus $v\not\in\rsp(Y)$ (by convention)
%and $v=a'S'$ as asserted. Also $T'Y'=[v_j^{-1} v]=S'$ so $\calY'$ extends $\calY$. 
%
%\medskip
%We may therefore assume that $m\geq1$. During the \textbf{for} loop the algorithm `cleans out the positions in the copy
%$w$ of $v$ corresponding to leading columns of the rows of $S$'. That is to say, at the end of the loop,
%$a=(v_{l_1},\dots,v_{l_m})$, and $w=v-\sum_{i=1}^m a_i S[i] =v-aS$ has a zero entry in position $l_i$,
%for each $i\leq m$. 
%If at this stage $w=0$, then we have $v=aS\in\rsp(S)=\rsp(Y)$, and the algorithm returns $(${\sc True}, $\calY,a)$. Suppose then that 
%$w\ne0$ after the \textbf{for} loop. Then the leftmost position
%$j$ in $w$ such that $w_j\ne0$ is found, and $w$ is `divided' by the scalar $w_j$. 
%The resulting vector $w' := w_j^{-1} \cdot(v - aS)$ becomes  
%row $m+1$ of $S'$ with $j$ the leading column of row $m+1$. Since $v$ is the
%new last row of $Y'$ we have $T'Y' = S'$, because 
%$-w_j^{-1} \cdot aTY + w_j^{-1} v = w_j^{-1} \cdot (v-aS) = w'$.
%By construction, $T'$ is an invertible lower triangular matrix and
%$S'$ is in row semi echelon form. Thus $\calY'=(Y',S',T',l')$ is a
%semi echelon data sequence of length $m+1$ extending $\calY$. 
%The algorithm returns  $(${\sc False}, $\calY',a')$, and $v=w+aS= a'S'$.
The \textbf{for} loop needs $2mn$ field operations if we count both multiplications
and additions. If $v\in\rsp(Y)$ then the algorithm terminates after this loop.
On the other hand, if  $v\not\in\rsp(Y)$, then 
Algorithm~\ref{clean} needs one inversion
of the scalar $w_j$ plus $2 \cdot \sum_{i=1}^m i = m(m+1)$
field operations for the vector times matrix multiplication
$aT$, because $T$ is a lower triangular matrix. This is altogether $m(m+1)+1$ operations. 
Finally, the scalar negation of $w_j^{-1}$ and the multiplication of $aT$ by 
$-w_j^{-1}$ needs another $m+1$ field operations, and a further $n$ operations 
are needed for the computation of $w_j^{-1} w$ in $S'$. Thus the
total number of field operations is at most $2mn + (m+1)^2 + 1 + n$.
\proofend

\smallskip
Having Algorithm~\ref{clean} at hand we can now present
Algorithm~\ref{algordpoly}, which computes relative order polynomials.
Since a (non-relative) order polynomial may be regarded as a relative order
polynomial with respect to the zero subspace, Algorithm~\ref{algordpoly}
can also be used to compute order polynomials, starting with the trivial semi
echelon data sequence, (see Remark~\ref{rem:seds}~(b)).

\begin{algorithm}[t]
\caption{$\quad$ \sc RelativeOrdPoly}
\label{algordpoly}
\index{RelativeOrdPoly@\textsc{RelativeOrdPoly}}%
\begin{algorithmic}
\STATE \textbf{Input:} A semi echelon data sequence $\calY=(Y,S,T,l)$ with 
$Y,S \in \F^{m \times n}$ (possibly $m=0$),
$v \in \F^n$, 
and $M \in \F^{n \times n}$ such that $W:=\rsp(Y)$ is $M$-invariant.
\STATE \textbf{Output:} A triple $(p,\calY',b)$ consisting of the
relative order polynomial $p := \ord_M(v+W)$ of degree $d$,
a semi echelon data sequence $\calY'=(Y',S',T',l')$ of length $m+d$ equal
to or extending $\calY$, and a vector $b \in \F^{m+d}$ such that $vM^d =
bY'$.

\vspace*{2mm}
\STATE $(Y',S',T',l') := (Y,S,T,l)$ \hspace*{2mm}
\COMMENT{the primed variables change during the algorithm}
\STATE $v' := v$
\STATE $m' := m$ \hspace*{3.08cm}  \COMMENT{can be zero!}
\LOOP
    \STATE $(c,(Y',S',T',l'),a) := \mbox{\sc CleanAndExtend}((Y',S',T',l'),v')$  
    \STATE \hspace*{5cm} 
\COMMENT{$T'Y'=S'$, $v'=aS'$}
    \IF { $ c = \mbox{\sc True} $ }
        \STATE \textbf{leave loop}
    \ENDIF
    \STATE $v' := v' \cdot M$
     \STATE $m' := m' +1$	
\ENDLOOP \hspace*{1.5cm}  \COMMENT{at this stage $c=$ {\sc True}, $v'=aS'$, $m'=\len(\calY')$}
\STATE $d := m'-m$
\STATE $b := a\cdot T'$
\STATE $p :=x^d-\sum_{i=0}^{d-1} b_{m+1+i} x^i$
\STATE \textbf{return} $(p, (Y',S',T',l'),b)$
\end{algorithmic}
\end{algorithm}

\begin{Prop}[Correctness and complexity of Alg.~\ref{algordpoly}:
{\sc RelativeOrdPoly}]
\label{proprelorderpol}
\index{RelativeOrdPoly@\textsc{RelativeOrdPoly}}%
Let $\calY=(Y,S,T,l)$ be a semi echelon data sequence with $Y,S \in \F^{m \times
n}$ (possibly $m=0$), $v \in \F^n$, and $M \in \F^{n \times n}$
such that $W:=\rsp(Y)$ is $M$-invariant. 
The output of Algorithm 2 satisfies the Output specifications, and
moreover if $d>0$, then
rows $m+1, \ldots, m+d$ of $Y'$ are equal to $v,vM,\ldots,vM^{d-1}$
respectively.
Algorithm~\ref{algordpoly} requires at most
\begin{eqnarray*}
2dn^2 &+& (n+2)d +2(m+d)n 
+ 2(n+1)s^{(1)}(m+d-1,m-1) +  \\
 &+& s^{(2)}(m+d-1,m-1) 
+ 2s^{(1)}(m+d,0)
\end{eqnarray*}
elementary field operations where $s^{(1)}$ and $s^{(2)}$ are the
functions defined in (\ref{si}).
\end{Prop}

\begin{Rem}
Note that, if $d=0$ then $S'=S$, so $v=b'Y\in W$, and in this case $p=1$. 
Algorithm~\ref{algordpoly} successively considers the vectors $v+W,
vM+W, \ldots, vM^d+W$  (those are the successive values of
$v'$) until $vM^d+W$ lies in the subspace of $V/W$ 
generated by the vectors $v+W, vM+W, \ldots, vM^{d-1}+W$. 
The given matrix $S$ together with Algorithm~\ref{clean} defines a
direct sum decomposition of the $\F$-vector space $V := \F^n = W \oplus W'$
where $W'$ is the subspace of vectors having $0$ in all positions
occurring in the list $l$. Since $W' \cong V/W$, 
Algorithm~\ref{algordpoly} effectively computes in $V/W$ by always `cleaning
out' vectors using $S$ first. 
\end{Rem}

\proofof{Proposition~\ref{proprelorderpol}} 
We again leave the proof of correctness of Algorithm~\ref{algordpoly}
to the reader.
%By Proposition~\ref{PropCleanAndExtend}, after each run of 
%{\sc CleanAndExtend} in the \textbf{loop}, the tuple 
%$(Y',S',T',l')$ is a semi echelon 
%data sequence equal to or extending $\calY$, so $T'Y'=S'$, and $v'=aS'$. 
%Moreover, at the beginning of the $i^{th}$ run of the  \textbf{loop} (since $c=$ 
%{\sc False} at the end of each previous run), we have
%$m'=m+i-1, v'=vM^{i-1}$,  and $Y'=Y$ if $i=1$ and otherwise $Y'$ is $Y$ 
%followed by additional rows $v,vM,\dots,vM^{i-2}$. Thus the  \textbf{loop}
%runs $d+1$ times (possibly $d=0$) and at the end of the $(d+1)^{st}$ run 
%we have $c=$ {\sc True}, $Y'$ is as claimed, and $v'=vM^d=aS'$. Now $b=aT'=(b_1,\dots,b_{m+d})$ satisfies
%\[
%vM^d=aS'=aT'Y'=bY'=b'Y+\sum_{i=0}^{d-1}b_{m+i+1}vM^i,
%\]
%that is, $vp(M)=b'Y\in W$, where $b'=(b_1,\dots,b_m)$. Moreover, $d$ is minimal 
%such that an expression of this type exists, so $p=\ord_M(v+W)$.  
Algorithm~\ref{algordpoly} calls Algorithm~\ref{clean} ({\sc CleanAnd\-Extend}) exactly
$d+1$ times with the lengths of the input semi echelon data
sequences being $m,m+1, \ldots, m+d$. After each but the last call to
Algorithm~\ref{clean} the value of 
$c$ returned is {\sc False}, and after the last call the value of $c$ 
is {\sc True}. Thus, by
Proposition~\ref{PropCleanAndExtend}, the number of steps needed 
for the $d+1$ runs of Algorithm~\ref{clean} is at most
\[
\left(\sum_{i=m}^{m+d-1} ((2i+1)n+(i+1)^2+1)\right)  +  2(m+d)n  \]
\[
   = s^{(2)}(m+d-1,m-1) + (2n+2)s^{(1)}(m+d-1,m-1) + (n+2)d +2(m+d)n.
\]
In addition, we have to do $d$ multiplications of $v'$ with $M$, which require
$2n^2$ elementary field operations each, and finally the computation of $b$
requires $2s^{(1)}(m+d,0)$ elementary field operations, again since 
$T'$ is a lower triangular matrix. Summing up gives the expression in 
the statement. 
\proofend

\smallskip
We conclude this section with two lemmas that are used to
compute absolute order polynomials using relative ones. We again
view $V$ as an $\F[x]$-module by letting $x$ act like $M$. For
$\{v^{(1)},\dots,v^{(m)}\} \subseteq V$, we denote by $\la
v^{(1)},\dots,v^{(m)}\ra_M$ the submodule of $V$ generated
by the vectors $\{v^{(1)},\dots,v^{(m)}\}$, that is, the smallest $M$-invariant
subspace containing $\{v^{(1)},\dots,v^{(m)}\}$. If $m=1$ then $\la
v^{(1)}\ra_M$ is the $\F$-span of the set $\{v^{(1)},v^{(1)}M,\dots,
v^{(1)}M^{n-1}\}$. We call $\la v^{(1)}\ra_M$ a \emph{cyclic
subspace} relative to $M$.

\begin{Lemm}[Order polynomials in cyclic subspaces]
\label{ordpolcyclic}
Let $v\in V$, $W = \left< v \right>_M < V$, and
$p := \ord_M(v)$ with $d := \deg(p)$. 
Then for each $w \in W$, there is a unique polynomial
$q \in \F[x]$ of degree less than $d$ such that $w = vq(M)$.
Moreover,
\[ \ord_M(w) = \frac{p}{\gcd(p,q)}. \]
\end{Lemm}
\noindent We omit the routine proof for the sake of brevity. \proofend
%\proofbeg
%The existence and uniqueness of $q$ follows from identifying $V$ as an 
%$\F[x]$-module. Let $f := \ord_M(w)$ and $r := \gcd(p,q)$, so that $wf(M)=0$, and 
%$p=rp', q=rq'$ for some $p',q'\in\F[x]$ such that $\gcd(p',q')=1$. 
%Then  $w p'(M) = v q(M) p'(M) = v p(M) 
%q'(M) = 0$, and hence $f$  divides $p'$.
%
%Assume first that $q$ is a divisor of $p$, so that $r = q$ and $p=qp'$. 
%Then since $0=wf(M)=vq(M)f(M)$, it follows
%that $p = \ord_M(v)$ divides $qf$. Thus $p'$ divides $f$, and so
%$f=p'$ in this case, as required.
%
%In the general case we have $a,b \in \F[x]$ such that 
%$r = qa+pb$. Then since $wa(M) = vq(M)a(M) = vr(M) - vp(M)b(M) = vr(M)$,
%it follows that $\ord_M(vr(M))$ divides $f$. However, $\ord_M(vr(M)) = p'$ 
%by the previous paragraph, and hence $f = p'$.
%\proofend

%\textbf{Or this one:}
%
%\proofbeg The cyclic module $v\F[x]$ is isomorphic
%to $\F[x]/(p\F[x])$ because $p\F[x] = \ann_{\F[x]}(v)$. Under this isomorphism
%$w$ is mapped to $q+p\F[x]$. Thus 
%$\ord_M(w)\F[x] = \ann_{\F[x]}(q+p\F[x])$. Since $\lcm(p,q) = qp/\gcd(p,q)$
%in the unique factorisation domain $\F[x]$ the statement follows
%as all occurring polynomials here are monic.
%\proofend

\begin{Lemm}[Absolute and relative order polynomials]
\label{absordpoly}
Let $W$ be an $M$-invariant subspace of $V$, $v \in V$ and 
$q := \ord_M(v+W) \in \F[x]$. Then
\[ \ord_M(v) = q \cdot \ord_M(vq(M)). \]
\end{Lemm}
\noindent We omit the routine proof for the sake of brevity. \proofend
%\proofbeg
%Let $p := \ord_M(v) \in \F[x]$. Then $vp(M) \in W$ (since $vp(M)=0$),
%and hence by Lemma~\ref{relorderpol}, $q$ divides $p$. If $q=p$
%then $vq(M)=0$ so $\ord_M(vq(M)) =1$ and the result follows.
%Otherwise $\deg(q)< \deg(p)$, $w:=vq(M)\in\la v\ra_M$, and so by
%Lemma~\ref{ordpolcyclic}, $p=\ord_M(w)\cdot \gcd(p,q)=\ord_M(w)\cdot q$.
% On the
% other hand, $p$ divides $q \cdot \ord_M(vq)$ since the latter is a
% polynomial annihilating $v$. Letting now $r \in \F[x]$ such that $p=qr$,
% we get that $r$ divides $\ord_M(vq)$ and since $0 = vqr$
% that $\ord_M(vq)$ divides $r$. Since both polynomials are monic it
% follows, that they are equal thus proving the claim.
%\proofend

\section{Computing the characteristic polynomial}

\label{charpoly}

In this section we present a version of a standard algorithm for computing
the characteristic polynomial of a matrix together with its analysis. 
It differs from the standard version in its use of randomisation. 

\subsection{Random vectors}\label{random}
\index{RandomVector@\textsc{RandomVector}}%
\index{RandomVector*@\textsc{RandomVector*}}%
Our characteristic polynomial algorithm, and later ones, 
make use of the algorithms {\sc Ran\-dom\-Vec\-tor} and 
{\sc RandomVector*} that produce independent uniformly 
dis\-tri\-bu\-ted random vectors, and independent uniformly distributed
random non-zero vectors,             
respectively, in a given finite vector space for which a basis is known.    
The algorithms are invoked for spaces
$\F^s$, for $s \in \N$, and for subspaces of $V$ of the form
%\[
%V(\neg\, l)=\{ v\,|\, v_{j}=0\quad\mbox{for}\quad  j\ne l_i, 1\leq i\leq m\}
%\quad\mbox{where}\quad l=(l_1,\dots,l_m).
%\]
\[
V(l)=\{ v\,|\, v_{l_i}=0\quad\mbox{for}\quad  1\leq i\leq m\}
\quad\mbox{where}\quad l=(l_1,\dots,l_m).
\]
If $l$ is the empty sequence then $V(l)=V$.
For a semi echelon data sequence $\calY = (Y,S,T,l)$, the 
vector space $V$ is the sum $V=V(l) \oplus \rsp(S)$. 
If $b=\mbox{\sc RandomVector}(\F^{\len(\calY)})$, then $bS$ is a
uniformly distributed random vector of $\rsp(S)$. 
Moreover we assume that for the disjoint spaces 
$\F^{\len(\calY)}$ and $V(l)$ the 
algorithms {\sc RandomVector} and {\sc RandomVector*} are applied 
independently so that in particular, 
if $a=\mbox{\sc RandomVector*}(V(l))$ 
then the sum 
$a + bS$ is a uniformly distributed random vector of $V\setminus \rsp(S)$. 

{\sc RandomVector} and, if we neglect the possibility of obtaining
the zero vector, also {\sc RandomVector*}, could proceed by selecting
independent uniformly distributed random field elements as coefficients
of the basis vectors. For the subspace $V(l)$, we could put zeros into
the entries occurring in $l$ and make random selections of elements from
$\F$ for each entry not in $l$.
For this reason we denote by $\xi_{r}$ an upper bound 
for the cost of {\sc RandomVector} or {\sc RandomVector*} applied to an 
$r$-dimensional space for one of these cases.
If $r<s$ then $\xi_r\leq\xi_s$ and $\xi_{r_1}+\xi_{r_2} \leq \xi_{r_1+r_2}$, and we would 
expect $\xi_r$ to vary linearly with $r$. In 
practical implementations the cost is much less than the cost of the field operations involved 
in the algorithm below.


\subsection{Characteristic polynomial algorithm}


\noindent The characteristic polynomial algorithm below would terminate           
successfully without making random selections of vectors. However, the  
use of randomisation is key to our application of this algorithm for    
finding minimal polynomials. As in previous sections, let $M$ be a      
matrix in $F^{n \times n}$ acting naturally on $V := \F^n$.             




\begin{algorithm}
\caption{$\quad$ \sc CharPoly}
\label{algcharpoly}
\index{CharPoly@\textsc{CharPoly}}%
\begin{algorithmic}
\STATE \textbf{Input:} $M \in \F^{n\times n}$
\STATE \textbf{Output:} 
A tuple $(k, (p^{(j)})_{1\leq j\leq k}, \calY, (b^{(j)})_{1\leq j\leq
k})$, where  each $p^{(j)}\in\F[x]$ and $\prod_{i=1}^k p^{(i)} =
\chi_{M,V}$ is the characteristic polynomial
of $M$ in its action on $V$,  each  $b^{(j)}\in\F^n$ and $\calY$ is a
semi echelon data sequence of length $n$ with the properties
specified in Proposition~\ref{propcharpoly}.

\vspace*{2mm}
\STATE $i := 0$
\STATE $\calY^{(0)}:=$ a trivial semi echelon data sequence
\WHILE {$\len(\calY^{(i)}) < n$}
    \STATE $i := i + 1$
    \STATE $a := \mbox{\sc RandomVector}(\F^{\len(\calY^{(i-1)})})$
    \STATE $c := \mbox{\sc RandomVector*}(V(l^{(i-1)}))$
    \STATE $v^{(i)} := aS^{(i-1)} + c$
	\STATE 
       \hspace*{10mm} \COMMENT{ $v^{(i)}\not\in\rsp(S^{(i-1)})$ where $\calY^{(i-1)}=(Y^{(i-1)},S^{(i-1)},
                              T^{(i-1)},l^{(i-1)})$}
    \STATE $(p^{(i)},\calY^{(i)},b^{(i)}) :=$
   % \STATE \hspace*{3cm}
             $\mbox{\sc RelativeOrdPoly}(\calY^{(i-1)},v^{(i)},M)$
\STATE  \hspace*{10mm} \COMMENT{$b^{(i)}\in\F^{\len(\calY^{(i)})}$; we add $n-\len(\calY^{(i)})$ zeros to make $b^{(i)}\in\F^{n}$}
\ENDWHILE
\STATE $k := i$
\STATE \textbf{return} $(k,(p^{(j)})_{1 \le j \le k}, 
                       \calY^{(k)}, 
(b^{(j)})_{1 \le j \le k})$
\end{algorithmic}
\end{algorithm}

\begin{Prop}[Correctness and complexity of Algorithm~\ref{algcharpoly}]
%: {\sc CharPoly}]
\label{propcharpoly}
\index{CharPoly@\textsc{CharPoly}}%
Algorithm~\ref{algcharpoly} satisfies
the Output specifications, and furthermore
$\calY=(Y,S,T,l)$ where
$Y \in \F^{n \times n}$ is invertible with rows
\[ v\!^{(1)}, v\!^{(1)}M, \ldots, v\!^{(1)}M^{d_1-1}, 
v\!^{(2)}, v\!^{(2)}M, \ldots, v\!^{(2)}M^{d_2-1},
   \ldots, v\!^{(k)}, v\!^{(k)}M, \ldots, v\!^{(k)} M^{d_k-1} 
\]
where $d_i := \deg(p^{(i)})$ for $1 \le i \le k$. Further, for $1 \le i \le k$,
$v^{(i)}$ is a uniformly distributed random element of\/ $V\setminus W_{i-1}$, 
$v^{(i)} M^{d_i} = b^{(i)} Y$ and $p^{(i)} = \ord_M(v^{(i)} + W_{i-1})$,
where $W_{i-1} := \left< v^{(1)}, \ldots, v^{(i-1)}\right>_M$ (for $i>1$), 
an $M$-invariant subspace of $V$ of dimension $s_{i-1}:=\sum_{j=1}^{i-1} d_j$, and 
$W_0=0$ of dimension $s_0=0$.
Moreover, Algorithm~\ref{algcharpoly} requires at most
%\[ \frac{10}{3}n^3
%   +\frac{3}{2}n^2
%   +\frac{7}{6}n
%   +2n\sum_{i=1}^k s_{i-1}
%   +2n\sum_{i=1}^k s_i
%   +\sum_{i=1}^k s_i(s_i+1) \]
%elementary field operations, where  $s_i := \sum_{j=1}^i d_j$ (so $s_0=0$), 
\[
\frac{33}{6}n^3+4n^2+\frac{3}{2}n 
\]
elementary field operations, plus
$k\xi_{n}$ for the $k$ calls to {\sc RandomVector*} and {\sc RandomVector}. Neglecting 
the latter cost this  is less than $6n^3$ elementary field operations, for sufficiently large $n$.
\end{Prop}

\begin{Rem}
We denote the semi echelon data sequences 
$\calY^{(i)}$ in the algorithm using indices to enable us to speak 
more easily about the intermediate results. However in practice
we have only one variable $\calY=(Y,S,T,l)$, the entries of which are 
growing during the execution of the algorithm.
\end{Rem}
%The vector $b^{(i)}$ comes into existence as a vector of length
%$s_i$, which is the number of rows of $Y^{(i)}$.
%The statement $v^{(i)}M^{d_i} = b^{(i)} Y$ in the proposition is to be 
%understood by padding $b^{(i)}$ with zeros up to length $n$.

%\emph{Remark 2:} Since $S^{(i-1)}$ is in row semi echelon form, choosing
%a random $v^{(i)}$ not contained in $W_{i-1}$ is easily achieved by
%putting zeros into the positions occurring in $l^{(i-1)}$ --- which are
%the leading columns of the rows of $S^{(i-1)}$ --- and randomising the
%other entries, such that the resulting vector is non-zero.

\begin{Rem}
Note that we do not multiply together the factors of 
$\chi_{M,V}$ because in our application of Algorithm~\ref{algcharpoly} 
we do not need the product itself.
\end{Rem}

\proofof{Proposition~\ref{propcharpoly}}
Most statements in the proposition follow immediately from 
Proposition~\ref{proprelorderpol}: note that, because of the conventions
explained in Remark \ref{rem:seds}(b), in the first run of the
`while' loop $v^{(1)}=c$ is a random non-zero vector of $V$ and
$p^{(1)}=\ord)M(v^{(1)})$, and more generally,
in the $i^{th}$ run of the 
`while' loop, Algorithm~\ref{algcharpoly}
chooses a  vector $v^{(i)}$ that is a uniformly distributed random element 
of $V\setminus \rsp(S^{(i-1)})$
% = V \setminus \rsp(Y^{(i-1)})$, 
and applies Algorithm~\ref{algordpoly}.
This immediately establishes all statements about
$(Y,S,T,l)$ including the one about the invertibility and the
rows of $Y$. Also it is clear that $p^{(i)} = \ord_M(v^{(i)} + W_{i-1})$.

Next we show that $\prod_{i=1}^k p^{(i)} = \chi_{M,V}$. 
This follows by considering
the matrix $YMY^{-1}$, which has the same characteristic polynomial
as $M$. Considering the action of $M$ with respect to the 
ordered basis of $\F^n$ given by the rows of $Y$, it follows from
the construction that $YMY^{-1}$ 
(written with respect to the standard basis) is equal to
\[ \left[\begin{array}{cccc}
 C_1       &   0  & \cdots & 0 \\
 B^{(2)}_1 &  C_2 & \ddots & \vdots \\
 \vdots    &\ddots& \ddots & 0 \\
 B^{(k)}_1 &\cdots& B^{(k)}_{k-1} & C_k
\end{array} \right] \]
where the matrix $C_i$ is the companion matrix of the polynomial $p^{(i)}$,
%that is, for $p^{(i)} = x^{d_i} - \sum_{j=0}^{d_i-1} a_j x^j$,
%\[ C_i = \left[ \begin{array}{ccccc}
%  0      & 1      & 0      & \cdots    & 0 \\
%  0      & 0      & 1      & \ddots    & \vdots \\
%  \vdots & \vdots & \ddots & \ddots    & 0 \\
%  0      & 0      & \cdots & 0         & 1 \\
%  a_0    & a_1    & \cdots & a_{d_i-2} & a_{d_i-1}
%    \end{array} \right] \in \F^{d_i \times d_i}. \]
and the $B^{(i)}_j$, for $2 \le i \le k$ and $1 \le j \le i-1$, are matrices
in $\F^{d_i \times d_j}$
with one non-zero row at the bottom and all other rows zero.
If $b^{(i)} = (b^{(i)}_1,\dots,b^{(i)}_n)$, then the bottom row of 
$B^{(i)}_j$ is $(b^{(i)}_{s_{j-1}+1},\dots,b^{(i)}_{s_j})$.
With this format at hand it is clear that the characteristic polynomial
of $YMY^{-1}$ is equal to the product $\prod_{i=1}^k p^{(i)}$ because
the $C_i$ are companion matrices.

Finally we derive the statement about the number of elementary field operations
needed by Algorithm~\ref{algcharpoly}.
In the $i^{th}$ run of the {\bf while} loop, the cost of constructing the
random vectors $a$ and $c$ is at most 
\[ \xi_{n-\len(\calY^{(i-1)})}+\xi_{\len(\calY^{(i-1)})}
\leq \xi_{n}, 
\]
(see Subsection~\ref{random}). The cost to compute $v^{(i)}$ is 
at most $2s_{i-1}n$ elementary field operations,
where $s_0=0$, and for $i\geq1$, $s_{i}=\sum_{j=1}^{i}d_j$ with 
$d_j=\deg p^{(j)}$. The cost 
of applying Algorithm {\sc RelativeOrdPoly}
is, by Proposition~\ref{proprelorderpol}, at most
\[ 
2d_in^2 + (n+2)d_i +2s_{i}n + 2(n+1)s^{(1)}(s_i-1,s_{i-1}-1) 
+ s^{(2)}(s_i-1,s_{i-1}-1) + 2s^{(1)}(s_i,0) 
\]
elementary field operations, noting that the value 
of `$d$' is $d_i$, the value of `$m$' is $s_{i-1}$, $s_{i-1}+d_i=s_i$,
and $s^{(1)}$, $s^{(2)}$ are the functions defined in (\ref{si}).

We consider the different terms
one by one, summing each over $i$ from $1$ to $k$. 
The total cost of constructing the random vectors is at most $k\xi_{n}$.
Summing the terms $2 s_{i-1} n$ gives $2n\sum_{i=1}^{k} s_{i-1}$, and
summing the terms $2d_in^2$ gives $2n^3$ since $\sum_{i=1}^k d_i=n$. 
Similarly, summing the terms
$(n+2)d_i$ gives $(n+2)n$. From the terms $2s_in$ we
get a contribution of 
$2n \sum_{i=1}^{k} s_i$.
The next two expressions involving the functions $s^{(1)}$ and $s^{(2)}$
sum to $2(n+1)s^{(1)}(n-1,0) = n(n+1)(n-1)$ and $s^{(2)}(n-1,0) = 
\frac{(n-1)n(2n-1)}{6}$ respectively, using (\ref{formels1}) and
(\ref{formels2}) and the properties noted
above it. 
Finally, the terms $2s^{(1)}(s_i,0)$ sum to 
$2\sum_{i=1}^k s^{(1)}(s_i,0) 
= \sum_{i=1}^k s_i(s_i+1)$.
Thus in total we obtain $k\xi_n$ plus
\begin{eqnarray*}
 2n^3
   &+&n(n+1)(n-1)
   +\frac{(n-1)n(2n-1)}{6}
   +n(n+2)
   +2n\sum_{i=1}^k s_{i-1} \\
   &+&2n\sum_{i=1}^k s_i
   +\sum_{i=1}^k s_i(s_i+1) 
\end{eqnarray*}
elementary field operations. The first four of these terms sum to $\frac{10}{3}n^3
   +\frac{3}{2}n^2
   +\frac{7}{6}n$. 
Using Lemma~\ref{estimates},
\[ 
   2n\sum_{i=1}^k s_{i-1}
   +2n\sum_{i=1}^k s_i
   +\sum_{i=1}^k s_i(s_i+1)\leq 2n^2(n+1)+\frac{n(n+1)(n+2)}{6} \]
so the total cost is at most 
\[
\frac{33}{6}n^3+4n^2+\frac{3}{2}n +k\xi_n.
\]
For sufficiently large $n$ this is less than $6n^3+k\xi_n$.
\proofend


\section{Probability estimates using the structure theory for modules}
\label{probest}

The basic idea of our minimal polynomial Algorithm~\ref{algminpolymc}
is to compute the order polynomials of a few
random vectors under the action of a given matrix $M$ and to prove that, 
with high probability, their least common multiple is
equal to the minimal polynomial of $M$.
The purpose of this section is to use the structure theory 
of $V = \F^n$ as an $\F[x]$-module to derive probability
estimates to be used in that proof.

First suppose that the characteristic polynomial of $M$ is written as a product
$\chi_{M,V} = \prod_{i=1}^t q_i^{e_i}$ with pairwise distinct
irreducible polynomials $q_i \in \F[x]$ and positive integer
multiplicities $e_i$.

Using \cite[Theorem 3.12]{Jacob1} we can then write the $\F[x]$-module $V$
as a direct sum of primary cyclic modules
\begin{equation} \label{primary}
V \cong \bigoplus_{i=1}^t \bigoplus_{j=1}^{m_i} w_{i,j} \F[x] 
\end{equation}
such that $\ord_M(w_{i,j}) = q_i^{f_{i,j}}$ with
$e_i \ge f_{i,1} \ge f_{i,2} \ge \cdots \ge f_{i,m_i} \ge 1$
and $\sum_{j=1}^{m_i} f_{i,j}=e_i$ for $1 \le i \le t$.

The minimal polynomial $\mu_{M,V}$ is the least
common multiple of the order polynomials of the vectors 
$(w_{i,j})_{1 \le i \le t, 1 \le j \le m_i}$, and hence is 
$\mu_{M,V} = \prod_{i=1}^t (q_i)^{f_{i,1}}$.

We use this structural description to derive the
first probability bound for the case where $\F = \F_q$ is a finite field
with $q$ elements.

\begin{Prop}[Probability that a %given 
$q_i$ has equal mult.~in $\mu_{M,V}$
and $\ord_M(v)$]
\label{ProbOneMult}

Let $\F = \F_q$ be a finite field with $q$ elements, let $V=\F^n$, let
$U$ be a (possibly zero) $M$-invariant subspace such that the multiplicity of
$q_i$ in $\mu_{M,U}$ is strictly smaller than in $\mu_{M,V}$, and let 
$v$ be a uniformly distributed random element of $V\setminus U$. Then 
the multiplicity of $q_i$ is the same in $\ord_M(v)$ and $\mu_{M,V}$ 
with probability greater than
$1-q^{-\deg q_i}$.
\end{Prop}
\proofbeg
By assumption the multiplicity of $q_i$ in $\mu_{M,U}$ 
is less than its multiplicity $f:=f_{i,1}$ in $\mu_{M,V}$. 
Let $w:=w_{i,1}$, with $w_{i,1}$ as in (\ref{primary}), so that $V=X\oplus Y$ with
$X, Y$ invariant under $M$ and $X=\la w\ra_M$. Then $\mu_{M,X}=
q_i^{f}$. We may identify the primary cyclic $\F[x]$-module 
$X$ with  $w\F[x]$, which is
isomorphic to the module $\F[x]/(q_i^{f}\F[x])$,
and in turn this is uniserial with composition series
\[ 
0 <   \frac{q_i^{f-1}\F[x]}{q_i^{f}\F[x]}
     <   \frac{q_i^{f-2}\F[x]}{q_i^{f}\F[x]} <
\cdots <  \frac{q_i\F[x]}{q_i^{f}\F[x]} 
       < \frac{\F[x]}{q_i^{f}\F[x]}. 
\]
Thus, $X$ has a unique maximal $\F[x]$-submodule, namely  $X':=\la w q_i(M)\ra_M$,
and $X'$ has codimension $r:=\deg(q_i)$ in $X$. 

As discussed above, each vector $v\in V$ has a 
unique expression as $v=x+y$ with $x\in X, y\in Y$. Moreover
$\ord_M(v)$ is the least common multiple of $\ord_M(x)$ and $\ord_M(y)$.
In particular, if $x\not\in X'$, then $\ord_M(x)=q_i^{f}$ and hence
the multiplicity of $q_i$ in $\ord_M(v)$ and $\mu_{M,V}$ is the same.
The number of vectors $v=x+y$ with $x\not\in X'$ is 
\[
|X\setminus X'|\cdot |Y|=(1-\frac{1}{q^r})|X|\cdot|Y|=(1-\frac{1}{q^r})q^n.
\]
Each of these vectors $v$ lies in $V\setminus U$ since the multiplicity of
$q_i$ in $\mu_{M,U}$ is less than $f$. Thus the probability, for
a uniformly distributed random $v\in V\setminus U$, that the multiplicity 
of $q_i$ in $\ord_M(v)$ and $\mu_{M,V}$ is the same is at least
\[
(1-\frac{1}{q^r})\frac{q^n}{|V\setminus U|} > 1-\frac{1}{q^r}.
\]
\proofend

\begin{Rem}
If for some irreducible factor $q_i$ we have $m_i > 1$
and $f_{i,1} = f_{i,2}$, then the above probability is even higher,
because we can apply the above argument independently to two or more summands
$w_{i,1}\F[x]$ and $w_{i,2}\F[x]$.
\end{Rem}


\smallskip
We now give a second probability bound which will be crucial in our
Monte Carlo algorithm to compute the minimal polynomial. In 
that algorithm we choose  a sequence of vectors 
$v^{(1)}, \dots v^{(u)}$ such that $v^{(1)}$ is a uniformly distributed
random element of $V\setminus\{0\}$, and for $i\geq2$ we choose
$v^{(i)}$ as a uniformly distributed random element of 
$V \setminus U$, where $U=\left< v^{(1)}, \ldots, v^{(i-1)} \right>_M$.
We hope to find $\mu_{M,V}$ as the least common multiple of the
orders of these vectors.


\begin{Prop}[Probability that an lcm of order polynomials equals
$\mu_{M,V}$]
\label{ProbAllMult}
Let\/ $\F = \F_q$ be a finite field with $q$ elements.
Suppose a sequence of vectors $v^{(1)}, \ldots, v^{(u)} \in V$ is chosen
as follows: $v^{(1)}$ is a uniformly
distributed random element of $V\setminus\{0\}$, and for $i>1$,
$v^{(i)}$ is a  uniformly distributed random element of  
$V \setminus \left< v^{(1)}, \ldots, v^{(i-1)} \right>_M$. 
Let
\[ 
f := \lcm( \ord_M(v^{(1)}), \ord_M(v^{(2)}), \ldots, 
\ord_M(v^{(u)}) ). 
\]
Then the probability that  $f = \mu_{M,V}$  is greater than
\[ 1-\sum_{i=1}^t q^{-u\deg q_i}. \]
\end{Prop}
\proofbeg
Consider the random experiment described in the statement. We first
examine one irreducible factor $q_i$. Let $E_i$ denote the event
that the multiplicity of $q_i$ in $f$ is strictly smaller than 
the multiplicity $f_{i,1}$ of $q_i$ in $\mu_{M,V}$. Furthermore, for
$1 \le j \le u$, let $F_j$  be the event that the multiplicity of $q_i$ in
$\ord_M(v^{(j)})$ is strictly smaller than $f_{i,1}$. 

Note that
the $F_j$ are not stochastically independent since we choose
$v^{(j)}$ outside of the space $\left< v^{(1)}, \ldots, v^{(j-1)}\right>_M$.
However, $E_i = F_1 \cap F_2 \cap \cdots \cap F_u$ because $f$ is the
least common multiple of the order polynomials of the $v^{(j)}$.
By Proposition~\ref{ProbOneMult} applied with $U=\{0\}$, 
the probability $\Prob(F_1)$
is less than $q^{-\deg q_i}$. Moreover, in the situation that 
$F_1 \cap \cdots \cap F_j$ holds and $j<u$,
we apply Proposition~\ref{ProbOneMult} with the
subspace $U := \left< v^{(1)}, \ldots, v^{(j)} \right>_M$ to conclude
that the conditional probability $\Prob(F_{j+1} | F_1 \cap \cdots \cap F_j)$
is less than $q^{-\deg q_i}$.
Thus we have
\begin{eqnarray*}
\Prob(E_i) = \Prob(F_1)&\cdot& \Prob(F_2 | F_1) \cdot \Prob(F_3 | F_1 \cap F_2)
   \cdot \cdots \\
    && \cdots \cdot \Prob(F_u | F_1 \cap \cdots \cap F_{u-1}) 
   < q^{-u\deg q_i}.
\end{eqnarray*}
Finally we consider all the different irreducible factors $q_i$. 

Even though
the events $E_1,\ldots,E_t$ may not be stochastically independent, we have
\[ \Prob( E_1 \cup \cdots \cup E_t ) \le \sum_{i=1}^t \Prob(E_i)
   < \sum_{i=1}^t q^{-u\deg q_i} \]
as claimed.
\proofend

\section{Computing minimal polynomials}
\label{minpoly}

Our minimal polynomial algorithm runs Algorithm~\ref{algcharpoly} as its
first step. So assume, from now on,
that we have already run Algorithm~\ref{algcharpoly}
and obtained all the output it produces, in particular
the basis given by the rows of the matrix $Y$ (as in Proposition~\ref{propcharpoly}),
\[ 
(v^{(1)}, v^{(1)}M, \ldots, v^{(1)} M^{d_1-1}, \ldots, v^{(k)}, 
v^{(k)} M, \ldots, v^{(k)} M^{d_k-1}) 
\]
the relative order polynomials $p^{(i)} = \ord_M(v^{(i)}
+ W_{i-1})$, and the vectors $b^{(i)}$ for $1 \le i \le k$.
Also assume that we have factorised all the polynomials $p^{(i)}$
as products $p^{(i)} = \prod_{j=1}^t q_j^{e_{i,j}}$ of irreducible 
polynomials $(q_j)_{1 \le j \le t}$.

The matrices $M$ and $YMY^{-1}$ have the same characteristic
and minimal polynomials. Also the order polynomials $\ord_M(v)$ and
$\ord_{YMY^{-1}}(vY^{-1})$ are equal for every $v \in V$ and thus also
the order polynomials $\ord_M(vY)$ and $\ord_{YMY^{-1}}(v)$ are equal for every
$v \in V$.

For the convenience of the reader we display the matrix
$YMY^{-1}$ in Figure~\ref{bigmat}. Note in particular that the matrix is
sparse, provided that the degrees $d_i$ are not too small.
Due to the special form of $YMY^{-1}$ it is much more efficient to compute
the images of vectors under $YMY^{-1}$, than under $M$. 
This is crucial in the analysis of our algorithms. 
Therefore we will from now on do all computations of order polynomials
with respect to $YMY^{-1}$.

Set $M' := YMY^{-1}$ and $W'_i := W_i Y^{-1}$ for $1 \le i \le k$.
Note that we have $v^{(i)} = e^{(s_{i-1}+1)} Y$ for $1 \le i \le k$
where $e^{(1)}, \ldots, e^{(n)}$ is the standard basis of $\F^n$. (That is,  
 $e^{(i)}$ contains exactly one $1$ in position $i$ and otherwise zeros. Recall
that $s_i = \sum_{j=1}^i d_j$ with $s_0 = 0$.) Furthermore, for $1 \le i \le k$,
the space $W_i = \left< v^{(1)}, \ldots, v^{(i)} \right>_M$ is equal to the space
$\{ vY \mid v \in \F^n \mbox{ with } v_j = 0 \mbox{ for } j > s_i \}$. 
Thus, the space $W'_i$ is the $\F$-linear
span $\left< e^{(1)}, e^{(2)}, \ldots, e^{(s_i)}\right>_\F$ and we have a
filtration
\[ 0 = W'_0 < W'_1 < W'_2 < \cdots < W'_k = V \]
such that each quotient $W'_i/W'_{i-1}$ is an $M'$-cyclic space generated 
by the coset represented by the standard basis vector $e^{(s_{i-1}+1)}$.

We begin by presenting Algorithm~\ref{algordpolabs} which computes the
absolute order polynomial of a vector with respect to the matrix $YMY^{-1}$,
using all the data acquired during Algorithm~\ref{algcharpoly}. 
We will apply this later
in the minimal polynomial algorithm to the first few of the vectors 
$e^{(s_{i-1}+1)}$ 
produced during a run of Algorithm~\ref{algcharpoly}. Note that for 
the analysis it is crucial that a number $z$ such that
the vector $v$ lies in $W'_z$ 
is given as input to the algorithm.

\begin{figure}
\caption{Overview of the matrix $YMY^{-1}$}
\label{bigmat}
\[ \left[ \begin{array}{cccccc|cccccc|c|cccccc}
\cline{1-6}
  0 & 1 &      &      &   &   \\
    & 0 & 1    &      & 0 &   \\
    &   &\ddots&\ddots&   &   \\
    & 0 &      &   0  & 1 &   \\
    &   &      &      & 0 & 1 \\
  {}* & * &   *  &   *  & * & * \\
\cline{1-12}
  &&&&&& 0 & 1 &      &      &   &   \\
  &&&&&&  & 0 & 1    &      & 0 &   \\
  &&&&&&  &   &\ddots&\ddots&   &   \\
  &&&&&&  & 0 &      &   0  & 1 &   \\
  &&&&&&  &   &      &      & 0 & 1 \\
  {}*&*&*&*&*&*&* & * &   *  &   *  & * & * \\
\cline{1-13}
  &&&\vdots&&& &&&\vdots&&& \ddots \\
\cline{1-19}
  &&&&&& &&&&&& & 0 & 1 &      &      &   &   \\
  &&&&&& &&&&&& & & 0 & 1    &      & 0 &   \\
  &&&&&& &&&&&& & &   &\ddots&\ddots&   &   \\
  &&&&&& &&&&&& & & 0 &      &   0  & 1 &   \\
  &&&&&& &&&&&& & &   &      &      & 0 & 1 \\
  {}*&*&*&*&*&*& *&*&*&*&*&*&* &*& * &   *  &   *  & * & * \\
\hline
\end{array} \right] \]
\end{figure}

\begin{algorithm}
\caption{$\quad$ \sc OrdPoly}
\label{algordpolabs}
\index{OrdPoly@\textsc{OrdPoly}}%
\begin{algorithmic}
\STATE \textbf{Input:} $M$, $k$, $(Y,S,T,l)$, $(p^{(j)})_{1 \le j \le k}$,
$(b^{(j)})_{1 \le j \le k}$ as returned by {\sc CharPoly}, an integer
$z$ with $1 \le z \le k$,
$v \in W'_z$, and the factorisation  
$p^{(j)} = \prod_{r=1}^t q_r^{e_{j,r}}$ for all $j\leq k$
\STATE \textbf{Output:} A list of factorised polynomials, the product
of which is $\ord_{YMY^{-1}}(v)$ 
\vspace*{2mm}
\STATE $i := z$ \hspace*{1cm} \COMMENT{will run down to $1$}
\STATE $f := [\ ]$ \hspace*{1cm} \COMMENT{empty list}
\REPEAT
    \STATE $h := \sum_{j=1}^{d_i} v_{s_{i-1}+j} x^{j-1}$
    \IF {$h \neq 0$}
    \STATE $\hat g := p^{(i)}/\gcd(h,p^{(i)})$ \hspace*{1cm} \COMMENT{factorised}
        \STATE \textbf{add} $\hat g$ to list $f$
        \STATE \textbf{compute} product $g$ of factors in $\hat g$
        \IF {$i > 1$}
            \STATE $v := v \cdot g(YMY^{-1})$ \hspace*{1cm} 
            \COMMENT{see Proposition~\ref{propordpol} for this computation}
        \ENDIF
    \ENDIF
    \STATE $i := i - 1$
\UNTIL {$i=0$}
\STATE \textbf{return} $f$
\end{algorithmic}
\end{algorithm}

\begin{Prop}[Correctness and complexity of Algorithm~\ref{algordpolabs}:
{\sc OrdPoly}]
\label{propordpol}
\index{OrdPoly@\textsc{OrdPoly}}%
Let $\F = \F_q$ be a field with $q$ elements.
The output of Algorithm~\ref{algordpolabs} satisfies the Output specifications.
Moreover, Algorithm~\ref{algordpolabs}
requires at most
\[
\sum_{j=1}^{z} 
  \left( 4 d_j^2  + 3d_j s_j + 2d_j \sum_{r=1}^j s_r \right)
\ \leq\ (\frac{z}{2}+9)s_z^2
\]
elementary field operations, 
where $d_j=\deg p^{(j)}$, $s_j=\sum_{r=1}^j d_j$ for $j\geq1$ and
$s_0=0$; and this is less than $n^3$ for $n$ sufficiently large.

\end{Prop}
\proofbeg
Since we are computing an order polynomial with respect to the matrix
$M' = YMY^{-1}$ we can always use the form of this matrix as displayed
in Figure~\ref{bigmat}.

The basic idea of Algorithm~\ref{algordpolabs} is to use
Lemmas~\ref{ordpolcyclic} and \ref{absordpoly} applied to the filtration
\[ 0 = W'_0 < W'_1 < W'_2 < \cdots < W'_k = V. \]

Starting with $i := z$ and the original $v$ lying in the space 
$W'_z$, the variable $i$ runs downwards until $1$.
In each step, Algorithm~\ref{algordpolabs}
computes the relative order polynomial $g := \ord_{M'}(v+W'_{i-1})$
for the then current value of $v \in W'_i$.
This assertion follows from
Lemma~\ref{ordpolcyclic} noting that,
by our discussion above, $p^{(i)} = \ord_M(v^{(i)}+W_i)= \ord_{M'}(e^{(s_{i-1}+1)}+W'_{i-1})$.
Next $v g(M')$ is evaluated, which lies in $W'_{i-1}$ by
Lemma~\ref{relorderpol}, and the induction can go on with $i$
replaced by $i-1$. The product of the polynomials in the list $f$ 
returned is the product of all 
the relative order polynomials computed in the repeat loop, and this is equal to
$\ord_{M'}(v)$, by Lemma~\ref{absordpoly}.

To count the number of elementary field operations is a bit complicated
here. Note first that by assumption we already know a factorisation of all the
$p^{(i)} = \prod_{j=1}^t q_j^{e_{i,j}}$ into
irreducible factors. Now  $\gcd(h,p^{(i)})$ is equal to the product of
the greatest common divisors $\gcd(h,q_j^{e_{i,j}})$, for $j\leq t$. Since the degrees of the
polynomials  $q_j^{e_{i,j}}$ sum up to the degree $d_i$ of $p^{(i)}$, finding these gcd's, by 
Proposition~\ref{standardgcd}, requires at most
\[ 
2(\deg(h)+1) \cdot \sum_{j=1}^t \left(\deg(q_j) e_{i,j} + 1\right) 
\] 
field operations, which is at most $4d_i^2$ since
$\deg(q_j) e_{i,j} + 1 \le 2 \deg(q_j) e_{i,j}$. Note that this is a rather
crude estimate. At this stage we know all multiplicities of the
$q_j$ in $\gcd(h,p^{(i)})$ and thus in $g := p^{(i)}/\gcd(h,p^{(i)})$.
Thus we have computed $g$ in factorised form, which is denoted by
$\hat g$ in Algorithm~\ref{algordpolabs}.

Now we discuss the number of operations needed to evaluate $v g(M')$.
Due to the sparseness of $M'$, a multiplication of a vector $w$ of
$W_i'$ from the right by $M'$ needs only a shift (which we neglect here)
and an addition of a multiple of the non-zero
part of $b^{(r)}$ for $1 \le r \le i$ requiring $2s_r$ operations for
each $r$. Thus computing $wM'$ requires at most $\sum_{r=1}^i 2s_r $
elementary operations. Note that $wM'$ lies in $W_i'$ still. If
$f(x)\in\F_q[x]$ of degree $d$, say $f(x)=\sum_{r=0}^dc_rx^r$, then we
can compute $wf(M')=\sum_{r=0}^dc_rw(M')^r$ by first computing
$w(M')^r\in W_i'$ for $1\leq r\leq d$ at a cost of at most
$2d\sum_{r=1}^i s_r $, next computing $c_rw(M')^r$ for $0\leq r\leq d$
at a cost of at most $(d+1)s_i$, and then adding these vectors at a
further cost of at most
$ds_i$, making a total cost to compute $wf(M')$ of at most
$(2d+1)s_i +2d\sum_{r=1}^i s_r $ elementary field operations.

The polynomial $g(x)$ is available in factorised form,
say $g(x)=\prod_{s=1}^uf_s(x)$ with $\deg f_s=m_s$,
where $\sum_{s=1}^um_s=\deg g=d_i$. From the previous
paragraph we see that $vg(M')$ can be computed at a cost of at most
\[
\sum_{s=1}^u\left((2m_s+1)s_i +2m_s\sum_{r=1}^i s_r \right)
=(2d_i+u)s_i+2d_i\sum_{r=1}^is_r \leq 3d_is_i+2d_i\sum_{r=1}^is_r.
\]
Subsequent runs of the repeat loop require similar numbers of elementary
operations, with $i$ replaced by $j$ where $i-1\geq j\geq 1$.
Thus Algorithm 4 needs at most
\[
\sum_{j=1}^z \left( 4d_j^2  + 3d_j s_j + 2d_j 
\sum_{r=1}^j s_r \right)
\]
elementary field operations, as claimed in the proposition.

To find a simpler upper bound we look at the terms one by one. The
last and most important term can be bounded by
\begin{eqnarray*}
  2\sum_{j=1}^z \left( d_j \sum_{r=1}^j s_r \right)
    &=& 2 \sum_{r=1}^z s_r \left(\sum_{j=r}^z d_j\right)
    = 2 \sum_{r=1}^z s_r (s_z - s_{r-1}) \\
    &=& 2 \sum_{r=1}^z s_r (s_z - s_r) + 2 \sum_{r=1}^z s_r d_r
    \le (\frac{z}{2}+2) \cdot s_z^2
\end{eqnarray*}
since the function $t(s_z-t)$ has maximum value $s_z^2/4$
for $t$ in the interval $[0,s_z]$.

The second term $3\sum_{j=1}^z d_j s_j$ is at most $3s_z^2$,
and the term
$4 \sum_{j=1}^z d_j^2$ is at most $4s_z\sum_{j=1}^z d_j = 4s_z^2$.

Altogether this amounts to a bound of $(\frac{z}{2}+9)s_z^2$ as claimed.
Asymptotically, this is bounded above by $n^3$ in the worst case as
$n \to \infty$.
\proofend

\smallskip
Now we present our main procedure, Algorithm~\ref{algminpolymc}. 

\begin{algorithm}
\caption{$\quad$ \sc MinPolyMC}
\label{algminpolymc}
\index{MinPolyMC@\textsc{MinPolyMC}}%
\begin{algorithmic}
\STATE \textbf{Input:} $M \in \F_q^{n \times n}$, $\varepsilon$ with
$0 < \varepsilon < 1/2$.
\STATE \textbf{Output:} A tuple $(b,f)$ where $b$ is either {\sc True}
or {\sc Uncertain} and $f \in \F_q[x]$
\STATE \hspace*{0mm} \phantom{\textbf{Output:}} (see
Proposition~\ref{propminpoly} for details).
\vspace*{2mm}
\STATE
       $((p^{(j)})_{1 \le j \le k},(Y,S,T,l),(b^{(j)})_{1 \le j \le k})
       := \mbox{\sc CharPoly}(M)$
\STATE Factorise all $p^{(j)} = \prod_{r=1}^t q_r^{e_{j,r}}$ 
\STATE Determine the least $u \in \N$ such that
 $\sum_{r=1}^t q^{-u\deg q_r} \le \varepsilon$
\STATE $u := \min\{ u,k \}$
\STATE $f := \lcm(p^{(1)}, \ldots, p^{(k)})$ 
\FOR {$i = 2$ to $u$}
    \STATE $f := \lcm(f,\mbox{\sc OrdPoly}
           (M,k,(Y,S,T,l),(p^{(j)})_{1 \le j \le k}, 
           (b^{(j)})_{1 \le j \le k},i,e^{(s_{i-1}+1)}))$
\ENDFOR
\IF {$u = k$ or $\deg f = n$}
    \STATE \textbf{return} $(\mbox{\sc True},f)$
\ELSE
    \STATE \textbf{return} $(\mbox{\sc Uncertain},f)$
\ENDIF
\end{algorithmic}
\end{algorithm}

\begin{Prop}[Correctness and complexity of Algorithm~\ref{algminpolymc}:
{\sc MinPolyMC}]\label{propminpoly}
\index{MinPolyMC@\textsc{MinPolyMC}}%
Given a matrix $M \in \F_q^{n \times n}$ and a number
$\varepsilon$ with $0 < \varepsilon < 1/2$, Algorithm~\ref{algminpolymc}
returns a tuple $(b,f)$, where $b$ is either {\sc True} or {\sc Uncertain}
and $f \in \F_q[x]$ is a polynomial. With probability at least $1-\varepsilon$
the polynomial $f=\mu_{M,\F_q^n}$, and if
\/ $b = \mbox{\sc True}$ then
$f = \mu_{M,\F_q^n}$ is guaranteed. Moreover, 
if\/ $f \ne \mu_{M,\F_q^n}$, then 
$f$ is a proper divisor of $\mu_{M,\F_q^n}$ and
every irreducible factor of $\mu_{M,\F_q^n}$ divides $f$.

The number of elementary field operations needed by 
Algorithm~\ref{algminpolymc} is bounded above by
\[ 
{\rm char}(n,q) + {\rm fact}(n,q) + 
  \sum_{i=1}^u (\frac{i}{2}+9)s_i^2
\]
where ${\rm char}(n,q)$ is an upper bound for the number of elementary field operations needed to
compute the characteristic polynomial (see
Proposition~\ref{propcharpoly}), ${\rm fact}(n,q)$ is an upper bound for 
the number of elementary field
operations needed to factorise each of a set of polynomials over $\F_q$ whose degrees sum to $n$
(see Subsection~{\rm\ref{polyfactn}}). Moreover either $u=k$, or $u<k$ and
$\sum_{j=1}^t q^{-u\deg q_j} \le \varepsilon$.

For $n$ sufficiently large and fixed $\varepsilon$, this is less than
\[ 6n^3 + {\rm fact}(n,q) + 
\frac{1}{3} \lceil\frac{\log n-\log \varepsilon}{\log q}\rceil^2 \cdot n^2 
\]
which is less than $7n^3 + {\rm fact}(n,q)$ (plus the cost of computing at most $n$ random 
vectors in Algorithm~\ref{algcharpoly}).
\end{Prop}


\begin{Rem}
Note that if we use a randomised polynomial factorisation algorithm
(necessary for large $q$), then the algorithm can be modified to allow
for a possible failure of factorisation of the `Factorise' step (line
2). Thus Theorem~\ref{main} follows from Proposition~\ref{algminpolymc}.
An upper bound for the term {\rm fact}$(n,q)$ in the complexity
bound is given in Remark~{\rm\ref{rem:polyfactn}}, and this yields an
upper bound in Proposition~{\rm\ref{propminpoly}} of $O(n^3\log^3 q)$
for $n$ sufficiently large and fixed $\varepsilon$.
\end{Rem}

\begin{Rem}
    \label{algdeterm}
    Algorithm~\ref{algminpolymc} can be changed into a deterministic
    one by running the `$i$-loop' for $i$ up to $k$, instead of $u$. An upper
    bound of the cost is then given by replacing $u$ by $k$ in the
    formula for the cost in Proposition~\ref{propminpoly}.

If $k>u$, then these additional $k-u$ runs of the `$i$-loop' may be viewed as 
a `verification  algorithm'. By Proposition~\ref{algminpolymc}, 
\index{verification}%
the additional cost of these extra runs is
\[
\sum_{i=u+1}^k(\frac{i}{2}+9)s_i^2 \leq s_k^2\left(\frac{(k+u+1)(k-u)}{4}+9(k-u)\right)
= s_k^2\frac{(k-u)(k+u+37)}{4}
\]
and for sufficiently large $n$ this cost is less than 
$n^4/4$ field operations.
\end{Rem}

\proofof{Proposition~\ref{propminpoly}}
Algorithm~\ref{algminpolymc} first computes the characteristic
polynomial of $M$ in its action on $\F_q^n$ and its factorisation.
This computation provides firstly the irreducible factors $q_j$ of
the minimal polynomial that allow us to determine $u$, and secondly
the input needed for running Algorithm~\ref{algordpolabs} to compute
the order polynomials of $v^{(2)}, \ldots, 
v^{(u)}$. Thirdly, it also yields a nice base change matrix $Y$ such 
that these order polynomials with respect to the matrix $M$ can in
fact be determined using Algorithm~\ref{algordpoly} for the vectors
$e^{(s_1+1)},\ldots,e^{(s_{u-1}+1)}$ since we have
$\ord_M(v^{(i)}) = \ord_{YMY^{-1}}(e^{(s_{i-1}+1})$ for $1 \le i \le
k$. Note that
$p^{(1)} = \ord_{YMY^{-1}}(e^{(1)}) = \ord_M(v^{(1)})$. 
By Proposition~\ref{propcharpoly},
the vector $v^{(j)}$ is a uniformly distributed random element of
$V\setminus\{0\}$ if $j=1$, or $V\setminus\left< v^{(1)}, \ldots,
v^{(j-1)}\right>_M$ if $j>1$. Hence, by Propositions~\ref{ProbAllMult}
and~\ref{propordpol}, the probability that $f$ after termination of
Algorithm~\ref{algminpolymc} is equal to $\mu_{M,\F_q^n}$ is at least
$1-\varepsilon$.

  From the discussion at the beginning of Section~\ref{probest},
$\mu_{M,\F_q^n}$ is the least common multiple of the $k$ polynomials
$\ord_M(v^{(1)}), \dots,\ord_M(v^{(k)})$, and hence if $u=k$ then
$f = \mu_{M,\F_q^n}$. This also implies, since the initial value
of $f$ is $\lcm(p^{(1)}, \ldots, p^{(k)})$, that the returned
polynomial $f$ divides $\mu_{M,\F_q^n}$ and every irreducible factor of
$\mu_{M,\F_q^n}$ divides $f$. In particular, if $\deg f =n$ then we must
have $f=\chi_{M,\F_q^n}=\mu_{M,\F_q^n}$. Thus if $(\mbox{\sc True},f)$
is returned then $f = \mu_{M,\F_q^n}$ is guaranteed.


The number of elementary field operations needed follows from 
Propositions~\ref{propcharpoly} and~\ref{propordpol} and summing. Note that,
after the factorisations computed in line 2 of the algorithm, we neglect
the forming of least common multiples and the products here, because
all results from Algorithm~\ref{algordpolabs} come already factorised
into irreducible factors. We can thus compute the least common multiples
by taking maximums of multiplicities. Hence the first displayed upper
bound is proved.

For the asymptotic complexity bound we have to consider the initial value of the number $u$,
namely the least integer $u$ such that $\sum_{j=1}^t q^{-u \deg q_j} \le \varepsilon$.
The largest value of this sum occurs when all the $q_j$ have degree 1, and as there are 
then at most $n$ such polynomials,  $\sum_{j=1}^t q^{-u \deg q_j}\le nq^{-u}$. 
Thus $u$ is at most the least integer such that $nq^{-u}\le \varepsilon$, namely 
\[
u_0:=\lceil \frac{\log n + \log (\varepsilon^{-1})}{\log q}\rceil
\]
and the value of $u$ used in the algorithm is at most $\min\{u_0,k\}\leq u_0$.
By Proposition~\ref{propcharpoly}, the asymptotic value of char$(n,q)$ is
less than $6n^3$ for $n$ sufficiently large, (plus the cost $k\xi_n$ 
of making $k$ random selections of vectors). 
By Proposition~\ref{propordpol},  the number of elementary
field operations used for the computation of the  $u\le u_0$ order polynomials
is at most
\[
\sum_{i=1}^u(\frac{i}{2}+9)s_i^2 \leq \frac{u(u+1)}{4}s_u^2+9us_u^2
\le u_0\left(\frac{u_0+37}{4}\right) n^2.
\]
which, for sufficiently large $n$ and fixed $\epsilon$, is less than
$\frac{1}{3}u_0^2n^2<n^3$.
\proofend

\section{Deterministic verification}
\label{verify}
\index{verification}%

In this section we explain how the probabilistic result of our Monte
Carlo algorithm can be verified deterministically. We begin by
discussing cases that can be handled rather cheaply, before we present
several general verification procedures, all of which, unfortunately,
have a worst-case cost of $O(n^4)$ field operations.

All notation from previous sections remains in force. The first result 
follows immediately from Proposition~\ref{propminpoly}.

\begin{Prop}[Cases, in which the result is already proven to be correct]
If the output polynomial of Algorithm~\ref{algminpolymc} is  $\chi_{M,\F_q^n}$,
then the output is $(${\sc True}, $\chi_{M,\F_q^n})$
and is correct.
%If $\mu_{M,\F_q^n} = \chi_{M,\F_q^n}$ then either the output of 
%Algorithm~\ref{algminpolymc} is incorrect, or the output is 
%$(${\sc True}, $\mu_{M,\F_q^n})$.
%That is, if the correct minimal polynomial is computed, this is
%automatically proved in the process.
\end{Prop}

For the next result observe that, if Algorithm~\ref{algminpolymc} 
is modified so that the {\bf for} loop is run $k$ times, then the resulting 
polynomial $f$ is guaranteed to be the minimal polynomial, giving a deterministic
algorithm with proven result. (Proof of correctness 
is given in the proof of Proposition~\ref{propminpoly}.)

\begin{Prop}[Case of few random vectors chosen during comp.~of
$\chi_{M,\F_q^n}$]
\label{veryfewvectors}
If $k \le \sqrt{n}$,  and the {\bf for} loop in 
Algorithm~\ref{algminpolymc} is run $k$ times, then the output polynomial
is $\mu_{M,\F_q^n}$. The 
%total cost for this loop is bounded by
%\[ n^3 + 10 n^{5/2} + n^2 \]
%elementary field operations such that the overall cost is less than
overall cost of this modification of Algorithm~\ref{algminpolymc} is at most
\[ {\rm char}(n,q) + {\rm fact}(n,q) + \frac{1}{4}n^3+\frac{37}{4} n^{5/2} \]
elementary field operations.
\end{Prop}
\proofbeg The only change 
to the complexity estimate is for the number of elementary field operations 
in the second last line of the proof of Proposition~\ref{propminpoly}:
\[
\sum_{i=1}^k(\frac{i}{2}+9)s_i^2 \leq \frac{k(k+1)}{4}s_k^2+9ks_k^2
\leq \frac{k(k+1)}{4}n^2+9kn^2\le \frac{1}{4}n^3+\frac{37}{4} n^{5/2}.
\]
The rest follows from Proposition~\ref{propminpoly}.
\proofend

\medskip
For the case of large $k$, we may use the procedure suggested in
Remark~\ref{algdeterm} as a verification algorithm, at a cost of
$O(k^2n^2)$ field operations. Two alternative verification procedures
\index{verification}%
are given below. The first involving evaluation on vectors is given in
Proposition~\ref{eval}, and the second using null space computations is
given in Proposition~\ref{propverify}.

\begin{Prop}[Verification by evaluation on vectors]\label{eval}
\index{verification}%
    For the output $(\textsc{Uncertain},f)$ of Algorithm~\ref{algminpolymc} 
    one can verify $f = \mu_{M,\F_q^n}$ using
    at most $dn(k-u)(k+u+4)$ elementary field operations where $u$ and
    $k$ are as in Proposition~\ref{propminpoly} and $d = \deg f$.
\end{Prop}
\proofbeg
The idea here is to check whether $e^{(s_{i-1}+1)} f(YMY^{-1})$ is equal to zero, for 
$u+1 \le i \le k$,
by direct evaluation using the techniques described in
the proof of Proposition~\ref{propordpol}. Recall first that the
result $f$ comes in factorised form. Since $e^{(s_{i-1}+1)}$ lies
in $W'_i$ the arguments in the proof of Proposition~\ref{propordpol}
show that this evaluation can be done using at most
$3d s_i + 2d \sum_{r=1}^i s_r$ elementary field operations. Thus, an
upper bound for the total cost for all these evaluations is
\[ 3d \sum_{i=u+1}^k s_i + 2d \sum_{i=u+1}^k \sum_{r=1}^i s_r. \]
The first term is bounded above by $3dn(k-u)$. As to the second
term, for $1 \le j \le u+1$, the value $s_j$ occurs in this expression
with coefficient $2d(k-u)$, while for $u+2 \le j \le k$, it occurs
with coefficient $2d(k-j+1)$. Thus the second term is at most
\[ 
2dn(u+1)(k-u) + 2dn(k-u)(k-u-1)/2 = dn(k-u)(k+u+1). 
\]
Adding this to the upper bound for the first term we get at most
$dn(k-u)(k+u+4)$ as claimed in the proposition.
\proofend

\medskip
For the following discussion we need a lemma:

\begin{Lemm}[Cost of evaluation of a polynomial at a matrix]
\label{costpolyeval}
Let $M \in \F^{n \times n}$ be a matrix and $f \in \F[x]$ a polynomial 
with degree $d < n$. Then the evaluation $f(M)$ can be computed using
at most $2dn^3$ elementary field operations.
\end{Lemm}

\proofbeg We take $2n^3$ elementary field operations as an upper bound for a matrix
multiplication. The computation of the powers $M^2, M^3, \ldots, M^d$
needs at most $2(d-1)n^3$ elementary field operations. The
multiplication, for each $i=1,\dots,d$, of $M^i$ by a coefficient of $f$
and addition of the result to the already computed matrix (the sum of previous terms) 
needs another $2dn^2$ elementary field operations. Finally, the
constant term of $f$ has to be added along the diagonal, which is yet
another $n$ elementary field operations. Since $d+1\le n \le 2 n^2$, 
this is altogether at most $2dn^3$
as claimed.
\proofend

\smallskip
Of course, this immediately implies:

\begin{Cor}[Small degree minimal polynomial]
If $\deg \mu_{M,\F_q^n} < n$, then the output of Algorithm~\ref{algminpolymc} 
can be verified by evaluation using at most 
\[ 2\cdot n^3 \cdot \deg \mu_{M,\F_q^n} \] 
elementary field operations.
\end{Cor}

\begin{Rem}
Note that using \cite[Theorem 2]{AC97} we could lower the complexity
in Lemma~\ref{costpolyeval} to $O(\sqrt d n^3)$ provided we 
stored $O(\sqrt d)$ matrices in memory at the same time. However, since
storing a matrix in $\F^{n \times n}$ needs $O(n^2)$ of memory, this
approach would often become impractical before  a
concrete problem would become intractable because of time constraints. 
We use our estimates in Lemma~\ref{costpolyeval} because of
these practical considerations.
However, in some practical situations, an improved polynomial evaluation
algorithm using more memory may be suitable.
\end{Rem}

We now present Algorithm~\ref{algminpolyverify} that can be run after
Algorithm~\ref{algminpolymc} to verify the correctness of the resulting polynomial
deterministically.

\begin{algorithm}
\caption{$\quad$ \sc MinPoly verification}
\label{algminpolyverify}
\index{verification}\index{MinPoly Verification@\textsc{MinPoly Verification}}%
\begin{algorithmic}
\STATE \textbf{Input:} $M \in \F^{n \times n}$, $\chi_{M,V} = \prod_{i=1}^t q_i^{e_i}$ (factorised), 
and a candidate $\prod_{i=1}^t q_i^{f_i}$ for $\mu_{M,\F_q^n}$ (factorised),
all data from Algorithm~\ref{algminpolymc}
\STATE \textbf{Output:} {\sc True} or a positive number $j$ (see
Proposition~\ref{propverify} for details).
\vspace*{2mm}
\FOR {$i=1$ to $t$}
    \IF {$f_i < e_i$}
        \STATE $M' := q_i(YMY^{-1})^{f_i}$
        \STATE $d := \dim_\F( \ker(M') )$
        \IF {$d < \deg(q_i) \cdot e_i$} 
            \STATE \textbf{return} $i$
        \ENDIF
    \ENDIF
\ENDFOR
\STATE \textbf{return} {\sc True}
\end{algorithmic}
\end{algorithm}

\begin{Prop}[Deterministic minimal polynomial verification]
\label{propverify}
\index{verification}\index{MinPoly Verification@\textsc{MinPoly Verification}}%

If Algorithm~\ref{algminpolyverify} is called with candidate minimal polynomial
$\prod_{i=1}^t q_i^{f_i}$ from Algorithm~\ref{algminpolymc}, 
then it either returns {\sc True} or a 
positive integer~$j$.
In the former case, $\mu_{M,\F_q^n}=\prod_{i=1}^t q_i^{f_i}$,
while in the latter case the
multiplicity of $q_j$ in  $\mu_{M,\F_q^n}$ is greater than $f_j$.
The number of elementary field operations required by 
Algorithm~\ref{algminpolyverify} is at most
\[ 
%\sum_{i=1}^t \left( 2r_in^3 + 2\log(f_i) \, n^3 + n^3 \right) =
 n^3 \cdot \sum_{i=1}^t \left( 2\deg q_i + 2\lceil \log f_i \rceil+1 \right).
\] 
\end{Prop}

\proofbeg Let $r_i := \deg q_i$ for $i = 1, \ldots, t$.
We again view $\F^n$ as $\F[x]$-module as in Section~\ref{probest} by
letting $x$ act as right multiplication by $M$. By 
\cite[Theorem~3.12]{Jacob1}, it is isomorphic to a direct sum of 
primary cyclic $\F[x]$-modules
\[ 
\F^n \cong \bigoplus_{i=1}^t \bigoplus_{j=1}^{m_i} w_{i,j} \F[x], 
\]
such that $\ord_M(w_{i,j}) = q_i^{f_i,j}$ with 
$e_i \ge f_{i,1} \ge f_{i,2} \ge \cdots \ge f_{i,m_i} \ge 1$ and
$\sum_{j=1}^{m_i} f_{i,j} = e_i$. Thus, for each $i$,
$q_i$ occurs in  $\mu_{M,\F_q^n}$ with multiplicity $f_{i,1}$,
and so in particular $f_i\le f_{i,1}$.
The element $q_i^{f_i}$ acts
invertibly on all direct summands $w_{i',j} \F[x]$ with $i' \neq i$
since $q_i$ is irreducible and every order polynomial of a non-zero vector
in such a direct summand is a power of $q_{i'}$, by 
Lemma~\ref{ordpolcyclic}. For $i' = i$ however, the dimension of the kernel 
of the action of $q_i^{f_i}$ on $w_{i,j} \F[x]$ is 
$r_i \cdot \min\{f_i,f_{i,j}\}$.
Thus the dimension of the kernel of the action of $q_i^{f_i}$ on the
whole of $\F^n$ is equal to 
\[
r_i\sum_{j=1}^{m_i}\min\{f_i,f_{i,j}\}
\le r_i\sum_{j=1}^{m_i}f_{i,j}=r_ie_i
\]
with equality if and only if $f_i\ge f_{i,1}$. 
Since $f_i\le f_{i,1}$, equality holds above if and only if $f_i$
is equal to the multiplicity $f_{i,1}$ of $q_i$ in  $\mu_{M,\F_q^n}$.
Therefore, Algorithm~\ref{algminpolyverify} always returns the
result as stated in the Proposition.

As to the cost, Algorithm~\ref{algminpolyverify} evaluates $q_i$ at 
$YMY^{-1}$ which needs at most $2r_i n^3$ elementary field operations
by Lemma~\ref{costpolyeval}. It then takes the result to the
$f_i^{\mathrm{th}}$ power, which can be done by repeated squaring
with at most $2n^3\lceil \log f_i \rceil$ elementary field operations, and 
finally computes the dimension of a null space, which can be done with at most
$n^3$ elementary field operations (compute a semi echelon basis
of the row space of the matrix). Note that we are not using the 
sparseness of $YMY^{-1}$ here.
\proofend

\begin{Rem}
The cost in Proposition~\ref{propverify} is much smaller than $n^4$ in
many cases. One of the worst cases occurs when $\chi_{M,\F^n}$ contains lots of
different factors of degree $1$ each occurring with multiplicity $3$, and 
all the $f_i$ are equal to $2$. Then Algorithm~\ref{algminpolyverify} has to
square about $n/3$ matrices and compute the null spaces of the results. 
This amounts to about
$2n^4/3$ elementary field operations, which is only about twice as fast as 
directly evaluating the minimal polynomial at $M$. Note that even in this case
only about every sixth entry of\/ $YMY^{-1}$ is different from zero. 
\end{Rem}

\begin{Rem}
 As in each of our procedures there are some simplifications we could make in practice
which do not reduce the worst case complexity estimates. For example, in Algorithm~\ref{algminpolyverify}, there is no need to compute the kernel of $q_i(YMY^{-1})^{f_i}$ if the irreducible $q_i$ does not divide any of the relative order polynomials $p^{(j)}$ for $u<j\leq k$. 
\end{Rem}



\section{Performance in practice}
\label{performance}

In this section we give some experimental evidence concerning 
the performance of Algorithm~\ref{algminpolymc} in comparison with 
that of algorithms currently implemented in the
{\sf GAP} library (see \cite{GAP4}). 

All computations were done on a machine with an Intel Core 2 Quad CPU Q6600
running at 2.40 GHz with 8 GB of main memory and two times 4 MB of second level
cache. 

We were unable to confirm that {\MAGMA} \cite{Magma} 
uses an algorithm  based on the canonical forms algorithm of Alan
Steel presented in~\cite{Steel} for computing minimal 
polynomials, although this is indicated in~\cite[Abstract]{Steel} and on the
web\\ %
(see \texttt{http://magma.maths.usyd.edu.au/magma/htmlhelp/text347.htm}).

Our colleague Colva Roney-Dougal kindly
ran the Baby Monster example matrix $M_2$ in the {\MAGMA} system and the resulting
times were roughly equivalent to the timing in the column ``Lib'' of
Figure~\ref{timings}, suggesting that this is indeed the case. Since the
minimal polynomial algorithm in the {\sf GAP} library is also based on 
the algorithm in \cite{Steel}, we did not conduct extensive comparison tests of our
algorithm on {\MAGMA}.

\subsection{Guide to the test data}
The timing results
are in Figure~\ref{timings}, all times are in seconds. 
The column marked ``$n$'' contains the
dimension of the matrix, the column marked ``$q$'' the number of elements
of the base field. 
The columns marked ``Lib'' and ``AS'' contain the times needed for one run of 
the minimal polynomial algorithm based on \cite{Steel} as implemented in 
the {\sf GAP} library, and as implemented (by the first author) in the
{\sf GAP} language, respectively.
The column ``MC'' contains the total time for our 
Monte Carlo algorithm as presented in Algorithm~\ref{algminpolymc}.
The next three columns marked ``Spin'', ``Fact'' and ``OrdP''
contain the times for the three phases of this algorithm respectively, 
namely the first phase to compute the characteristic polynomial via
relative order polynomials, the second phase to factor all factors of
the characteristic polynomial and count multiplicities, and the third
phase to compute some absolute order polynomials to guess the minimal
polynomial. Finally, the last column marked ``Ver.'' contains
the time for the deterministic verification via 
\index{verification}%
Algorithm~\ref{algminpolyverify}.
The maximal error probability for our Monte Carlo algorithm was
$\varepsilon = 1/100$ for all runs.


\subsection{The test matrices}
Next, we describe the matrices $M_1, \ldots, M_{10}$ we used.

(a) \quad The matrices $M_1$ and $M'_1$ were purely random matrices from
$\F_3^{1000\times 1000}$ with all entries
chosen with uniform distribution from the field $\F_3$. Such matrices
are with very high probability cyclic, that is, their characteristic and
minimal polynomials are equal. Usually, Algorithm~\ref{algcharpoly} only
has to pick very few random vectors for such matrices. The {\bf for} loop
of Algorithm~\ref{algminpolyverify} quickly checks whether the least common multiple
of the relative order polynomials (which is the input candidate polynomial)
already has degree $n$. It turned out that $M_1'$ was cyclic but not $M_1$, 
and this explains the
big differences in the runtimes for these matrices.

(b)\quad The matrix $M_2$ is one coming from actual applications. Namely, it is
the matrix $a+b+ab$ where the two matrices $a,b \in \F_2^{4370 \times
4370}$ describe the action of two standard generators of the Baby monster
sporadic simple group on its smallest faithful simple module over $\F_2$.
The matrices $a$ and $b$ were downloaded from the WWW-Atlas of group
\index{WWW-Atlas of group representations}%
representations (see \cite{WWWAtlas}). The matrix $a+b+ab$ is
interesting because it is one of the algebra words that is used
in the {\sc MeatAxe} (see \cite{MeatAxeParker} and \cite{MeatAxeHoltRees})
to compute composition series of modules and
we could very well imagine using the minimal polynomial instead of
the characteristic polynomial in some places in the {\sc MeatAxe}.

The reason why the standard algorithm for the minimal polynomial
performed rather badly on this matrix is that its characteristic polynomial
has irreducible factors of degrees $1$, $1$, $2$, $4$, $6$, $88$,
$197$, $854$ and $934$ with respective multiplicities
$2$, $2277$, $4$, $1$, $1$, $1$, $1$, $1$ and $1$.
Therefore the standard algorithm spins up large subspaces
many times.

(c)\quad The matrices $M_3$ to $M_7$ were constructed in the following way:
In the language of $\F[x]$-modules we chose the order polynomials of
the generators of their primary cyclic submodules, that is we chose the
minimal polynomials on the primary cyclic submodules. For irreducible factors
of degree one this amounts to choosing the sizes and numbers of the
Jordan blocks occurring in the Jordan normal form of the matrix.
After writing down the corresponding normal form of the matrix we
conjugated it with a random element of the general linear group to
get a dense matrix with the same normal form.

For $M_3 \in \F_5^{600 \times 600}$ we chose one cyclic summand with
minimal polynomial $(x-\zeta_5)^{300}$ plus 300 summands with minimal
polynomial $x-\zeta_5$, where $\zeta_5 \in \F_5$ is a primitive root.
This is a typical case in which our Monte Carlo algorithm and
the deterministic verification both perform very well in comparison
\index{verification}%
with older techniques. The reason for this is that the high dimensional
cyclic subspace is spun up many times in the standard minimal polynomial
algorithm as for the matrix $M_2$.

For $M_4 \in \F_3^{1200 \times 1200}$ we chose 400 cyclic summands with
minimal polynomial $(x-\zeta_3)^2$ plus 400 cyclic summands with
minimal polynomial $(x-\zeta_3)$, where again $\zeta_3 \in \F_3$ is
a primitive root. In contrast with the matrix $M_3$,  our algorithms 
performed very well in this case but they were not
much faster than the older techniques, since the standard algorithm
run on $M_4$ does not spin up many large cyclic subspaces.

For $M_5 \in \F_{251}^{600 \times 600}$ we chose 200 different linear
factors $x-\alpha$ and for each added one cyclic space with minimal polynomial
$(x-\alpha)^2$ and one with $x-\alpha$. This example originally was a
worst case scenario for our deterministic verification. However, since $k=2$ is 
\index{verification}%
quite small, a deterministic verification can be done relatively
cheaply as described in Proposition~\ref{veryfewvectors} and
Remark~\ref{algdeterm}, even though the integer $u$ in
Algorithm~\ref{algminpolymc} is only $1$. The deterministic
verification Algorithm~\ref{algminpolyverify} ran very slowly 
(more than 300 seconds) in this example.

For $M_6 \in \F_2^{2391 \times 2391}$ we chose the irreducible polynomial
$f(x) = x^3+x^2+1 \in \F_2[x]$ of degree $3$ and added cyclic spaces with 
respective minimal polynomials $f^{400}$, $f^{200}$, $f^{100}$,
$f^{50}$, $f^{25}$, $f^{12}$, $f^6$, $f^3$ and $f$.

For $M_7 \in \F_{3^4}^{220 \times 220}$ we chose an irreducible polynomial
$f(x) \in \F_{10}[x]$ of degree $10$ and added cyclic spaces with
respective minimal polynomials $f^{10}$, $f$, $f^2$, $f^3$, $f$, $f^2$ and
$f^3$.

(d)\quad The matrices $M_8$ and $M_9$ were standard generators of 
${\rm GL}(400,17)$, 
conjugated by the pseu\-do-ran\-dom element $M_{10}$ of the same group.
Note that $M_8$ had
order 16 while $M_9$ and $M_{10}$ had very high order and were cyclic 
matrices. We chose these examples because they may be typical of 
difficult cases in an application of the minimal polynomial algorithm
for computing the projective order of a matrix. 

Our algorithm %ran into the case that it 
very quickly discovered that the least common multiple of the relative order
polynomials was already equal to the characteristic polynomial.

\begin{figure}
\caption{Timings for minimal polynomial computation}
\label{timings}
\begin{center}
\hspace*{-5mm}
\begin{tabular}{|c|r|r|r|r|r|r|r|r|r|r|}
\hline
M & $q$ & $n$ & Lib & AS & MC & Spin & Fact & OrdP & Ver. & k \\
\hline
\hline
$M_1$  & 3   & 1000 & 1.95$^*$ & 0.65 & 13.7 & 0.33 & 13.3 & 0.05 & 0 & 2 \\
$M_1'$ & 3   & 1000 & 1.31$^*$ & 0.68 & 0.32 & 0.32 & 0 & 0 & 0 & 2 \\
$M_2$  & 2   & 4370 & 12975 & 3098 & 5.74 & 3.80 & 1.10 & 0.83 & 3.02 & 2212 \\
$M_3$  & 5   &  600 & 59.5 & 21.0 & 0.33 & 0.16 & 0.08 & 0.08 & 0.19 & 301 \\
$M_4$  & 3   & 1200 & 2.00$^*$& 0.45 & 0.44 & 0.38 & 0.06 & 0.01 & 0.06 & 800 \\
$M_5$  & 251 &  600 & 2.9 & 3.3 & 3.26 & 2.82 & 0.55 & 0.04 & 0 & 2 \\ 
$M_6$  & 2   & 2391 & 14.6 & 3.3 & 2.25  & 0.91 & 0.18 & 1.15 & 1.02 & 9 \\
$M_7$  & 243 &  220 & 0.77 & 0.88 & 0.36 & 0.34 & 0.01 & 0.01 & 0.21 & 7 \\
$M_8$  & 17  &  400 & 0.46 & 0.20  & 0.048 & 0.032 & 0.012 & 0.004 & 0.00 & 399 \\
$M_9$  & 17  &  400 & 0.26 & 0.20 & 0.23 & 0.23 & 0 & 0 & 0 & 1 \\
$M_{10}$& 17  &  400 & 0.26 & 0.19 & 0.22 & 0.22 & 0 & 0 & 0 & 1 \\
% old:
% $M_1$  & 3   & 1000 & 4.24$^*$ & 1.74 & 10.1 & 0.92 & 9.20 & 0 & 0 & 2 \\
% $M_1'$ & 3   & 1000 & 3.65$^*$ & 2.14 & 1.13 & 1.13 & 0 & 0 & 0 & 1 \\
% $M_2$  & 2   & 4370 & 25613 & 7696 & 10.9 & 7.64 & 1.96 & 1.28 & 9.22 & \\
% $M_3$  & 5   &  600 & 121 & 50.2 & 0.45 & 0.28 & 0.02 & 0.15 & 0.58 & \\
% $M_4$  & 3   & 1200 & 3.70$^*$& 1.19 & 1.20 & 1.03 & 0.15 & 0.02 & 0.27 & \\
% $M_5$  & 251 &  600 & 8.5 &16.0& 10.26 & 8.69 & 0.73 & 0.83 & 636 & \\ 
% $M_6$  & 2   & 2391 & 31.5& 8.2& 4.6  & 2.30 & 0.35 & 1.95 & 2.46 & \\
% $M_7$  & 243 &  220 & 2.52 & 3.15& 1.06 & 1.01 & 0.03 & 0.018 & 0.80 & \\
% $M_8$  & 17  &  400 & 0.980 & 0.450  & 0.069 & 0.035 & 0.027 & 0.007 & 0.064 & \\
% $M_9$  & 17  &  400 & 0.611 & 0.500 & 0.581 & 0.581 & 0 & 0 & 0 & \\
% $M_{10}$& 17  &  400 & 0.607 & 0.504 & 0.583 & 0.583 & 0 & 0 & 0 & \\
\hline
\end{tabular}
\hspace*{-5mm}

\medskip
$^*$ averaged over 10 runs
\end{center}
\end{figure}

\bigskip\noindent
{\bf Acknowledgements:}\quad 
We would like to thank an anonymous referee for invaluable
suggestions that improved and streamlined the exposition and in particular encouraged us to think about and improve
the deterministic verification procedures.
\index{verification}%

%\footnote{
This research forms part of a Discovery
Project and Federation Fellowship Grant of the  second author funded by the
Australian Research Council.%}

%\bibliographystyle{alpha}
%\bibliographystyle{jcm}

%\bibliography{minpoly}

% is is a part of the habilitation thesis of Max Neunhoeffer

\chapter{Further linear algebra algorithms}
\label{chap:linalg}

In this chapter we collect for the sake of completeness some
linear algebra algorithms together with their complexity analysis
that are used in later chapters. 

\section{Order and projective order of a matrix}
\label{sec:orders}

This section is about the computation of the order and the projective
order of a matrix $M \in \F_q^{n \times n}$. After defining these 
terms we describe the 
ideas of the method in \cite{CellLeedOrder} in particular to show
two things: Firstly that computing the minimal polynomial of a matrix
is a crucial step in the computation of its order, and secondly that
integer factorisations of some of the numbers $q^i-1$ for $1 \le i \le n$ 
can be needed in the process.

\begin{DefProp}[Order and projective order]
Let $\F$ be a field and $M \in \F^{n \times
n}$ an invertible matrix. The \emph{order} of $M$ is the least natural
number $o$ such that $M^o$ is equal to the identity matrix\/ $\one$.
The \emph{projective order} of $M$ is the least natural number $p$
such that $M^p$ is a scalar matrix, that is, a scalar multiple of the
identity matrix. It follows immediately by division with remainder in
the exponent that $o$ divides every natural number $k$ for which 
$M^k$ is equal to\/ $\one$ and that $p$ divides every natural number $k$
for which $M^k$ is a scalar multiple of\/ $\one$. Thus $p$ divides in
particular $o$ and it follows immediately, that if $M^p = \lambda \cdot
\one$ with $\lambda \in \F$, then $o$ is $p$ times the order of the
scalar $\lambda$. \proofend
\end{DefProp}

For a monic polynomial $f \in \F[X]$ with non-zero constant term
we define the \emph{order} (\emph{projective order} resp.) of 
$X + f\F[X]$ in $\F[X]/f\F[X]$ modulo $f$ as the least natural 
number $p$ such that $X^p$ is congruent 
to $1$ (a scalar resp.) modulo $f$ or equivalently, that $X^p-1$ 
($X^p-\lambda$ resp.) is
divisible by $f$ (for some $\lambda \in \F$). 
The order and projective order
of a matrix $M$ as above and that of $X$ modulo the minimal polynomial 
of $M$ are linked by the following lemma:

\begin{Lemm}[Orders and projective orders]
Let $\F$ be a field and $M \in \F^{n \times n}$ an invertible matrix.
Then the (projective) order of $M$ is equal to the (projective) order of
$X + \mu_M \F[X]$ modulo the minimal polynomial $\mu_M$ of $M$.
\end{Lemm}
\proofbeg 
The polynomial $X^p-\lambda$ is divisible by $\mu_M$ if and
only if $M^p = \lambda \one$ for all $\lambda \in \F$.
\proofend

In the following we use this lemma to switch between matrices and
polynomials as seems appropriate for the argumentation.

We now want to discuss the computation of both the order and the
projective order of a matrix or its minimal polynomial respectively.

Let $f \in \F[X]$ be a monic polynomial with non-zero constant term. 
By the Chinese Remainder theorem the factor ring
$\F[X]/f\F[X]$ is isomorphic to
\[ \F[X]/f\F[X] \cong
   \prod_{i=1}^k \F[X]/f_i^{e_i} \F[X] \]
where $f = \prod_{i=1}^k f_i^{e_i}$ is the factorisation of $f$ into
its pairwise distinct irreducible factors $f_i$. Using this
isomorphism the (projective) order of $X+f\F[X]$ is equal to the least 
common multiple of the (projective) orders of the $X + f_i^{e_i}
\F[X]$ in $\F[X]/f_i^{e_i}\F[X]$. Thus, as a first step we factorise
$f$ completely and now determine the (projective) order of
$X+g^e\F[X]$ for an irreducible monic polynomial $g$ of degree $d$ 
with non-zero constant term.

From now on we switch again to matrices. Let $C \in \F^{d \times d}$ be 
the companion matrix of $g$ and $N$ the $(de \times de)$-block matrix with 
$C$ along the main block diagonal, $(d \times d)$-identity matrices
along the block diagonal directly above the main diagonal and zero blocks
elsewhere:
\[ N = \left[ \begin{array}{ccccc}
    C      & \one   & 0      & \cdots & 0 \\
    0      & C      & \one   & \ddots & \vdots \\
    \vdots & \ddots & \ddots & \ddots & 0 \\
    \vdots & \ddots & \ddots & C      & \one \\
    0      & \cdots & \cdots & 0      & C
\end{array} \right]. \]
The minimal polynomial of $N$ is $g^e$ which can be seen as follows:
Let $K$ be the splitting field $K$ of $g$ over $\F$. Since $g$ is
irreducible and thus has no multiple roots, the 
matrix $C$ is similar to a diagonal matrix over $K$, that is, there is an
invertible $T \in K^{d \times d}$ such that $TCT^{-1}$ is diagonal
with pairwise disjoint eigenvalues.
Thus, conjugating $N$ with the block matrix having $T$ along the
main block diagonal and permuting rows and columns suitably shows that
the Jordan normal form of $N$ over $K$ consists of $d$ blocks of
size $e$, thus every eigenvalue of $C$ occurs in the minimal
polynomial of $N$ with multiplicity $e$. Thus the minimal polynomial
of $N$ over $K$ is $g^e$ and thus also $\mu_N = g^e$.

We assume from now on that $\F$ is the finite field $\F_q$ with $q$
elements.
Writing $N = \tilde C + \tilde \one$ where $\tilde C$ is the matrix
with only $C$ along the main block diagonal and $\tilde \one$
accordingly we have $\tilde C \cdot \tilde \one = \tilde \one \cdot \tilde C$
and thus $N^k = \sum_{i=0}^k {k \choose i} \tilde C^{k-i} \tilde
\one^i$. This immediately implies that the $(d \times d)$-block
of $N^k$ in position $(j,j+i)$ is ${k \choose i}C^{k-i}$ for all
$1 \le j \le d-i$. 

Therefore the (projective) order of $N$ can be determined in the
following way: The number $\lceil \log_p(e)\rceil$ is the number
of factors $p$ that have to occur in $k$, such that all binomial
coefficients ${k \choose i} = 0$ for $1 \le i \le e$ (where we set ${k
\choose i} = 0$ for $i > k$). Let $l$ be the (projective) order of
$X+g\F[X]$ in the field $\F[X]/g\F[X]$. Then the (projective) order of
$N$ is the product $p^{\lceil \log_p(e) \rceil} \cdot l$. Note that
determining the number $l$ involves computing in the field extension
$\F[X]/g\F[X]$.

In practice one takes $q^d-1$ (resp.~$\frac{q^d-1}{q-1}$) as an upper bound
for the (projective) order of $X+g\F[X]$ and then uses the ``bounded order
algorithm'' described in \cite[Section 2]{CellLeedOrder} to get the
actual (projective) order. However, this algorithm depends on the
integer factorisation of $q^d-1$ since it uses a divide-and-conquer
approach using this factorisation.

Altogether, we can conclude the following result:

\begin{Prop}[Computing the (projective) order of a matrix]
Let $M$ be a matrix in $\F_q^{n \times n}$. Assume that the 
minimal polynomial of $M$ is known and completely factorised into
irreducible factors, and that all integer factorisations of $q^i-1$
for $1 \le i \le n$ are known. 

Then the (projective) order of $M$ can be
computed in $O(n^3 \cdot \log_2(q) \cdot \log_2(n\log_2(q)))$
field operations.
\end{Prop}
\proofbeg Use the same arguments as in the proof of 
\cite[\textsc{Order Algorithm}]{CellLeedOrder}. 
\proofend

\begin{Rem}
This shows that every improvement of the algorithm to compute the
minimal polynomial also helps to compute the (projective) order.
\end{Rem}

\section{Solving systems of linear equations}
\label{sec:syslineq}

This section describes and analyses the problem of solving a system
of linear equations. This can be formulated as follows:

Let $\F$ be a field, $Y \in \F^{m \times n}$ and $b \in \F^{1 \times n}$.
Find all $x \in \F^{1 \times m}$ with $xY = b$.

Sometimes the problem has to be solved for more than one $b$ and sometimes
it is enough to find one solution $x$. In particular the homogeneous case 
$b=0$ is important, here the set of solutions is called the \emph{nullspace}
of $M$.

The general approach to this problem is standard, thus we are mostly
interested in the complexity. The basic procedure is
Algorithm~\ref{semiechelonise}: By running
all rows of $Y$ successively through Algorithm~\ref{clean} we
find three things in the process: Firstly a subsequence
$t=(t_1,t_2, \ldots, t_{r})$ of $(1,2,\ldots, m)$ with
$1 \le t_1 < t_2 < \cdots < t_r \le m$ such that the matrix
$Y' \in \F^{r \times n}$ consisting of the rows with numbers 
$t_1, t_2, \ldots, t_r$ of $Y$ has rank $r$. Secondly we get a
semi echelon data sequence $\calY = (Y',S,T,l)$ for $Y'$ as defined
in Definition~\ref{semiecheseq}. Thirdly, for every row of $Y$ that
is a linear combination of the rows above it in $Y$, we find a linear 
relation of the rows of $Y$ and all these relations span the nullspace
of $Y$ in the end.

We leave the details and the proof that this procedure does what
we claim to the reader. The following proposition analyses the cost
of Algorithm~\ref{semiechelonise}.

\begin{algorithm}
\caption{$\quad$ \sc SemiEchelonise}
\label{semiechelonise}
\begin{algorithmic}
\STATE \textbf{Input:} A matrix $Y \in \F^{m \times n}$.
\STATE \textbf{Output:} Indices $1 \le t_1 < t_2 < \cdots < t_r$,
a matrix $Y' \in \F^{r \times n}$ with semi echelon data sequence
\STATE \mbox{}\phantom{\textbf{Output:}}
$\calY = (Y',S,T,l)$, and a matrix $N \in \F^{(m-r) \times n}$
whose rows span the nullspace of $Y$
\STATE $\calY := $ empty semi echelon sequence for rows with length $n$
\STATE $N := $ empty matrix for rows with length $n$
\STATE $t := $ empty sequence
\FOR {$i=1$ to $m$}
    \STATE $(f,\calY,a) := \textsc{CleanAndExtend}(\calY,Y[i])$
    \IF {$f = \textsc{True}$}
        \STATE Append a row to $N$ using $a$
    \ELSE
        \STATE Append $i$ to sequence $t$
    \ENDIF
\ENDFOR
\STATE \textbf{Return:} $(t,\calY,N)$
\end{algorithmic}
\end{algorithm}

\begin{Prop}[Complexity of Algorithm~\ref{semiechelonise}]
\label{semiechelon}
If $r$ is the rank of $M$, then Algorithm~\ref{semiechelonise} needs
at most
\[ \frac{r(r+1)(2r+1)}{6} + nr^2 + 2rn(m-r) + r \]
elementary field operations. This is $O(nm^2)$.
\end{Prop}
\proofbeg
We add up the maximal number of elementary field operations needed
by the calls to \textsc{CleanAndExtend} using
Proposition~\ref{PropCleanAndExtend}. We start with the calls that
extend the semi echelon data sequence:
\begin{eqnarray*}
\sum_{i=0}^{r-1} \left( (2i+1)n + (i+1)^2 + 1 \right) 
  &=& \frac{r(r-1)}{2} \cdot 2n + rn + \frac{r(r+1)(2r+1)}{6} + r \\
  &=& nr^2 + \frac{r(r+1)(2r+1)}{6} + r,
\end{eqnarray*}
where we use Formulas \ref{formels1} and \ref{formels2}. In the worst
case that the first $r$ rows of $Y$ are linearly independent the
number of operations used to clean the rows that do not extend
the semi echelon data sequence is at most $(m-r)\cdot 2rn$.
For the asymptotic statement we use $r \le \min\{ m,n \}$.
\proofend

\medskip
After Algorithm~\ref{semiechelonise} has completed the rows of the
matrix $N$ span the solution space of the system of linear equations
$xY = 0$. For a given $0 \neq b \in \F^{1 \times n}$ we have to run
the ``cleaning'' part of Algorithm~\ref{clean} once.
The result is either that the system has no solutions or a vector
$a \in \F^{1 \times r}$ such that $b = aS = aTY'$ for the matrix $S$ in
$\calY$ that is in row semi echelon form. Since $Y'$ consists of a 
selection of the rows of $Y$ the vector $aT$ contains all the necessary
information to put together a solution $x$ with $b = xY$. The set
of all solutions is then equal to $\{ x+n \mid n \in \rsp(N) \}$.

\begin{Cor}[Solving a linear system of equations]
\label{solvelinsys}
The set of solutions of the system of linear equations $xY=0$ can be
read off from the output of Algorithm~\ref{semiechelonise}. For
the system $xY=b$ it can be derived using
$2rn + r(r+1)$ elementary field operations.

Thus, such a system can be solved in $O(nm^2)$ elementary field operations.
\end{Cor}
\proofbeg
The above method needs one call to \textsc{CleanAndExtend} using at most
$2rn$ elementary field operations and the vector matrix multiplication
$aT$ with the lower triangular matrix $T$ needs at most
$2\cdot \frac{r(r+1)}{2}$ elementary field operations.
\proofend

\section{Inverting matrices}
\label{sec:invert}

Inverting a matrix $M \in \F^{n \times n}$ is the same as solving the
system of linear equations $xM = e_i$ for all the standard basis
vectors $e_i \in \F^{1 \times n}$. By Proposition~\ref{semiechelon}
and Corollary~\ref{solvelinsys} we get:

\begin{Prop}[Inverting a matrix]
Let $M \in \F^{n \times n}$ be an invertible matrix. Then the inverse
can be computed using at most
\[ \frac{n(n+1)(2n+1)}{6} + 4n^3 + n^2 + n \]
elementary field operations, which is asymptotically less than $5n^3$
and this is $O(n^3)$.
\end{Prop}
\proofbeg
We first run Algorithm~\ref{semiechelonise} using at most
\[ \frac{n(n+1)(2n+1)}{6} + n^3 + n \]
elementary field operations by Proposition~\ref{semiechelon}. Then
we solve $xM = e_i$ for $1 \le i \le n$ using a total of
\[ n(2n^2+n(n+1)) = 3n^3+n^2 \]
elementary field operations. Summing up gives the claim.
\proofend

\section{The discrete logarithm problem}
\label{thedlp}

State problem, cite currently best possible solution, give overview
over where it can be solved efficiently.


% this is a part of the habilitation thesis of Max Neunhoeffer

\chapter{The Meat-Axe}
\label{chap:meataxe}

This chapter basically describes, what the Meat-Axe can do for us.

\section{Composition series}

\section{Homomorphisms}

\section{Socles and radicals}

\section{Complexity analysis of the Meat-Axe}

% this is a part of the habilitation thesis of Max Neunhoeffer

\chapter{Composition trees}

After lots of preparations in the previous chapters
in this chapter the main part of the whole book begins. We begin to talk
about the recognition of matrix groups. We start by formulating
the concept of ``constructive recognition'' and explain the reasoning
behind it in Section~\ref{constrrecog}. In this section we develop
a preliminary formulation of the main problem which will be refined
in the next Section~\ref{recapproach}. There we explain the fundamental
approach to achieve
constructive recognition by means of a ``composition tree''.
The refinement of the formulation of the main
problem there is necessary to allow for an efficient recursive solution.
We chose to ``develop'' the problem formulation in this way in the hope
to provide both a gentle introduction as well as a precise final
formulation with good reasons for its complex structure.

In the following sections we develop a
framework for group recognition that is not only suitable for matrix
groups and projective groups but also allows for the implementation of
the asymptotically best algorithms to handle permutation groups. In
this framework we can switch between different representations of
groups within one composition tree which allows for example to use 
permutation group methods during matrix group recognition, provided
we find some set our matrix group is acting upon.

The general approach to build a composition tree is not new and is
already described in \cite{MatGrpProj}. What is new in our approach
is first the abstraction to allow for different representations
intermixed in the same composition tree and secondly the fact that
we change the generating set in each node of the composition tree
which dramatically decreases the length of the resulting straight
line programs. 

The contents of this chapter stem from joint work with \'Akos Seress
and are an elaboration on the article \cite{AkosMaxISSAC}.

\section{Constructive recognition}
\label{constrrecog}

There are at least two fundamentally different ways to represent groups on
a computer. The first uses a presentation of the group and then
expresses group elements as words in a free group representing
cosets of the normal subgroup generated by the relations. The second
approach uses an ambient group whose elements can be represented and
multiplied directly on the computer, and the group is defined by giving
a list of generators. As ambient groups one can use symmetric groups, general
linear groups or projective groups, since we can store and manipulate
permutations and matrices efficiently on a computer.

In this book we concentrate on the second approach. To formalise our
problem we first write down our assumptions about the ambient group
and then formulate the fundamental problem.

\begin{Hyp}[Ambient group]
\label{ambient}
When we speak about an \emph{ambient group} we mean a finite group that can
be represented on a computer such that we can perform the following tasks:
\begin{itemize}
\item Store and compare group elements using a bounded amount of memory
per element.
\item Multiply group elements.
\item Invert group elements.
\item Compute the order of a group element.
\end{itemize}
\end{Hyp}

\begin{Rem}
All finite symmetric groups fulfil the hypotheses in Section~\ref{ambient}.
We call subgroups of finite symmetric groups \emph{permutation groups}.

For a prime power $q$ the groups $\GL(n,q)$ and $\PGL(n,q)$ also fulfil
the hypotheses. We call subgroups of $\GL(n,q)$ \emph{matrix groups}
and subgroups of $\PGL(n,q)$ \emph{projective groups}.
\end{Rem}

\begin{Problem}[Constructive recognition --- first formulation]
\label{ProbCR1}
Let $\GG$ be an ambient group in the sense of Hypothesis~\ref{ambient} and 
assume that we are given a generating tuple $(g_1, \ldots, g_k) \in
\GG^k$ for a group
$G := \left< g_1, \ldots, g_k \right> \le \GG$. 

We say that we have \emph{recognised $G$ constructively} if we have 
computed $|G|$ and
prepared a procedure that does the following: Given $g \in \GG$,
decide whether $g \in G$ and if so, express $g$ as a straight line program
in $(g_1, \ldots, g_k)$. This latter procedure is called \emph{constructive
membership test}.
\proofend
\end{Problem}

Note that this formulation will be refined in Sections~\ref{ProbCR2} 
and~\ref{ProbCR3}.

\smallskip
The problem as posed in Problem~\ref{ProbCR1} can be solved easily by just
enumerating the finite group $G$ completely. However, this is neither
a sensible approach nor very interesting. The crucial point missing
in Problem~\ref{ProbCR1} is that we want to do constructive recognition
\emph{efficiently}. To express what we mean by that we use complexity
theory. The class of problems is given by all possible tuples of
generators of subgroups of our ambient groups. We have to specify
what we mean by ``input size'':

\begin{Def}[Input size parameters]
\label{inputsize}
Let $\GG$ be an ambient group in the sense of Hypothesis~\ref{ambient}
and $(g_1, \ldots, g_k) \in \GG^k$. If $\GG$ is either $\GL(n,q)$ or
$\PGL(n,q)$, then the input size of our problem is measured by $n$, $k$
and $\log_2(q)$. If $\GG$ is the symmetric group on $n$ points then the
input size is measured by $n$ and $k$. We silently set $q := 1$ in that
case such that we can uniformly speak of the input size parameters $n$, $k$
and $q$ in all cases.
\end{Def}

\begin{Rem}
Note that we use $\log_2(q)$ as one of the parameters of the input size
rather than $q$ itself. The reason behind this is that one needs
$\log_2(q)$ bits of data to store one finite field element of $\F_q$.
This means that the amount of storage needed for the input matrices is
proportional to $kn^2\log_2(q)$. Of course, the fact that the discrete
logarithm problem in $\F_q$ prevents us from finding a real polynomial
time algorithm for constructive recognition stems from this decision
(compare Section~\ref{thedlp}).
\proofend
\end{Rem}

There is another reason why the formulation in Problem~\ref{ProbCR1} is not
practical. Namely, the straight line programs for elements $g \in G$
written in terms of the original generators can be extremely long. In
addition, for most generating tuples of $G$ it can be tedious if not
impossible to find a method to express group elements in $G$ as straight
programs in these generators. Usually, a recognition procedure first finds
another ``nice'' generating tuple for $G$ for which a good constructive
membership testing procedure is available. To solve the original problem,
such recognition procedures remember, how they got the nice generating
tuple from the original generating tuple by means of a single, rather long
straight line program.

With the above notion of efficiency and these arguments in mind we can now 
formulate a preliminary version of the constructive recognition problem:

\begin{Problem}[Constructive recognition --- preliminary formulation]
\label{ProbCR2}
Let $\GG$ be an ambient group in the sense of Hypothesis~\ref{ambient} and 
assume we are given a generating tuple $(g_1, \ldots, g_k) \in
\GG^k$ for a group $G := \left< g_1, \ldots, g_k \right> \le \GG$. 

We say that we have \emph{recognised $G$ constructively} if we have 
computed $|G|$ and a
generating tuple $( g'_1, \ldots, g'_l )$ for $G$, for which we have
prepared a procedure that does the following: Given $g \in \GG$,
decide whether $g \in G$ and if so, express $g$ as a straight line program
in $(g'_1, \ldots, g'_l)$. The new generating tuple may or may not be the
same as the original one.

We assume that we have factorised all integers $q^i-1$ for $1 \le i \le n$
and that we can solve the discrete logarithm problem in all fields
$\F_{q^i}$ for $1 \le i \le n$ efficiently. 

We call the first phase until $G$ is
recognised constructively the \emph{recognition phase} and the second
phase the \emph{(constructive) membership test phase}.

The aim is that the algorithms for both phases have a runtime that is proved
to be bounded by a fixed polynomial in the input size parameters $n$, $k$
and $q$ in the sense of Definition~\ref{inputsize}.
\proofend
\end{Problem}

Note that this formulation will be refined in Problem~\ref{ProbCR3}.

\begin{Rem}[Randomised algorithm]
We are content with using Las Vegas or Monte Carlo algorithms (see
Section~\ref{montevegas}) provided that we can later on verify our results
deterministically, even if this takes longer than the randomised
recognition step. However, also the verification phase should have a
runtime that is proved to be bounded by a fixed polynomial in the input
size parameters $n$, $k$ and $q$.
\end{Rem}

\section{A recursive approach --- reductions}
\label{recapproach}

Traditionally, the constructive recognition problem for permutation
groups is solved by computing a stabiliser chain and a set of strong
generators using the Schreier-Sims method (see~\cite{Si} and \cite{Ser}). 

Although an analogous method can be used for matrix and projective groups,
this becomes infeasible already for very small values of $n$ since the
occurring orbit lengths can be very large. Also for large base permutation
groups (for a formal definition see Section~\ref{permgrps}) like the
symmetric and alternating groups in their natural representation there
are asymptotically better methods because the stabiliser chains are very long.

Therefore one seeks \emph{reductions} in these cases. A reduction is
a surjective group homomorphism $\varphi : G \to H$ that can be computed
explicitly and for which $H$ is again a subgroup of a possibly different
ambient group. In addition, the homomorphism $\varphi$ must either have
a non-trivial kernel or the group $H$ must be ``easier to handle'' in some
sense. This can for example mean that one of the values $n$, $k$ and
$q$ is smaller for $H$ than for $G$ while the others remain unchanged,
or that the expression $kn^2\log_2(q)$ becomes smaller such that we need
less memory to store the generators for $H$ than those for $G$.
Another possibility is that for a matrix group $G$ the homomorphic image
$H$ is a permutation group.

Having found a reduction $\varphi : G \to H$ one first tries to recognise
$H$ constructively, then computes generators for the kernel $N$ of
$\varphi$, recognises $N$ constructively and then puts everything together
to achieve a constructive recognition of $G$ from all other intermediate
results. This procedure can be repeated recursively for the image $H$ and
the kernel $N$ such that we end up with a tree (see
Figure~\ref{comptreefig}), where the nodes correspond
to subfactors of the original $G$ and where we have to recognise
the groups in the leaf nodes constructively by other means without
a further reduction.

\begin{figure}
\begin{center}
\includegraphics[width=2.5in]{comptree}
\end{center}
\caption{A composition tree}
\label{comptreefig}
\end{figure}

To make this recursion work we have to say quite carefully what has to
happen at a non-leaf node $G$ such that we actually have solved the
constructive recognition problem for the group $G$ at this node provided
we have solved the corresponding problem for the homomorphic image
$H=\varphi(G)$ and the kernel $N = \ker(\varphi)$. Note that the problem
in our first formulation in Problem~\ref{ProbCR1} could be solved
recursively in this way. However, for our formulation in Problem~\ref{ProbCR2}
this is not possible since we can only express group elements in the nice
generators and not in the original ones. This is the reason why we have to
refine the formulation once more. This time we present the final
formulation:

\begin{Problem}[Constructive recognition --- final formulation]
\label{ProbCR3}
Let $\GG$ be an ambient group in the sense of Hypothesis~\ref{ambient} and 
assume that we are given a generating tuple $(g_1, \ldots, g_k) \in
\GG^k$ for a group $G := \left< g_1, \ldots, g_k \right> \le \GG$. 

We say that we have \emph{recognised $G$ constructively} if we have 
\begin{itemize}
\item computed $|G|$ and
\item a generating tuple $( g'_1, \ldots, g'_l )$ for $G$, for which we have
\item 
prepared a procedure that does the following: Given $g \in \GG$,
decide whether $g \in G$ and if so, express $g$ as a straight line program
in $(g'_1, \ldots, g'_l)$. Also, we have
\item prepared a procedure that computes, given preimages $(p_1, \ldots,
p_k)$ of $(g_1, \ldots, g_k)$ under some (surjective) homomorphism $\psi
: \hat G \to G$ for some group $\hat G$, preimages under $\psi$ of the
nice generators $(g'_1, \ldots, g'_l)$.
\end{itemize}
The new generating tuple may or may not be the same as the original one.
The last point can for example be achieved by storing a straight line
program that expresses $(g'_1, \ldots, g'_l)$ in terms of $(g_1, \ldots,
g_k)$.

We assume that we have factorised all integers $q^i-1$ for $1 \le i \le n$
and that we can solve the discrete logarithm problem in all fields
$\F_{q^i}$ for $1 \le i \le n$ efficiently. 

We call the first phase until $G$ is
recognised constructively the \emph{recognition phase} and the second
phase the \emph{(constructive) membership test phase}.

The aim is that the algorithms for both phases have a runtime that is proved
to be bounded by a fixed polynomial in the input size parameters $n$, $k$
and $q$ in the sense of Definition~\ref{inputsize}.

In this book we refer to this problem as the \emph{constructive recognition
problem (CRP)}.
\proofend
\end{Problem}

We claim that this final formulation of the constructive recognition
problem has the property that if $\varphi : G \to H$ is a reduction, then
having solved the constructive recognition problem for $H$ and $N :=
\ker(\varphi)$ in fact suffices to solve the constructive recognition
problem for $G$. To this end we have to
clearly say what the original generators and the new, ``nice'' generators
of our groups $G$, $H$ and $N$ are. This is the purpose of the following
definition.

\begin{Def}[Reduction node]
\label{reducnode}
Let $G = \left< g_1, \ldots, g_k \right> \le \GG$ be as in 
Problem~\ref{ProbCR3}.
We say that we have set up a \emph{reduction node} for $G$, if we have
performed the following steps:
\begin{itemize}
\item[(1)] Finding a surjective group homomorphism $\varphi : G \to H$ for
which we have a procedure to compute $\varphi(g)$ for all $g \in G$.
If this procedure is given an element $g \in \GG \setminus G$ then it
returns either {\fail} or an arbitrary element in the ambient group of $H$.
\item[(2)] Solving Problem~\ref{ProbCR3} for $H = \left< \varphi(g_1),
\ldots, \varphi(g_k) \right>$ with nice generators $(h_1, \ldots, h_s)$.
\item[(3)] Computing preimages $(h'_1, \ldots, h'_s) \in G^s$ under $\varphi$
of the nice generators $(h_1, \ldots, h_s)$ of $H$ using the solution
of (2).
\item[(4)] Computing generators $(n_1, \ldots, n_t) \in N^t$ with $N :=
\ker(\varphi)$ together with a straight line program expressing
$(n_1, \ldots, n_t)$ in terms of $(g_1, \ldots, g_k)$.
\item[(5)] Solving Problem~\ref{ProbCR3} for $N = \left< n_1, \ldots, n_t
\right>$ with nice generators $(n'_1, \ldots, n'_u) \in N^u$.
\end{itemize}
For a reduction node we call the tuple $(h'_1, \ldots, h'_s, n'_1, \ldots,
n'_u)$ its nice generators.
\end{Def}

\begin{Prop}[The recursion works]
Let $G = \left< g_1, \ldots, g_k \right> \le \GG$ be as in 
Problem~\ref{ProbCR3}.
If we have set up a reduction node for $G$ (in the sense of
Definition~\ref{reducnode}), then we have solved
Problem~\ref{ProbCR3} for $G$.
\end{Prop}
\proofbeg
The group order $|G|$ is equal to the product $|H| \cdot |N|$ since
$\varphi$ is surjective and $N = \ker(\varphi)$.

Recall that we call the tuple $(h'_1, \ldots, h'_s, n'_1, \ldots, n'_u)$
the nice generators for $G$. Since 
\[ (\varphi(h'_1), \ldots, \varphi(h'_s))
= (h_1, \ldots, h_s) \] 
generates the epimorphic image $H$ of $G$ and
$(n'_1, \ldots, n'_u)$ generates the kernel $N$ the tuple of nice
generators is a generating tuple for $G$.

Given $g \in \GG$, we can test membership in $G$ constructively as follows:
First we try to map $g$ with $\varphi$. If this fails, the element $g$
does not lie in $G$ by our assumptions in \ref{reducnode}.(1). Otherwise
call the image $h \in H$. Note that in this case it can still be that
$g$ is not contained in $G$, regardless whether $h$ lies in $H$ or not! 

We then try to express $h$ as a straight 
line program $s_1$ in the nice generators $(h_1, \ldots, h_s)$ of $H$. If this
fails, then $h$ is not contained in $H$ and thus $g$ is not contained in
$G$. Otherwise we evaluate $s_1$ with inputs
$(h'_1, \ldots, h'_s)$, the preimages of the nice generators of $H$ under
$\varphi$. If $g$ is contained in $G$, then the result $h'$ is a preimage 
under $\varphi$ of $h = \varphi(g)$. Thus $n := h'^{-1}\cdot g$ is contained
in the kernel $N$ of $\varphi$. We try to express $n$ as a straight
line program $s_2$ in $(n'_1, \ldots, n'_u)$. If this fails, then $n$
is not contained in $N$ and thus $g$ is not contained in $G$. Otherwise,
we put together the straight line programs $s_1$ and $s_2$ to form
one big straight line program $s$ that reaches $h'n$ from the input 
$(h'_1, \ldots, h'_s, n'_1, \ldots, n'_u)$. In this case we have proved
that $g$ lies in $G$.
\proofend

\medskip
We conclude this section with a short overview over the rest of the chapter
and an outlook onto the following ones:
We have now found our final formulation of the constructive
recognition problem and we have shown a general approach for its solution
by means of reduction. In the next Section~\ref{findkernel} we explain how
to find the kernel $N$ of a homomorphism as in Definition~\ref{reducnode}
once we have constructively recognised the epimorphic image $H$. After that
in Section~\ref{findhom} we describe a generic framework to organise the
finding of reductions using a database of different methods. Finally we
explain how this whole framework can be used to implement the
asymptotically best algorithms for arbitrary permutation groups.
The following chapters cover the question how to actually find reductions
and solve the constructive recognition problem for matrix groups and
projective groups.

\section{Finding the kernel of a homomorphism}
\label{findkernel}

In this section we assume that we are in the situation described in
Problem~\ref{ProbCR3} and want to construct a reduction node in the sense
of Definition~\ref{reducnode}. We assume that we can produce evenly
distributed random elements in $G$ (see Section~\ref{randomelts}).

Assume that we have already found a homomorphism
$\varphi : G \to \HH$ into some ambient group $\HH$ in the sense of
Hypothesis~\ref{ambient}, such that we can map arbitrary group elements
$g \in G$ with $\varphi$. Then we set $H := \left< \varphi(g_1), \ldots,
\varphi(g_k) \right>$ and constructively recognise $H$ which produces
among other things nice generators $(h_1, \ldots, h_s)$ of $H$. Another
result of the recognition is that we can compute preimages $(h'_1, \ldots,
h'_s) \in G^s$ of the $(h_1, \ldots, h_s)$ under $\varphi$.

Equipped with all this data and methods we can now do the following
for an arbitrary element $g \in G$: Map it with $\varphi$ into $H$
and call $h := \varphi(g)$. Then find a straight line program $s$ reaching
$h$ from $(h_1, \ldots, h_s)$ and evaluate it on $(h'_1, \ldots, h'_s)$
to get an element $h' \in G$ with the property that
$\varphi(g)=\varphi(h') = h$, thus $h'^{-1}\cdot g$ lies in the kernel $N$
of $\varphi$. For this construction we have:

\begin{Prop}[Even distribution of kernel elements]
\label{evendistker}
If $x_1, x_2, \ldots, x_m$ are evenly distributed random elements in $G$,
then the above procedure produces evenly distributed random elements in
$N$.
\end{Prop}
\proofbeg
Since $H \cong G/N$, the elements of $H$ correspond to cosets of $N$ in
$G$. Our constructive membership test in $H$ produces a unique straight
line program for every element $h \in H$ and the evaluation of those
programs on the preimages $(h'_1, \ldots, h'_s)$ thus chooses exactly one
element of $G$ from each coset of $N$. Since the $x_i$ are evenly
distributed in the whole of $G$ their images under $\varphi$ are
distributed evenly in $H$. Multiplying $x_i$ with the inverse of the
chosen coset representative in the coset $x_iN$ amounts to mapping
$x_i N$ bijectively onto $N$. Again by the even distribution of the $x_i$
it follows, that the elements $y_i$ are evenly distributed in $N$.
\proofend

\smallskip
Proposition~\ref{evendistker} now allows to produce evenly distributed
elements in $N$. To find a generating tuple of $N$ we simply produce
a certain amount of random elements. Since every proper subgroup of $N$ 
has index at least $2$ in $N$, the probability to increase the
subgroup that is generated by the already produced elements is at least 
$1/2$ for every new element. Thus the probability to find a generating
set for the whole of $N$ is very high provided we produce enough random
elements.

\medskip
Sometimes, we not only recognise $H$ constructively in the sense of
Problem~\ref{ProbCR3} but also find a presentation in terms of the 
nice generators. This happens mostly but not exclusively if $H$ is an 
almost simple group. In that case, we have the following result:

\begin{Prop}[Kernel if a presentation of $H$ is known]
\label{kernelpres}
Assume that in the above situation $H$ is isomorphic to the
finitely presented group generated by generators $(h_1, \ldots, h_s)$
subject to relations $(r_1, \ldots, r_m)$ given as a straight line
program $s$ in terms of the $(h_1, \ldots, h_s)$. Let $(r'_1, \ldots,
r'_m)$ be preimages of the $(r_1, \ldots, r_m)$ under $\varphi$. 
Let further $x_i$ for $1 \le i \le k$ be elements in $\left< h'_1, \ldots, 
h'_s \right>$ with $\varphi(x_i) = \varphi(g_i)$ obtained by the procedure 
described directly before Proposition~\ref{evendistker}, and set 
$y_i := x_i^{-1} \cdot g_i \in N$. Then the kernel $N$ of
$\varphi$ is the normal closure in $G$ of the group generated by
$(y_1, \ldots, y_k, r'_1, \ldots, r'_m)$.
\end{Prop}
\proofbeg
Clearly, all the elements $y_i$ and $r'_i$ lie in $N$. Since $N$ is a
normal subgroup of $G$, the normal closure $\tilde N$ of the group generated
by $(y_1, \ldots, y_k, r'_1, \ldots, r'_m)$ is contained in $N$.

On the other hand, every element $n$ of $N$ is a product of the generators
$g_i$. Thus, setting $n = g_{i_1} \cdots g_{i_t}$ for some numbers
$i_j \in \{ 1, \ldots, k \}$, we get:
\[ n = x_{i_1} x_{i_1}^{-1} g_{i_1} \cdots x_{i_t} x_{i_t}^{-1} g_{i_t}
     = x_{i_1} y_{i_1} \cdots x_{i_t} y_{i_t}
     = \tilde n \cdot x \]
where $\tilde n$ is a product of some $G$-conjugates of the $y_{i_j}$
and $x$ is a product of the $h'_i$ that lies in $N$. Since the $h'_i$
are preimages of the generators $h_i$ of $H$ and $r'_i$ are preimages
of the relations $r_i$ the element $x$ lies in the normal closure
of the subgroup generated by the $r'_i$ and thus $\tilde N = N$.
\proofend

\begin{Rem}[Computing the kernel with a given presentation of $H$]
Using the results of the constructive recognition of $H$ together we
can explicitly compute preimages $r'_i$ of the relations $r_i$ since
the latter are given as straight line programs in terms of the nice
generators of $H$. Thus we can explicitly compute all the generators in
Proposition~\ref{kernelpres}. The normal closure can be computed using
the methods in \cite[Chapter 2]{Ser}.
% FIXME: add reference
\end{Rem}

In this section we have shown that to set up a reduction node only needs
an explicitly computable homomorphism $\varphi$ from $G$ into some
ambient group $\HH$. Given this, one can constructively recognise the
image, compute generators for the kernel and constructively recognise it,
thereby fulfilling all requirements for a reduction node.


\section{Finding homomorphisms and nice generators}
\label{findhom}

During the recursive recognition procedure we have to solve
Problem~\ref{ProbCR3} at every node of the composition tree, either
by finding a reduction and setting up a reduction node or by solving 
the constructive recognition problem directly. In this section we describe
a generic framework to organise an algorithm to achieve this.

We have to explore every group $G$ as in Problem~\ref{ProbCR3} 
occurring in our tree to find out whether we 
can solve the constructive recognition 
problem, or what kind of homomorphism can be 
applied to it. To this end, the framework holds a collection
of so-called ``find homomorphism'' methods in stock. A find homomorphism
method's objective is either to find a homomorphism $\varphi: G \to H$
onto some subgroup $H$, thereby setting up a reduction node and 
creating a new non-leaf node, or to
solve the constructive recognition 
problem directly, which can but does not always have to find
an isomorphism to some known group.

For each type of groups (permutation groups, matrix groups, and black-box
groups), the system has a database of find homomorphism methods.
We call the procedure that decides, which methods to try and in which order 
the ``method selection''. For this purpose we define a very simple and yet 
versatile algorithm, which we will describe now. It is usable independently
from group recognition, but we will explain it here in the context of
our generic recognition procedure. Note that this new method selection
procedure is not to be confused with the {\GAP} (see \cite{GAP4}) method
selection. 

The group recognition procedure for a node just calls the generic method
selection procedure with the database of find homomorphism methods
corresponding to the type of the group.

The methods in each database are ranked, thereby defining a total
order. The method selection procedure calls the methods one after another,
starting with high ranks. A find homomorphism method reports back to
the generic procedure by returning one of the four values in
Table~\ref{methselresults}.

\begin{table}[ht]
\begin{tabular}{lp{4in}}
\texttt{true}: &
   The method was successful and has either set up a
   leaf or a non-leaf node. For details see below. \\
\texttt{fail}: &
   The method has failed to find a homomorphism or
   to solve the constructive recognition problem, at least temporarily. \\
\texttt{false}: &
   The method has failed and will do so always for
   the group $G$ in question, such that there is no point in trying
   this method again for the group $G$. \\
\texttt{NotApplicable}: &
   The method is currently not applicable
   but it might become applicable later, provided new knowledge is
   found out about the group $G$.
\end{tabular}
\caption{Possible results of a find homomorphism method}
\label{methselresults}
\end{table}

The first case is the only one that terminates the recognition procedure
for the current node in the composition tree.
All other cases make it necessary to try other methods. The difference
between these latter three cases lies in the fact, how the generic
procedure chooses the next method called. If a method returns
\texttt{NotApplicable}, then the method selection just calls the next
method in the database. In the other two cases \texttt{false} and 
\texttt{fail}, the method selection again starts with the highest ranked
method, but skipping all methods that have already been tried and
have failed by returning either \texttt{false} or \texttt{fail}.

When all available methods either have declared themselves
\texttt{NotApplicable} or have failed, then the method selection 
starts all over again, now calling methods again that have returned
\texttt{fail} once but of course skipping methods that have returned
\texttt{false}. This whole process is repeated until each method has
failed a configurable number of times, when the method selection
finally gives up.

We hope that this design is simple enough to keep an overview of what is
tried in which order to prove the whole algorithm to work correctly,
and versatile enough to implement a wide range of different algorithms,
in which ``trying different methods'' is involved. We now explain
the rationale behind some of the features of this procedure.

The idea behind the fact that after a method having returned
\texttt{fail} or \texttt{false} the method selection starts again
with the highest ranked method is that even a failed method might
have found out new information about the group, thereby making
higher ranked methods, that have refused to work earlier by
returning \texttt{NotApplicable}, newly applicable.

In the {\GAP} system, information about a group $G$ is collected in
so-called {\em attributes} (for example,
a permutation group object may store an attribute whether it is transitive
or not), but we may also acquire the information that 
$G$ is simple, or that it is solvable, or we may know $|G|$, etc. 
Note that at any given time, the value of an attribute for a given group
object can already be computed or not, and the group object can learn new
information about itself during its lifetime. This feature is already
part of the {\GAP} system library. 

Further attributes may be
computed when the method selection tries to apply
different find homomorphism methods while processing the current group $G$.
Therefore, a find homomorphism method can just look whether or not
a certain attribute is already known and then decide if it starts
to work or declares itself \texttt{NotApplicable}. With new attributes
being computed, this decision might be changed. By convention, a method
should never use much computation time to find out that it is not
applicable.

The idea behind a method returning \texttt{fail} is that often
randomized algorithms are used that have the potential to fail,
but still may succeed when tried again. The ranking and the failure 
probabilities of course have to be tuned carefully to assemble
a sensible recognition system.

Once a method returns \texttt{true} it reports whether it has found a
reduction or has solved the constructive recognition problem by other
means. In the first case the generic machinery sets up a recognition node
as described in Sections~\ref{recapproach} and \ref{findkernel}. In the
other case a leaf in the composition tree is built and the find
homomorphism method that returned \texttt{true} is responsible for the
data and methods to do constructive membership testing in the group 
of the node. In this way, a composition tree is built recursively using the
database of find homomorphism methods together with the generic code
organising the recursive framework.

Another important feature of this framework is the following. Quite often,
a find homomorphism method finds out information about the group in
question that can be used further down in the composition tree most notably
in the homomorphic image or the kernel of the group homomorphism found.
For example it might already know that it has set up an isomorphism such
that the kernel is trivial or it might know that the homomorphic image of a
matrix group is a permutation group. To communicate such information to the
methods used further down in the tree there is an infrastructure to hand
down so-called ``hints''. These hints might for example consist of additional
knowledge about the group such that certain find homomorphism methods
are particularly well suited. Having this type of information available
further down in the tree helps choosing the best find homomorphism method
first. Furthermore, even a failed method can store already 
obtained data about the group in the data structure of the node it tried
to work on. In this way methods that are called later in the method
selection process can profit from this additional information. In the
chapters to come in this book we describe find homomorphism methods
and mention such hints that can help other methods along with the
description of the methods.

We illustrate the generic method selection described in this section
in the next Section~\ref{permgrps} where we explain how it is used to
glue together the asymptotically best algorithms currently known for
the handling of permutation groups to set up a constructive recognition
algorithm for the case that the ambient group is a symmetric group.


\section{Asymptotically best algorithms for permutation groups}
\label{permgrps}

Traditionally, the constructive recognition problem for permutation
groups is solved by computing a stabiliser chain and a set of strong
generators using the Schreier-Sims method (see~\cite{Si} and \cite{Ser}). 

However, this method does not work very efficiently for so-called
``large base groups''. We start by developing the necessary language
from the complexity theory of permutation groups.

\begin{Def}[Base of a permutation group]
Let $G \le S_n$ be a subgroup of the symmetric group on $\{1,\ldots,n\}$.
A tuple $B \in \{ 1, \ldots, n\}^l$ is called a \emph{base of $G$ with
length $l$}, if only the identity of $G$ fixes every point in $B$: 
\[ \{ g \in G \mid b_i^g = b_i \mbox{ for all } 1 \le i \le l \} = \{ \id \}. \]
The \emph{stabiliser chain} belonging to a base $B$ is the chain of
subgroups
\[ G \ge G_{b_1} \ge G_{b_1,b_2} \ge \cdots \ge G_{b_1,\ldots,b_k} =
\{\id\} \]
where $G_{b_1, \ldots, b_j} =
\{ g \in G \mid b_i^g = b_i \mbox{ for all } 1 \le i \le j \}$.
\end{Def}

Obviously, the length of every stabiliser chain for a permutation group 
$G$ is at least as big as the length of a shortest base of $G$.
This is the reason why computing stabiliser chains is not efficient
for groups without a ``reasonably short'' base. Of course, the
Schreier-Sims method does not necessarily find a shortest possible base.
However, one can implement the method such that in the resulting
stabiliser chain all inclusions are proper.
Therefore, we consider not the shortest possible base length but
an upper bound for the length of a base:

\begin{Prop}[Maximal base length]
If $G \le S_n$ then every base of $G$, for which all inclusions in its
stabiliser chain are proper, has length at most $\lceil \log_2(|G|) \rceil$.
\end{Prop}
\proofbeg
The index of one stabiliser in the previous one in the stabiliser chain is
at least $2$.
\proofend

To formulate precise complexity statements we have to talk about families
of permutation groups:

\begin{Def}[Families of small and large base groups]
Let $\calF$ be a family of permutation groups, all embedded into
possibly different symmetric groups $S_n$ and given by generating tuples. 
For a $G \in \calF$ we denote 
by $n(G)$ the $n$ of the symmetric group $S_n$ into which $G$ is embedded
and by $k(G)$ the number of generators by which $G$ is defined.

The family $\calF$ is called \emph{a family of small base groups} if there
is a positive constant $c$ such that $\log_2(|G|) \le \log^c_2(n(G))$ for all
$G \in \calF$. If there is no such constant then $\calF$ is called
\emph{a family of large base groups}.
\end{Def}

\begin{Rem}[A single permutation group]
Note that the terms ``large base group'' and ``small base group'' are
\emph{not defined} for a single permutation group $G$ although one is
sometimes tempted to call a member of a family of large base groups
a ``large base group''. However, complexity statements about algorithms
only make sense for families of inputs and thus it is only sensible to
talk about ``small'' or ``large'' bases in the context of families of
permutation groups.
\end{Rem}

The motivation for the terms ``family of large/small base permutation
groups'' is the following result:

\begin{Theo}[Complexity of the Schreier-Sims method]
Let $\calF$ be a family of small base groups. Then a randomised version
of the Schreier-Sims method computes a base and strong generators
for any $G \in \calF$ in nearly-linear time in the input size $k(G)\cdot
n(G)$, that is the runtime is bounded by a function in
$O(k(G)n(G) \cdot \log^c_2(n(G)))$ for some constant $c$.
\end{Theo}
\proofbeg See \cite{nearlylin} or \cite[Theorem 4.5.5]{Ser}. \proofend

\medskip
If a permutation group $G \le S_n$ is given by a list of generators, it
is at first glance not clear, whether it has a short base. Of course,
we want to avoid to compute a base by an application of the Schreier-Sims
method if there is no short base.

To solve Problem~\ref{ProbCR3} for a permutation group we use the method
selection procedure described in Section~\ref{findhom}. In the following we
describe the methods that are used in this process. An overview of these 
methods is given in Table~\ref{permdb}. The basic idea is to recognise
and handle alternating composition factors without computing a stabiliser 
chain.

\begin{table}[ht]
\begin{center}
\begin{tabular}{|c|l|l|c|}
\hline
Rank & Name & Action & Hom/Leaf \\
\hline
\hline
50 & \texttt{NonTransitive} & Restrict to orbit & Hom \\
\hline
40 & \texttt{Giant} $A_n/S_n$ & Find standard generators & Leaf \\
\hline
30 & \texttt{Imprimitive} & Find block system & Hom \\
\hline
20 & \texttt{Jellyfish} & Find standard generators & Leaf \\
\hline
10 & \texttt{StabilizerChain} & Compute a base and strong generators & Leaf \\
\hline
\end{tabular}
\end{center}
\caption{Find homomorphism methods for permutation groups}
\label{permdb}
\end{table}

The method with the highest rank $50$ is the reduction method called
\texttt{NonTransitive} for intransitive groups. It first tests whether
$G$ is transitive, and if not constructs an explicitly
computable homomorphism by restricting the action to one of the orbits in the
permutation domain. If $G$ turns out to be transitive, then the
\texttt{NonTransitive} method immediately returns \texttt{false}
indicating, that it can never be successful for the group $G$.

The next method \texttt{Giant} with rank $40$ tries to decide, whether the
group $G \le S_n$ is a ``giant'', that is, it is either equal to
$A_n$ or to $S_n$ itself. In these two cases, there are better methods than
computing a stabiliser chain to solve the constructive recognition problem.
These methods are described in detail in \cite[Section~10.2]{Ser}, see
in particular \cite[Section~10.2.4]{Ser}. The algorithm described there
is one-sided Las Vegas in the following sense:  If it recognises $A_n$ or
$S_n$, the result is guaranteed to be correct. If $G$ is not a giant, it
fails very quickly. With a small error probability it can fail even if
$G$ is equal to $A_n$ or $S_n$. Thus, the method \texttt{Giant} returns
\texttt{fail} in case of failure meaning that it could be called later
on to try again. However, in the setup described here this will not
happen, since later methods will always succeed, even if they take a
very long time.

The third method \texttt{Imprimitive} with rank $30$ tries to find a block
system using the algorithm described in \cite[Section~5.5.1]{Ser}. If it 
finds one, it constructs a homomorphism onto the action on the set of
blocks. Since the algorithm is deterministic, it returns \texttt{false}
if no block system is found and the group $G$ is proved to be primitive.

Note that in the case that the \texttt{Imprimitive} method finds a block
system, it hands the information about the blocks down to the node that is
set up for the kernel. This allows firstly to use a different method to
create generators for the kernel since one needs often much more elements
to generate a subgroup of a repeated direct product than usual. Secondly,
the system can use a special find homomorphism
method for the kernel that produces a balanced composition tree, which 
is more efficient both during the recognition phase and the membership
testing phase. Here, the infrastructure in our framework to hand down 
hints to kernel and homomorphic image of a homomorphism is used 
extensively.

The fourth find homomorphism method \texttt{Jellyfish} with rank $20$
handles another special case in which the Schreier-Sims method does
not work efficiently. This case is that an alternating or symmetric
group on $\{1,\ldots,m\}$ acts on $n = {m \choose k}^r$ $r$-tuples
of $k$-subsets of $\{1,\ldots,m\}$. Of course, the action is not
given in this way but simply as an action on the set $\{1,\ldots,n\}$
and the problem is to identify each number in $\{1,\ldots,n\}$ with
an $r$-tuple of $k$-subsets of $\{ 1, \ldots, m\}$. In fact, the
algorithm does not solve this latter problem but tries to guess
$m$, $k$ and $r$ and directly find standard generators for $A_m$ or
$S_m$. A deterministic algorithm to solve this problem is described
in \cite[Section~4]{fastmanag} and a faster randomised one in
\cite{Jellyfish}.

Finally, if everything that has previously been tried has failed, we
compute a base and strong generators using the Schreier-Sims method. This
is done by the method called \texttt{StabilizerChain} with rank $10$.

This setup tries the right things in the right order to handle groups
with small as well as with large bases and thus implements a framework to
handle all permutation groups with the asymptotically best algorithms
known.


% this is a part of the habilitation thesis of Max Neunhoeffer

\chapter{Finding homomorphisms}
\label{chap:findhom}

In the previous Chapter~\ref{chap:comptree} we have formulated the problem
of constructive recognition of groups in Problem~\ref{ProbCR3} and have
explained what a reduction is. This chapter describes how to find
reductions for matrix groups and projective groups and how to prove
that a certain collection of methods will for any given
matrix group or projective group either find a reduction
or show that the constructive 
recognition problem can be solved efficiently by other means. By the 
arguments in Chapter~\ref{chap:comptree} this solves
Problem~\ref{ProbCR3} whenever the ambient group is $\GL(n,q)$ or
$\PGL(n,q)$.

We use two fundamental theoretical tools. The first is Aschbacher's theorem
about the subgroup structure of the classical groups, which we describe
in detail for the case of $\GL(n,q)$ in Section~\ref{sect:aschbacher},
and the second is the classification of finite simple groups. We are
content with the statements in Aschbacher's theorem about the general
linear group since they suffice for the purposes of constructive
recognition and the statements and arguments are quite a bit simpler
than for the other classical groups. For details see \cite{aschbacher}.

Roughly speaking, Aschbacher's theorem states that every subgroup of $\GL(n,q)$
is either a member of at least one of $7$ concretely given classes 
\CC1 to \CC7 of 
subgroups, or it contains a classical group in its natural representation, 
or it is an almost-simple group modulo scalars.

All the classes \CC1 to \CC7 are somehow defined in a geometric way (see
Sections~\ref{descC1} to \ref{descC7}) and thus promise some kind of
reduction. The two other cases are covered by two further classes \CC8
and \CC9, which are described in Sections~\ref{descC8} and \ref{descC9}.
For members of the latter two classes one will usually have to solve the
constructive recognition problem without further reduction.
% FIXME

The idea is to provide efficient algorithms for all the classes \CC1 to
\CC7 to recognise whether a given matrix group lies in the class, and if
so, to find a reduction using this information. If none of these algorithms
succeeds, Aschbacher's theorem shows that the group must be a member of
\CC8 or \CC9. In that case the constructive recognition problem has to be
solved by different means, usually by first finding out which classical
or almost simple group it is and then using this information to do the
constructive recognition in a special case, for example using standard
generators (see Sections~\ref{solveC8} and \ref{solveC9}).

The purpose of this chapter is to explain the statement of Aschbacher's 
theorem for $\GL(n,q)$ in detail, to give a proof and to give an
overview over the known methods to deal with the different classes
together with references into the literature. An algorithm to recognise
and handle classes \CC3 and \CC5 provided that the group does not lie in
class \CC1 is described in detail in Chapter~\ref{chap:subsemi}.

\section{A variant of Aschbacher's Theorem}
\label{sect:aschbacher}

Note again that we restrict ourselves to the general linear group
throughout, which is only a special case of Aschbacher's Theorem.

\begin{Not}
For this section we fix $n \in \N$ and $q=p^e$ for a prime $p$ and
talk about the group $\GL(n,q)$. We denote the vector space $\F_q^{1
\times n}$ by $V$ and note that $\GL(n,q)$ acts from the right on $V$ by
vector-matrix multiplication.
\end{Not}

For the original formulation of his theorem in \cite{aschbacher}, Aschbacher 
defines $8$ classes
of subgroups of $\GL(n,q)$ and proves that every subgroup $G$ is either
a subgroup of some member of one of these $8$ classes or has a certain
list of properties. We change this formulation in the following
way: Instead of the original 8 Aschbacher class \CC 1 to \CC 8
we define different classes \DD 1 to \DD 8
(for details see the descriptions in Sections \ref{descC1} to
\ref{descC8}).
Furthermore, we collect
the subgroups $G \le \GL(n,q)$ that fulfil the properties in the statement of 
Aschbacher's Theorem in the class \DD9. The slight modifications 
to the class definitions on the one hand stem from our proof of the
theorem, on the other hand they are motivated in the following way: We try
to increase the number of subgroups contained in classes that can be
handled algorithmically well and try to decrease the number of subgroups
contained in classes for which the known algorithms are not yet completely
satisfying, mostly with respect to their complexity analysis.
In addition we try to reduce the overlap between the classes.

We formulate a variant of Aschbacher's Theorem in the following way:

\begin{Theo}[Variant of Aschbacher's Theorem, specialised to\/ $\GL(n,q)$]
\label{Asch}
Let $G$ be a subgroup of\/ $\GL(n,q)$ with $n \ge 2$.
Then $G$ is contained in at least one of the
classes \DD1 to \DD9 of subgroups described in Sections~\ref{descC1}
to \ref{descC9}.
\end{Theo}
\proofbeg Compare \cite[Appendix 2, Theorem 1]{RobPhd}, \cite{aschbacher}, 
\cite{kleilieb} and 
\cite[Theorem~1]{smashprim}. For a proof see Section~\ref{AschProof}.
\proofend

%\begin{Rem}
%In our description of the classes \CC1 to \CC9 we follow
%\cite{kleilieb}. Kleidman and Liebeck change the definition 
%slightly in comparison to Aschbacher but argue that Theorem~\ref{Asch}
%remains true with their definitions (see \cite[Chapter~4]{kleilieb}).
%FIXME
%\end{Rem}

\medskip
We proceed with our definition of the subgroup classes \DD1 to \DD9. Throughout,
we denote for a subgroup $G < \GL(n,q)$ its subgroup of scalar matrices
by $Z$, that is, $Z := Z(\GL(n,q)) \cap G$. For each class we either
give an alternate structural description or at least give the structure 
of some ``typical'' example groups in that class.

\subsection{Description of class \DD1: ``reducible''}
\label{descC1}

\newcommand{\desc}[1]{\begin{center}\fbox{\parbox{5.3in}{#1}}\end{center}}
\newcommand{\diffasch}[1]{\textbf{Differences to Aschbacher's class \CC#1:}}
\newcommand{\stru}{\textbf{Alternate structure description:}\par}
\newcommand{\exmemb}{\textbf{Example members:}\par}

\desc{
A group $G \le \GL(n,q)$ is a member of \DD1 if there is a subspace
$0 < W < V$ that is stabilised by $G$, that is, $Wg = W$ for all $g \in G$.
}

\diffasch1
Our class \DD1 consists of all members of \CC1 and all their
subgroups.

\smallskip
\stru
A group lies in \DD1 if and only if it is conjugate in $\GL(n,q)$ 
to a subgroup of a group
\[ P_m := \left\{ \left[ \begin{array}{cc} A & 0 \\ C & D \end{array} \right]
           \mid A \in \GL(m,q), D \in \GL(n-m,q) \mbox{ and }
           C \in \F_q^{n-m \times m} \right\} \]
for some $0 < m < n$. The group $P_m$ is called a \emph{maximal parabolic
subgroup} and is a semidirect product
of the normal $p$-subgroup
\[ U_m := \left\{ \left[ \begin{array}{cc} \one_m & 0 \\ C & \one_{n-m} 
           \end{array} \right]
           \mid 
           C \in \F_q^{n-m \times m} \right\} \]
and $\GL(m,q) \times \GL(n-m,q)$, the factors being embedded on the
diagonal blocks. 
Here, $\one_m$ is the $(m \times m)$-identity matrix and $\one_{n-m}$ is the
$(n-m) \times (n-m)$-identity matrix.

\subsection{Description of class \DD2: ``imprimitive''}
\label{descC2}

\desc{
A group $G \le \GL(n,q)$ is a member of \DD2 if the natural module $V$ is
absolutely irreducible and there is a decomposition
of $V$ as a direct sum of $m$-dimensional subspaces 
$V = V_1 \oplus \cdots \oplus V_t$
such that the summands are permuted transitively by $G$. That is, 
for every $1 \le i \le t$ and every 
$g \in G$ there is a $j$ with $V_i g = V_j$ and the action on the
summands is transitive.
}

\diffasch2
We include in \DD2 the subgroups of the members of \CC2 that
permute the direct summands transitively, but we exclude all groups
acting not absolutely irreducibly.

\smallskip
\exmemb
The groups $\GL(m,q) \wr S_t$, where the $\GL(m,q)$ factors act on
the direct summands $V_i$ and the symmetric group on top permutes the
subspaces, all lie in \DD2. Subgroups of these groups belong to \DD2
if they act irreducibly.


\subsection{Description of class \DD3: ``semilinear''}
\label{descC3}

\desc{
A group $G \le \GL(n,q)$ lies in \DD3 if the natural module $V$ is
irreducible and there is a finite field
extension $\F_{q^s}$ of $\F_q$,
for which we can extend the $\F_q$-vector space structure of
$V$ to an $\F_{q^s}$-vector space structure of dimension
$n/s$, such that for every $g \in G$ there exists 
an automorphism $\alpha_g$ of $\F_{q^s}$ with
\[ (v+\lambda w)\cdot g = v \cdot g + \lambda^{\alpha_g} \cdot w \cdot
g\]
for all $v,w \in V$ and all $\lambda \in \F_{q^s}$. This means that we
can interpret $V$ as an $\F_{q^s}$-vector space for which the natural action
of $G$ is $\F_{q^s}$-semilinear, so $G$ is a subgroup of $\GGL(n/s,q^s)$.
}

Note that the automorphisms of $\F_{q^s}$ occurring in the semilinear
actions of group elements will automatically fix every element of $\F_q$, 
since the original action is $\F_q$-linear. Therefore they are elements of
the Galois group $\Gal(\F_{q^s}/\F_q)$. Note further that
the group $G$ lies in \DD3 
with trivial automorphisms $\alpha_g$ for all $G$ if and only if
$V$ is irreducible but not absolutely irreducible (see \cite[(29.13)]{CR0}).

\medskip
\diffasch3
We include in \DD3 all subgroups of the members of \CC3, but we exclude all
groups acting reducibly.

\smallskip
\stru
A group lies in \DD3 if and only if it acts irreducibly on the natural
module and is conjugate in $\GL(n,q)$ to a subgroup of $\GGL(n/s,q^s)$ for
some prime $s \mid n$, realised as $(n \times n)$-matrices over $\F_q$
by choosing an $\F_q$-basis of $\F_{q^s}$.


\subsection{Description of class \DD4: ``tensor-decomposable''}
\label{descC4}

%%% % Old version:
%%% \desc{
%%% A group $G \le \GL(n,q)$ lies in class \DD4 if there is a decomposition
%%% of $V = V_1 \otimes V_2$ into a tensor product with 
%%% $1 < d_1 := \dim_{\F_q}(V_1) < n$
%%% and $d_2 := \dim_{\F_q}(V_2)$ that is preserved by $G$, 
%%% that is, for every $g \in
%%% G$ there are elements $g_1 \in \End_{\F_q}(V_1)$ and $g_2 \in
%%% \End_{\F_q}(V_2)$ such that $(v_1 \otimes v_2) g = v_1 g_1 \otimes v_2 g_2$
%%% for all $v_1 \in V_1$ and $v_2 \in V_2$.
%%% }
\desc{
A group $G \le \GL(n,q)$ lies in class \DD4 if the natural module $V$ is
absolutely irreducible and $G$ has a normal subgroup $N$
such that $V|_N$ is isomorphic as an $\F_q N$-module to a direct sum of
$k \ge 2$ modules which are all isomorphic to a single absolutely irreducible 
$\F_q N$-module $W$.
}

\diffasch4
We define \DD4 including more conditions on the structure of the
group and its natural representation than Aschbacher. The tensor
product decomposition in the definition of \CC4 follows from our
conditions, see Proposition~\ref{tensorprop}. On the other
hand we allow the dimensions of the tensor factors to be equal.

\smallskip
\exmemb
The group $\GL(d_1,q) \circ \GL(d_2,q)$ is the central
product of $\GL(d_1,q)$ and $\GL(d_2,q)$ for $d_1 \cdot d_2 = n$ and
thus is contained in $\GL(n,q)$. It is
the set of Kronecker products of a matrix in $\GL(d_1,q)$ and one in
$\GL(d_2,q)$. Those groups are members in \DD4.


\subsection{Description of class \DD5: ``subfield''}
\label{descC5}

\desc{
A group $G \le \GL(n,q)$ lies in \DD5 if the natural module $V$ is
absolutely irreducible and there exists a proper subfield $\F_{q_0}$
of $\F_q$, a matrix $T \in \GL(n,q)$, and scalars $(\beta_g)_{g \in
G}$ with $\beta_g \in \F_q$ such that $\beta_g \cdot T^{-1} g T \in
\GL(n,q_0)$ for all $g \in G$.}

\diffasch5
We include in \DD5 subgroups of the members of \CC5 but we exclude
all groups acting not absolutely irreducibly.

\smallskip
\stru
A group lies in \DD5 if and only if it is conjugate in $\GL(n,q)$ to a
subgroup of $\GL(n,q_0) \cdot \F_q^*$ where $\F_{q_0}$ is a 
proper subfield of $\F_q$.

\subsection{Description of class \DD6: ``extraspecial''}
\label{descC6}

\desc{
A group $G \le \GL(n,q)$ lies in \DD6 if the natural module $V$ is absolutely
irreducible, $n=r^m$ for a prime $r$ and
\begin{itemize}\setlength{\itemsep}{0pt}\setlength{\parskip}{0pt}
    \item \textbf{either} $r$ is odd and $G$ has a normal subgroup $E$ that 
is an extraspecial $r$-group of order $r^{1+2m}$ and exponent $r$,
\item \textbf{or} $r=2$ and $G$ has a normal subgroup $E$ that is either
    extraspecial of order $2^{1+2m}$ or a central product of a cyclic
    group of order $4$ with an extraspecial group of order $2^{1+2m}$,
\end{itemize}
\textbf{and} in both cases the linear action of $G$ on the
$\F_r$-vector space $E/Z(E)$ of dimension $2m$ is irreducible. }

\diffasch6
We have added the condition about the irreducible $\F_r$-linear action of $G$ on
$E/Z(E)$ because it comes out of our proof easily and might help to
devise algorithms to find a reduction for groups in this class. On the
other hand we include subgroups of members of \CC6 if they fulfil this
condition.

\smallskip
\exmemb
The following subgroups of $\GL(n,q)$ for suitable $(n,q)$ lie in \DD6:

$r^{1+2m}.\Sp(2m,r)$ and $2_+^{1+2m}.O^+(2m,2)$ and 
$2_-^{1+2m}.O^-(2m,2)$ and $(4 \circ 2^{1+2m}).\Sp(2m,2)$.


\subsection{Description of class \DD7: ``tensor-induced''}
\label{descC7}

\desc{
A group $G \le \GL(n,q)$ lies in \DD7 if it acts absolutely irreducibly on the 
natural module $V$ and, for some $k$, it has a normal 
subgroup $N$ containing $Z = Z(G)$ that is isomorphic to a central product 
$T \circ \cdots \circ T$ of
$k$ copies of a group $T$ which is a central extension of a non-abelian
simple group $\bar T$ by $Z$, such that: 

The restricted
module $V|_N$ is isomorphic to an outer tensor product $W_1 \otimes_{\F_q}
\cdots \otimes_{\F_q} W_k$ of $k$
absolutely irreducible $\F_q T$-modules of the same dimension on which 
$Z$ acts as scalars, and $G/N$ acts on $N$ by permuting the tensor factors
transitively.}

By the term ``outer tensor product'' we mean that every factor
$T$ in the central product $N$ acts on exactly one of the tensor factors $W_i$.
Note that the $\F_q T$-modules $W_i$ need not necessarily be
isomorphic to each other.

\medskip
\diffasch7 
We define \DD7 including more conditions on the structure of the group
and its natural representation than Aschbacher. On the other hand we
have added some subgroups of the members of \CC7.

\smallskip
\textbf{Note:} 
If $G$ is a
group in \DD7, then its natural projective representation
is tensor-induced,
see Proposition~\ref{tensorindprop}. However, the natural
representation of $G$ needs not be tensor-induced as is shown in
Remark~\ref{nottensorind} by an example.

\smallskip
\exmemb
The following subgroups of $\GL(n,q)$ lie in \DD7:
$\GL(r,q)^{\otimes k}.S_k$ where $n=r^k$ and $\GL(r,q)^{\otimes k}$ 
denotes the central product of $k$ copies of $\GL(r,q)$.

\subsection{Description of class \DD8: ``classical''}
\label{descC8}

\desc{
A group $G \le \GL(n,q)$ lies in \DD8 if $G/Z$ contains a classical simple
group in its natural representation in one of the following ways:
\begin{itemize}\setlength{\itemsep}{0pt}\setlength{\parskip}{0pt}
\item $G/Z$ contains $\PSL(n,q)$ and $(n,q) \notin \{(2,2),(2,3)\}$,
\item $n$ is even, $G$ is contained in $\Sp(n,q)$ for some non-singular 
symplectic form, $G/Z$ contains $\PSp(n,q)$ and $(n,q) \notin           
\{(2,2),(2,3),(4,2)\}$,
\item $q$ is a square, $G$ is contained in $U(n,q)$ for some non-singular 
Hermitian form, $G/Z$ contains $\PSU(n,q)$ and $(n,q) \notin \{ (2,4),
(2,9), (3,4) \}$,
\item $G$ is contained in $O^\epsilon(n,q)$, the corresponding
    $\POmega^\epsilon(n,q)$ is simple and contained in $G/Z$.
    $\POmega^\epsilon(n,q)$ is simple if and only if
    \begin{itemize}\setlength{\itemsep}{0pt}\setlength{\parskip}{0pt}
        \item[*] $n\ge 3$, and
        \item[*] $q$ is odd if $n$ is odd, and
        \item[*] $\epsilon$ is -- if $n=4$, and
        \item[*] $(n,q) \notin \{ (3,3), (4,2) \}$.
    \end{itemize}
\end{itemize}
}

\textbf{Note:} For the orthogonal groups the given restrictions on $(n,q)$
are necessary for $\POmega^{(\pm)}(n,q)$ to be simple, see
\cite[Section~2.4]{ATLAS} for details.

\medskip
\diffasch8
We only include groups in \DD8 that modulo scalars contain a simple 
classical group in its natural representation defined over $\F_q$.


\subsection{Description of class \DD9: ``almost-simple''}
\label{descC9}

\desc{
A group $G \le \GL(n,q)$ lies in \DD9, if it is not in \DD8 and
there is a non-abelian simple
group $\bar N$ and a group $T$ with $\bar N \subseteq T \subseteq \Aut(\bar N)$ 
such that $G/Z$ is isomorphic to $T$ (in this case $G/Z$ is called
``almost-simple'') and
the natural module $V$ gives rise to an
absolutely irreducible projective representation for $T$ that
is not realisable over a proper subfield of~$\F_q$.}

\textbf{Note:} The existence of the projective representation given by $V$
limits the possibilities for $\bar N$ and $T$ for given $(n,q)$ and thus
provides an interesting application for the representation theory of finite
simple groups, their Schur covers and automorphism groups.

\medskip
\diffasch9
Aschbacher does not call this class \CC9 but in the literature many
authors have called this ``Aschbacher class \CC9``.

\smallskip
\exmemb
The absolutely irreducible projective representations of the
quasi-simple groups provide examples of groups in \DD9.


\section{A proof of the $\mathbf{\GL}$-version of Aschbacher's Theorem}
\label{AschProof}

The contents of this section are a variation on ``The Theory behind
\textsc{Smash}'' from \cite[Section~2]{smashnormal} and provide a proof for
Theorem~\ref{Asch}. The part about the semilinear action is copied
from Section~\ref{subsec:semilin} and thus from 
\cite[Section~6.4]{subfieldpaper}.

Let $G$ be any subgroup of $\GL(n,q)$ acting from the right on the
natural module $V := \F_q^{1 \times n}$.

If the natural module $V$ is reducible, then $G$ is contained in class \DD1
as defined in Section~\ref{descC1}.

From now on we assume that $V$ is irreducible.

If $V$ is not absolutely irreducible, then by \cite[(29.13)]{CR0} and the 
usual Wedderburn Theorems
the endomorphism ring $\End_{\F_q G}(V)$ is a proper finite extension field
$\F_{q^s}$ of $\F_q$. This automatically extends the $\F_q$-vector
space structure of $V$ to an $\F_{q^s}$-vector space structure
such that the $G$-action is $\F_{q^s}$-linear. Thus $G$ lies in class
\DD3 (see \ref{descC3}) with trivial Galois automorphisms.

From now on we assume that $V$ is absolutely irreducible. Let $Z$ be the
subgroup of scalar matrices contained in $G$, i.e.~$Z := G \cap
Z(\GL(n,q))$. Since $V$ is absolutely irreducible it follows that $Z$ is
the centre of $G$.

At this stage of the proof we mention that obviously $G$ can lie
in \DD5 (see \ref{descC5}).

From now on we assume that $G$ does not lie in \DD5. In the sequel this 
assumption will be used to rule out classical groups defined over
smaller fields and achieve the corresponding statement in class \DD9
(see \ref{descC9}).

We assume first that $G/Z$ is a simple group. If $G/Z$ were cyclic of prime 
order, then $G$ would be abelian and $V$ could not be absolutely
irreducible contrary to our assumptions.

There are two possibilities:
The first is that $G/Z$ is one of the
classical simple groups in its natural representation defined over $\F_q$
and $G$ lies in \DD8 (see
\ref{descC8}). Note that by our assumption of $G$ not lying in \DD5 we can
exclude classical simple groups defined over a proper subfield. Otherwise
$G$ lies in \DD9 (see \ref{descC9}) with
$\bar N =T \cong G/Z$ since $V$ gives rise to an absolutely irreducible
projective representation of the finite simple group $G/Z$ that
is by assumption not realisable over a proper subfield of $\F_q$.

We assume from now on that $G/Z$ is not simple, let $\bar N$
be a non-trivial minimal normal subgroup of $G/Z$ and $N$ be
the corresponding non-scalar normal subgroup of $G$ with $Z < N
\triangleleft G$.

We now use Clifford theory applied to the natural module $V$. By Clifford's
theorem (see \cite[(49.2) and (49.7)]{CR0}) the restricted $\F_q N$-module
$V|_N$ is a direct sum of irreducible $\F_q N$-modules which are all
$G$-conjugates of one irreducible $\F_q N$-submodule $W \le V|_N$. In
particular, all these irreducible summands have the same dimension. We now
distinguish several cases.

If there is more than one homogeneous component (i.e.~not all conjugates of
$W$ are isomorphic to $W$ as $\F_q N$-modules), then $G$ lies in \DD2
(see \ref{descC2}),
since the homogeneous components provide a direct sum decomposition that is
preserved by $G$ and the summands are permuted transitively.

From now on we assume additionally that there is only one homogeneous
component, that is, all $Wg$ are isomorphic to $W$ as $\F_q N$-modules.
Then $\dim W > 1$ because $N$ is non-scalar.

We first assume that $W$ is not absolutely irreducible including both
cases $W = V$ and $W < V$. Then $\End_{\F_q N}(W)$ is a proper finite
extension field of $\F_q$, say $\F_{q^s}$, and $W$ and $V|_N$
both can be considered as $\F_{q^s}$-vector spaces such that the action of
$N$ on them is $\F_{q^s}$-linear, because $V|_N$ is isomorphic to a direct
sum of copies of $W$. We can embed $\F_{q^s} \le \End_{\F_q N}(V) \le \GL(n,q)$.

We claim that the action of $G$ is $\F_{q^s}$-semilinear proving that $G$ lies
in \DD3 (see \ref{descC3}). Let $c \in \GL(n,q)$ generate the
multiplicative group of $\F_{q^s}$. Then, for all $h \in N$ and $g
\in G$, we have $hc=ch$ by definition and thus $h^g c^g = c^g h^g =
h' c^g = c^g h'$, for some $h' \in N$. As $h$ varies over $N$, the
element $h'$ takes every value in $N$, therefore $\left< c \right>
= \left< c^g \right>$ and so $c^g = c^k$ for some $k$. Suppose that
$c^i + c^j = c^l$, then $(c^i)^g + (c^j)^g = (c^l)^g$ so $g$ acts as
field automorphism on $\F_{q^s}$. We have thus proved that $G$ 
lies in \DD3 (see \ref{descC3}).

From now on we assume that $W$ is absolutely irreducible.

We first assume that $W$ is a proper subspace of $V$. Let $d :=
\dim_{\F_q}(W)$ such that $1 < d < n$. Since by assumption 
all $G$-conjugates of $W$ are isomorphic to $W$ as $\F_q N$-modules,
we have shown that $G$ lies in \DD4 (see \ref{descC4}).

We now move on to the case that $W = V|_N$, that is, the restriction
$V|_N$ is irreducible. We are still assuming that $W$ is
absolutely irreducible.

Recall that by assumption $\bar N$ is a minimal normal subgroup of $G/Z$.
As such, by \cite[Theorem 4.3A.(iii)]{DixonMort}, it is
a direct product $\bar T_1 \times \cdots \times \bar T_k$ of copies of a 
simple group $\bar T$ which are all conjugate under $G/Z$. Thus $N$ is
a central product of the corresponding preimages $T_1, \ldots, T_k$
under the natural map $G \to G/Z$. We now distinguish three cases:
\begin{itemize}\setlength{\itemsep}{0pt}\setlength{\parskip}{0pt}
\item[(i)] $\bar T$ is cyclic of prime order $r$, 
\item[(ii)] $\bar T$ is non-abelian simple with $k > 1$ and
\item[(iii)] $N/Z$ is non-abelian simple.
\end{itemize}
We now consider case (i) that $\bar T$ is cyclic of prime order $r$, then
$\bar N = N/Z$ is an elementary-abelian $r$-group of order $r^k$.
Since $N$ is not abelian, we have $k > 1$.
The commutator subgroup $N'$ of $N$ is contained in $Z$ and
we recall that $Z$ is the centre of $N$ since $V|_N$ is absolutely
irreducible. So $N$ is nilpotent and thus the direct product of its
Sylow subgroups. Let $R$ be the $r$-Sylow subgroup of $N$, all other
Sylow subgroups of $N$ consist of scalar matrices since $N/Z$ is
elementary abelian. Thus the module $V|_R$ is absolutely irreducible
and $R$ is not abelian.
For $x,y \in R$, we have $1 = [x^r,y] = [x,y]^r$ since $x^r$ and 
$[x,y]$ lie in $Z(R) = R \cap Z$. Therefore $R'=N'$ has exponent $r$ and
because it is contained in the cyclic group $Z(R) \le Z$ we have $|R'| =
r = |N'|$.

Assume first that $r$ is odd. For $x,y \in R$ we have $(xy)^r =
x^ry^r[y,x]^{r(r-1)/2} = x^ry^r$ since $R'$ has order $r$, thus the
elements of $R$ whose order divides $r$ form a characteristic 
subgroup of $N$ and thus a normal subgroup $E$ of $G$. Since for
$r$ odd, an $r$-group containing a unique subgroup of order $r$ is
cyclic, $E$ is not contained in $Z$ and thus by the minimality
of $N/Z$ we have $Z E = N$. It immediately follows that $V|_E$
is absolutely irreducible, $E$ is not abelian, 
$E \cap Z = Z(E) = E' = R' = N'$ has $r$ elements and 
$N/Z = ZE/Z \cong E/(E \cap Z) = E/Z(E)$. Thus the Frattini subgroup
$\Phi(E)$ of $E$ is equal to $Z(E)$ and $E$ is shown to be extraspecial
of order $r^{1+k}$ and exponent $r$. It follows using
\cite[V.16.14]{Hup} that $k$ is even and
$n=r^{k/2}$ because this holds for the faithful absolutely irreducible
representations of an extraspecial group. The linear action of $G$ on
the $\F_r$-vector space $E/Z(E)$ is irreducible because every minimal
normal subgroup of $N/Z \cong E/Z(E)$ is conjugated to a full basis by
$G$. Hence $G$ lies in \DD6 with odd $r$.

Consider now $r=2$. Recall $R' \le Z$ with $|R'| = 2$ and
let $E$ be the set of elements $x \in R$ with $x^2 \in R'$. 
For $x,y \in E$ we have $(xy)^2 = x^2y^2[y,x] \in R'$ proving that
$E$ is a characteristic subgroup of $R$ and thus a normal subgroup of
$G$. We claim that $E$ is not contained in $Z$: If $4$ does not divide
$|Z(R)|$ then this is trivial. If $4$ divides $|Z(R)|$ and $xZ(R)$ and $yZ(R)$
are different elements of $R/Z(R)$, then at least one of
$x^2$, $y^2$ and $(xy)^2$ is a square in the cyclic group $Z(R)$ and
thus one of $x$, $y$ and $xy$ can be multiplied by an element of
$Z(R)$ to get an involution. Therefore $R$ and thus $E$ both contain a 
non-central element whose square is contained in $R'$.
By the minimality of $N/Z$ we have $Z E = N$ as above. We immediately
get that $V|_E$ is absolutely irreducible, $E$ is not abelian,
$E \cap Z = Z(E)$ is cyclic with either $2$ or $4$ elements and
$E' = R' = N'$ has $2$ elements and is contained in $Z(E)$.
Furthermore, $N/Z = ZE/Z \cong E/(E \cap Z) = E/Z(E)$. Thus the
Frattini subgroup $\Phi(E)$ of $E$ is equal to $E'$ and $E$ is either
extraspecial of order $2^{1+k}$ or a central product of a cyclic group
of order $4$ consisting of scalar matrices and an extraspecial group 
of order $2^{1+k}$. In both cases it follows again using
\cite[V.16.14]{Hup} that $k$ is even and
$n = 2^{k/2}$. As above the linear action of $G$ on the $\F_2$-vector space
$E/Z(E)$ is irreducible and thus $G$ lies in \DD6 with $r=2$.

This concludes case (i) that $\bar T$ is cyclic of prime order $r$.

We now consider case (ii) that $\bar N$ is a direct product of more than
one copies of a non-abelian simple group $\bar T$. Thus $N$ is a
central product of the groups $T_1, \ldots, T_k$, each of which is
isomorphic to a single group $T$, which is a central extension of
$\bar T$ by $Z$. The absolutely irreducible representations
of $N$ in which $Z$ acts as scalars are just tensor products of 
absolutely irreducible representations of $T$ in which $Z \le T$ acts as
scalars. Furthermore, since $T_1, \ldots, T_k$ are all conjugate under
$G$, the natural module $V|_N$ must be isomorphic to an outer tensor product
$W_1 \otimes_{\F_q} \cdots \otimes_{\F_q} W_k$ 
of $k$ absolutely irreducible $\F_q T$-modules $W_i$ on which $Z$
acts as scalars. Since $G/N$ acts on $N$ by permuting the factors
transitively, all modules $W_i$ have the same dimension. So $G$ lies in
\DD7 concluding case~(ii).

We finally consider case (iii) that $\bar N = N/Z$ is a non-abelian finite 
simple group. Since in this case the centraliser $C_{\bar N}(\bar N)$ is 
trivial, $G/Z$ is contained in the automorphism group of $\bar N$. 
Thus either $N$ is a classical simple group in its natural
representation defined over $\F_q$ and
$G$ lies in \DD8, or $G$ lies in \DD9. Note that the assumptions we
picked up along the way of this proof now allow to conclude that $V$
gives rise to an absolutely irreducible projective representation for
$G$ which is not realisable over a subfield (since $G$ is not \DD5).
\proofend

\medskip
In the rest of this section we present results relating our class
definitions to those of Aschbacher and to other concepts of group
representations.

\begin{Prop}[\DD4 implies tensor decomposability]
    \label{tensorprop}
If a group $G \le \GL(n,q)$ lies in \DD4, then it is tensor decomposable,
by which we mean the following:
there is a decomposition
of\/ $V = V_1 \otimes V_2$ into a tensor product with 
$1 < d_1 := \dim_{\F_q}(V_1) < n$
and $d_2 := \dim_{\F_q}(V_2)$ that is preserved by $G$, 
that is, for every $g \in
G$ there are elements $g_1 \in \End_{\F_q}(V_1)$ and $g_2 \in
\End_{\F_q}(V_2)$ such that $(v_1 \otimes v_2) g = v_1 g_1 \otimes v_2 g_2$
for all $v_1 \in V_1$ and $v_2 \in V_2$.
\end{Prop}
\proofbeg
This proof is copied from Section~\ref{subsec:tensor} and thus from 
\cite[Section~6.6]{subfieldpaper}.

We assume that $G$ lies in \DD4, so the natural module $V$ is
absolutely irreducible and $G$ has a normal subgroup $N$ such that
$V|_N$ is isomorphic to a direct sum of $k>2$ modules which are all
isomorphic to a single absolutely irreducible $\F_q N$-module $W$
of dimension $d < n$.

It immediately follows that we can choose a basis of $V$ such that all
elements of $N$ are block diagonal matrices in which all diagonal blocks
are identical of size $d$.

As $N \triangleleft G$, for all $h \in N$ and $g \in G$,
 the product $g^{-1}hg \in N$ and thus $g^{-1} h g$ is also 
a block diagonal matrix in which all $d \times d$-blocks along the diagonal
are identical. Fixing $g$, we conclude that $g\cdot (g^{-1}hg) = hg$ for all
$h \in N$. If we now cut $g$ into $d \times d$-blocks, we get:

\begin{eqnarray*}
   &g \cdot (g^{-1}hg) & 
 = \left[ \begin{array}{c|c|c|c}
      g_{1,1} & g_{1,2} & \cdots & g_{1,n/d} \\ \hline
      g_{2,1} & g_{2,2} & \cdots & g_{2,n/d} \\ \hline
      \vdots  & \vdots  & \ddots & \vdots    \\ \hline
      g_{n/d,1}&g_{n/d,2}& \cdots& g_{n/d,n/d} \end{array} \right]
\cdot \left[ \begin{array}{c|c|c|c}
      D^g(h) & 0   & \cdots &      0    \\ \hline
         0   &D^g(h)&\cdots &      0    \\ \hline
      \vdots  & \vdots  & \ddots & \vdots    \\ \hline
         0    &    0    & \cdots& D^g(h) \end{array} \right] \\
 &=& \left[ \begin{array}{c|c|c|c}
      D(h)    & 0       & \cdots &     0    \\ \hline
         0    &D(h)     &\cdots &      0    \\ \hline
      \vdots  & \vdots  & \ddots & \vdots    \\ \hline
         0    &    0    & \cdots& D(h)   \end{array} \right]
\cdot \left[ \begin{array}{c|c|c|c}
      g_{1,1} & g_{1,2} & \cdots & g_{1,n/d} \\ \hline
      g_{2,1} & g_{2,2} & \cdots & g_{2,n/d} \\ \hline
      \vdots  & \vdots  & \ddots & \vdots    \\ \hline
      g_{n/d,1}&g_{n/d,2}& \cdots& g_{n/d,d/n} \end{array} \right]
 = hg
\end{eqnarray*}
where the $g_{i,j}$ are $d \times d$-matrices, $D(h)$ is a matrix
representing $h$ on the module $W$ and $D^g(h) = D(g^{-1}hg)$ is the
same representation twisted by the element $g$. By the block diagonal
structure of the matrices in $N$ we get 
$g_{i,j} \cdot D^g(h) = D(h) \cdot g_{i,j}$ for all $i$ and $j$ and 
all $h \in N$.

But by hypothesis, the matrix representations $D$ and $D^g$ of $N$
are isomorphic. Thus there is a nonzero matrix $T \in \F_q^{d \times d}$ with
$T \cdot D^g(h) = D(h) \cdot T$ for all $h \in N$. By Schur's lemma and
since the representation $D$ is absolutely irreducible,
the matrix $T$ is invertible and unique 
up to multiplication by an element in $C_{\GL(d,q)}(D(N))$, which
consists only of the scalar matrices.

This shows that for every pair $(i,j) \in \{ 1, \ldots, d/n \} \times 
\{ 1, \ldots, d/n \}$ there
is a unique element $e_{i,j} \in \F_q$ (possibly $0$) with 
$g_{i,j} = T \cdot e_{i,j}$. Thus we have shown that with respect to
the above choice of basis, every element $g$ is equal to a Kronecker 
product of some matrix in $U \in \F_q^{n/d \times n/d}$ with a matrix
$T \in \F_q^{d \times d}$. Since $g$ is invertible both 
$U$ and $T$ are invertible. 

This provides an $\F_q$-linear isomorphism of 
$\F_q^n$ and the tensor product $\F_q^{n/d} \otimes_{\F_q} \F_q^d$
such that all $g \in G$ act as a Kronecker product proving the
proposition.
\proofend

\begin{DefProp}[{Tensor induction of (projective) representations}]
    For details and proofs see \cite[13A]{CRI} and \cite[Section~2]{kovacs}.

    Let $G$ be a group, $K$ an arbitrary field, $H<G$ a subgroup of
    index $k$ and $G=\bigcup_{i=1}^k g_i H$ with $g_1,\ldots,g_k \in
    G$ and $g_1=1$. Let $\varphi : G \to H \wr S_k$ be the group
    monomorphism defined by $\varphi(x) = \pi \cdot (h_1,\ldots,h_k)$
    for $x \in G$, where $xg_i = g_{\pi(i)}h_i$ with $\pi \in S_k$
    and $h_i \in H$ for $1 \le i \le k$.
    Let $\kappa$ be the group homomorphism $\kappa: \GL(d,K)
    \wr S_k \to \GL(d^k,K)$ where the direct factors of the base group
    act on the tensor factors of $(K^d)^{\otimes k} \cong K^{d^k}$ and
    the top group permutes them. Let $\bar\kappa : \PGL(d,K) \wr S_k
    \to \PGL(d^k,K)$ be the corresponding map for the projective linear
    group.

    For a representation $\rho: H \to \GL(d,K)$ of $H$
    there is a corresponding map $(\rho \wr S_k) : H \wr S_k \to
    \GL(d,K) \wr S_k$ mapping the base group component-wise by $\rho$. 
    The \emph{tensor-induced representation}
    $\rho\!\uparrow^{\otimes G} : G \to \GL(d^k,K)$ is then the
    composite map $\kappa \circ (\rho \wr S_k) \circ \varphi$. The
    equivalence class of this representation does not depend on the
    choices in this construction.

    Similarly,
    for a projective representation $\bar\rho : H \to \PGL(d,K)$ of
    $H$ there is a corresponding map $(\bar\rho \wr
    S_k) : H \wr S_k \to \PGL(d,K)$ mapping the base group
    component-wise by $\bar\rho$.
    The \emph{tensor-induced projective representation}
    $\bar\rho\!\uparrow^{\otimes G} : G \to \PGL(d^k,K)$ is then the
    composite map $\bar\kappa \circ (\bar\rho \wr S_k) \circ \varphi$.
    The equivalence class of this projective representation does not
    depend on the choices in this construction.
\end{DefProp}

\begin{Prop}[\DD7 implies ``projective tensor-induced'']
    \label{tensorindprop}
If a group $G \le \GL(n,q)$ lies in \DD7, then the projective
representation afforded by its natural module is tensor-induced.
\end{Prop}
\proofbeg
Let $G$ be in \DD7 and let $N$, $T$, $Z$ and $V$ be as in
Section~\ref{descC7}.
We can use the proof of \cite[Tensor Induction Theorem]{kovacs}. 
Although the hypotheses of
the theorem contain that the natural module is not induced from a any
proper subgroup, this is only used in the proof to ensure that the
constituents of the restricted natural module $V|_N$ are all
equivalent and absolutely irreducible. However, we have this by
assumption and the proof goes through.
\proofend

\begin{Rem}[{\DD7 does \emph{not} imply tensor-induced}]
    \label{nottensorind}
    Let $G$ be the semidirect product 
    $(\SL(2,7) \otimes_{\F_7} \SL(2,7)).S_2 < \GL(4,7)$ where the
    $S_2$ permutes the tensor factors. The natural module $V := \F_7^4$ is
    absolutely irreducible also when restricted to the normal subgroup
    $N := \SL(2,7) \otimes_{\F_7} \SL(2,7)$ of index $2$. The center 
    $Z := \left< -\id \right>$ of $G$ has $2$ elements and $N/Z$ is a
    minimal normal subgroup of $G/Z$ which is a direct product of two
    copies of the non-abelian simple group $\PSL(2,7)$. Thus $G$ lies
    in \DD7 by construction.

    However, $G$ does not have a subgroup of index $4$, the only 
    subgroup of $G$ of index $2$ is $N$ and $N$ does
    not have an irreducible representation of dimension $2$ over
    $\F_7$. Thus the natural module is not tensor-induced.

    Of course, using Proposition~\ref{tensorindprop} we conclude that
    the projective representation afforded by the natural module is in
    fact tensor-induced from the two-dimensional projective
    absolutely irreducible representation of $N$ over $\F_7$.
\end{Rem}

\section{Finding reductions}
\label{findred}

The purpose of this section is to give an overview how reductions can
be found algorithmically provided that a matrix group
lies in one of the classes \DD1 to \DD7. We do not give full details
here but rather sketch the methods used and collect references into the
literature.

We continue to use the notation for a group $G < \GL(n,q)$ that
$Z := Z(\GL(n,q)) \cap G$ is the subgroup of scalar matrices and that
$V := \F_q^{1 \times n}$ is the natural right module. For the
complexity statements we assume that $G$ is given by $m$ generators
$g_1, \ldots, g_m \in \GL(n,q)$.

\subsection{Finding a reduction in the reducible case: \DD1}
\label{solveC1}

Using the MeatAxe (see \cite{MeatAxeHoltRees, IL, MeatAxeRP}) we can
decide efficiently whether the natural module $V$ is
irreducible. If not, we find a proper invariant subspace $0<W<V$
thereby finding a reduction as described in Section~\ref{descC1}. 
If $V$ is irreducible, we either prove
that it is absolutely irreducible or find an element that generates
the endomorphism ring as a field.

Under the assumptions in Section~\ref{hypMtx} the MeatAxe provides a
Las Vegas algorithm which terminates
after at most $O(mn^3\log \delta^{-1})$ elementary field operations where $m$
is the number of generators of $G$ and $0 < \delta < 1/2$ is an upper
bound for the failure probability (see Lemma~\ref{MeatAxe}).


\subsection{Finding a reduction in the semilinear or subfield case:
\DD3/\DD5}
\label{solveC3C5}

Chapter~\ref{chap:subsemi} deals with the case that $G$ does not lie
in \DD1.


\subsection{Finding a reduction in the extraspecial case: \DD6}
\label{solveC6}

\subsection{Finding a reduction in the imprimitive case: \DD2}
\label{solveC2}

\subsection{Finding a reduction in the tensor-decomposable case: \DD4}
\label{solveC4}

\subsection{Finding a reduction in the tensor-induced case: \DD7}
\label{solveC7}




% this is a part of the habilitation thesis of Max Neunhoeffer

\chapter{Leaves of the composition tree}
\label{chap:leaves}
\index{leaf}%

This chapter is about constructive recognition (see
Problem~\ref{ProbCR3}) of the leaves of a
composition tree (see Section~\ref{recapproach}). By the arguments in
Chapter~\ref{chap:findhom} this means that we have to be
able to solve Problem~\ref{ProbCR3} for the subgroups $G \le \GL(n,q)$
that lie in classes \DD8 and \DD9 (see Sections~\ref{descD8} and
\ref{descD9}). Because of the homomorphism from $\GL(n,q) \to
\PGL(n,q)$ and our setup of the composition tree we only have
to work with the projective version $\bar G \le \PGL(n,q)$.

We do not want to describe the details of the methods used here but
instead give an overview and refer the reader to the literature.
One reason for this is that the current state of the art does not seem
to be final, another is that the author has not contributed to the
development of algorithms in this area up to now.

This chapter is structured as follows. We begin by describing
methods based on permutation groups, which we call ``direct methods
for constructive recognition'' in Section~\ref{solvedirect}. The
next Section~\ref{nonconstructive} explains the concept of
``non-constructive recognition'' which means to determine
the isomorphism type of a given group. Once this is known, the
concept of standard generators applies, which is introduced in
Section~\ref{standardgens}. These concepts in turn are usually needed to apply
more specialised methods for constructive recognition, of which an
overview is given in Section~\ref{solveD8} for classical groups in
their natural representation and in
Section~\ref{solveD9} for almost simple groups.

\section{Direct methods for constructive recognition}
\label{solvedirect}

For ``small groups'' one can use permutation group methods to solve
Problem~\ref{ProbCR3} for a group $\bar G \le \PGL(n,q)$. The basic idea is
to find a permutation action. This immediately gives a homomorphism
into a permutation group which will be an isomorphism if the group is
simple. Even if there is a non-trivial kernel the composition tree setup
will take care of this.

Matrix groups and projective groups of course act on their natural
module and thus on vectors and subspaces. So, finding some
action is not difficult. Achieving good performance however can be
tricky. To this end we want to find short orbits. Heuristic methods
for this for matrix groups can be found in \cite{shortorbits}.

Another possibility is low index methods.
Here one would guess the point stabiliser of a point with a short
orbit, restrict the natural module to it and use the MeatAxe to find a
proper invariant subspace. The orbit of this subspace would then be
relatively short. However, although this approach looks promising and
occasionally works, it is not clear how to find generators of such a
point stabiliser, even with random methods.

Once we have an action and thus a homomorphism into a permutation 
group we can either use the methods described in Section~\ref{permgrps}
or immediately compute a stabiliser chain for the projective or matrix
group using the Schreier-Sims method (see \cite{nearlylin} or
\cite{Ser}). Both approaches solve the constructive recognition
problem \ref{ProbCR3} even if the group is not a simple group.

Note that for larger dimensions and in particular for classical groups
we do not even want to try these direct methods because any orbit we
can possible find would be prohibitively large.


\section{Non-constructive recognition}
\label{nonconstructive}
\index{non-constructive recognition}%

The term ``non-constructive recognition'' for a group means finding
the isomorphism type. In situations where direct methods for constructive 
recognition (see the previous Section~\ref{solvedirect}) fail, the usual
approach is to determine the isomorphism type of the group first and
then use additional knowledge about the group in question to do the
constructive part of the recognition. In particular to use the
technique of standard generators (see the next
Section~\ref{standardgens}), the isomorphism type has to be known in
advance.

For the non-constructive recognition problem we can for example use
element order statistics. We produce uniformly distributed elements in
the group and compute their orders. If we see an element order that
does not occur in a group of a certain isomorphism type, we can
immediately rule out this type. If we fail to see an element order that
occurs very frequently in a certain isomorphism type after some tries,
we can rule out this type with a known error probability.

See Sections~\ref{solveD8} and \ref{solveD9} for an overview of the
methods for non-constructive recognition for classical
groups in their natural representation (class \DD8) and for groups in class \DD9
respectively.

\section{Standard generators}
\label{standardgens}
\index{standard generators}%

In this section we give a definition of the term
``standard generators''. As in this whole chapter we do not want to go
into too much detail but instead give the reader an idea of the
concept.

Once the isomorphism type of a group $G = \left< g_1, \ldots,
g_k\right>$ is known (after successful non-con\-struc\-tive recognition as in
Section~\ref{nonconstructive}), one wants to find an explicit
isomorphism of $G$ to a ``standard copy $\hat G$''. Note that such an
isomorphism is not automatically ``computable'' in the sense that 
we can map group elements back and forth, as we will see below.
However, in the end we strive to use previously acquired and stored
knowledge about $\hat G$ and transfer it over to $G$ via that
isomorphism to eventually solve the constructive recognition problem (see
\ref{ProbCR3}) for $G$.

The concept of standard generators serves this purpose.

\begin{Def}[Standard generators]
\index{standard generators}%
    Let $G$ be a finite group and $\Aut(G)$ its automorphism group. Then
    $\Aut(G)$ acts on tuples of elements of $G$ componentwise. We choose
    one orbit of this action that contains tuples whose entries
    generate $G$ as a group, and call exactly those tuples in this
    orbit \emph{standard generators} for $G$. This choice is done once
    and forever for every isomorphism type of finite group and the
    chosen orbit is described by giving a set of properties of the
    tuples that uniquely determines the orbit.
\end{Def}

\begin{Rem}
Note that this choice has to be done individually for every
isomorphism type of finite groups and the properties have to be
determined intelligently, such that finding a tuple of standard
generators is possible efficiently (see Section~\ref{goodstandgens}).
\end{Rem}

This rather vague definition can best be filled with life by an
example:

\begin{Exa}[Standard generators for the sporadic simple Mathieu group
    $M_{11}$]
    \label{ExaM11}
    This description is taken from \cite[$M_{11}$ page]{WWWAtlas} and
\index{WWW-Atlas of group representations}%
    is derived in \cite[Example~11]{standgens}.

    Standard generators of $M_{11}$ are $(a,b)$ where $a$ has order $2$, 
$b$ has order $4$, $ab$ has order $11$ and $ababababbababbabb$ has 
order $4$. Note that it is a theorem that these properties uniquely
determine an orbit of $\Aut(M_{11})$ on pairs of elements of $M_{11}$.

\smallskip
To find standard generators for $M_{11}$:
\begin{enumerate}
        \setlength{\parskip}{0pt}
    \item Find an element of order $4$ or $8$. This powers up to $x$ of order 
        $2$ and $y$ of order $4$.

      [The probability of success at each attempt is $3$ in $8$.]
\item Find a conjugate $a$ of $x$ and a conjugate $b$ of $y$ such that $ab$ 
    has order $11$.

      [The probability of success at each attempt is $16$ in $165$.]
  \item If $ababbabbb$ has order $3$, then replace $b$ by its inverse.
  \item Now $ababbabbb$ has order $5$, and standard generators of $M_{11}$ 
      have been obtained.
\end{enumerate}
It is a theorem that this procedure produces standard generators and the
probabilities can be read off the character table of $M_{11}$. Of course 
we assume uniformly distributed random elements for these
probabilities to be correct.
Note that choosing random conjugates of $x$ that are uniformly distributed in the
conjugacy class of $x$ can be achieved by conjugating $x$ with a random element
that is uniformly distributed in the group.
\end{Exa}

\begin{Prop}[The virtue of standard generators]
\index{standard generators}%
If $(s_1, \ldots, s_m) \in G^m$ and $(t_1, \ldots, t_m) \in G^m$ are 
both standard generators for a group $G$, then the equations
\[ \varphi (s_i) = t_i \qquad\mbox{for all } 1 \le i \le m \]
uniquely define an automorphism $\varphi$ of $G$.

That is, if we have a tuple of standard generators $(s_1, \ldots, s_m)$ for $G$
and $\hat G$ is a standard isomorphic copy of $G$ for which we know
a tuple of standard generators $(u_1, \ldots, u_m)$, then the
equations
\[ \psi (s_i) = u_i \qquad\mbox{for all } 1 \le i \le m \]
uniquely define an explicit isomorphism $\psi$ from $G$ to $\hat G$.
\end{Prop}
\proofbeg
Since by definition both tuples lie in the same orbit under $\Aut(G)$
and both tuples generate $G$, there is exactly one automorphism
$\varphi \in \Aut(G)$ mapping $s_i$ to $t_i$ for all $1 \le i \le m$.
The hypothesis that $\hat G$ is isomorphic to $G$ takes care of the
second statement.
\proofend

\begin{Rem}[The problem of mapping elements]
Note that even if we have found standard generators in $G$, the above
definition of the explicit isomorphism $\psi$ to $\smash{\hat G}$ does in fact 
\emph{not}
enable us to map arbitrary elements of $G$ via $\psi$, because for
this we would have to express an arbitrary element of $G$ as a
straight line program in $(s_1, \ldots, s_m)$, which is exactly the
\index{straight line program}\index{SLP}%
constructive recognition problem we want to solve!

However, if we want to store certain elements or subgroups of $\hat
G$ beforehand, we can store them as straight line programs in $(u_1,
\ldots, u_m)$ and can then evaluate these straight line programs
in $(s_1, \ldots, s_m)$ to actually get their images under $\psi^{-1}$
in $G$. This fact helps to transfer previously acquired knowledge from
$\hat G$ to $G$.
\end{Rem}

\begin{Rem}[Good standard generators]
\label{goodstandgens}
\index{standard generators}%

We want to comment only briefly on this topic. Basically, the choice
of the standard generators for an isomorphism type of group, that is 
the choice of the $\Aut(G)$-orbit 
in the tuples of elements of $G$, is ``good'', if it is relatively
easy to find a tuple of standard generators by random methods. The
example in Section~\ref{ExaM11} exhibits this. The probabilities to find
the right elements in the algorithm presented there are quite good,
such that very few random elements will usually lead to success.

A large collection of such good choices of standard generators together 
with algorithms to find them can be found on the WWW-Atlas of group
\index{WWW-Atlas of group representations}%
representations, see \cite{WWWAtlas}.
\end{Rem}

\begin{App}[Storing hints for stabiliser chains]
\label{hintsstabchains}
\index{hints for stabiliser chains}%
One immediate application of standard generators is the following. If
a group has a subgroup $U$ with a relatively low index, then we can store
generators for this subgroup as a straight line program in standard
generators. Once we have recognised the isomorphism type of $G$
using non-constructive recognition and have found standard generators
in $G$, we can evaluate this straight line program, get a
generating set of a subgroup $U$ of $G$ with this low index, restrict
the natural module to $U$ and find a proper invariant subspace. Provided
that $G$ acts irreducibly on its natural module, the
$G$-orbit of this subspace in the natural action on subspaces then
gives a permutation action of $G$ which is isomorphic to the one on
cosets of $U$. Note that the existence of such a proper invariant subspace 
of course depends not only on the isomorphism type of $G$ but rather
also on the actual representation $G$ comes in.
Thus, using standard generators in this way, we can collect hints
to find ``good'' actions for the different absolutely irreducible
matrix representations of a group.

Eamonn O'Brien and Robert Wilson have for example done exactly this
for the sporadic simple groups. Their hints data is available in the
{\MAGMA} system (see \cite{Magma}) and will be used in the composition tree
implementation in the {\GAP} system as well.
\end{App}
\index{standard generators}%

\section{The classical case in natural representation: \DD8}
\label{solveD8}

The seminal paper by Neumann and Praeger \cite{neumann-praeger} which
presents an algorithm to decide whether a given group $G \le \GL(n,q)$
contains the special linear group was the starting point of a whole
industry of papers concerned with non-constructive and constructive
recognition of groups.

An algorithm to recognise classical groups in their
natural representation non-constructively is given in
\cite{classicalnonconstructive}. Once this is done, other algorithms
apply: In \cite{slrecogconstr} an algorithm to recognise $\SL(n,q)$
in its natural representation constructively is presented with
effective cost $O(n^4q)$. For arbitrary classical groups in their
natural representation the results in \cite{peteconstructiveclassical}
give algorithms to solve the constructive recognition problem with
effective cost $O(n^5 \log^2 q)$, however these algorithms need an
$\SL(2,q)$ oracle, that is, they rely on a solution of the constructive
recognition problem for $\SL(2,q)$. Recently a new preprint (see
\cite{recogclassicalodd}) has appeared which handles the case of odd
characteristic.

The basic idea of all these constructive recognition algorithms is to find
a tuple of standard generators and then perform a base change
such that linear algebra methods can be used to express arbitrary
elements as straight line programs in the standard generators.


\section{The almost simple case: \DD9}
\label{solveD9}
\index{almost simple}%

This part of the whole group recognition project is probably the one
for which the currently known methods are least satisfying. In particular
because many isomorphism types of groups have to be dealt with in a case by
case fashion a lot of work still needs to be done.
In this section we try to give an overview of the known methods by
means of references to the literature. We intentionally leave out
complexity results for the sake of brevity and because the last word on
these does not yet seem to be spoken. A more detailed account of
the state of the art can be found in \cite{OB}.

We begin with a discussion of the problem with the ``almost'' in
``almost simple''.

If a group $G \le \GL(n,q)$ with $Z := G \cap Z(\GL(n,q))$ 
is contained in class \DD9, there is a
non-abelian simple group $\bar N$ and a group $T$ with $\bar N
\leq T \leq \Aut(\bar N)$ and $G/Z \cong T$. The first problem for
both constructive and non-constructive recognition is that $G/Z$ is
itself not necessarily simple, after all, $G/Z$ is only ``almost simple''.
However, the Schreier conjecture, which follows from the classification
of finite simple groups, says that $\Aut(\bar N) / \bar N$ is solvable
for all finite simple groups~$\bar N$. 

In fact, this outer automorphism
group is rather small for groups occurring in practice. 
If $\bar N$ is alternating or sporadic, then
$|\Out(\bar N)| = 2$ except for $|\Out(A_6)| = 4$.
In \cite[Lemma 1.4]{LucchiniMorigi} it is shown
that for a simple group $\bar N$ of Lie type in cross-characteristic, 
$|\Out(\bar N)| \le \beta \log n$ for some global constant
$\beta$. If $\bar N$ is a simple group of Lie type in its defining
characteristic, then $|\Out(\bar N)| \le 2(n+1)\log q$ by
\cite[Proof of Lemma 1.3]{LucchiniMorigi}.

So the first step for recognition is to find the simple subgroup
isomorphic to $\bar N$ of $G/Z$, we denote this by $N/Z$ in the following. 
One approach is to go down the derived series
until a group is reached which is ``probably perfect''. To test for
the latter, one uses the algorithm by Leedham-Green and O'Brien in
\cite[5.3]{RecogTensInd}. This algorithm estimates the order of an element in a
factor group if the group and the normal subgroup are given.
This technique produces generators for $N$ efficiently.

Most publications in this area seem to concentrate on the
non-constructive and constructive recognition of simple groups.
However, to completely solve Problem~\ref{ProbCR3} for all groups in
\DD9 one has to be able to do constructive recognition for all
almost simple groups. There seems to be no method known to get hold of
the factor group $G/N$ in general since in most cases
the restriction of the natural module to $N$ is absolutely
irreducible.  We will ignore this problem here, in particular since $G/N$
is very small in most cases, such that solving the constructive recognition
problem for $N$ together with a few coset case distinctions can solve the
problem in $G$ satisfactorily.

Next we try to give an overview of the known techniques for
non-constructive recognition of simple groups. Alternating groups and
sporadic groups can be recognised non-constructively by looking at
element orders of random elements. For Lie type groups there is an
algorithm for non-constructive recognition in \cite{blackboxlienonconstr}.
One outstanding case in this paper is covered by methods in
\cite{altseimer}. However, both algorithms need to know the defining
characteristic of the Lie type group in advance. To determine this,
there are three concurrent methods, one is described in
\cite{primpowgraphs}, a newer one in \cite{findingcharlie} and thirdly there
is unpublished work by Seress which uses statistics about the two largest 
projective element orders occurring.

In 2001 Malle and O'Brien developed a practical implementation of all 
the algorithms for non-constructive recognition known at the time,
which is distributed with the {\MAGMA} system.

Assuming from now on that the non-constructive recognition is achieved we
conclude this section with a list of references to work that has been done
to solve the constructive recognition problem for classes of simple groups.
Note that many of these solutions include the corresponding almost simple
groups thereby solving specific cases of \DD9 groups completely.

An algorithm to recognise the alternating group $A_k$ and the symmetric
group $S_k$ constructively in an arbitrary 
representation is described in \cite{bbsymaltconstr}. An alternative
algorithm was developed in \cite{bratuspak}.

For classical groups in another than their natural representation (see
Section~\ref{solveD8} for the \DD8 case) there is an algorithm in
\cite{bbclassical}. The particularly important case of the special linear
group $\SL(2,q)$ of rank $2$ in non-natural but defining characteristic
representation is dealt with in \cite{classicallargefield}
and \cite{psl2qconstr} in the sense that an algorithm is described to find
an explicit computable epimorphism to $\PSL(2,q)$ in its natural
representation. Further work to improve the algorithms for simple classical 
groups in arbitrary representations can be found in \cite{bbomega},
\cite{bbunitary}, \cite{bbpsldq}, \cite{computingmatrix} and
\cite{bbortho}. Recently the preprint \cite{smalldegreegl} appeared which
deals with recognising small degree representations of general linear
groups specifically.

For Lie type groups work on constructive recognition has appeared in
\cite{recogSL3}, \cite{rybaid} and \cite{suzukiconstr}. An article about 
constructive recognition of exceptional groups of Lie type by Kantor and Magaard is
in preparation. However, this area is still work in progress.

For the sporadic simple groups the data collected by O'Brien and Wilson
about subgroup chains (see Section~\ref{hintsstabchains}) can be used to
solve the constructive recognition problem after non-con\-struc\-tive
recognition using the direct methods mentioned in
Section~\ref{solvedirect}. 

We finish this section by mentioning two other approaches to the
constructive part of the recognition problem.

One is published
in the preprint \cite{bbconstrmember}. It uses involution centralisers
and the fact that generators for them can be computed efficiently (see
\cite{BrayInv}). One instance of the constructive membership test problem
for a group $G$ is translated into three instances of the corresponding
problem in the centralisers of certain involutions which come up during the
run of the algorithm.

Finally we want to mention the paper \cite{gensift}, which introduces a
generic framework for constructive membership testing in a group whose
isomorphism type is known. It relies on standard generators (see
Section~\ref{standardgens}) and prepared subgroup chains stored as straight
line programs in the standard generators.
\index{straight line program}\index{SLP}%

% REFERENCE: Kantor, Seress: Black box classical groups???

% this is a part of the habilitation thesis of Max Neunhoeffer

\chapter{Recognising subfield and semilinear}
\label{chap:subsemi}

to be written

% this is a part of the habilitation thesis of Max Neunhoeffer

\chapter{Further algorithms}
\label{chap:furtheralg}



\appendix

% this is a part of the habilitation thesis of Max Neunhoeffer

\chapter{Addendum}

to be written


%\mbox{}
\thispagestyle{fancy}

\chapter{Notations}

\label{NotationIndex}
\begin{longtable}{|lll|}
\hline\endfoot
\hline\endhead
\hline
\multicolumn{3}{|l|}{Symbols:}\\
\hline
$\emptyset$             & the empty set
                        & \\
$\subseteq$             & is contained in
                        & \\
$\supseteq$             & contains
                        & \\
$\subsetneq$            & is contained properly
                        & \\
$\supsetneq$            & contains properly
                        & \\
$\circ$                 & concatenation of mappings
                        & %\ref{conventions} 
                        \\
$\circlearrowleft$      & commutative diagram 
                        & \\
$\hookrightarrow$       & injective mapping
                        & \\
$\twoheadrightarrow$    & surjective mapping
                        & \\
$\cong$                 & isomorphism
                        & \\
$\equiv$                & congruence
                        & \\
$\lceil x \rceil$       & smallest integer $\ge x$
                        & \\
$(-|-)$                 & bilinear form
                        & \\
$\displaystyle\prod$    & product
                        & \\
$\displaystyle\coprod$  & coproduct
                        & \\
$\displaystyle\bigoplus$ & direct sum of modules
                        & \\
$\left< - \right>_A$    & $A$-span
                        & \\
$\oplus$                & direct sum
                        & \\
$\triangleleft$         & is ideal in 
                        & \\
$\varphi^{-1}(N)$       & full preimage of the set $N$ under $\varphi$
                        & \\
$\mathbb{C}$            & set of complex numbers
                        & \\
$\det M$                & determinante of the matrix $M$
                        & \\
$E_d$                   & $d \times d$ identity matrix
                        & \\
$\End_G(M)$             & set of endomorphisms of the $G$-module $M$
                        & \\
$\F_q$                  & field with $q$ elements
                        & \\
$\GL_n(q)$              & group of invertible matrices in $\F_q^{n \times n}$
                        & \\
$\Hom_G(M,N)$           & set of homomorphism between the $G$-modules $M$
                          and $N$
                        & \\
$\id_M$                 & identity mapping $M \to M$
                        & \\
$\Ima f$                & image of the mapping $f$
                        & \\
$\ker f$                & kernel of the mapping $f$
                        & \\
$\N$                    & set of natural numbers, $0 \notin \N$
                        & \\
$\Q$                    & set of rational numbers
                        & \\
$\R$                    & set of real numbers
                        & \\
$\rad(R)$, $\rad(M)$    & Jacobson-radical of a ring $R$ or module $M$
                        & %\ref{conventions} 
                          \\
$S_n$                   & symmetric group on $n$ points
                        & \\
$\soc(M)$               & socle of a module $M$
                        & \\
$\Z$                    & set of rational integers 
                        & \\
\hline
\end{longtable}

\phantomsection{}
\label{NotationIndexEnd}

\clearpage
\markboth{Appendix C: List of figures}{Appendix C: List of figures}

\newcommand{\friss}[1]{}
\newcommand{\myitem}[2]{\rlap{#1}\hspace*{\cftchapnumwidth}{#2}}

\phantomsection{}
\addcontentsline{toc}{chapter}{\myitem{C}{List of figures}}
\vspace*{10mm}
{\huge\bf Appendix C}

\listoffigures

\phantomsection{}
\addcontentsline{toc}{chapter}{\myitem{D}{List of tables}}
\vspace*{28mm}
{\huge\bf Appendix D}

\listoftables

\stepcounter{chapter}
\stepcounter{chapter}

\markboth{Appendix C: List of figures}{Appendix C: List of figures}

%\newpage\mbox{}
%\clearpage\mbox{}\thispagestyle{fancy}

{\sloppy
\bibliographystyle{alpha}
\bibliography{habil}
}

\printindex

\end{document}

