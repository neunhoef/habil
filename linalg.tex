% is is a part of the habilitation thesis of Max Neunhoeffer

\chapter{Further linear algebra algorithms}

In this chapter we collect for the sake of completeness some
linear algebra algorithms together with their complexity analysis
that are used in later chapters. 

\section{Order and projective order of a matrix}
\label{sec:orders}

This section is about the computation of the order and the projective
order of a square matrix. 

Let $\F$ be a field and $M \in \F^{n \times
n}$ an invertible matrix. The \emph{order} of $M$ is the least natural
number $o$ such that $M^o$ is equal to the identity matrix $\mathbf{1}$.
The \emph{projective order} of $M$ is the least natural number $p$
such that $M^p$ is a scalar matrix, that is, a scalar multiple of the
identity matrix. It follows immediately by division with remainder in
the exponent that $o$ divides every natural number $k$ for which 
$M^k$ is equal to $\one$ and that $p$ divides every natural number $k$
for which $M^k$ is a scalar multiple of $\one$. Thus $p$ divides in
particular $o$ and it follows immediately, that if $M^p = \lambda \cdot
\one$ with $\lambda \in \F$, then $o$ is $p$ times the order of the
scalar $\lambda$.

For a monic polynomial $f \in \F[X]$ with non-zero constant term
we define \emph{order} (\emph{projective order} resp.) of 
$X + f\F[X]$ in $\F[X]/f\F[X]$ modulo $f$ as the least natural 
number $p$ such that $X^p$ is congruent 
to $1$ (a scalar resp.) modulo $f$ or equivalently, that $X^p-1$ 
($X^p-\lambda$ resp.) is
divisible by $f$ (for some $\lambda \in \F$). 
The order and projective order
of a matrix $M$ as above and that of $X$ modulo its minimal polynomial are 
linked by the following lemma:

\begin{Lemm}[Orders and projective orders]
Let $\F$ be a field and $M \in \F^{n \times n}$ an invertible square matrix.
Then the (projective) order of $M$ is equal to the (projective) order of
$X + \mu_M \F[X]$ module its minimal polynomial $\mu_M$.
\end{Lemm}
\proofbeg 
The polynomial $X^p-\lambda$ is divisible by $\mu_M$ if and
only if $M^p = \lambda \one$ for all $\lambda \in \F$.
\proofend

In the following we use this lemma to switch between matrices and
polynomials as seems appropriate for the argumentation.

We now want to discuss the computation of both the order and the
projective order of a matrix or its minimal polynomial respectively.

Let $f \in \F[X]$ be a monic polynomial with non-zero constant term. 
By the Chinese Remainder theorem the factor ring
$\F[X]/f\F[X]$ is isomorphic to
\[ \F[X]/f\F[X] \cong
   \prod_{i=1}^k \F[X]/f_i^{e_i} \F[X] \]
where $f = \prod_{i=1}^k f_i^{e_i}$ is the factorisation of $f$ into
its pairwise distinct irreducible factors $f_i$. Using this
isomorphism the (projective) order of $X+f\F[X]$ is equal to the least 
common multiple of the (projective) orders of the $X + f_i^{e_i}
\F[X]$ in $\F[X]/f_i^{e_i}\F[X]$. Thus, as a first step we factorise
$f$ completely and now determine the (projective) order of
$X+g^e\F[X]$ for an irreducible monic polynomial $g$ of degree $d$ 
with non-zero constant term.

From now on we switch again to matrices. Let $C \in \F^{d \times d}$ be 
the companion matrix of $g$ and $N$ the $(de \times de)$-block matrix with 
$C$ along the main block diagonal, $(d \times d)$-identity matrices
along the block diagonal directly above the main diagonal and zero blocks
elsewhere:
\[ N = \left[ \begin{array}{ccccc}
    C      & \one   & 0      & \cdots & 0 \\
    0      & C      & \one   & \ddots & \vdots \\
    \vdots & \ddots & \ddots & \ddots & 0 \\
    \vdots & \ddots & \ddots & C      & \one \\
    0      & \cdots & \cdots & 0      & C
\end{array} \right]. \]
The minimal polynomial of $N$ is $g^e$ which can be seen as follows:
Let $K$ be the splitting field $K$ of $g$ over $\F$. Since $g$ is
irreducible and thus has no multiple roots, the 
matrix $C$ is similar to a diagonal matrix over $K$, that is, there is an
invertible $T \in K^{d \times d}$ such that $TCT^{-1}$ is diagonal
with pairwise disjoint eigenvalues.
Thus, conjugating $N$ with the block matrix having $T$ along the
main block diagonal and permuting rows and columns suitably shows that
the Jordan normal form of $N$ over $K$ consists of $d$ blocks of
size $e$, thus every eigenvalue of $C$ occurs in the minimal
polynomial of $N$ with multiplicity $e$. Thus the minimal polynomial
of $N$ over $K$ is $g^e$ and thus also $\mu_N = g^e$.

We assume from now on that $\F$ is a finite field $\F_q$ with $q$
elements.
Writing $N = \tilde C + \tilde \one$ where $\tilde C$ is the matrix
with only $C$ along the main block diagonal and $\tilde \one$
accordingly we have $\tilde C \cdot \tilde \one = \tilde \one \cdot \tilde C$
and thus $N^k = \sum_{i=0}^k {k \choose i} \tilde C^{k-i} \tilde
\one^i$. This immediately implies that the $(d \times d)$-block
of $N^k$ in position $(j,j+i)$ is ${k \choose i}C^{k-i}$ for all
$1 \le j \le d-i$. 

Therefore the (projective) order of $N$ can be determined
in the following way: Let $k$ be the least natural number, such that
all binomial coefficients ${k \choose i} = 0$ for $1 \le i \le e$
(where we set ${k \choose i} = 0$ for $i > k$. Let further $l$ be the
(projective) order of $X+g\F[X]$ in the field $\F[X]/g\F[X]$. Then the 
(projective) order of $N$ is the least common multiple of $k$ and $l$.



\section{Solving systems of linear equations}
\label{sec:syslineq}

\section{Inverting matrices}
\label{sec:invert}

