\documentclass[11pt]{article}

\usepackage[a4paper,top=1in,bottom=1in]{geometry}
\usepackage[mtpluscal]{mathtime}
\usepackage{times}

\begin{document}
\title{Summary of content}
\author{Max Neunh\"offer}
\maketitle

The matrix group recognition project has produced many results but is
not yet fully completed. This book gives an overview of the current
state of the art and shows some contributions of the author. The
collaborative nature of the whole project lends itself to joint papers
by more than one person. Therefore, some of the author’s work in the
context of this project, which can be found in Chapters III, V and VII,
are in fact collaborations with other authors.

A second major part of the contributions of the author to the project
lies in actual implementations. Together with \'{A}kos Seress
we have started an implementation of the best known algorithms
for group recognition in the GAP computer algebra system. This
implementation comes in the form of two GAP packages \textsf{recogbase}
and \textsf{recog} which were first published in 2009. The first
provides a generic framework to implement composition trees in GAP
for arbitrary types of groups. One interesting feature of the generic
framework is that within a single composition tree there can be groups
in different representations, that is, there can be a mixture of
permutation groups, matrix groups and projective groups. The second
package tries to collect the best known methods for group recognition
for each of these types of groups. Thus, everybody who contributes
implementations will be an author of the recog package.

This book describes a major part of the algorithms used in both
packages. For some methods, mainly for the leaves of the composition
tree, the reader is however referred to the literature. Since the
matrix group recognition project is not yet finished, the present work
cannot give a complete description of a solution to the constructive
recognition problem. In particular for the leaves of the composition
tree a lot of work and improvement of algorithms still has to be done,
as is explained in some detail in Chapter VIII.

We have intentionally left out a number of topics which are related,
mostly because research on them is not completed or not even in a state
for a satisfactory description. One is the whole field of algorithms
for matrix groups that build on constructive recognition. Derek Holt
and Mark Stather have recently published a paper pointing in that
direction and Mark Stather’s PhD thesis also lies in this area.
Another topic left out is the constructive recognition of black box
groups, because we wanted to concentrate on matrix groups and projective
groups. We only cover the verification phase very briefly, although
this phase is necessary to have computational proofs of the results.
The reason for this is that in fact very little work has been done on
the actual implementation of verification routines. What is needed for
this is basically good presentations for the groups occurring in the
leaves. Finally, a global analysis of the whole procedure of building a
composition tree is still needed. This seems achievable using estimates
on the length of a composition series but is not done yet.

Now we turn to the contents of this book and outline its structure.
Chapter I introduces briefly some concepts from computer science, which
apply to computational group theory, and describes a way to produce
nearly uniformly distributed random elements in a finite group. Chapter
II introduces a new method to implement matrices over finite fields
on a computer. Having an efficient implementation of the finite field
arithmetic and linear algebra routines is obviously an indispensable
foundation for implementing matrix group algorithms. The contents of
this chapter are not yet published elsewhere.

Chapter III contains a new randomised method to compute the minimal
polynomial of a square matrix over a finite field. It is basically
a copy of a paper jointly written by Cheryl Praeger and the author.
Computing minimal polynomials of invertible matrices is an important
ingredient to compute the order and projective order of such matrices,
which are computations that are used throughout nearly all matrix group
algorithms.

Chapter IV completes the description of the infrastructure for
implementing matrix group algorithms by explaining how to compute the
order and projective order of a matrix and how to perform higher level
linear algebra computations like solving systems of linear equations
and inverting matrices. In addition the two major obstacles for
polynomial time algorithms, the discrete logarithm problem and integer
factorisation, are introduced. This chapter is nothing new but is needed
for the sake of completeness.

In Chapter V we explain the basic problem attacked by the matrix group
recognition project, namely the constructive recognition problem. We
give a gentle introduction starting with a rough formulation of the
problem followed by two refinements. Then the fundamental approach using 
composition trees and a generic framework for group recognition
are described. We explain in detail the idea and purpose of reduction
homomorphisms. The contents of this chapter are a variation and
extension of a paper by \'{A}kos Seress and the author, in which the
composition tree approach is refined by allowing for a change in
the generating set of the group to be recognised. This refinement can
dramatically increase the performance, because the resulting straight
line programs are much shorter than in the traditional version. Finally,
the chapter closes with a description how the currently best known
algorithms for the constructive recognition of permutation groups fit in
nicely with the proposed framework.

The next Chapter VI explains a variant of Aschbacher’s theorem on
subgroups of classical groups restricted to the general linear group
case. A relatively short complete proof is given. Our variant changes
the definition of the occurring classes of subgroups slightly to make
it easier to devise algorithms for finding reduction homomorphisms
for groups in some of these classes. This approach already seems
to bear fruit in the last part of the chapter, where we present an
overview of algorithms to find reduction homomorphisms for groups in the
different classes. For our classes $\mathcal{D}_2$, $\mathcal{D}_4$ and
$\mathcal{D}_7$ new ideas to tackle groups in these classes are given
as well as references to the literature for the currently best known
methods.

Chapter VII is a copy of a paper which is joint work of the author
with Jon Carlson and Colva Roney-Dougal. It presents new completely
analysed randomised algorithms to find a reduction for the case that
$G \le \mathrm{GL}(n, q)$ acts irreducibly on its natural module and
lies in at least one of the semilinear or subfield Aschbacher classes
$\mathcal{C}_3$ and $\mathcal{C}_5$.

Chapter VIII finally tries to summarise the state of the art for
algorithms to do constructive recognition for the leaves of the
composition tree, that is for groups in the Aschbacher classes
$\mathcal{C}_8$ and $\mathcal{C}_9$. We do not try to explain any of the
best known methods there, because the final word on them seems not to be
spoken at the time of this writing. Rather, we explain the concepts of
non-constructive recognition and standard generators and give references
to the literature.

\end{document}
