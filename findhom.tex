% this is a part of the habilitation thesis of Max Neunhoeffer

\chapter{Finding homomorphisms}
\label{chap:findhom}

In the previous Chapter~\ref{chap:comptree} we have formulated the problem
of constructive recognition of groups in Problem~\ref{ProbCR3} and have
explained what a reduction is. This chapter describes how to find
reductions for matrix groups and projective groups and how to prove
that a certain collection of methods will for any given
matrix group or projective group either find a reduction
or show that a situation is on hand, in which the constructive 
recognition problem can be solved efficiently by other means. By the 
arguments in Chapter~\ref{chap:comptree} this solves
Problem~\ref{ProbCR3} whenever the ambient group is $\GL(n,q)$ or
$\PGL(n,q)$.

We use two fundamental theoretical tools. The first is Aschbacher's theorem
about the subgroup structure of the classical groups, which we describe
in detail for the case of $\GL(n,q)$ in Section~\ref{sect:aschbacher},
and the second is the classification of finite simple groups. We are
content with the statements in Aschbacher's theorem about the general
linear group since they suffice for the purposes of constructive
recognition and the statements are quite a bit simpler than for the
other classical groups. For details see \cite{aschbacher}.

Aschbacher's theorem states that every subgroup of $\GL(n,q)$
is either a member of at least one of $7$ concretely given classes 
\CC1 to \CC7 of 
subgroups, or it contains a classical group in its natural representation, 
or it is an almost-simple group. IS THAT RIGHT?
% FIXME

All the classes \CC1 to \CC7 are somehow defined in a geometric way (see
Sections~\ref{descC1} to \ref{descC7}) and thus promise some kind of
reduction. The two other cases are covered by two further classes \CC8
and \CC9, which are described in Sections~\ref{descC8} and \ref{descC9}.
For members of the latter two classes one will usually have to solve the
constructive recognition problem without further reduction.
% FIXME

The idea is to provide efficient algorithms for all the classes \CC1 to
\CC7 to recognise whether a given matrix group lies in the class, and if
so, to find a reduction using this information. If none of these algorithms
succeeded, Aschbacher's theorem shows that the group must be a member of
\CC8 or \CC9. In that case the constructive recognition problem has to be
solved by different means, usually by first finding out which classical
or almost simple group it is and then using this information to do the
constructive recognition in a special case, for example using standard
generators (see Sections~\ref{solveC8} and \ref{solveC9}).

The purpose of this chapter is to explain the statement of Aschbacher's 
theorem for $\GL(n,q)$ in detail and to give an overview over the known
methods to deal with the different classes together with references into
the literature. An algorithm to recognise and handle classes \CC3 and \CC5
provided that the group does not lie in class \CC1 is described in detail
in Chapter~\ref{chap:subsemi}.

\section{Aschbacher's Theorem}
\label{sect:aschbacher}

Note again that we restrict ourselves to the general linear group
throughout, which is only a special case of Aschbacher's Theorem.

\begin{Not}
For this section we fix $n \in \N$ and $q=p^e$ for a prime $p$ and
talk about the group $\GG := \GL(n,q)$. We denote the vector space $\F_q^{1
\times n}$ by $V$ and note that $\GG$ acts from the right on $V$ by
vector-matrix multiplication.
\end{Not}

For the original formulation of his theorem in \cite{aschbacher}, Aschbacher 
defines $8$ classes
of subgroups of $\GG$ and proves that every subgroup $G \le \GG$ is either
a subgroup of some member of one of these $8$ classes or has a certain
amount of properties. We want to change this formulation in the following
way: We say that a group $G \le \GG$ lies in ``Aschbacher class \CC i''
for $1 \le i \le 8$ if and only if it is a subgroup of a member of
the class with number $i$ in \cite{aschbacher}. Furthermore, we collect
the subgroups $G \le \GG$ that fulfil the properties in the statement of 
Aschbacher's Theorem in the class \CC9. Therefore, we can formulate
Aschbacher's Theorem in the following way:

\begin{Theo}[Aschbacher's Theorem, specialised to\/ $\GL(n,q)$]
\label{Asch}
Let $G$ be a subgroup of\/ $\GG = \GL(n,q)$.
Then $G$ is contained in at least one of the
classes \CC1 to \CC9 of subgroups described in Sections~\ref{descC1}
to \ref{descC9}.
\end{Theo}
\proofbeg See \cite{aschbacher}, \cite{kleilieb} and
\cite[Theorem~1]{smashprim}. Note that \cite[Section~2]{smashprim}
contains a great part of a proof for the special case of $G < \GL(n,q)$.
\proofend

\begin{Rem}
%In our description of the classes \CC1 to \CC9 we follow
%\cite{kleilieb}. Kleidman and Liebeck change the definition 
%slightly in comparison to Aschbacher but argue that Theorem~\ref{Asch}
%remains true with their definitions (see \cite[Chapter~4]{kleilieb}).
% FIXME
\end{Rem}

We go on with the description of the Aschbacher classes. For each class, we
also give the structure of the maximal members, this information is from
\cite[Table 3.5.A]{kleilieb}.

\subsection{Description of class \CC1: ``reducible''}
\label{descC1}

\newcommand{\desc}[1]{\begin{center}\fbox{\parbox{5in}{#1}}\end{center}}
\newcommand{\stru}{\textbf{Structure of maximal members:}\par}

\desc{
A subgroup $G \le \GG$ is a member of \CC1 if there is a subspace
$W < V$ that is stabilised by $G$, that is, $Wg = W$ for all $g \in G$.
}

\stru
Every such group is conjugate in $\GG$ to a subgroup of a group
\[ P_m := \left\{ \left[ \begin{array}{cc} A & 0 \\ C & D \end{array} \right]
           \mid A \in \GL(m,q), D \in \GL(n-m,q) \mbox{ and }
           C \in \F_q^{n-m \times m} \right\} \]
for some $0 < m < n$. The group $P_m$ is called a \emph{maximal parabolic
subgroup} and is a semidirect product
of the normal $p$-subgroup
\[ U_m := \left\{ \left[ \begin{array}{cc} \one_m & 0 \\ C & \one_{n-m} 
           \end{array} \right]
           \mid 
           C \in \F_q^{n-m \times m} \right\} \]
and $\GL(m,q) \times \GL(n-m,q)$, the factors being embedded on the
diagonal blocks. 
Here, $\one_m$ is the $(m \times m)$-identity matrix and $\one_{n-m}$ is the
$((n-m) \times (n-m))$-identity matrix.

\subsection{Description of class \CC2: ``imprimitive''}
\label{descC2}

\desc{
A subgroup $G \le \GG$ is a member of \CC2 if there is a decomposition
of $V$ as a direct sum of $m$-dimensional subspaces 
$V = V_1 \oplus \cdots \oplus V_t$
that is preserved by $G$, that is, for every $1 \le i \le t$ and every 
$g \in G$ there is a $j$ with $V_i g = V_j$.
}

\stru
Every such group is conjugate in $\GG$ to a subgroup of $\GL(m,q) \wr S_t$.
In this wreath product the $\GL(m,q)$ factors act on the direct summands
$V_i$ and the symmetric group on top permutes the subspaces.

Note that if $m = 1$ and $q \le 4$ or if $m=q=2$, then $\GL(m,q) \wr S_t$
is not maximal in $\GL(n,q)$ but still a maximal element in our class
\CC2.

\subsection{Description of class \CC3: ``semilinear''}
\label{descC3}

\desc{
A subgroup $G \le \GG$ lies in class \CC3 if there is a prime $s \mid n$,
for which we can extend the $\F_q$-vector space structure of
$V$ to an $\F_{q^s}$-vector space structure of dimension
$n/s$, such that there exists for every generator 
$g_i$ with $1 \le i \le k$ 
an automorphism $\alpha_i$ of $\F_{q^s}$ with
\[ (v+\lambda w)\cdot g_i = v \cdot g_i + \lambda^{\alpha_i} \cdot w \cdot
g_i\]
for all $v,w \in V$ and all $\lambda \in \F_{q^s}$. This means that we
can interpret $V$ as an $\F_{q^s}$-vectorspace for which the natural action
of $G$ is $\F_{q^s}$-semilinear, so $G$ is a subgroup of $\GGL(n/s,q^s)$.
}

Note that the automorphisms of $\F_{q^s}$ occurring in the semilinear
actions of group elements will automatically fix every element of $\F_q$, 
since the original action is $\F_q$-linear. Therefore they are elements of
the Galois group $\Gal(\F_{q^s}/\F_q)$.

\medskip
\stru
Every such group is conjugate in $\GG$ to a subgroup of $\GGL(n/s,q^s)$
for some prime $s \mid n$, 
realised as $(n \times n)$-matrices over $\F_q$ by choosing an 
$\F_q$-basis of $\F_{q^s}$.

\subsection{Description of class \CC4: ``tensor-decomposable''}
\label{descC4}

\desc{
A subgroup $G \le \GG$ lies in class \CC4 if there is a decomposition
of $V = V_1 \otimes V_2$ into a tensor product with $d_1 := \dim_{\F_q}(V_1)
\neq d_2 := \dim_{\F_q}(V_2)$ and $1 < d_1 < n$ that is preserved by $G$, 
that is, for every $g \in
G$ there are elements $g_1 \in \End_{\F_q}(V_1)$ and $g_2 \in
\End_{\F_q}(V_2)$ such that $(v_1 \otimes v_2) g = v_1 g_1 \otimes v_2 g_2$
for all $v_1 \in V_1$ and $v_2 \in V_2$.
}

\stru
Every such group is conjugate in $\GG$ to a subgroup of $\GL(d_1,q)
\circ \GL(d_2,q)$, by which we mean the central product of $\GL(d_1,q)$ and
$\GL(d_2,q)$. This central product is the set of Kronecker products of
a matrix in $\GL(d_1,q)$ and one in $\GL(d_2,q)$.

Note that for $d_1=2=q$ or $d_2 = 2 = q$ this central product is not 
maximal in $\GL(n,q)$ but maximal in our class \CC4.

\subsection{Description of class \CC5: ``subfield''}
\label{descC5}

\desc{
A subgroup $G < \GG$ lies in class \CC5 if there exists a subfield 
$\F_{q_0}$ of $\F_q$ with prime index, a matrix $T \in \GL(n,q)$, and 
scalars $\beta_1, \ldots, \beta_m
\in \F_q^*$, such that $T \cdot g_i \cdot T^{-1} = \beta_i \cdot h_i$ with
$h_i \in \GL(d,q_0)$.}

\stru
Every such group is conjugate to a subgroup of $\GL(d,q_0) \cdot \F_q^*$.

\subsection{Description of class \CC6: ``extraspecial''}
\label{descC6}

\subsection{Description of class \CC7: ``tensor-induced''}
\label{descC7}

\subsection{Description of class \CC8: ``classical''}
\label{descC8}

\subsection{Description of class \CC9: ``almost-simple''}
\label{descC9}

\section{Finding a reduction in the reducible case: \CC1}
\label{solveC1}

\section{Finding a reduction in the semilinear or subfield case: \CC3/\CC5}
\label{solveC3C5}

\section{Finding a reduction in the extraspecial case: \CC6}
\label{solveC6}

\section{Finding a reduction in the imprimitive case: \CC2}
\label{solveC2}

\section{Finding a reduction in the tensor-decomposable case: \CC4}
\label{solveC4}

\section{Finding a reduction in the tensor-induced case: \CC7}
\label{solveC7}



