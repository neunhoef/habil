% this is a part of the habilitation thesis of Max Neunhoeffer

\chapter{Finding homomorphisms}
\label{chap:findhom}

In the previous Chapter~\ref{chap:comptree} we have formulated the problem
of constructive recognition of groups in Problem~\ref{ProbCR3} and have
explained what a reduction is. This chapter describes how to find
reductions for matrix groups and projective groups and how to prove
that a certain collection of methods will for any given
matrix group or projective group either find a reduction
or show that a situation is on hand, in which the constructive 
recognition problem can be solved efficiently by other means. By the 
arguments in Chapter~\ref{chap:comptree} this solves
Problem~\ref{ProbCR3} whenever the ambient group is $\GL(n,q)$ or
$\PGL(n,q)$.

We use two fundamental theoretical tools. The first is Aschbacher's theorem
about the subgroup structure of the classical groups, which we describe
in detail for the case of $\GL(n,q)$ in Section~\ref{sect:aschbacher},
and the second is the classification of finite simple groups. We are
content with the statements in Aschbacher's theorem about the general
linear group since they suffice for the purposes of constructive
recognition and the statements are quite a bit simpler than for the
other classical groups. For details see \cite{aschbacher}.

Aschbacher's theorem states that every subgroup of $\GL(n,q)$
is either a member of at least one of $7$ concretely given classes 
\CC1 to \CC7 of 
subgroups, or it contains a classical group in its natural representation, 
or it is an almost-simple group. IS THAT RIGHT?
% FIXME

All the classes \CC1 to \CC7 are somehow defined in a geometric way (see
Sections~\ref{descC1} to \ref{descC7}) and thus promise some kind of
reduction. The two other cases are covered by two further classes \CC8
and \CC9, which are described in Sections~\ref{descC8} and \ref{descC9}.
For members of the latter two classes one will have to solve the
constructive recognition problem without further reduction.

The idea is to provide efficient algorithms for all the classes \CC1 to
\CC7 to recognise whether a given matrix group lies in the class, and if
so, to find a reduction using this information. If none of these algorithms
succeeded, Aschbacher's theorem shows that the group must be a member of
\CC8 or \CC9. In that case the constructive recognition problem has to be
solved by different means, usually by first finding out which classical
or almost simple group it is and then using this information to do the
constructive recognition in a special case, for example using standard
generators (see Sections~\ref{solveC8} and \ref{solveC9}).

The purpose of this chapter is to explain the statement of Aschbacher's 
theorem for $\GL(n,q)$ in detail and to give an overview over the known
methods to deal with the different classes together with references into
the literature. An algorithm to recognise and handle classes \CC3 and \CC5
provided that the group does not lie in class \CC1 is described in detail
in Chapter~\ref{chap:subsemi}.

\section{Aschbacher's theorem}
\label{sect:aschbacher}

\begin{Not}
For this section we fix $n \in \N$ and $q=p^e$ for a prime $p$ and
talk about the group $\GG := \GL(n,q)$. We denote the vector space $\F_q^{1
\times n}$ by $V$ and note that $\GG$ acts from the right on $V$ by
vector-matrix multiplication.
\end{Not}

\begin{Theo}[Aschbacher, specialised to $\GL(n,q)$]
\label{Asch}
Let $G$ be a subgroup of\/ $\GG = \GL(n,q)$.
Then $G$ is contained in a member of at least one of the
classes \CC1 to \CC9 of subgroups described in Sections~\ref{descC1}
to \ref{descC9}.
\end{Theo}
\proofbeg See \cite{aschbacher} and \cite{kleilieb}.
\proofend

\begin{Rem}
In our description of the classes \CC1 to \CC9 we follow
\cite{kleilieb}. Kleidman and Liebeck change the definition 
slightly in comparison to Aschbacher but argue that Theorem~\ref{Asch}
remains true with their definitions (see \cite[Chapter~4]{kleilieb}).
\end{Rem}

\subsection{Description of Class \CC1: ``reducible''}
\label{descC1}

A subgroup $G < \GG$ is a member of $\CC1$ if there is a subspace
$W < V$ that is stabilised by $G$, that is, $Wg = W$ for all $g \in G$.

\subsection{Description of Class \CC2: ``imprimitive''}
\label{descC2}

\subsection{Description of Class \CC3: ``semi-linear''}
\label{descC3}

\subsection{Description of Class \CC4: ``tensor-decomposable''}
\label{descC4}

\subsection{Description of Class \CC5: ``subfield''}
\label{descC5}

\subsection{Description of Class \CC6: ``extraspecial''}
\label{descC6}

\subsection{Description of Class \CC7: ``tensor-induced''}
\label{descC7}

\subsection{Description of Class \CC8: ``classical''}
\label{descC8}

\subsection{Description of Class \CC9: ``almost-simple''}
\label{descC9}

\section{Finding a reduction in the reducible case: \CC1}
\label{solveC1}

\section{Finding a reduction in the semi-linear or subfield case: \CC3/\CC5}
\label{solveC3C5}

\section{Finding a reduction in the extraspecial case: \CC6}
\label{solveC6}

\section{Finding a reduction in the imprimitive case: \CC2}
\label{solveC2}

\section{Finding a reduction in the tensor-decomposable case: \CC4}
\label{solveC4}

\section{Finding a reduction in the tensor-induced case: \CC7}
\label{solveC7}



