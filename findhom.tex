% this is a part of the habilitation thesis of Max Neunhoeffer

\chapter{Finding homomorphisms}
\label{chap:findhom}

In the previous Chapter~\ref{chap:comptree} we have formulated the problem
of constructive recognition of groups in Problem~\ref{ProbCR3} and have
explained what a reduction is. This chapter describes how to find
reductions for matrix groups and projective groups and how to prove
that a certain collection of methods will for any given
matrix group or projective group either find a reduction
or show that a situation is on hand, in which the constructive 
recognition problem can be solved efficiently by other means. By the 
arguments in Chapter~\ref{chap:comptree} this solves
Problem~\ref{ProbCR3} whenever the ambient group is $\GL(n,q)$ or
$\PGL(n,q)$.

We use two fundamental theoretical tools. The first is Aschbacher's theorem
about the subgroup structure of the classical groups, which we describe
in detail for the case of $\GL(n,q)$ in Section~\ref{sect:aschbacher},
and the second is the classification of finite simple groups. We are
content with the statements in Aschbacher's theorem about the general
linear group since they suffice for the purposes of constructive
recognition and the statements are quite a bit simpler than for the
other classical groups. For details see \cite{aschbacher}.

Aschbacher's theorem states that every subgroup of $\GL(n,q)$
is either a member of at least one of $7$ concretely given classes 
\CC1 to \CC7 of 
subgroups, or it contains a classical group in its natural representation, 
or it is an almost-simple group. IS THAT RIGHT?
% FIXME

All the classes \CC1 to \CC7 are somehow defined in a geometric way (see
Sections~\ref{descC1} to \ref{descC7}) and thus promise some kind of
reduction. The two other cases are covered by two further classes \CC8
and \CC9, which are described in Sections~\ref{descC8} and \ref{descC9}.
For members of the latter two classes one will usually have to solve the
constructive recognition problem without further reduction.
% FIXME

The idea is to provide efficient algorithms for all the classes \CC1 to
\CC7 to recognise whether a given matrix group lies in the class, and if
so, to find a reduction using this information. If none of these algorithms
succeeded, Aschbacher's theorem shows that the group must be a member of
\CC8 or \CC9. In that case the constructive recognition problem has to be
solved by different means, usually by first finding out which classical
or almost simple group it is and then using this information to do the
constructive recognition in a special case, for example using standard
generators (see Sections~\ref{solveC8} and \ref{solveC9}).

The purpose of this chapter is to explain the statement of Aschbacher's 
theorem for $\GL(n,q)$ in detail and to give an overview over the known
methods to deal with the different classes together with references into
the literature. An algorithm to recognise and handle classes \CC3 and \CC5
provided that the group does not lie in class \CC1 is described in detail
in Chapter~\ref{chap:subsemi}.

\section{Aschbacher's Theorem}
\label{sect:aschbacher}

Note again that we restrict ourselves to the general linear group
throughout, which is only a special case of Aschbacher's Theorem.

\begin{Not}
For this section we fix $n \in \N$ and $q=p^e$ for a prime $p$ and
talk about the group $\GG := \GL(n,q)$. We denote the vector space $\F_q^{1
\times n}$ by $V$ and note that $\GG$ acts from the right on $V$ by
vector-matrix multiplication.
\end{Not}

For the original formulation of his theorem in \cite{aschbacher}, Aschbacher 
defines $8$ classes
of subgroups of $\GG$ and proves that every subgroup $G \le \GG$ is either
a subgroup of some member of one of these $8$ classes or has a certain
amount of properties. We want to change this formulation in the following
way: We say that a group $G \le \GG$ lies in ``Aschbacher class \CC i''
for $1 \le i \le 8$ if and only if it is a subgroup of a member of
the class with number $i$ in \cite{aschbacher}. Furthermore, we collect
the subgroups $G \le \GG$ that fulfil the properties in the statement of 
Aschbacher's Theorem in the class \CC9. Therefore, we can formulate
Aschbacher's Theorem in the following way:

\begin{Theo}[Aschbacher's Theorem, specialised to\/ $\GL(n,q)$]
\label{Asch}
Let $G$ be a subgroup of\/ $\GG = \GL(n,q)$ with $n \ge 2$.
Then $G$ is contained in at least one of the
classes \CC1 to \CC9 of subgroups described in Sections~\ref{descC1}
to \ref{descC9}.
\end{Theo}
\proofbeg See \cite{aschbacher}, \cite{kleilieb} and
\cite[Theorem~1]{smashprim}. For a full proof see Section~\ref{AschProof}.
\proofend

\begin{Rem}
%In our description of the classes \CC1 to \CC9 we follow
%\cite{kleilieb}. Kleidman and Liebeck change the definition 
%slightly in comparison to Aschbacher but argue that Theorem~\ref{Asch}
%remains true with their definitions (see \cite[Chapter~4]{kleilieb}).
FIXME
\end{Rem}

We go on with the description of the Aschbacher classes. Throughout, we denote for a
subgroup $G < \GG$ its subgroup of scalar matrices by $Z$, that is,
$Z := Z(\GL(n,q)) \cap G$. For each class, we also give the structure
of the maximal members, this information is from \cite[Table~3.5.A]{kleilieb}.

\subsection{Description of class \CC1: ``reducible''}
\label{descC1}

\newcommand{\desc}[1]{\begin{center}\fbox{\parbox{5in}{#1}}\end{center}}
\newcommand{\stru}{\textbf{Structure of maximal members:}\par}

\desc{
A subgroup $G \le \GG$ is a member of \CC1 if there is a subspace
$0 < W < V$ that is stabilised by $G$, that is, $Wg = W$ for all $g \in G$.
}

\stru
Every such group is conjugate in $\GG$ to a subgroup of a group
\[ P_m := \left\{ \left[ \begin{array}{cc} A & 0 \\ C & D \end{array} \right]
           \mid A \in \GL(m,q), D \in \GL(n-m,q) \mbox{ and }
           C \in \F_q^{n-m \times m} \right\} \]
for some $0 < m < n$. The group $P_m$ is called a \emph{maximal parabolic
subgroup} and is a semidirect product
of the normal $p$-subgroup
\[ U_m := \left\{ \left[ \begin{array}{cc} \one_m & 0 \\ C & \one_{n-m} 
           \end{array} \right]
           \mid 
           C \in \F_q^{n-m \times m} \right\} \]
and $\GL(m,q) \times \GL(n-m,q)$, the factors being embedded on the
diagonal blocks. 
Here, $\one_m$ is the $(m \times m)$-identity matrix and $\one_{n-m}$ is the
$((n-m) \times (n-m))$-identity matrix.

\subsection{Description of class \CC2: ``imprimitive''}
\label{descC2}

\desc{
A subgroup $G \le \GG$ is a member of \CC2 if there is a decomposition
of $V$ as a direct sum of $m$-dimensional subspaces 
$V = V_1 \oplus \cdots \oplus V_t$
such that the summands are permuted transitively by $G$. That is, 
for every $1 \le i \le t$ and every 
$g \in G$ there is a $j$ with $V_i g = V_j$ and the action on the
summands is transitive.
}

\stru
Every such group is conjugate in $\GG$ to a subgroup of $\GL(m,q) \wr S_t$.
In this wreath product the $\GL(m,q)$ factors act on the direct summands
$V_i$ and the symmetric group on top permutes the subspaces.

Note that if $m = 1$ and $q \le 4$ or if $m=q=2$, then $\GL(m,q) \wr S_t$
is not maximal in $\GL(n,q)$ but still a maximal element in our class
\CC2.

\subsection{Description of class \CC3: ``semilinear''}
\label{descC3}

\desc{
A subgroup $G \le \GG$ lies in class \CC3 if there is a prime $s \mid n$,
for which we can extend the $\F_q$-vector space structure of
$V$ to an $\F_{q^s}$-vector space structure of dimension
$n/s$, such that there exists for every
$g \in G$
an automorphism $\alpha_g$ of $\F_{q^s}$ with
\[ (v+\lambda w)\cdot g = v \cdot g + \lambda^{\alpha_g} \cdot w \cdot
g\]
for all $v,w \in V$ and all $\lambda \in \F_{q^s}$. This means that we
can interpret $V$ as an $\F_{q^s}$-vectorspace for which the natural action
of $G$ is $\F_{q^s}$-semilinear, so $G$ is a subgroup of $\GGL(n/s,q^s)$.
}

Note that the automorphisms of $\F_{q^s}$ occurring in the semilinear
actions of group elements will automatically fix every element of $\F_q$, 
since the original action is $\F_q$-linear. Therefore they are elements of
the Galois group $\Gal(\F_{q^s}/\F_q)$. Note further that for irreducible
$V$, the group $G$ lies in \CC3 
with trivial automorphisms $\alpha_g$ for all $G$ if and only if
$V$ is not absolutely irreducible (see \cite[(29.13)]{CR0}).

\medskip
\stru
Every such group is conjugate in $\GG$ to a subgroup of $\GGL(n/s,q^s)$
for some prime $s \mid n$, 
realised as $(n \times n)$-matrices over $\F_q$ by choosing an 
$\F_q$-basis of $\F_{q^s}$.

\subsection{Description of class \CC4: ``tensor-decomposable''}
\label{descC4}

\desc{
A subgroup $G \le \GG$ lies in class \CC4 if there is a decomposition
of $V = V_1 \otimes V_2$ into a tensor product with $d_1 := \dim_{\F_q}(V_1)
\neq d_2 := \dim_{\F_q}(V_2)$ and $1 < d_1 < n$ that is preserved by $G$, 
that is, for every $g \in
G$ there are elements $g_1 \in \End_{\F_q}(V_1)$ and $g_2 \in
\End_{\F_q}(V_2)$ such that $(v_1 \otimes v_2) g = v_1 g_1 \otimes v_2 g_2$
for all $v_1 \in V_1$ and $v_2 \in V_2$.
}

\stru
Every such group is conjugate in $\GG$ to a subgroup of $\GL(d_1,q)
\circ \GL(d_2,q)$, by which we mean the central product of $\GL(d_1,q)$ and
$\GL(d_2,q)$. This central product is the set of Kronecker products of
a matrix in $\GL(d_1,q)$ and one in $\GL(d_2,q)$.

Note that for $d_1=2=q$ or $d_2 = 2 = q$ this central product is not 
maximal in $\GL(n,q)$ but maximal in our class \CC4.

\subsection{Description of class \CC5: ``subfield''}
\label{descC5}

\desc{
A subgroup $G < \GG$ lies in class \CC5 if there exists a subfield 
$\F_{q_0}$ of $\F_q$ with prime index, a matrix $T \in \GL(n,q)$, and 
scalars $\beta_1, \ldots, \beta_m
\in \F_q^*$, such that $T \cdot g_i \cdot T^{-1} = \beta_i \cdot h_i$ with
$h_i \in \GL(n,q_0)$.}

\stru
Every such group is conjugate in $\GG$ to a subgroup of $\GL(n,q_0)
\cdot \F_q^*$.

\subsection{Description of class \CC6: ``extraspecial''}
\label{descC6}

\subsection{Description of class \CC7: ``tensor-induced''}
\label{descC7}

\subsection{Description of class \CC8: ``classical''}
\label{descC8}

\desc{
A subgroup $G < \GG$ lies in class \CC8 if $G/Z$ contains a classical simple
group in one of the following ways:
\begin{itemize}
\item $G/Z$ contains $\PSL(n,q)$ and $(n,q) \notin \{(2,2),(2,3)\}$,
\item $n$ is even, $G$ is contained in $\Sp(n,q)$ for some non-singular 
symplectic form $G/Z$ contains $\PSp(n,q)$ and $(n,q) \notin           ,
\{(2,2),(2,3),(4,2)\}$                                                 ,
\item $q$ is a square, $G$ is contained in $U(n,q)$ for some non-singular 
Hermitian form, $G/Z$ contains $\PSU(n,q)$ and $(n,q) \notin \{ (2,4),
(2,9), (3,4) \}$,
\item $G$ is contained in $O^+(n,q)$ or $O^-(n,q)$ (for $n$ even) or 
$O(n,q)$ (for $n$ odd) respectively and $G/Z$ contains the corresponding 
simple group $\POmega^{(\pm)}(n,q)$.
\end{itemize}
}

\textbf{Note:} For the orthogonal groups certain restrictions on $(n,q)$
apply for $\POmega^{(\pm)}(n,q)$ to be simple, for which the reader is kindly
asked to consult \cite[Section~2.4]{ATLAS}.

\stru
The maximal members in \CC8 are $\GL(n,q)$, $Sp(n,q)$, $U(n,q)$ and the
appropriate orthogonal groups.

\subsection{Description of class \CC9: ``almost-simple''}
\label{descC9}

\desc{
A subgroup $G < \GG$ lies in class \CC9, if there is a non-abelian simple
group $S$ and a group $T$ with $S \subseteq T \subseteq \Aut(S)$ such that
$G/Z$ is isomorphic to $T$ (in this case $G/Z$ is called
``almost-simple'') and
the $G$-module $V$ gives rise to an
absolutely irreducible projective representation for $T$ which
is not realisable over a proper subfield of $\F_q$.}

\textbf{Note:} The existence of the projective representation given by $V$
limits the possibilities for $S$ and $T$ for given $(n,q)$ and thus
provides an interesting application for the representation theory of finite
simple groups, their Schur covers and automorphism groups.

\medskip
\stru
The absolutely irreducible projective representations of the automorphism
groups of the non-abelian finite simple groups provide the maximal members
of class \CC9.


\section{A proof of the $\mathbf{\GL}$-version of Aschbacher's Theorem}
\label{AschProof}

The contents of this section are a variation on ``The Theory behind
\textsc{Smash}'' from \cite[Section~2]{smashnormal} and provide a proof for
Theorem~\ref{Asch}.

Let $G$ be any subgroup of $\GG = \GL(n,q)$ acting from the right on the
natural module $V := \F_q^{1 \times n}$.

If the natural module $V$ is reducible, then $G$ is contained in class \CC1
as defined in Section~\ref{descC1}.

From now on we assume that $V$ is irreducible.

If $V$ is not absolutely irreducible, then by \cite[(29.13)]{CR0} and the 
usual Wedderburn Theorems
the endomorphism ring $E := \End_{\F_q G}(V)$ is a proper finite extension field
of $\F_q$. This automatically extends the $F_q$-vector space structure of
$V$ to an $\F_{q^s}$-vector space structure for any intermediate field 
$\F_q \subsetneq \F_{q^s} \subseteq E$ such that the $G$-action is
$\F_{q^s}$-linear. Thus $G$ lies in class \CC3 (see \ref{descC3})
with trivial Galois
automorphisms.

From now on we assume that $V$ is absolutely irreducible. Let $Z$ be the
subgroup of scalar matrices contained in $G$, i.e.~$Z := G \cap
Z(\GL(n,q))$. Since $V$ is absolutely irreducible it follows that $Z$ is
the center of $G$.

At this stage of the proof we mention that obviously $G$ can be
realisable over a proper subfield and thus lie in \CC5 (see
\ref{descC5}).

From now on we assume that $G$ does not lie in \CC5. In the sequel this 
assumption will be used to rule out classical groups defined over
smaller fields and achieve the corresponding statement in class \CC9
(see \ref{descC9}).

We assume first that $G/Z$ is a simple group. Then there are two possibilities:
The first is that $G/Z$ is one of the
classical simple groups defined over $\F_q$ and $G$ lies in \CC8 (see
\ref{descC8}), because by our assumption of $G$ not lying in \CC5 we can
exclude classical simple groups defined over subfields. Otherwise
$G$ lies in \CC9 (see \ref{descC9}) with
$S=T \cong G/Z$ since $V$ gives rise to an absolutely irreducible
projective representation of the finite simple group $G/Z$ that
is by assumption not realisable over a proper subfield of $\F_q$.

We assume from now on that $G/Z$ is not simple, let $\bar N$
be a non-trivial minimal normal subgroup of $G/Z$ and $N$ be
the corresponding non-scalar normal subgroup of $G$ with $Z < N
\triangleleft G$.

We now use Clifford theory applied to the natural module $V$. By Clifford's
theorem (see \cite[(49.2) and (49.7)]{CR0}) the restricted $\F_q N$-module
$V|_N$ is a direct sum of irreducible $\F_q N$-modules which are all
$G$-conjugates of one irreducible $\F_q N$-submodule $W \le V|_N$. In
particular, all these irreducible summands have the same dimension. We now
distinguish several cases.

If there is more than one homogeneous component (i.e.~not all conjugates of
$W$ are isomorphic to $W$ as $\F_q N$-modules), then $G$ lies in \CC2
(see \ref{descC2}),
since the homogeneous components provide a direct sum decomposition that is
preserved by $G$ and the summands are permuted transitively.

From now on we assume additionally that there is only one homogeneous
component, that is, all $Wg$ are isomorphic to $W$ as $\F_q N$-modules.
Then $\dim W > 1$ because $N$ is non-scalar.

We first assume that $W$ is not absolutely irreducible including both
cases $W = V$ and $W < V$. Then $\End_{\F_q N}(W)$ is a proper finite
extension field of $\F_q$, say $\F_{q^s}$, and $W$ and $V|_N$
both can be considered as $\F_{q^s}$-vector spaces such that the action of
$N$ on them is $\F_{q^s}$-linear, because $V|_N$ is isomorphic to a direct
sum of copies of $W$. We can embed $\F_{q^s} \le \End_{\F_q N}(V) \le \GL(n,q)$.

We claim that the action of $G$ is $\F_{q^s}$-semilinear proving that $G$ lies
in \CC3 (see \ref{descC3}). Let $c \in \GL(n,q)$ generate the
multiplicative group of $\F_{q^s}$. Then, for all $h \in N$ and $g
\in G$, we have $hc=ch$ by definition and thus $h^g c^g = c^g h^g =
h' c^g = c^g h'$, for some $h' \in N$. As $h$ varies over $N$, the
element $h'$ takes every value in $N$, therefore $\left< c \right>
= \left< c^g \right>$ and so $c^g = c^k$ for some $k$. Suppose that
$c^i + c^j = c^l$, then $(c^i)^g + (c^j)^g = (c^l)^g$ so $g$ acts as
field automorphism on $\F_{q^s}$. By using a subfield of $\F_{q^s}$ if
necessary we can choose $s$ to be prime and thus have proved that $G$
lies in \CC3 (see \ref{descC3}).

From now on we assume that $W$ is absolutely irreducible.

We first assume that $W$ is a proper subspace of $V$. Let $d :=
\dim_{\F_q}(W)$ such that we have $1 < d < n$. Since by assumption 
all $G$-conjugates of $W$ are isomorphic to $W$ as $\F_q N$-modules,
we can choose a basis of $V$ such that all elements of $N$ are block
diagonal matrices in which all diagonal blocks are identical of size
$d$.

As $N \triangleleft G$, for all $h \in N$ and $g \in G$,
 the product $g^{-1}hg \in N$ and thus $g^{-1} h g$ is also 
a block diagonal matrix in which all $d \times d$-blocks along the diagonal
are identical. Fixing $g$, we conclude that $g\cdot (g^{-1}hg) = hg$ for all
$h \in N$. If we now cut $g$ into $d \times d$-blocks, we get:

\begin{eqnarray*}
   &g \cdot (g^{-1}hg) & 
 = \left[ \begin{array}{c|c|c|c}
      g_{1,1} & g_{1,2} & \cdots & g_{1,n/d} \\ \hline
      g_{2,1} & g_{2,2} & \cdots & g_{2,n/d} \\ \hline
      \vdots  & \vdots  & \ddots & \vdots    \\ \hline
      g_{n/d,1}&g_{n/d,2}& \cdots& g_{n/d,n/d} \end{array} \right]
\cdot \left[ \begin{array}{c|c|c|c}
      D^g(h) & 0   & \cdots &      0    \\ \hline
         0   &D^g(h)&\cdots &      0    \\ \hline
      \vdots  & \vdots  & \ddots & \vdots    \\ \hline
         0    &    0    & \cdots& D^g(h) \end{array} \right] \\
 &=& \left[ \begin{array}{c|c|c|c}
      D(h)    & 0       & \cdots &     0    \\ \hline
         0    &D(h)     &\cdots &      0    \\ \hline
      \vdots  & \vdots  & \ddots & \vdots    \\ \hline
         0    &    0    & \cdots& D(h)   \end{array} \right]
\cdot \left[ \begin{array}{c|c|c|c}
      g_{1,1} & g_{1,2} & \cdots & g_{1,n/d} \\ \hline
      g_{2,1} & g_{2,2} & \cdots & g_{2,n/d} \\ \hline
      \vdots  & \vdots  & \ddots & \vdots    \\ \hline
      g_{n/d,1}&g_{n/d,2}& \cdots& g_{n/d,d/n} \end{array} \right]
 = hg
\end{eqnarray*}
where the $g_{i,j}$ are $d \times d$-matrices, $D(h)$ is a matrix
representing $h$ on the module $W$ and $D^g(h) = D(g^{-1}hg)$ is the
same representation twisted by the element $g$. By the block diagonal
structure of the matrices in $N$ we get 
$g_{i,j} \cdot D^g(h) = D(h) \cdot g_{i,j}$ for all $i$ and $j$ and 
all $h \in N$.

But by hypothesis, the matrix representations $D$ and $D^g$ of $N$
are isomorphic. Thus there is a nonzero matrix $T \in \F_q^{d \times d}$ with
$T \cdot D^g(h) = D(h) \cdot T$ for all $h \in N$. By Schur's lemma and
since the representation $D$ is absolutely irreducible,
the matrix $T$ is invertible and unique 
up to multiplication by an element in $C_{\GL(d,q)}(D(N))$, which
consists only of the scalar matrices.

This shows that for every pair $(i,j) \in \{ 1, \ldots, d/n \} \times 
\{ 1, \ldots, d/n \}$ there
is a unique element $e_{i,j} \in \F_q$ (possibly $0$) with 
$g_{i,j} = T \cdot e_{i,j}$. Thus we have shown that with respect to
the above choice of basis, every element $g$ is equal to a Kronecker 
product of some matrix in $U \in \F_q^{n/d \times n/d}$ with a matrix
$T \in \F_q^{d \times d}$. Since $g$ is invertible both 
$U$ and $T$ are invertible. 

This provides an $\F_q$-linear isomorphism of 
$\F_q^d$ and the tensor product $\F_q^{n/d} \otimes_{\F_q} \F_q^d$
such that all $g \in G$ act as a Kronecker product proving that
$G$ lies in \CC4 (see \ref{descC4}).

We now move on to the case that $W = V$, that is, the restriction
$V|_N$ is irreducible. Recall that we are still assuming that $W$ is
absolutely irreducible.

If $\bar N$ elementary abelian: \CC6 with $p=2$ separately.

If $\bar N$ direct product of non-abelian simple groups: \CC7

If $\bar N$ simple, then $G$ almost simple and \CC8 or \CC9

\section{Finding reductions}
\label{findred}

\subsection{Finding a reduction in the reducible case: \CC1}
\label{solveC1}

\subsection{Finding a reduction in the semilinear or subfield case: \CC3/\CC5}
\label{solveC3C5}

\subsection{Finding a reduction in the extraspecial case: \CC6}
\label{solveC6}

\subsection{Finding a reduction in the imprimitive case: \CC2}
\label{solveC2}

\subsection{Finding a reduction in the tensor-decomposable case: \CC4}
\label{solveC4}

\subsection{Finding a reduction in the tensor-induced case: \CC7}
\label{solveC7}



