% this is a part of the habilitation thesis of Max Neunhoeffer

\chapter{Finding homomorphisms}
\label{chap:findhom}
\index{finding homomorphism!for matrix groups}%
\index{finding homomorphism!for projective groups}%

In the previous Chapter~\ref{chap:comptree} we formulated the problem
of constructive recognition of groups in Problem~\ref{ProbCR3} and 
explained what a reduction is. This chapter describes how to find
reductions for matrix groups and projective groups and how to prove
that a certain collection of methods will for any given
matrix group or projective group either find a reduction
or show that the constructive 
recognition problem can be solved efficiently by other means. Together
with the methods described in the next Chapter~\ref{chap:leaves}
and using the 
arguments in Chapter~\ref{chap:comptree} this solves
Problem~\ref{ProbCR3} whenever the ambient group is $\GL(n,q)$ or
$\PGL(n,q)$.

We use two fundamental theoretical tools. The first is a variant of
Aschbacher's theorem
about the subgroup structure of the classical groups, which we describe
in detail for the case of $\GL(n,q)$ in Section~\ref{sect:aschbacher},
and the second is the classification of finite simple groups together
with the known results about their absolutely irreducible modular matrix
representations. We are
content with the statements in Aschbacher's theorem about the general
linear group since they suffice for the purposes of constructive
recognition and the statements and arguments are simpler
than for the other classical groups. For details see \cite{aschbacher}.

Roughly speaking, Aschbacher's theorem states that every subgroup of $\GL(n,q)$
is either a subgroup of a member of at least one of $7$ concretely given
classes \CC1 to \CC7 of subgroups, or it contains a classical group
in its natural representation, or it is an almost simple group modulo
scalars.

For the classes \CC1 to \CC7 the geometric way in which they are defined
(compare
Sections~\ref{descD1} to \ref{descD7}) promises some kind of
reduction. The two other cases are covered by two further classes \CC8
and \CC9, which are described in Sections~\ref{descD8} and \ref{descD9}.
For members of the latter two classes one will usually have to solve the
constructive recognition problem without further reduction.
% FIXME

Rather than presenting the original version of Aschbacher's theorem
for $\GL(n,q)$, we present a variant, for which we define slightly
different classes \DD1 to \DD9 of subgroups of $\GL(n,q)$ (see 
Sections~\ref{descD1} to \ref{descD9}). The statement
then says that every subgroup of $\GL(n,q)$ is contained in at least one
of the classes \DD1 to \DD9. 

The general idea is to provide efficient algorithms for all the classes \DD1 to
\DD7 to recognise whether a given matrix group lies in the class, and if
so, to find a reduction using this information. If none of these algorithms
succeeds, our variant of Aschbacher's theorem shows that the group must
be a member of \DD8 or \DD9. In this case the constructive recognition problem
has to be
solved by different means, usually by first finding out which classical
or almost simple group it is and then by using this information to do the
constructive recognition in a special case, for example using standard
generators (see Sections~\ref{solveD8} and \ref{solveD9}).

Our classes \DD1 to \DD9 are similar to the classes \CC1 to \CC9 but
slightly different. First of all we include many but not all subgroups of
class \CC i into \DD i such that we can replace the formulation ``$G$ is a
subgroup of a member of \CC i'' by ``$G$ is contained in \DD i''. Many
authors have used this formulation for the Aschbacher classes \CC1 to
\CC9 anyway, thereby more or less implicitly changing the definition. 
But we have also added further conditions on the groups
in our classes \DD1 to \DD9 for three reasons. The first is that we
wanted those classes that were difficult to handle algorithmically
to be smaller, in order to make it easier to devise algorithms for
the new classes. The second is that we wanted the classes to be
``more disjoint'', so that fewer groups lie in more than one
class. The third reason is that some more conditions on the groups in
the classes immediately come out of the proof in the $\GL(n,q)$ case.

The purpose of this chapter is to explain the statement of our variant 
of Aschbacher's 
theorem for $\GL(n,q)$ in detail, to give a proof and to give an
overview of the known methods to deal with the different classes
together with references to the literature. Furthermore we present new 
ideas to handle the classes \DD2, \DD4 and \DD7, which seem to work very
well in practice, but are not yet fully analysed with respect to their
applicability and complexity. It seems that our changed definition of
these classes helped to derive better algorithms than was previously
possible for the more general classes \CC2, \CC4 and \CC7.
An algorithm to recognise
and handle the classes \DD3 and \DD5 provided that the group does not lie in
class \DD1 is described in detail in Chapter~\ref{chap:subsemi}.
An overview of the handling of groups in classes \DD8 and \DD9 is
given in the next Chapter~\ref{chap:leaves}.

\section{A variant of Aschbacher's theorem}
\label{sect:aschbacher}

Note again that we restrict ourselves to the general linear group
throughout, which is only a special case of Aschbacher's theorem.

\begin{Not}
For this section we fix $n \in \N$ and $q=p^e$ for a prime $p$.
We denote the vector space $\F_q^{1
\times n}$ by $V$ and note that $\GL(n,q)$ acts from the right on $V$ by
vector-matrix multiplication.
\end{Not}

For the original formulation of his theorem in \cite{aschbacher} Aschbacher 
defines $8$ classes
of subgroups of $\GL(n,q)$ and proves that every subgroup $G$ is either
a subgroup of some member of one of these $8$ classes or has a certain
list of properties, the latter groups are usually denoted to lie in
a class \CC9. We change this formulation in the following
way: Instead of the original 8 Aschbacher class \CC 1 to \CC 8
we define different classes \DD 1 to \DD 8
(for details see the descriptions in Sections \ref{descD1} to
\ref{descD8}).
Furthermore we collect
the subgroups $G \le \GL(n,q)$ that fulfil the properties in the statement of 
Aschbacher's theorem in the class \DD9. The slight modifications 
to the class definitions on the one hand stem from our proof of the
theorem, on the other hand they are motivated in the following way: We try
to increase the number of subgroups contained in classes that can be
handled algorithmically well and try to decrease the number of subgroups
contained in classes for which the known algorithms are not yet completely
satisfying, mostly with respect to their complexity analysis.
In addition we try to reduce the overlap between the classes.

We formulate a variant of Aschbacher's theorem in the following way.

\begin{Theo}[Variant of Aschbacher's theorem, specialised to\/ $\GL(n,q)$]
\label{Asch}
\index{Aschbacher's theorem}%
Let $G$ be a subgroup of\/ $\GL(n,q)$ with $n \ge 2$.
Then $G$ is contained in at least one of the
classes \DD1 to \DD9 of subgroups described in Sections~\ref{descD1}
to \ref{descD9}.
\end{Theo}
\proofbeg Compare \cite[Appendix 2, Theorem 1]{RobPhd}, \cite{aschbacher}, 
\cite{kleilieb} and 
\cite[Theorem~1]{smashprim}. For a proof see Section~\ref{AschProof}.
\proofend

%\begin{Rem}
%In our description of the classes \CC1 to \CC9 we follow
%\cite{kleilieb}. Kleidman and Liebeck change the definition 
%slightly in comparison to Aschbacher but argue that Theorem~\ref{Asch}
%remains true with their definitions (see \cite[Chapter~4]{kleilieb}).
%FIXME
%\end{Rem}

\medskip
We proceed with our definition of the subgroup classes \DD1 to \DD9. Throughout
we denote for a subgroup $G \le \GL(n,q)$ its subgroup of scalar matrices
by $Z$, that is, $Z := Z(\GL(n,q)) \cap G$. For each class we either
give an alternative, more structural definition, or give the
structure of some ``typical'' example groups in that class.

\subsection{Description of class \DD1: ``reducible''}
\label{descD1}
\index{reducible groups}%
\index{D1@\DD1}%

\newcommand{\desc}[1]{\begin{center}\fbox{\parbox{5.3in}{#1}}\end{center}}
\newcommand{\diffasch}[1]{\textbf{Differences to Aschbacher's class \CC#1:}}
\newcommand{\stru}{\textbf{Alternative definition:}\par}
\newcommand{\exmemb}{\textbf{Example members:}\par}

\desc{
A group $G \le \GL(n,q)$ is a member of \DD1 if there is a subspace
$0 < W < V$ that is stabilised by $G$, that is, $Wg = W$ for all $g \in G$.
}

\diffasch1
Our class \DD1 consists of all members of \CC1 and all their
subgroups.

\smallskip
\stru
A group lies in \DD1 if and only if it is conjugate in $\GL(n,q)$ 
to a subgroup of a group
\[ P_m := \left\{ \left[ \begin{array}{cc} A & 0 \\ C & D \end{array} \right]
           \mid A \in \GL(m,q), D \in \GL(n-m,q) \mbox{ and }
           C \in \F_q^{(n-m) \times m} \right\} \]
for some $0 < m < n$. The group $P_m$ is called a \emph{maximal parabolic
subgroup} and is a semidirect product
\index{maximal parabolic subgroup}\index{parabolic subgroup}%
of the normal $p$-subgroup
\[ U_m := \left\{ \left[ \begin{array}{cc} \one_m & 0 \\ C & \one_{n-m} 
           \end{array} \right]
           \mid 
           C \in \F_q^{(n-m) \times m} \right\} \]
and $\GL(m,q) \times \GL(n-m,q)$, the factors being embedded on the
diagonal blocks. 
Here $\one_m$ is the $(m \times m)$-identity matrix and $\one_{n-m}$ is the
$(n-m) \times (n-m)$-identity matrix.

\subsection{Description of class \DD2: ``imprimitive''}
\label{descD2}
\index{imprimitive groups}%
\index{D2@\DD2}%

\desc{
A group $G \le \GL(n,q)$ is a member of \DD2 if the natural module $V$ is
absolutely irreducible and $G$ has a normal subgroup $N$ containing $Z$ 
such that $V|_N$ is isomorphic as an $\F_q N$-module to a direct sum of
absolutely irreducible modules that are not all isomorphic.
}

Note that 
by Clifford theory this immediately implies that there is a decomposition
of $V$ as a direct sum of $m$-dimensional subspaces 
$V = V_1 \oplus \cdots \oplus V_t$
such that the summands are permuted transitively by $G$. That is, 
for every $1 \le i \le t$ and every 
$g \in G$ there is a $j$ with $V_i g = V_j$ and the action on the
summands is transitive. The $V_i$ are the homogeneous components
of $V|_N$.

\medskip
\diffasch2
We include in \DD2 the subgroups of the members of \CC2 that
permute the homogeneous components of $V|_N$ transitively, but we exclude all groups
acting not absolutely irreducibly and we add the condition that the
irreducible $\F_q N$-summands of $V|_{N}$ are absolutely irreducible.
We also exclude groups in \CC2 for which the kernel of the permutation
action is contained in the centre of $G$.

\smallskip
\exmemb
The groups $\GL(m,q) \wr S_t$, where the $\GL(m,q)$ factors act on
the direct summands $V_i$ and the symmetric group on top permutes the
subspaces, all lie in \DD2. Subgroups of these groups belong to \DD2
if they act absolutely irreducibly.


\subsection{Description of class \DD3: ``semilinear''}
\label{descD3}
\index{semilinear groups}%
\index{D3@\DD3}%

\desc{
A group $G \le \GL(n,q)$ lies in \DD3 if the natural module $V$ is
irreducible and there is a finite field
extension $\F_{q^s}$ of $\F_q$,
for which we can extend the $\F_q$-vector space structure of
$V$ to an $\F_{q^s}$-vector space structure of dimension
$n/s$, such that for every $g \in G$ there exists 
an automorphism $\alpha_g$ of $\F_{q^s}$ with
\[ (v+\lambda w)\cdot g = v \cdot g + \lambda^{\alpha_g} \cdot w \cdot
g\]
for all $v,w \in V$ and all $\lambda \in \F_{q^s}$. This means that we
can interpret $V$ as an $\F_{q^s}$-vector space for which the natural action
of $G$ is $\F_{q^s}$-semilinear, so $G$ is a subgroup of $\GGL(n/s,q^s)$.
}

Note that the automorphisms of $\F_{q^s}$ occurring in the semilinear
actions of group elements will automatically fix every element of $\F_q$, 
since the original action is $\F_q$-linear. Therefore they are elements of
the Galois group $\Gal(\F_{q^s}/\F_q)$. Note further that
the group $G$ lies in \DD3 
with trivial automorphisms $\alpha_g$ for all $G$ if and only if
$V$ is irreducible but not absolutely irreducible (see \cite[(29.13)]{CR0}).

\medskip
\diffasch3
We include in \DD3 subgroups of the members of \CC3 excluding all
groups acting reducibly.

\smallskip
\stru
A group lies in \DD3 if and only if it acts irreducibly on the natural
module and is conjugate in $\GL(n,q)$ to a subgroup of $\GGL(n/s,q^s)$ for
some $s \mid n$, realised as $(n \times n)$-matrices over $\F_q$
by choosing an $\F_q$-basis of $\F_{q^s}$.


\subsection{Description of class \DD4: ``tensor decomposable''}
\label{descD4}
\index{tensor decomposable groups}%
\index{D4@\DD4}%

%%% % Old version:
%%% \desc{
%%% A group $G \le \GL(n,q)$ lies in class \DD4 if there is a decomposition
%%% of $V = V_1 \otimes V_2$ into a tensor product with 
%%% $1 < d_1 := \dim_{\F_q}(V_1) < n$
%%% and $d_2 := \dim_{\F_q}(V_2)$ that is preserved by $G$, 
%%% that is, for every $g \in
%%% G$ there are elements $g_1 \in \End_{\F_q}(V_1)$ and $g_2 \in
%%% \End_{\F_q}(V_2)$ such that $(v_1 \otimes v_2) g = v_1 g_1 \otimes v_2 g_2$
%%% for all $v_1 \in V_1$ and $v_2 \in V_2$.
%%% }
\desc{
A group $G \le \GL(n,q)$ lies in class \DD4 if the natural module $V$ is
absolutely irreducible and $G$ has a normal subgroup $N$
such that $V|_N$ is isomorphic as an $\F_q N$-module to a direct sum of
$k \ge 2$ modules which are all isomorphic to a single absolutely irreducible 
$\F_q N$-module $W$ with $\dim_{\F_q}(W) > 1$.
}

\diffasch4
We define \DD4 including more conditions on the structure of the
group and its natural module than Aschbacher. In particular, we omit
simple groups acting on an irreducible tensor product of two
irreducible modules. The original
definition of \CC4 uses a tensor product decomposition of $V$
which is invariant under $G$.
This follows from our conditions, see Proposition~\ref{tensorprop}. 
Contrary to \CC4, we allow for \DD4 the dimensions of the tensor factors 
to be equal.

\smallskip
\exmemb
The group $\GL(d_1,q) \circ \GL(d_2,q)$ is the central
product of $\GL(d_1,q)$ and $\GL(d_2,q)$ for $d_1 \cdot d_2 = n$ and
thus is contained in $\GL(n,q)$. It is
the set of Kronecker products of a matrix in $\GL(d_1,q)$ and one in
$\GL(d_2,q)$. These groups are members in \DD4 if $1 < d_1 < n$.


\subsection{Description of class \DD5: ``subfield''}
\label{descD5}
\index{subfield groups}%
\index{D5@\DD5}%

\desc{
A group $G \le \GL(n,q)$ lies in \DD5 if the natural module $V$ is
absolutely irreducible and there exists a proper subfield $\F_{q_0}$
of $\F_q$, a matrix $T \in \GL(n,q)$, and scalars $(\beta_g)_{g \in
G}$ with $\beta_g \in \F_q$ such that $\beta_g \cdot T^{-1} g T \in
\GL(n,q_0)$ for all $g \in G$.}

\diffasch5
We include in \DD5 subgroups of the members of \CC5 excluding
all groups acting not absolutely irreducibly.

\smallskip
\stru
A group lies in \DD5 if and only if it is conjugate in $\GL(n,q)$ to a
subgroup of $\GL(n,q_0) \cdot \F_q^*$, where $\F_{q_0}$ is a 
proper subfield of $\F_q$.

\subsection{Description of class \DD6: ``extraspecial''}
\label{descD6}
\index{extraspecial groups}%
\index{D6@\DD6}%

\desc{
A group $G \le \GL(n,q)$ lies in \DD6 if the natural module $V$ is absolutely
irreducible, $n=r^m$ for a prime $r$ and
\begin{itemize}\setlength{\itemsep}{0pt}\setlength{\parskip}{0pt}
    \item \textbf{either} $r$ is odd and $G$ has a normal subgroup $E$ that 
is an extraspecial $r$-group of order $r^{1+2m}$ and exponent $r$,
\item \textbf{or} $r=2$ and $G$ has a normal subgroup $E$ that is either
    extraspecial of order $2^{1+2m}$ or a central product of a cyclic
    group of order $4$ with an extraspecial group of order $2^{1+2m}$,
\end{itemize}
\textbf{and} in both cases the linear action of $G$ on the
$\F_r$-vector space $E/Z(E)$ of dimension $2m$ is irreducible. }

\diffasch6
We have added the condition about the irreducible $\F_r$-linear action of $G$ on
$E/Z(E)$ because it is an easy by-product of our proof and might help to
devise algorithms to find a reduction for groups in this class. On the
other hand we include subgroups of members of \CC6 if they fulfil this
condition.

\smallskip
\exmemb
The following subgroups of $\GL(n,q)$ for suitable $(n,q)$ lie in \DD6:

$r^{1+2m}.\Sp(2m,r)$ and $2_+^{1+2m}.\mathrm{O}^+(2m,2)$ and 
$2_-^{1+2m}.\mathrm{O}^-(2m,2)$ and $(4 \circ 2^{1+2m}).\Sp(2m,2)$.


\subsection{Description of class \DD7: ``tensor induced''}
\label{descD7}
\index{tensor induced groups}%
\index{D7@\DD7}%

\desc{
A group $G \le \GL(n,q)$ lies in \DD7 if it acts absolutely irreducibly on the 
natural module $V$ and, for some $k$, it has a normal 
subgroup $N$ containing $Z = Z(G)$ that is isomorphic to a central product 
$T \circ \cdots \circ T$ of
$k$ copies of a group $T$ which is a central extension of a non-abelian
simple group $\bar T$ by $Z$, such that: 

The restricted
module $V|_N$ is isomorphic to an outer tensor product $W_1 \otimes_{\F_q}
\cdots \otimes_{\F_q} W_k$ of $k$
absolutely irreducible $\F_q T$-modules of the same dimension on which 
$Z$ acts as scalars, and $G/N$ acts on $N$ by permuting the tensor factors
transitively.}

By the term ``outer tensor product'' we mean that every factor
$T$ in the central product $N$ acts on exactly one of the tensor factors $W_i$.
Note that the $\F_q T$-modules $W_i$ need not necessarily be
isomorphic to each other.

\medskip
\diffasch7 
We define \DD7 including more conditions on the structure of the group
and its natural representation than Aschbacher. On the other hand we
have added some subgroups of the members of \CC7.

\smallskip
\textbf{Note:} 
If $G$ is a
group in \DD7, then its natural projective representation
is tensor induced,
see Proposition~\ref{tensorindprop}. However, the natural
representation of $G$ needs not be tensor induced as is shown in
Remark~\ref{nottensorind} by an example.

\smallskip
\exmemb
The following subgroups of $\GL(n,q)$ lie in \DD7:

$\GL(r,q)^{\otimes k}\colon S_k$, where $n=r^k$ and $\GL(r,q)^{\otimes k}$ 
denotes the central product of $k$ copies of $\GL(r,q)$.

\subsection{Description of class \DD8: ``classical''}
\label{descD8}
\index{classical groups}%
\index{D8@\DD8}%

\desc{
A group $G \le \GL(n,q)$ lies in \DD8 if $G/Z$ contains a classical simple
group in its natural representation in one of the following ways:
\begin{itemize}\setlength{\itemsep}{0pt}\setlength{\parskip}{0pt}
\item $G/Z$ contains $\PSL(n,q)$ and $(n,q) \notin \{(2,2),(2,3)\}$,
\item $n$ is even, $G$ is contained in $N_{\GL(n,q)}(\Sp(n,q))$ 
for some non-singular 
symplectic form, $G/Z$ contains $\PSp(n,q)$ and $(n,q) \notin           
\{(2,2),(2,3),(4,2)\}$,
\item $q$ is a square, $G$ is contained in 
$N_{\GL(n,q)}(\SU(n,q^{1/2}))$ 
for some non-singular 
Hermitian form, $G/Z$ contains $\PSU(n,q^{1/2})$ and $(n,q^{1/2}) 
\notin \{ (2,2), (2,3), (3,2) \}$,
\item $G$ is contained in $N_{\GL(n,q)}(\Omega^\epsilon(n,q))$, 
the corresponding
    $\POmega^\epsilon(n,q)$ is simple and contained in $G/Z$.
    The group $\POmega^\epsilon(n,q)$ is simple if and only if
    \begin{itemize}\setlength{\itemsep}{0pt}\setlength{\parskip}{0pt}
        \item[*] $n\ge 3$, and
        \item[*] $q$ is odd if $n$ is odd, and
        \item[*] $\epsilon$ is -- if $n=4$, and
        \item[*] $(n,q) \notin \{ (3,3), (4,2) \}$.
    \end{itemize}
\end{itemize}
}

\textbf{Note:} For the orthogonal groups the given restrictions on $(n,q)$
are necessary for $\POmega^{\epsilon}(n,q)$ to be simple, see
\cite[Section~2.4]{ATLAS} for details.

\medskip
\diffasch8
We only include groups in \DD8 that modulo scalars contain a simple 
classical group in its natural representation defined over $\F_q$.


\subsection{Description of class \DD9: ``almost simple''}
\label{descD9}
\index{almost simple groups}\index{almost simple}%
\index{D9@\DD9}%

\desc{
A group $G \le \GL(n,q)$ lies in \DD9, if it is not in \DD8 and
there is a non-abelian simple
group $\bar N$ and a group $T$ with $\bar N \leq T \leq \Aut(\bar N)$ 
such that $G/Z$ is isomorphic to $T$ (in this case $G/Z$ is called
``almost simple'') and
the natural module $V$ gives rise to an
absolutely irreducible projective representation for $T$ that
is not realisable over a proper subfield of~$\F_q$.}

\textbf{Note:} The existence of the projective representation given by $V$
limits the possibilities for $\bar N$ and $T$ for given $(n,q)$ and thus
provides an interesting application for the representation theory of finite
simple groups, their Schur covers and automorphism groups.

\medskip
\diffasch9
Aschbacher does not define a class \CC9 but in the literature many
authors have called the groups which fulfill the conditions in
his main theorem the ``Aschbacher class \CC9''.
For the definition of \DD9 we have left out the condition that the
simple group leaves a form invariant.

\smallskip
\exmemb
The absolutely irreducible projective representations of the
quasi-simple groups provide examples of groups in \DD9.


\section{A proof of the $\mathbf{\GL}$-version of Aschbacher's theorem}
\label{AschProof}

The contents of this section are a variation on ``The Theory behind
\textsc{Smash}'' from \cite[Section~2]{smashnormal} and provide a proof for
\index{Smash@\textsc{Smash}}%
Theorem~\ref{Asch}. The part about the semilinear action is taken
from Section~\ref{subsec:semilin} and thus from 
\cite[Section~6.4]{subfieldpaper}.

Let $G$ be any subgroup of $\GL(n,q)$ acting from the right on the
natural module $V := \F_q^{1 \times n}$.

If the natural module $V$ is reducible, then $G$ is contained in class \DD1
as defined in Section~\ref{descD1}.

From now on we assume that $V$ is irreducible.

If $V$ is not absolutely irreducible, then by \cite[(29.13)]{CR0} and the 
usual Wedderburn theorems
the endomorphism ring $\End_{\F_q G}(V)$ is a proper finite extension field
$\F_{q^s}$ of $\F_q$. This automatically extends the $\F_q$-vector
space structure of $V$ to an $\F_{q^s}$-vector space structure
such that the $G$-action is $\F_{q^s}$-linear. Thus $G$ lies in class
\DD3 (see \ref{descD3}) with trivial Galois automorphisms.

From now on we assume that $V$ is absolutely irreducible. Let $Z$ be the
subgroup of scalar matrices contained in $G$, i.e.~$Z := G \cap
Z(\GL(n,q))$. Since $V$ is absolutely irreducible it follows that $Z$ is
the centre of $G$.

At this stage of the proof we mention that obviously $G$ can lie
in \DD5 (see \ref{descD5}).

From now on we assume that $G$ does not lie in \DD5. In the sequel this 
assumption will be used to rule out classical groups defined over
smaller fields and achieve the corresponding statement in class \DD9
(see \ref{descD9}).

We assume first that $G/Z$ is a simple group. If $G/Z$ were cyclic of prime 
order, then $G$ would be abelian and $V$ would not be absolutely
irreducible, contrary to our assumptions.

There are two possibilities:
The first is that $G/Z$ is one of the
classical simple groups in its natural representation defined over $\F_q$
and $G$ lies in \DD8 (see
\ref{descD8}). Note that by our assumption of $G$ not lying in \DD5 we can
exclude classical simple groups defined over a proper subfield. Otherwise
$G$ lies in \DD9 (see \ref{descD9}) with
$\bar N =T \cong G/Z$ since $V$ gives rise to an absolutely irreducible
projective representation of the finite simple group $G/Z$ that
is by assumption not realisable over a proper subfield of $\F_q$.

We assume from now on that $G/Z$ is not simple, let $\bar N$
be a minimal normal subgroup of $G/Z$ and $N$ be
the corresponding non-scalar normal subgroup of $G$ with $Z < N
\unlhd G$.

Next we use Clifford theory applied to the natural module $V$. By Clifford's
theorem (see \cite[(49.2) and (49.7)]{CR0}) the restricted $\F_q N$-module
$V|_N$ is a direct sum of irreducible $\F_q N$-modules which are all
$G$-conjugates of one irreducible $\F_q N$-submodule $W \le V|_N$. In
particular all these irreducible summands have the same dimension. 

First we assume that $W$ is not absolutely irreducible.
Then $\End_{\F_q N}(W)$ is a proper finite
extension field of $\F_q$, say $\F_{q^s}$, and $W$ and $V|_N$
both can be considered as $\F_{q^s}$-vector spaces of smaller dimension
such that the action of
$N$ on them is $\F_{q^s}$-linear, because $V|_N$ is isomorphic to a direct
sum of $G$-conjugates of $W$. Note that the action of $c \in \End_{\F_q
N}(W)$ on $W$ can be transferred to a conjugate $Wg$ by setting
$(wg)c := (wc)g$ for all $w \in W$ and $g \in G$ showing that $Wg$ is
not absolutely irreducible as well.  
We can embed $\F_{q^s} \le \End_{\F_q N}(V) \le \GL(n,q)$.

We claim that the action of $G$ is $\F_{q^s}$-semilinear proving that $G$ lies
in \DD3 (see \ref{descD3}). Let $c \in \GL(n,q)$ generate the
multiplicative group of $\F_{q^s}$. Then, for all $h \in N$ and $g
\in G$, we have $hc=ch$ by definition and thus $h^g c^g = c^g h^g =
h' c^g = c^g h'$, for some $h' \in N$. As $h$ varies over $N$, the
element $h'$ takes every value in $N$, therefore $\left< c \right>
= \left< c^g \right>$ and so $c^g = c^k$ for some $k$. Suppose that
$c^i + c^j = c^l$, then $(c^i)^g + (c^j)^g = (c^l)^g$ so $g$ acts as
field automorphism on $\F_{q^s}$. We have thus proved that $G$ 
lies in \DD3 (see \ref{descD3}).

From now on we assume that $W$ is absolutely irreducible and
distinguish several cases.

If there is more than one homogeneous component (i.e.~not all conjugates of
$W$ are isomorphic to $W$ as $\F_q N$-modules), then $G$ lies in \DD2
(see \ref{descD2}),
since the homogeneous components provide a direct sum decomposition that is
preserved by $G$ and the summands are permuted transitively. The above
argument shows that all $G$-conjugates of $W$ are absolutely
irreducible because $W$ is.

We assume from now on additionally that there is only one homogeneous
component, that is, all $Wg$ are isomorphic to $W$ as $\F_q N$-modules.
Then $\dim W > 1$ because $N$ is non-scalar.

First we assume that $W$ is a proper subspace of $V$. Let $d :=
\dim_{\F_q}(W)$ such that $1 < d < n$. Since by assumption 
all $G$-conjugates of $W$ are isomorphic to $W$ as $\F_q N$-modules,
we have shown that $G$ lies in \DD4 (see \ref{descD4}).

We proceed to the case that $W = V|_N$, that is, the restriction
$V|_N$ is irreducible. We are still assuming that $W$ is
absolutely irreducible.

Recall that by assumption $\bar N$ is a minimal normal subgroup of $G/Z$.
As such, by \cite[Theorem 4.3A.(iii)]{DixonMort}, it is
a direct product $\bar T_1 \times \cdots \times \bar T_k$ of copies of a 
simple group $\bar T$ which are all conjugate under $G/Z$. Thus $N$ is
a central product of the corresponding preimages $T_1, \ldots, T_k$
under the natural map $G \to G/Z$. We distinguish three cases:
\begin{itemize}\setlength{\itemsep}{0pt}\setlength{\parskip}{0pt}
\item[(i)] $\bar T$ is cyclic of prime order $r$, 
\item[(ii)] $\bar T$ is non-abelian simple with $k > 1$ and
\item[(iii)] $\bar T \cong N/Z$ is non-abelian simple.
\end{itemize}
We now consider case (i) that $\bar T$ is cyclic of prime order $r$, then
$\bar N = N/Z$ is an elementary-abelian $r$-group of order $r^k$.
Since $N$ is not abelian, we have $k > 1$.
The derived subgroup $N'$ of $N$ is contained in $Z$ and
we recall that $Z$ is the centre of $N$ since $V|_N = W$ is absolutely
irreducible. So $N$ is nilpotent and thus the direct product of its
Sylow subgroups. Let $R$ be the $r$-Sylow subgroup of $N$, all other
Sylow subgroups of $N$ consist of scalar matrices since $N/Z$ is
elementary abelian. Thus the module $V|_R$ is absolutely irreducible
and $R$ is not abelian.
For $x,y \in R$, we have $1 = [x^r,y] = [x,y]^r$ since $x^r$ and 
$[x,y]$ lie in $Z(R) = R \cap Z$. Therefore $R'=N'$ has exponent $r$ and
because it is contained in the cyclic group $Z(R) \le Z$ we have $|R'| =
r = |N'|$.

Assume first that $r$ is odd. For $x,y \in R$ we have $(xy)^r =
x^ry^r[y,x]^{r(r-1)/2} = x^ry^r$ since $R'$ has order $r$, thus the
elements of $R$ whose order divides $r$ form a characteristic 
subgroup of $N$ and thus a normal subgroup $E$ of $G$. Since for
$r$ odd, an $r$-group containing a unique subgroup of order $r$ is
cyclic, $E$ is not contained in $Z$ and thus by the minimality
of $N/Z$ we have $Z E = N$. It immediately follows that $V|_E$
is absolutely irreducible, $E$ is not abelian, 
$E \cap Z = Z(E) = E' = R' = N'$ has $r$ elements and 
$N/Z = ZE/Z \cong E/(E \cap Z) = E/Z(E)$. Thus the Frattini subgroup
$\Phi(E)$ of $E$ is equal to $Z(E)$ and $E$ is shown to be extraspecial
of order $r^{1+k}$ and exponent $r$. It follows using
\cite[V.16.14]{Hup} that $k$ is even and
$n=r^{k/2}$ because this holds for the faithful absolutely irreducible
representations of an extraspecial group. The linear action of $G$ on
the $\F_r$-vector space $E/Z(E)$ is irreducible because every minimal
normal subgroup of $N/Z \cong E/Z(E)$ is conjugated to a full basis by
$G$. Hence $G$ lies in \DD6 with odd $r$.

Consider now $r=2$. Recall $R' \le Z$ with $|R'| = 2$ and
let $E$ be the set of elements $x \in R$ with $x^2 \in R'$. 
For $x,y \in E$ we have $(xy)^2 = x^2y^2[y,x] \in R'$ proving that
$E$ is a characteristic subgroup of $R$ and thus a normal subgroup of
$G$. We claim that $E$ is not contained in $Z$: If $4$~does not divide
$|Z(R)|$ then this is trivial. If $4$ divides $|Z(R)|$ and $xZ(R)$ and $yZ(R)$
are different elements of $R/Z(R)$, then at least one of
$x^2$, $y^2$ and $(xy)^2$ is a square in the cyclic group $Z(R)$ and
thus one of $x$, $y$ and $xy$ can be multiplied by an element of
$Z(R)$ to get an involution. Therefore $R$ and thus $E$ both contain a 
non-central element whose square is contained in $R'$.
By the minimality of $N/Z$ we have $Z E = N$ as above. We immediately
get that $V|_E$ is absolutely irreducible, $E$ is not abelian,
$E \cap Z = Z(E)$ is cyclic with either $2$ or $4$ elements and
$E' = R' = N'$ has $2$ elements and is contained in $Z(E)$.
Furthermore, $N/Z = ZE/Z \cong E/(E \cap Z) = E/Z(E)$. Thus the
Frattini subgroup $\Phi(E)$ of $E$ is equal to $E'$ and $E$ is either
extraspecial of order $2^{1+k}$ or a central product of a cyclic group
of order $4$ consisting of scalar matrices and an extraspecial group 
of order $2^{1+k}$. In both cases it follows using
\cite[V.16.14]{Hup} again that $k$ is even and
$n = 2^{k/2}$. As above the linear action of $G$ on the $\F_2$-vector space
$E/Z(E)$ is irreducible and thus $G$ lies in \DD6 with $r=2$.

This concludes case (i) that $\bar T$ is cyclic of prime order $r$.

We now consider case (ii) that $\bar N$ is a direct product of more than
one copies of a non-abelian simple group $\bar T$. Thus $N$ is a
central product of the groups $T_1, \ldots, T_k$, each of which is
isomorphic to a single group $T$, which is a central extension of
$\bar T$ by $Z$. The absolutely irreducible representations
of $N$ in which $Z$ acts as scalars are just tensor products of 
absolutely irreducible representations of $T$ in which $Z \le T$ acts as
scalars. Furthermore, since $T_1, \ldots, T_k$ are all conjugate under
$G$, the natural module $V|_N$ must be isomorphic to an outer tensor product
$W_1 \otimes_{\F_q} \cdots \otimes_{\F_q} W_k$ 
of $k$ absolutely irreducible $\F_q T$-modules $W_i$ on which $Z$
acts as scalars. Since $G/N$ acts on $N$ by permuting the factors
transitively, all modules $W_i$ have the same dimension. So $G$ lies in
\DD7 concluding case~(ii).

\enlargethispage{-1\baselineskip}
We finally consider case (iii) that $\bar N = N/Z$ is a non-abelian finite 
simple group. Since in this case the centraliser $C_{G/Z}(\bar N)$ is 
trivial (because $W=V|_N$ is absolutely irreducible), 
$G/Z$ is contained in the automorphism group of $\bar N$. 
Thus either $N$ is a classical simple group in its natural
representation defined over $\F_q$ and
$G$ lies in \DD8, or $G$ lies in \DD9. Note that the assumptions we
picked up during this proof allow us to conclude that $V$
gives rise to an absolutely irreducible projective representation for
$G$ which is not realisable over a subfield (since $G$ is not \DD5).
\proofend

\medskip
In the rest of this section we present results relating our class
definitions to those of Aschbacher and to other concepts of group
representations.

\begin{Prop}[\DD4 implies tensor decomposability]
    \label{tensorprop}
\index{tensor decomposable groups}%
If a group $G \le \GL(n,q)$ lies in \DD4, then it is tensor decomposable,
by which we mean the following:
There is a decomposition
of\/ $V = V_1 \otimes V_2$ into a tensor product with 
$1 < d_1 := \dim_{\F_q}(V_1) < n$
and $d_2 := \dim_{\F_q}(V_2)$ that is preserved by $G$, 
that is, for every $g \in
G$ there are elements $g_1 \in \End_{\F_q}(V_1)$ and $g_2 \in
\End_{\F_q}(V_2)$ such that $(v_1 \otimes v_2) g = v_1 g_1 \otimes v_2 g_2$
for all $v_1 \in V_1$ and $v_2 \in V_2$.
\end{Prop}
\proofbeg
This proof is taken from Section~\ref{subsec:tensor} and thus from 
\cite[Section~6.6]{subfieldpaper}.

We assume that $G$ lies in \DD4, so the natural module $V$ is
absolutely irreducible and $G$ has a normal subgroup $N$ such that
$V|_N$ is isomorphic to a direct sum of $k\ge 2$ modules which are all
isomorphic to a single absolutely irreducible $\F_q N$-module $W$
of dimension $d < n$.

It immediately follows that we can choose a basis of $V$ such that all
elements of $N$ are block diagonal matrices in which all diagonal blocks
are identical of size $d$.

As $N \unlhd G$, for all $h \in N$ and $g \in G$,
 the product $g^{-1}hg \in N$ and thus $g^{-1} h g$ is also 
a block diagonal matrix in which all $d \times d$-blocks along the diagonal
are identical. Fixing $g$, we conclude that $g\cdot (g^{-1}hg) = hg$ for all
$h \in N$. If we now cut $g$ into $d \times d$-blocks, we get:
\begin{eqnarray*}
   &g \cdot (g^{-1}hg) & 
 = \left[ \begin{array}{c|c|c|c}
      g_{1,1} & g_{1,2} & \cdots & g_{1,n/d} \\ \hline
      g_{2,1} & g_{2,2} & \cdots & g_{2,n/d} \\ \hline
      \vdots  & \vdots  & \ddots & \vdots    \\ \hline
      g_{n/d,1}&g_{n/d,2}& \cdots& g_{n/d,n/d} \end{array} \right]
\cdot \left[ \begin{array}{c|c|c|c}
      D^g(h) & 0   & \cdots &      0    \\ \hline
         0   &D^g(h)&\cdots &      0    \\ \hline
      \vdots  & \vdots  & \ddots & \vdots    \\ \hline
         0    &    0    & \cdots& D^g(h) \end{array} \right] \\
 &=& \left[ \begin{array}{c|c|c|c}
      D(h)    & 0       & \cdots &     0    \\ \hline
         0    &D(h)     &\cdots &      0    \\ \hline
      \vdots  & \vdots  & \ddots & \vdots    \\ \hline
         0    &    0    & \cdots& D(h)   \end{array} \right]
\cdot \left[ \begin{array}{c|c|c|c}
      g_{1,1} & g_{1,2} & \cdots & g_{1,n/d} \\ \hline
      g_{2,1} & g_{2,2} & \cdots & g_{2,n/d} \\ \hline
      \vdots  & \vdots  & \ddots & \vdots    \\ \hline
      g_{n/d,1}&g_{n/d,2}& \cdots& g_{n/d,d/n} \end{array} \right]
 = hg,
\end{eqnarray*}
where the $g_{i,j}$ are $d \times d$-matrices, $D(h)$ is a matrix
representing $h$ on the module $W$ and $D^g(h) = D(g^{-1}hg)$ is the
same representation twisted by the element $g$. By the block diagonal
structure of the matrices in $N$ we get 
$g_{i,j} \cdot D^g(h) = D(h) \cdot g_{i,j}$ for all $i$ and $j$ and 
all $h \in N$.

But by hypothesis, the matrix representations $D$ and $D^g$ of $N$
are isomorphic. Thus there is a nonzero matrix $T \in \F_q^{d \times d}$ with
$T \cdot D^g(h) = D(h) \cdot T$ for all $h \in N$. By Schur's lemma and
since the representation $D$ is absolutely irreducible,
the matrix $T$ is invertible and unique 
up to multiplication by an element in $C_{\GL(d,q)}(D(N))$, which
consists only of the scalar matrices.

This shows that for every pair $(i,j) \in \{ 1, \ldots, d/n \} \times 
\{ 1, \ldots, d/n \}$ there
is a unique element $u_{i,j} \in \F_q$ (possibly $0$) with 
$g_{i,j} = T \cdot u_{i,j}$. Thus we have shown that with respect to
the above choice of basis, every element $g$ is equal to a Kronecker 
product of some matrix in $U \in \F_q^{n/d \times n/d}$ with a matrix
$T \in \F_q^{d \times d}$. Since $g$ is invertible both 
$U$ and $T$ are invertible. 

This provides an $\F_q$-linear isomorphism of 
$\F_q^n$ and the tensor product $\F_q^{n/d} \otimes_{\F_q} \F_q^d$
such that all $g \in G$ act as a Kronecker product proving the
proposition.
\proofend

\begin{DefProp}[{Tensor induction of (projective) representations}]
\label{tensorinduction}
\index{tensor induction}%
    For details and proofs see \cite[13A]{CRI} and \cite[Section~2]{kovacs}.

    Let $G$ be a group, $K$ an arbitrary field, $H<G$ a subgroup of
    index $k$ and $G=\bigcup_{i=1}^k g_i H$ with $g_1,\ldots,g_k \in
    G$ and $g_1=1$. Let $\varphi : G \to H \wr S_k$ be the group
    monomorphism defined by $\varphi(x) = \pi \cdot (h_1,\ldots,h_k)$
    for $x \in G$, where $xg_i = g_{\pi(i)}h_i$ with $\pi \in S_k$
    and $h_i \in H$ for $1 \le i \le k$.
    Let $\kappa$ be the group homomorphism $\kappa: \GL(d,K)
    \wr S_k \to \GL(d^k,K)$, where the direct factors of the base group
    act on the tensor factors of $(K^d)^{\otimes k} \cong K^{d^k}$ and
    the top group permutes them. Let $\bar\kappa : \PGL(d,K) \wr S_k
    \to \PGL(d^k,K)$ be the corresponding map for the projective linear
    group.

    For a representation $\rho: H \to \GL(d,K)$ of $H$
    there is a corresponding map $(\rho \wr S_k) : H \wr S_k \to
    \GL(d,K) \wr S_k$ mapping the base group componentwise by $\rho$. 
    The \emph{tensor induced representation}
    $\rho\!\uparrow^{\otimes G} : G \to \GL(d^k,K)$ is then the
    composite map $\kappa \circ (\rho \wr S_k) \circ \varphi$. The
    equivalence class of this representation does not depend on the
    choices in this construction.

    Similarly,
    for a projective representation $\bar\rho : H \to \PGL(d,K)$ of
    $H$ there is a corresponding map 
    $(\bar\rho \wr S_k) : H \wr S_k \to \PGL(d,K) \wr S_k$ 
    mapping the base group
    componentwise by $\bar\rho$.
    The \emph{tensor induced projective representation}
    $\bar\rho\!\uparrow^{\otimes G} : G \to \PGL(d^k,K)$ is then the
    composite map $\bar\kappa \circ (\bar\rho \wr S_k) \circ \varphi$.
    The equivalence class of this projective representation does not
    depend on the choices in this construction.
\end{DefProp}

\begin{Prop}[\DD7 implies ``projective tensor induced'']
    \label{tensorindprop}
\index{tensor induced groups}%
If a group $G \le \GL(n,q)$ lies in \DD7, then the projective
representation afforded by its natural module is tensor induced.
\end{Prop}
\proofbeg
Let $G$ be in \DD7 and let $N$, $T$, $Z$ and $V$ be as in
Section~\ref{descD7}.
We can use the proof of \cite[Tensor Induction Theorem]{kovacs}. 
Although the hypotheses of
the theorem contain that the natural module is not induced from any
proper subgroup, this is only used in the proof to ensure that the
constituents of the restricted natural module $V|_N$ are all
equivalent and absolutely irreducible. However, we have this by
assumption and the proof goes through.
\proofend

\begin{Rem}[{\DD7 does \emph{not} imply tensor induced}]
    \label{nottensorind}
\index{tensor induced groups}%
    Let $G$ be the semidirect product 
    $(\SL(2,7) \otimes_{\F_7} \SL(2,7)).S_2 < \GL(4,7)$, where the
    $S_2$ permutes the tensor factors. The natural module $V := \F_7^4$ is
    absolutely irreducible also when restricted to the normal subgroup
    $N := \SL(2,7) \otimes_{\F_7} \SL(2,7)$ of index $2$. The centre 
    $Z := \left< -\id \right>$ of $G$ has two elements and $N/Z$ is a
    minimal normal subgroup of $G/Z$ which is a direct product of two
    copies of the non-abelian simple group $\PSL(2,7)$. Thus $G$ lies
    in \DD7 by construction.

    However, $G$ does not have a subgroup of index $4$, the only 
    subgroup of $G$ of index $2$ is $N$ and $N$ does
    not have an irreducible representation of dimension $2$ over
    $\F_7$. Thus the natural module is not tensor induced.

    Of course, using Proposition~\ref{tensorindprop} we conclude that
    the projective representation afforded by the natural module is in
    fact tensor induced from the two-dimensional projective
    absolutely irreducible representation of $N$ over $\F_7$.
\end{Rem}

\section{Finding reductions}
\label{findred}
\index{reduction}%

The purpose of this section is to give an overview of how reductions can
be found algorithmically, provided that a matrix group
lies in one of the classes \DD1 to \DD7. We do not give full details
here but instead sketch the methods used and collect references to the
literature.

We continue to use the notation for a group $G \le \GL(n,q)$ that
$Z := Z(\GL(n,q)) \cap G$ is the subgroup of scalar matrices and that
$V := \F_q^{1 \times n}$ is the natural right module. For the
complexity statements we assume that $G$ is given by $m$ generators
$g_1, \ldots, g_m \in \GL(n,q)$.

Note that for a matrix group there is always the determinant map
into the ground field and the natural homomorphism
$G \to G/Z$. We can compute both explicitly and efficiently.

In the following we assume that we have already used the determinant
homomorphism and then distinguish the cases, in which of the classes
\DD1 to \DD7 the group $G$ lies. Whenever it is convenient we can
switch over to the projective group $G/Z$ using the natural map 
as reduction.

\subsection{Finding a reduction in the reducible case: \DD1}
\label{solveC1}

Using the MeatAxe (see \cite{MeatAxeHoltRees, IL, MeatAxeRP}) we can
decide efficiently whether the natural module $V$ is
irreducible. If not, we find a proper invariant subspace $0<W<V$
thereby finding a reduction as described in Section~\ref{descD1}. 
If $V$ is irreducible, we either prove
that it is absolutely irreducible or find an element that generates
the endomorphism ring as a field.

Under the assumptions in Section~\ref{hypMtx} the MeatAxe provides a
Las Vegas algorithm which terminates
after at most $O(mn^3\log \delta^{-1})$ elementary field operations, where $m$
is the number of generators of $G$ and $0 < \delta < 1/2$ is an upper
bound for the failure probability (see Lemma~\ref{MeatAxe}).


\subsection{Finding a reduction in the semilinear or subfield case:
\DD3/\DD5}
\label{solveC3C5}

Chapter~\ref{chap:subsemi} deals with the case that $G$ does not lie
in \DD1 but is contained in \DD3 or in \DD5. The algorithms described
there sometimes even find a reduction if $G$ lies neither in \DD3 nor
in \DD5. Note that, as it is common in the literature, we say in that
chapter that a group ``lies in \CC i'' if it is a subgroup of one of
the groups in Aschbacher class \CC i. For \CC1, \CC3 and \CC5 this then 
amounts to the same as talking about \DD1, \DD3 and \DD5 respectively.

The algorithm presented in Chapter~\ref{chap:subsemi} is Las Vegas and is
guaranteed to terminate in 
\[ O(n^3\log(\delta^{-1})(m+\log(\delta^{-1} \log n))
   + n^4\log(\delta^{-1} \log n) \log q) \]
elementary field operations,
if $0 < \delta < 1/2$ is the prescribed upper bound for the failure
probability (see Section~\ref{subsemi:sec:complexity}).

We conclude this section by pointing to some earlier work regarding
reductions for groups in these classes. In \cite{MeatAxeHoltRees} it
is described how to test whether a matrix group acts absolutely
irreducibly and in \cite{smashnormal} the general case of a group
in \CC3 is considered.

For the class \CC5 there is an algorithm described in
\cite{GlasbyHowlett} which decides whether a matrix group can be
conjugated to one over a subfield. In \cite{GLGOB} the general case
of a group lying in \CC5 is considered.


\subsection{Finding a reduction in the imprimitive and tensor
cases: \DD2 and \DD4}
\label{solveD24}

The content of this section is a bit experimental and speculative. We
mainly describe a new idea to find reductions for the classes \DD2 and
\DD4. At the end of this section we give references to the literature
to describe the current state of the art.

The descriptions of the classes \DD2 and \DD4 have one property in
common: They involve a non-scalar normal subgroup $N$ of $G$ such
that the restriction of the natural module to $N$ is reducible. That
is, the MeatAxe can detect $N$ in the sense that if we have a list of
elements of $G$ that happen to generate $N$, the MeatAxe applied to
these elements will return a proper $\F_q N$-submodule. Since we can
compute normal closures (see \cite[Theorem~2.3.9]{Ser}) it is in fact
enough to produce one non-central element of $N$ to find a reduction for
these cases.

However, simply producing random elements in $G$ and hoping to hit elements
in $N$ does not work efficiently. Therefore we propose a new method to
produce non-central elements of a normal subgroup which is put together
from different well-known techniques and seems to work extremely well in
practice. We give a few heuristic reasons why this works so well below, but
a complete analysis of this method still has to be done at the time of 
this writing.
In the following description of Algorithm~\ref{alginvjump}
``\textsc{InvolutionJumper}'' we include a
slight variation which will be used in Section~\ref{solveD7} to resolve
the \DD7 case as well.

\begin{algorithm}
\caption{$\quad$ \sc InvolutionJumper}
\index{InvolutionJumper@\textsc{InvolutionJumper}}%
\label{alginvjump}
\begin{algorithmic}
\STATE \textbf{Input:} A group $G$, an involution
$x \in G$, an integer $t$, and a boolean $b$.
\STATE \textbf{Output:} An involution in $G$ or \textsc{Fail}.

\smallskip
\STATE $i := 0$
\REPEAT
    \STATE $i := i + 1$
    \STATE $y := \textsc{Random}(G)$ \hspace*{10mm} \COMMENT{produce a random
element of $G$}
    \STATE $c := [x,y]$ and
    $o := \textsc{Order}(c)$
    \IF {$o > 1$}
        \IF {$o$ is even}
            \STATE \textbf{Return} $c^{o/2}$
        \ELSIF {$b = \textsc{True}$}
            \STATE $z := y \cdot c^{(o-1)/2}$ and
            $o := \textsc{Order}(z)$
            \IF {$o$ is even}
                \STATE \textbf{Return} $z^{o/2}$
            \ENDIF
        \ENDIF
    \ELSIF {$b = \textsc{True}$}
        \STATE $o := \textsc{Order}(y)$
        \IF {$o$ is even}
            \STATE \textbf{Return} $y^{o/2}$
        \ENDIF
    \ENDIF
\UNTIL $i \ge t$
\STATE \textbf{Return} \textsc{Fail}
\end{algorithmic}
\end{algorithm}

\begin{Prop}[The Involution Jumper]
\index{InvolutionJumper@\textsc{InvolutionJumper}}%
Let $G$ be a finite group. If Algorithm~\ref{alginvjump}
``\textsc{InvolutionJumper}'' is called with $G$ and an
involution $x$, the return value $u$ is either an involution in $G$ with
$xu=ux$ or \textsc{Fail}. 
\end{Prop}
\proofbeg
This method is heavily inspired by the ``dihedral trick'' to compute
elements for involution centralisers (see \cite[2.2]{BrayInv}). We briefly
repeat some arguments from there:
The element $x$ is an
involution and for $c := x^{-1} y^{-1} x y$ we have 
$x c = c^{-1}x$ and thus
$xc^k = c^{-k}x$ for all $k$. Thus, if $c$ has even 
order $2k$, then
$c^k$ is an involution commuting with $x$ and $c^k$ lies in $[H,G]$ 
for every subgroup $H \le G$ with $x \in H$ by construction.

Now assume that $c$ has odd order $2k+1 > 1$,
then $xyc^k = yxcc^k = y c^{-k-1}x=yc^kx$ and $z := yc^k$ is an 
element in the 
involution
centraliser $C_{G}(x)$. In fact, Richard Parker has observed that
if $y$ is
uniformly distributed in $G$, then $z$ is uniformly distributed in
$C_{G}(x)$ in this case. Thus, if the order of $z$ is even,
then Algorithm~\ref{alginvjump} ``\textsc{InvolutionJumper}'' 
returns an involution as well.

If $c=1$ and thus has order $1$, then $y$ lies in
$C_G(x)$ by chance. If $y$ has even order $2k$, then $y^k$ is an
involution commuting with $x$.

The whole algorithm fails, if it fails to produce enough elements of
even order in $C_G(x)$. Note however that if $v \neq x$ is any
involution in $G$, then $\left< x,v \right>$ is a
dihedral subgroup of $G$ which contains at least one
involution commuting with and different from $x$.
\proofend

\begin{Rem}[The boolean argument $b$]
The boolean argument $b$ will usually
be \textsc{True}, however, it can be set to
\textsc{False}, in which case the algorithm will return a $u$ 
that is contained in the group $[H,G]$ for
every subgroup $H \le G$ containing $x$. 
%More precisely, with
%probability $1-|C_G(x)|/|G|$ (note that if $G$ acts absolutely
%irreducibly then $x$ is not central), the order of $c$ will
%not be equal to $1$. 
%Then, for $b = \textsc{False}$ the resulting $u$
%in fact lies in $[H,G]$ for every subgroup $H < G$ that contains $x$.
This option will be used in Section~\ref{solveD7} to handle the case
of a group in \DD7.
\end{Rem}

\begin{DefProp}[A Markov chain on involution classes]
\index{Markov chain!on involution classes}%
\label{markovinvclasses}
Let $G$ be a finite group of even order and $\mathcal{M}$ be the set of
conjugacy classes of involutions in $G$. Then
Algorithm~\ref{alginvjump} ``\textsc{InvolutionJumper}'' defines a Markov
chain with vertex set $\mathcal{M}$ in the following way: A step in
the Markov chain starting at an involution class $C \in \mathcal{M}$
involves choosing an arbitrary involution in $C$ and running
Algorithm~\ref{alginvjump} ``\textsc{InvolutionJumper}'' on it in the group $G$
until a value not equal to \textsc{Fail} is returned. The resulting
involution class is the conjugacy class of the result. The
probability to reach a certain involution class does not depend on the
choice of the involution in the starting class.
\end{DefProp}
\proofbeg
Only the last statement needs to be proved. The result of
\[ \textsc{InvolutionJumper}(G,x,\infty,\textsc{True}) \] can
be summarised in the following formula, which is evaluated for the
first random element $y$ generated in the algorithm for which it is
defined:
\[ f(x,y)
    := \left\{ 
    \begin{array}{l@{\mbox{ if }}l}
        [x,y]^k & \textsc{Order}([x,y]) = 2k \\
        (y[x,y]^k)^l & \textsc{Order}([x,y]=2k+1 \mbox{ for } k \ge 1
                       \mbox{ and } \textsc{Order}(y[x,y]^k) = 2l \\
        y^k          & xy=yx \mbox{ and } \textsc{Order}(y) = 2k 
    \end{array}
    \right. .
\]
Since we have $f(x^g,y^g) = f(x,y)^g$ whenever $f(x,y)$ is defined and
$y$ is supposed to be uniformly distributed in $G$ and thus in each
conjugacy class of $G$, this reasoning proves that the probability to
reach involution class $B \in \mathcal{M}$ from class $A \in
\mathcal{M}$ does in fact not depend on the choice of the
representative for $A$.
\proofend

\begin{Prop}[The Markov chain in Definition~\ref{markovinvclasses} is 
    irreducible and aperiodic]
\label{markovinvlimit}
Let $G$ be a finite group of even order.
If Algorithm~\ref{alginvjump} ``\textsc{InvolutionJumper}'' is called with
the boolean argument $b = \textsc{True}$, then the Markov chain on the 
involution classes $\mathcal{M}$ of $G$ defined in
Definition~\ref{markovinvclasses} is irreducible and
aperiodic and thus has a stationary distribution in which every vertex
has a non-zero probability.
\end{Prop}
\proofbeg
If $G$ has only one class of involutions there is nothing to show.

Let $x,v \in G$ be involutions in different classes. With non-zero 
probability the random element $y$ is equal to $v$. Assume that this
happens. If $x$ and $v$
commute then Algorithm~\ref{alginvjump} returns $v$ itself. 
If $x$ and $v$ do not commute, the subgroup
$\left< x,v \right>$ of $G$ is a dihedral group. In this case, the
order of $[x,v]$ is even, because if it were odd, all involutions
in $\left< x,v \right>$ were conjugate contradicting our assumption
that $x$ and $v$ are not conjugate in $G$. Thus the
central involution of $\left< x,v \right>$ is returned which commutes
with both $x$ and $v$, and in the next step $v$ could be returned with
non-zero probability. This proves that the Markov chain is
irreducible.

As to aperiodicity, this immediately follows from the fact that the
random element $y$ in the algorithm could be $x$ itself in which case
$x$ is returned.

Therefore, by \cite[Theorem 2.2.1]{Ser} the Markov chain has a
stationary distribution in which every vertex has a non-zero
probability, because every vertex can be reached from every vertex in
at most two steps with non-zero probability.
\proofend

\smallskip
Obviously, the Involution Jumper needs an
involution to start with. To this end we use a very simple minded
algorithm to find one, which is given as Algorithm~\ref{alginvmaker}
``\textsc{InvolutionMaker}''.

\begin{algorithm}
\caption{$\quad$ \sc InvolutionMaker}
\index{InvolutionMaker@\textsc{InvolutionMaker}}%
\label{alginvmaker}
\begin{algorithmic}
\STATE \textbf{Input:} A finite group $G$ and an integer $t$.
\STATE \textbf{Output:} An involution in $G$ or \textsc{Fail}.

\smallskip
\STATE $i := 0$
\REPEAT
    \STATE $i := i + 1$
    \STATE $x := \textsc{Random}(G)$ \hspace*{10mm} \COMMENT{produce a random
element of $G$}
    \STATE $o := \textsc{Order}(x)$
    \IF {$o$ is even}
        \STATE \textbf{Return} $x^{o/2}$
    \ENDIF
\UNTIL $i \ge t$
\STATE \textbf{Return} \textsc{Fail}
\end{algorithmic}
\end{algorithm}

\begin{Prop}[Involution Maker]
\index{InvolutionMaker@\textsc{InvolutionMaker}}%
Let $G$ be a finite group. Then the return value $u$ of 
Algorithm~\ref{alginvmaker} ``\textsc{InvolutionMaker}'' 
called with the group $G$ is either
an involution or \textsc{Fail}. The latter result occurs only if
only a small percentage of elements in $G$ has even order.
\end{Prop}
\proofbeg Omitted.
\proofend

\begin{Rem}
    Obviously, this algorithm can fail horribly if $G$ contains only
    very few elements of even order. This happens for example in
    groups like $\PSL(2,2^k)$ for large $k$.
\end{Rem}

To apply all this to our situation with $G \le \GL(n,q)$ and
$Z = G \cap Z(\GL(n,q))$ we simply compute in $G/Z$ using
comparison modulo scalars and 
the projective order of an element instead of the order.
To indicate this trick we call Algorithms~\ref{alginvjump} and
\ref{alginvmaker} with the group $G/Z$ but then consider the result as
an element of $G$ in the sequel.

The following definition helps in formulations.

\begin{Def}[Projective involution]
\index{involution!projective}\index{projective involution}%
Let $G \le \GL(n,q)$ and $Z := G \cap Z(\GL(n,q))$ be the subgroup of scalar
matrices. An element $x \in G \setminus Z$ with $x^2 \in Z$ is
called \emph{projective involution in $G$}.
\end{Def}

\begin{algorithm}
\caption{$\quad$ \textsc{FindHomD247}}
\index{FindHomD247@\textsc{FindHomD247}}%
\label{algreductionD24}
\begin{algorithmic}
\STATE \textbf{Input:} A group $G \le \GL(n,q)$ known to act absolutely
irreducibly.
\STATE \textbf{Output:} A reduction homomorphism or \textsc{Fail}.

\smallskip
\STATE $x := \textsc{InvolutionMaker}(G/Z,100)$
\IF {$x = \textsc{Fail}$}
    \STATE \textbf{Return} \textsc{Fail}
\ENDIF
\STATE $i := 0$
\WHILE {\textsc{True}} 
    \STATE $i := i + 1$
    \STATE Compute the normal closure $N := \left< x \right>^G$ in $G$
    \STATE Run the MeatAxe on $V|_N$
    \IF {$V|_N$ is reducible}
        \IF {$V|_N$ is semisimple and has more than one homogeneous component}
            \STATE \textbf{Return} action on homogeneous components (\DD2)
        \ELSIF {$V|_N$ is semisimple but has only one homogeneous component}
            \STATE \textbf{Return} tensor product decomposition (\DD4)
        \ENDIF
    \ELSE 
        \STATE Perform check for \DD7 (see Section~\ref{solveD7})
    \ENDIF
    \IF {$i \ge 9$}
        \STATE \textbf{Return} \textsc{Fail}
    \ENDIF
    \STATE $y := \textsc{InvolutionJumper}(G/Z,x,100,\textsc{True})$
    \IF {$y = \textsc{Fail}$}
        \STATE \textbf{Return} \textsc{Fail}
    \ENDIF
\ENDWHILE
\end{algorithmic}
\end{algorithm}

We summarise the method to find a reduction in the case that $G$ lies in
\DD2 or \DD4 in Algorithm~\ref{algreductionD24} ``\textsc{FindHomD247}''. 
This algorithm
already contains a line to check for \DD7 which we will explain in
Section~\ref{solveD7}. We do not analyse this algorithm here but only
describe some observations and give a few heuristic arguments to try to
explain why it works so well. The reader is kindly referred to future
publications of the author for a more detailed analysis.

\begin{Obs}[Algorithm~\ref{algreductionD24} ``\textsc{FindHomD247}'' works 
well in practice]
\index{FindHomD247@\textsc{FindHomD247}}%
We have implemented Algorithm~\ref{algreductionD24}
``\textsc{FindHomD247}'' in {\GAP} and have
performed many experiments. To generate random elements we use the
\textsc{Rattle} algorithm (a variant of product replacement, see
\index{Rattle@\textsc{Rattle}}%
\cite{LGMurray, LGO97} and Section~\ref{rattle}). 
In all cases of groups $G$ in \DD2 or \DD4 we
looked at, only very few steps of Algorithm~\ref{alginvjump}
``\textsc{InvolutionJumper}'' were needed to produce an element of the
normal subgroup $N$ and thus to find a reduction. In typical examples
like $(\GL(3,3)^{\times 6}).S_6 \le \GL(18,3)$ (imprimitive \DD2) or $\Sp(6,3)
\otimes \mathrm{O}(7,3) \le \GL(48,3)$ (tensor decomposable \DD4) 
in average only slightly more than one
step was needed to produce an element of the normal subgroup. Thus trying
$9$ times seems to be enough to rule out \DD2 and \DD4.
\end{Obs}

\begin{Expl}[Heuristic explanation for Algorithm~\ref{algreductionD24} 
``\textsc{FindHomD247}'']
\index{FindHomD247@\textsc{FindHomD247}}%
For being contained in a normal subgroup $N \unlhd G$ only the
conjugacy class of an element $x \in G$ is relevant since $N$ is a union of
conjugacy classes of $G$. The Involution Jumper (see Algorithm~\ref{alginvjump}) 
basically performs a random walk through the involution classes of $G/Z$,
as is explained in Definition~\ref{markovinvclasses}. 
The corresponding Markov chain seems to have very good properties with
respect to visiting each vertex. 
To begin with, Proposition~\ref{markovinvlimit} shows that it is irreducible
and has a stationary distribution in which every class has a non-zero 
probability. Even better, the vertices representing
involution classes in normal subgroups seem to attract the Involution
Jumper. The fact that we are computing commutators seems to add to this
effect because we can hope to go down along the derived series of $G$.
For example in all cases where the commutator $[x,y]$ of the previous
involution with a random element has even order but has odd order in
$G/N$, the Involution Jumper immediately jumps into $N$!

The experimental evidence suggests that it is feasible to analyse this
procedure and to come up with a proof that it is in fact a Las Vegas
algorithm that works with high probability for all or at least most groups
in \DD2 and \DD4. It certainly has a good complexity in terms of $n$, the
number $m$ of generators of $G$ and $\log q$ (ignoring as usual the
discrete logarithm (see Section~\ref{thedlp}) and integer factorisation 
(see Section~\ref{intfact}) problems in order computations). 
The major question to be
resolved is, for which groups in \DD2 or \DD4 this
does not work well or not at all.
\end{Expl}

\smallskip
We conclude this section with a few references to the existing
literature. In \cite{smashnormal} an algorithm is described to decide
whether or not a group is acting imprimitively. The ideas in that
paper have actually inspired lots of the methods presented in this
chapter. An algorithm to decide whether a matrix group preserves a
tensor decomposition of the natural module is presented in
\cite{tensprodproj} and \cite{LGO97}.


\subsection{Finding a reduction in the tensor induced case: \DD7}
\label{solveD7}

The definition of class \DD7 is similar to the ones for \DD2 and \DD4 in
the sense that a normal subgroup $N$ of $G \le \GL(n,q)$ with $Z \le N$
for $Z = G \cap Z(\GL(n,q))$ is involved.
However, the restriction $V|_N$ of the natural module $V$ to $N$ is not
reducible in the \DD7 case. Thus the approach described in
Section~\ref{solveD24} does not work immediately. However, the observation
that Algorithm~\ref{alginvjump} ``\textsc{InvolutionJumper}'' produces an
element of $N$ with relatively high probability also holds for the \DD7
case. To some extent it seems to work even better in this case, since the 
factor group $G/N$ has an action on $k$ points, where $n=r^k$ for some $r$
and $k$, which makes $k$ relatively small in real world situations.

\begin{algorithm}
\caption{$\quad$ \textsc{FindHomD7}}
\index{FindHomD7@\textsc{FindHomD7}}%
\label{algreductionD7}
\begin{algorithmic}
\STATE \textbf{Input:} A group $G \le \GL(n,q)$, a normal subgroup $N$ known 
to act absolutely irreducibly 
\STATE \mbox{}\phantom{\textbf{Input:}} and a projective involution $x \in N$.
\STATE \textbf{Output:} A reduction homomorphism or \textsc{Fail}.

\smallskip
\IF {$n$ is not a $k$-th power for some $k \ge 2$}
    \STATE \textbf{Return} \textsc{Fail}
\ENDIF
\FOR {$i=1, \ldots, 3$}
    \STATE $y := \textsc{InvolutionJumper}(N/Z,x,100,\textsc{False})$
    \IF {$y \neq \textsc{Fail}$}
        \STATE $x := y$
    \ENDIF
\ENDFOR
\STATE Compute the normal closure $T := \left< x \right>^N$ in $N$
\STATE Run the MeatAxe on $V|_T$
\IF {$V|_T$ irreducible or not semisimple or not homogeneous}
    \STATE \textbf{Return} \textsc{Fail}
\ENDIF
\STATE Let $W$ be an irreducible $\F_q T$-submodule of $V|_T$ and $d :=
\dim_{\F_q}(W)$
\IF {$n$ is not a power of $d$}
    \STATE \textbf{Return} \textsc{Fail}
\ENDIF
\STATE Compute $k$ with $n=d^k$
\STATE Enumerate the orbit of $T$ under the conjugation action of $G$
\IF {orbit length $\neq k$ or no proper action}
    \STATE \textbf{Return} \textsc{Fail}
\ENDIF
\STATE \textbf{Return} action homomorphism as reduction
\end{algorithmic}
\end{algorithm}

There are basically two problems. The first is that we cannot easily detect 
that the normal closure of our involution produced by the Involution
Jumper is equal to $N$, because $V|_N$ is absolutely irreducible. The
second is that even if we assume this, we have to find the action of $G/N$ on
the direct factors of $N/Z$ (recall that in the \DD7 case the group $N/Z$
is isomorphic to a direct product of pairwise isomorphic non-abelian simple
groups that are permuted transitively by the conjugation action of $G/N$ on
$N$). In this section we present a new idea to solve these two problems
which works very well in practice. As for the \DD2 and \DD4 cases, the reader is
kindly referred to future publications by the author for a complete analysis.

Algorithm~\ref{algreductionD24} ``\textsc{FindHomD247}'' already contains a 
statement ``Perform check for \DD7'' in a situation in which we have produced
a candidate for the normal subgroup $N$. We now describe how to perform
this check, which is summarised in Algorithm~\ref{algreductionD7}
``\textsc{FindHomD7}''.
\index{FindHomD7@\textsc{FindHomD7}}%

First of all, if $n$ does not have a decomposition as
$n=r^k$, then we do nothing since $G$ cannot lie in \DD7. Otherwise,
we need a relatively quick check whether $N/Z$ is a direct product of
pairwise isomorphic non-abelian simple groups acting irreducibly. To
this end, we simply play the same trick again, since if $G$ is in \DD7,
then $N$ is in \DD4. That is, we run Algorithm~\ref{alginvjump}
``\textsc{InvolutionJumper}'' on $N/Z$. Note that we already have an
involution in $N/Z$, namely the one we used to produce $N$.

In comparison to the method for \DD2 and \DD4, we change two things,
firstly, we set the boolean argument $b$ to \textsc{False}, and
secondly, we call the Involution Jumper three times in a row (whenever
it returns \textsc{Fail}, we stick to the previous involution).
Both changes should improve our chances to produce an involution in $N/Z$
that is trivial in all but one of the direct factors of $N/Z$. With $b$ set
to \textsc{False} we only take commutators with random elements of $N$.
This means that once a component in the direct decomposition is trivial it
will stay trivial. Due to the fact that the number of direct factors is
relatively small, doing three steps will usually suffice to produce an
involution whose normal closure in $N/Z$ is exactly one direct factor.

Let now $T$ be the normal closure of the final involution of $N/Z$ we
produce. We run the MeatAxe on the restriction $V|_T$ of the natural module
and immediately give up if $V|_T$ is irreducible, not semisimple or not
homogeneous. Otherwise, we determine the dimension $d$ of an irreducible
submodule. If $n=d^k$ for some $k$, then we assume that $T$ is in fact
equal to one of the $k$ central factors of $N$.

We then try to enumerate the orbit of $G$ acting on the central factors of
$T$, immediately giving up if this orbit has a different length than $k$.

The only problem to solve for this is how to represent the conjugates of
$T$ on the computer and how to compare them. Since we know that in the \DD7
case $T/Z$ is a non-abelian simple group, this is both relatively easy. We
represent $T$ and its conjugates by a short generating set. To start this,
we produce $3$ random elements of $T$ making sure that the first one does
not commute with the two others. We then conjugate three-tuples of elements
with the generators of $G$. To compare two such three-tuples, we simply
check, whether or not the first element of the first tuple commutes with
every element in the second tuple. If so, the two conjugates of $T$ commute
and thus are not equal. Otherwise, we assume that the two conjugates are
equal. Since in the \DD7 case, $T/Z$ is non-abelian simple, this test is
good enough and can be done very quickly.

This whole procedure finds the action of $G$ on the direct factors of $N/Z$
with high probability and thus returns a reduction provided $G$ is in \DD7
and $N$ is the normal subgroup from the definition of the class \DD7.

Of course, all of this is no proof that this method works in all cases and
even less a full complexity analysis. However, in real life examples this
method seems to work extremely well and quickly fails if $G$ is not a \DD7 group.
Further work to analyse these ideas is required.

\enlargethispage{2\baselineskip}
We conclude this section with a few references to the existing
literature. In \cite{RecogTensInd} an algorithm is described to
determine algorithmically whether or not a group is projectively
tensor induced. This paper uses the methods in \cite{tensprodproj} and
\cite{LGO97}. Note that this covers the more general case of a
subgroup of a group in Aschbacher class \CC7.


\subsection{Finding a reduction in the extraspecial case: \DD6}
\label{solveC6}

Efficient methods to find a reduction in the \DD6 case can be found in
\cite{C6FindHom}.

